\documentclass[a4paper,UKenglish,cleveref,autoref,numberwithinsect]{lipics-v2019}

\usepackage{stmaryrd}
\usepackage{proof}

%\graphicspath{{./graphics/}}%helpful if your graphic files are in another directory

\bibliographystyle{plainurl}% the mandatory bibstyle

\theoremstyle{definition}
\newtheorem{defn}[theorem]{Definition}

\newcommand{\Fomega}{\mathtt{F}_\omega}

\newcommand{\Typevars}{\mathcal{A}}
\newcommand{\Vars}{\mathcal{V}}
\newcommand{\Rules}{\mathcal{R}}
\newcommand{\World}{\mathcal{W}}
\newcommand{\Iterms}{\mathcal{I}}
\newcommand{\ITypes}{\mathcal{Y}}

\newcommand{\arrkind}{\Rightarrow}
\newcommand{\arrtype}{\rightarrow}
\newcommand{\quant}[2]{\forall #1.#2}

\newcommand{\abstraction}[2]{\backslash #1.#2}
\newcommand{\app}[2]{#1 \cdot #2}
\newcommand{\tapp}[2]{#1 * #2}
\newcommand{\subst}[2]{#1:=#2}

\newcommand{\abs}[2]{\lambda #1.#2}
\newcommand{\tabs}[2]{\Lambda #1.#2}
\newcommand{\pair}[2]{\langle #1,#2 \rangle}
\newcommand{\expair}[2]{[#1,#2]}

\newcommand{\arrW}{\leadsto}
\newcommand{\arr}[1]{\longrightarrow_{#1}}
\newcommand{\red}{\longrightarrow}

\newcommand{\nat}{\mathtt{nat}}
\newcommand{\flatten}{\mathtt{flatten}}
\newcommand{\lift}{\mathtt{lift}}

\newcommand{\typeinterpret}[1]{\llbracket #1 \rrbracket}
\newcommand{\interpret}[1]{\llbracket #1 \rrbracket}
\newcommand{\itp}[1]{\llbracket #1 \rrbracket}

\newcommand{\refsec}[1]{Section~\ref{sec:#1}}

\newcommand{\FTV}{\mathrm{FTV}}
\newcommand{\FV}{\mathrm{FV}}
\newcommand{\Tc}{\mathcal{T}}
\newcommand{\Vc}{\mathcal{V}}

\newcommand{\cl}{\mathcal{C}}
\newcommand{\dom}{\mathrm{dom}}
\newcommand{\nf}{\mathrm{nf}}

\newcommand{\da}{\mathord{\downarrow}}
\newcommand{\SN}{\mathrm{SN}}
\newcommand{\Cb}{\mathbb{C}}
\newcommand{\Nbb}{\mathbb{N}}
\newcommand{\val}[3]{\ensuremath{\llbracket#1\rrbracket_{#2}^{#3}}}
\newcommand{\gteq}[3]{\ensuremath{\ge_{#1}^{#2,#3}}}

\newcommand{\proves}{\vdash}

\newcommand{\Typemap}{\mathcal{T\!M}}
\newcommand{\Termmap}{\mathcal{J}}
\newcommand{\succinterpret}{\succ^{\Termmap}}
\newcommand{\succeqinterpret}{\succeq^{\Termmap}}

\newcommand{\List}{\mathtt{List}}
\newcommand{\nil}{\mathtt{nil}}
\newcommand{\cons}{\mathtt{cons}}
\newcommand{\fold}{\mathtt{fold}}
\newcommand{\xlet}[4]{\mathtt{let}_{#1}\,#2\,\mathtt{be}\,[#3]\,\mathtt{in}\,#4}

\newcommand{\CK}[1]{\textcolor{blue}{CK: #1}}
\newcommand{\CKchange}[1]{\textcolor{blue}{#1}}
\newcommand{\LC}[1]{\textcolor{red}{LC: #1}}

\title{Polymorphic Higher-order Termination}

%\titlerunning{Dummy short title}

\author{{\L}ukasz Czajka}{Faculty of Informatics, TU Dortmund, Germany \and \url{http://www.mimuw.edu.pl/~lukaszcz/} }{lukaszcz@mimuw.edu.pl}{https://orcid.org/0000-0001-8083-4280}{}

\author{Cynthia Kop}{Institute of Computer Science, Radboud University Nijmegen, Netherlands \and \url{https://www.cs.ru.nl/~cynthiakop/}}{c.kop@cs.ru.nl}{https://orcid.org/0000-0002-6337-2544}{}

\authorrunning{\L. Czajka and C. Kop}

\Copyright{{\L}ukasz Czajka and Cynthia Kop}

\ccsdesc[500]{Theory of computation~Rewrite systems}
\ccsdesc[500]{Theory of computation~Equational logic and rewriting}
\ccsdesc[300]{Theory of computation~Type theory}

\keywords{termination, polymorphism, higher-order rewriting, permutative conversions}

%\category{}%optional, e.g. invited paper

%\relatedversion{A full version of the paper is available at \url{...}.}
%\supplement{}%optional, e.g. related research data, source code, ... hosted on a repository like zenodo, figshare, GitHub, ...

%\acknowledgements{I want to thank \dots}%optional

%\nolinenumbers %uncomment to disable line numbering

%\hideLIPIcs  %uncomment to remove references to LIPIcs series (logo, DOI, ...), e.g. when preparing a pre-final version to be uploaded to arXiv or another public repository

%Editor-only macros:: begin (do not touch as author)%%%%%%%%%%%%%%%%%%%%%%%%%%%%%%%%%%
\EventEditors{John Q. Open and Joan R. Access}
\EventNoEds{2}
\EventLongTitle{42nd Conference on Very Important Topics (CVIT 2016)}
\EventShortTitle{CVIT 2016}
\EventAcronym{CVIT}
\EventYear{2016}
\EventDate{December 24--27, 2016}
\EventLocation{Little Whinging, United Kingdom}
\EventLogo{}
\SeriesVolume{42}
\ArticleNo{23}
%%%%%%%%%%%%%%%%%%%%%%%%%%%%%%%%%%%%%%%%%%%%%%%%%%%%%%

\begin{document}

\maketitle

\begin{abstract}
  We generalise the termination method of higher-order polynomial
  interpretations to a setting with impredicative
  polymorphism. Instead of using weakly monotonic functionals, we
  interpret terms in a suitable extension of System~$\Fomega$. In
  addition to enabling a direct interpretation of rewrite rules which
  make essential use of impredicative polymorphism, thanks to the
  possibility of encoding inductive data types in the polymorphic
  lambda-calculus, our generalisation eases the applicability of the
  method also in the non-polymorphic setting. As an illustration of
  the potential of our method, we prove termination of a substantial
  fragment of full intuitionistic second-order propositional logic
  with permutative conversions.
\end{abstract}

\section{Introduction}

Termination of higher-order term rewriting
systems~\cite[Chapter~11]{Terese2003} has been an active area of
research for several decades. Higher-order polynomial
interpretations~\cite{pol:96,Kop2012,FuhsKop2012}, introduced by Jaco
van de Pol~\cite{Pol1993,pol:96}, generalise polynomial
interpretations -- a termination method complete for first-order
rewriting.

In this paper, we further generalise higher-order polynomial
interpretations to a higher-order formalism with full impredicative
polymorphism. In the term rewriting literature, polymorphic
higher-order frameworks are usually restricted to shallow (rank-1,
weak) polymorphism, i.e., type quantifiers are effectively allowed
only at the top of a type. While shallow polymorphism often suffices
in functional programming practice, there do exist interesting
examples of rewrite systems which require higher-rank impredicative
polymorphism.

For instance, in recent extensions of Haskell or in some type-theory
based proof assistants (e.g.~Coq) one may define a type of
heterogenous lists.
\[
\begin{array}{l}
  \List : * \\
  \mathtt{nil} : \List \\
  \mathtt{cons} : \forall \alpha . \alpha \arrtype \List \arrtype \List \\
  \mathtt{foldl} : \forall \beta . (\forall \alpha . \beta \arrtype \alpha \arrtype \beta) \arrtype \beta \arrtype \List \arrtype \List \\
  \mathtt{foldl}_\sigma(f,a,\nil) \red a \\
  \mathtt{foldl}_\sigma(f,a,\cons_\tau(x,l)) \red \mathtt{foldl}_\sigma(f,f \tau a x,l)
\end{array}
\]
The above states that $\List$ is a type ($*$), states the types of its
two constructors $\nil$ and $\cons$, and defines the corresponding
fold-left function~$\mathtt{foldl}$. Each element of a heterogenous
list may have a different type. Together with the element, the
function argument of~$\mathtt{foldl}$ receives the element's type. The
$\forall$-quantifier binds type variables -- a term of type $\forall
\alpha . \tau$ takes a type~$\rho$ as an argument and the result is a
term of type~$\tau[\subst{\alpha}{\rho}]$.

Polymorphism in type theory is parametric, which means that a
polymorphic function behaves uniformly with respect to its generic
arguments, not depending on their actual types. This makes it
impossible to directly extract elements from a heterogenous list -- we
can only perform some operations on them under the assumption that we
know nothing about their types. In practice, one would constrain the
type variable~$\alpha$ above with a type class, in order to guarantee
the existence of certain functions on the elements of a list.

Impredicativity of polymorphism means that the type itself may be
substituted for its own type variable, e.g., if $\mathtt{f} : \forall
\alpha . \tau$ then $f (\forall \alpha . \tau) :
\tau[\subst{\alpha}{\forall\alpha.\tau}]$. This prevents a translation
into an infinite set of simply typed rules by instantiating the type
variables. The above example is not directly reducible to shallow
polymorphism as used in the~ML programming language or the polymorphic
higher-order rewriting formalisms of~\cite{jou:oka:91,jou:rub:99}.

We will use the above rules for fold on heterogenous lists as our
running example, but their termination could be shown by encoding them
in System~$\mathtt{F}$. In Section~\ref{sec:examples}, we study a more
complex example -- termination of a substantial fragment of~IPC2,
i.e., full intuitionistic second-order propositional logic with
permutative conversions. Permutative
conversions~\cite[Chapter~6]{TroelstraSchwichtenberg1996} are used in
proof theory to obtain ``good'' normal forms of natural deduction
proofs, which satisfy e.g.~the subformula property. Termination proofs
for systems with permutative conversions are notoriously tedious and
difficult, with some incorrect claims in the literature and no uniform
methodology. Pol and Schwichtenberg used higher-order polynomial
interpretations to prove termination of a fragment of first-order
logic with permutative conversions~\cite{PolSchwichtenberg1995}, in
the hope of providing a more perspicuous proof of this well-known
result. Notably, they did not treat disjunction. The
paper~\cite{SorensenUrzyczyn2010} depends on termination of~IPC2,
citing a proof from~\cite{Wojdyga2008}, which, however, later turned
out to be incorrect. To our knowledge, termination of the full
system~IPC2 remains an open problem.

\section{Preliminaries}\label{sec_preliminaries}

In this section we introduce System~$\Fomega$ (see
e.g.~\cite[Section~11.7]{SorensenUrzyczyn2006}), which will form a
basis both of our interpretations and of a general syntactic framework
for the investigated systems.

\begin{defn}\label{def_types}
  \emph{Kinds} are defined inductively: $*$ is a kind, and if
  $\kappa_1,\kappa_2$ are kinds then so is $\kappa_1 \arrkind
  \kappa_2$. We assume an infinite set~$\Vc_\kappa$ of \emph{type
    constructor variables} of each kind~$\kappa$. Variables of
  kind~$*$ are \emph{type variables}. We assume a fixed
  set~$\Sigma^T_\kappa$ of \emph{type constructor symbols} paired of
  kind~$\kappa$, denoted $c : \kappa$.

  We define the set~$\Tc_\kappa$ of \emph{type constructors} of
  kind~$\kappa$ by the following grammar. Type constructors of
  kind~$*$ are \emph{types}.
  \[
  \begin{array}{rcl}
    \Tc_{*} &::=& \Vc_{*}
    \mid \Sigma^T_{*} \mid
    \Tc_{\kappa\arrkind *}\Tc_{\kappa} \mid \forall\Vc_\kappa\Tc_* \mid \Tc_*\arrtype\Tc_* \\
    \Tc_{\kappa_1\arrkind\kappa_2} &::=& \Vc_{\kappa_1\arrkind\kappa_2}
    \mid \Sigma^T_{\kappa_1\arrkind\kappa_2} \mid
    \Tc_{\kappa\arrkind(\kappa_1\arrkind\kappa_2)}\Tc_{\kappa} \mid \lambda \Vc_{\kappa_1} \Tc_{\kappa_2}
  \end{array}
  \]

  We use the standard notations $\forall \alpha . \tau$ and $\lambda
  \alpha . \tau$. When $\alpha$ is of kind $\kappa$ then we use the
  notation $\forall \alpha : \kappa . \tau$. If not indicated
  otherwise, we assume~$\alpha$ to be a type variable. We treat type
  constructors up to $\alpha$-conversion.

  The compatible closure of the rule $(\lambda\alpha.\varphi)\psi \to
  \varphi[\alpha := \psi]$ defines $\beta$-reduction on type
  constructors. Note that type constructors are simply-typed
  lambda-terms, so $\beta$-reduction on type constructors terminates
  and is confluent, hence every type constructor~$\tau$ has a unique
  $\beta$-normal form~$\nf_\beta(\tau)$. A \emph{type atom} is a type
  in $\beta$-normal form which is not an arrow $\tau_1\arrtype\tau_2$
  or a quantification $\forall\alpha\tau$.

  We define $\FV(\varphi)$ -- the set of free type constructor
  variables of the type constructor~$\varphi$ -- in an obvious way by
  induction on~$\varphi$. A type constructor~$\varphi$ is
  \emph{closed} if $\FV(\varphi) = \emptyset$.

  We assume a fixed type symbol~$\chi_* \in
  \Sigma^T_\kappa$. For $\kappa=\kappa_1\arrkind\kappa_2$ we define
  $\chi_\kappa = \lambda \alpha:\kappa_1 . \chi_{\kappa_2}$.
\end{defn}

\begin{defn}\label{def_preterms}
  We assume given an infinite set $\Vars$ of variables, and let
  $\Gamma$ refer to a variable context -- a set of pairs $x : \tau$ of
  a variable and a type, such that each variable occurs at most
  once. We assume given a fixed set $\Sigma$ of \emph{function
    symbols}, each paired with a closed type, denoted $\mathtt{f} :
  \tau$.  Every function symbol $\mathtt{f}$ occurs only with one type
  declaration. We use $\Gamma,x:\tau$ for $\Gamma \cup \{x:\tau\}$. We
  set $\FTV(\Gamma) = \bigcup\{\FV(\tau) \mid (x : \tau) \in
  \Gamma\}$.

  The set of preterms consists of all expressions~$s$ such that
  $\Gamma \vdash s : \sigma$ can be inferred for some type $\sigma$
  and context $\Gamma$ by the following clauses:
  \begin{itemize}
  \item $\Gamma \vdash x : \sigma$ for every $(x : \sigma) \in \Gamma$.
  \item $\Gamma \vdash \mathtt{f} : \sigma$ for all
    $(\mathtt{f} : \sigma) \in \Sigma$.
  \item $\Gamma \vdash \abs{x:\sigma}{s} : \sigma \arrtype \tau$ if $x
    \in \Vars$ and $\Gamma, x : \sigma \vdash s : \tau$.
  \item $\Gamma \vdash \tabs{\alpha}{s} : \quant{\alpha}{\sigma}$ if
    $\Gamma \vdash s : \sigma$ and $\alpha \notin \FTV(\Gamma)$.
  \item $\Gamma \vdash \app{s}{t} : \tau$ if $\Gamma \vdash s : \sigma
    \arrtype \tau$ and $\Gamma \vdash t : \sigma$
  \item $\Gamma \vdash \tapp{s}{\tau} : \sigma[\subst{\alpha}{\tau}]$
    if $\Gamma \vdash s : \quant{\alpha:\kappa}{\sigma}$ and~$\tau$ is
    a type constructor of kind~$\kappa$,
  \item $\Gamma \vdash s : \tau$ if $\Gamma \vdash s : \tau'$ and
    $\tau =_\beta \tau'$.
  \end{itemize}
  The set of free variables of a preterm~$t$, denoted $\FV(t)$, is
  defined in the expected way. Analogously, we define the
  set~$\FTV(t)$ of type constructor variables occurring free
  in~$t$. We say that $t$ is \emph{closed} if $\FV(t) = \emptyset$ and
  $\FTV(t) = \emptyset$. If $\alpha$ is a type constructor variable of
  kind~$\kappa$ then we use the notation $\tabs{\alpha:\kappa}{t}$.
  We define the equivalence relation~$\equiv$ by: $s \equiv t$ iff $s$
  and $t$ are identical modulo $\beta$-conversion in types.
\end{defn}

\begin{lemma}
  If $\Gamma \vdash s : \tau$ and $s \equiv t$ then $\Gamma \vdash t :
  \tau$.
\end{lemma}

\begin{proof}
  Induction on~$s$.
\end{proof}

\begin{defn}\label{def_terms}
  The set of \emph{terms} is the set of the equivalence classes
  of~$\equiv$.
\end{defn}

Because $\beta$-reduction on types is confluent and terminating, every
term has a canonical preterm representative -- the one with all types
occurring in it $\beta$-normalized. We say that a term is
\emph{closed} if its canonical representative is. We define $\FTV(t)$
as the value of~$\FTV$ on the canonical representative of~$t$.

Because typing and term formation operations (abstraction,
application, \ldots) are invariant under~$\equiv$, we may denote terms
by their (canonical) representatives and informally treat them
interchangeably.

We will often abuse notation to omit $\cdot$ and $*$. Thus, $s t$ can
refer to both $\app{s}{t}$ and $\tapp{s}{t}$. This is not ambiguous
due to typing. We will also use $\abstraction{a}{s}$ for either
$\abs{a}{s}$ or $\tabs{a}{s}$.

\begin{lemma}[Substitution lemma]
  \begin{enumerate}
  \item If $\Gamma, x : \sigma \proves s : \tau$ and $\Gamma \proves t
    : \sigma$ then $\Gamma \proves s[\subst{x}{t}] : \tau$.
  \item If $\Gamma \proves t : \sigma$ then
    $\Gamma[\subst{\alpha}{\tau}] \proves t[\subst{\alpha}{\tau}] :
    \sigma[\subst{\alpha}{\tau}]$.
  \end{enumerate}
\end{lemma}

\begin{proof}
  Induction on the typing derivation.
\end{proof}

\begin{lemma}[Generation lemma]
  Assume $\Gamma \proves t : \sigma$ and let $\Vc = \FTV(t) \cup
  \FTV(\Gamma)$. Then there is a type~$\sigma'$ such that $\sigma'
  =_\beta \sigma$ and $\FV(\sigma') \subseteq \Vc$ and one of the
  following holds.
  \begin{itemize}
  \item $t \equiv x$ is a variable and $(x : \sigma') \in \Gamma$.
  \item $t \equiv \mathtt{f}$ is a function symbol with $\mathtt{f} :
    \sigma'$ in $\Sigma$.
  \item $t \equiv \abs{x:\tau_1}{s}$ and
    $\sigma'=\tau_1\arrtype\tau_2$ and $\Gamma, x : \tau_1 \vdash s :
    \tau_2$.
  \item $t \equiv \tabs{\alpha}{s}$ and $\sigma' =
    \quant{\alpha}{\tau}$ and $\Gamma \vdash s : \tau$ and $\alpha
    \notin \FTV(\Gamma)$.
  \item $t \equiv \app{t_1}{t_2}$ and
    $\Gamma \vdash t_1 : \tau \arrtype \sigma'$ and
    $\Gamma \vdash t_2 : \tau$ and $\FV(\tau) \subseteq \Vc$.
  \item $t \equiv \tapp{s}{\tau}$ and
    $\sigma' = \rho[\subst{\alpha}{\tau}]$ and
    $\Gamma \vdash s : \quant{(\alpha:\kappa)}{\rho}$ and~$\tau$ is a
    type constructor of kind~$\kappa$.
  \end{itemize}
\end{lemma}

\begin{proof}
  By induction on the derivation $\Gamma \proves t : \sigma$. Note
  that if $\alpha \notin \Vc$ is of kind~$\kappa$ and~$\Gamma' \proves
  s : \sigma'$ with~$\FTV(s) \cup \FTV(\Gamma') \subseteq \Vc$, then
  $\Gamma \proves s : \sigma'[\subst{\alpha}{\chi_\kappa}]$ by the
  substitution lemma.
\end{proof}

For convenience, we sometimes assume without loss of generality that
the terms are given in orthodox Church-style, i.e., instead of using
contexts we assume that each variable occurrence is annotated with a
type (where two occurrences of the same variable must be annotated
with the same type). Note that given a context~$\Gamma$ under which
all considered terms are typable, there is a natural isomorphism
between typed terms as defined above and terms given in orthodox
Church-style. We write $t : \tau$ if $t$ has type~$\tau$, in a fixed
implicit context~$\Gamma$, or equivalently with~$t$ considered as an
orthodox Church-style typed term.

\subsection{Encodings of inductive types}\label{sec_encodings}

The polymorphic lambda-calculus has a much greater expressive power
than the simply-typed lambda-calculus. Arbitrary inductive data types
may be encoded, together with their constructors and recursors with
appropriate derived reduction rules. This makes the polynomial
interpretation method easier to apply, even in the non-polymorphic
setting, thanks to more sophisticated ``programming'' in the
interpretations.

The reader is advised to consult e.g.~\cite[Chapter~11]{Girard1989}
for more background. Here, we only define abbreviations for a few
inductive types used later. The lemmas stated in this section follow
easily from definitions, after unfolding the abbreviations. We omit
the proofs. Type subscripts are dropped when clear or irrelevant.

\begin{defn}[Product types]
  \[
  \begin{array}{rcl}
    \sigma \times \tau &=& \forall p . (\sigma \arrtype \tau \arrtype p) \arrtype p \\
    \pair{t_1}{t_2}_{\sigma,\tau} &=& \tabs{p}{\abs{x:\sigma\arrtype\tau\arrtype p}{x t_1 t_2}} \\
    \pi^1_{\sigma,\tau}(t) &=& t \sigma (\abs{x:\sigma}{\abs{y:\tau}{x}}) \\
    \pi^2_{\sigma,\tau}(t) &=& t \tau (\abs{x:\sigma}{\abs{y:\tau}{y}})
  \end{array}
  \]
\end{defn}

\begin{lemma}
  \begin{enumerate}
  \item If $\Gamma \proves t_1 : \sigma$ and $\Gamma \proves t_2 :
    \tau$ then $\Gamma \proves \pair{t_1}{t_2} : \sigma \times \tau$.
  \item If $\Gamma \proves t : \tau_1 \times \tau_2$ then $\Gamma
    \proves \pi^i(t) : \tau_i$.
  \end{enumerate}
\end{lemma}

\begin{lemma}
  $\pi^i(\pair{t_1}{t_2}) \leadsto^* t_i$
\end{lemma}

\begin{lemma}
  If $t \succ t'$ then $\pi^i(t) \succ \pi^i(t')$.
\end{lemma}

\begin{defn}[Existential types]
  \[
  \begin{array}{rcl}
    \Sigma \alpha . \sigma &=& \forall p . (\forall \alpha . \sigma \arrtype p) \arrtype p \\
    \expair{\tau}{t}_{\Sigma\alpha.\sigma} &=& \tabs{p}{\abs{x:\forall\alpha.\sigma\arrtype p}{x \tau t}} \\
    \xlet{\rho}{t}{\alpha,x:\sigma}{s} &=& t \rho (\tabs{\alpha}{\abs{x:\sigma}{s}})
  \end{array}
  \]
\end{defn}

\begin{lemma}
  \begin{enumerate}
  \item If $\Gamma \proves t : \sigma[\subst{\alpha}{\tau}]$ then
    $\Gamma \proves \expair{\tau}{t} : \Sigma \alpha . \sigma$.
  \item If $\Gamma \proves t : \Sigma \alpha . \sigma$ and
    $\Gamma,x:\sigma \proves s : \rho$ and $\alpha \notin
    \FTV(\Gamma,\rho)$ then $\xlet{}{t}{\alpha,x}{s} : \rho$.
  \end{enumerate}
\end{lemma}

\begin{lemma}
  $\xlet{}{\expair{\tau}{t}}{\alpha,x}{s} \leadsto
  s[\subst{\alpha}{\tau}][\subst{x}{t}]$.
\end{lemma}

\begin{lemma}
  If $t \succ t'$ then $\xlet{}{t}{\alpha,x}{s} \succ
  \xlet{}{t'}{\alpha,x}{s}$.
\end{lemma}

\begin{defn}[Heterogenous lists]
  \[
  \begin{array}{rcl}
    \List &=& \forall p . (\forall \alpha . \alpha \arrtype p \arrtype p) \arrtype p \arrtype p \\
    \nil &=& \tabs{p}{\abs{f:\forall \alpha . \alpha \arrtype p \arrtype p}{\abs{a : p}{a}}} \\
    \cons_\rho(h,t) &=& \tabs{p}{\abs{f:\forall\alpha . \alpha \arrtype p \arrtype p}{\abs{a : p}{f \rho h (l p f a)}}} \\
    \fold_\rho(f,a,l) &=& l \rho f a
  \end{array}
  \]
\end{defn}

\begin{lemma}
  \begin{enumerate}
  \item $\Gamma \proves \nil : \List$.
  \item If $\Gamma \proves h : \rho$ and $\Gamma \proves t : \List$
    then $\Gamma \proves \cons_\rho(h,t) : \List$.
  \item If $\Gamma \proves l : \List$ and $\Gamma \proves f : \forall
    \alpha . \alpha \arrtype \rho \arrtype \rho$ and $a : \rho$ then
    $\Gamma \proves \fold_\rho(f,a,l) : \rho$.
  \end{enumerate}
\end{lemma}

\begin{lemma}
  \begin{enumerate}
  \item $\fold_\rho(f,a,\nil) \leadsto a$.
  \item $\fold_\rho(f,a,\cons_\tau(h,t)) \leadsto f \tau h
    (\fold_\rho(f,a,t))$.
  \end{enumerate}
\end{lemma}

\begin{lemma}
  If $l \succ l'$ then $\fold_\rho(f,a,l) \succ \fold_\rho(f,a,l')$.
\end{lemma}

\section{Polymorphic Algebraic Functional Systems}\label{sec_systems}

To represent the rewrite systems whose termination we are going to
analyse, we define the framework of Polymorphic Algebraic Functional
Systems (PAFS). We use a syntax based on System~$\Fomega$,
specialising Section~\ref{sec_preliminaries}.

\begin{defn}\label{def_pafs_types_terms}
  \emph{Kinds}, \emph{type constructors} and \emph{types} are defined
  like in Definition~\ref{def_types}, parameterised by a fixed
  set~$\Sigma^T = \bigcup_{\kappa}\Sigma^T_\kappa$ of type constructor
  symbols.

  Given a fixed set~$\Sigma$ of function symbols, we define \emph{PAFS
    terms} like in Definition~\ref{def_terms} (based on
  Definition~\ref{def_preterms}) with the following restrictions:
  \begin{itemize}
  \item if $\mathtt{f} : \sigma \in \Sigma$ then
    \[
    \sigma = \forall (\alpha_1 : \kappa_1) \ldots \forall (\alpha_n : \kappa_n)
    . \sigma_1 \arrtype \ldots \arrtype \sigma_k \arrtype \tau
    \]
    with~$\tau$ a type atom,
  \item for any subterm $\app{s}{u}$ of a term~$t$, we have $s =
    \mathtt{f} \rho_1 \ldots \rho_n u_1 \ldots u_m$ where
    \[
    \mathtt{f} : \forall (\alpha_1 : \kappa_1) \ldots
    \forall (\alpha_n : \kappa_n) . \sigma_1 \arrtype \ldots \arrtype
    \sigma_k \arrtype \tau
    \]
    with $\tau$ a type atom and $m < k$.
  \end{itemize}
\end{defn}

We use the notation
$\mathtt{f}_{\rho_1,\ldots,\rho_n}(s_1,\ldots,s_k)$ for
$\mathtt{f} \rho_1 \ldots \rho_n s_1 \ldots s_k$ when
\[
  \mathtt{f} : \forall (\alpha_1 : \kappa_1) \ldots
  \forall (\alpha_n : \kappa_n) . \sigma_1 \arrtype \ldots \arrtype
  \sigma_k \arrtype \tau
\]
is a function symbol in~$\Sigma$ with~$\tau$ a type atom, and $\rho_i$
is a type constructor of kind $\kappa_i$ for $i=1,\ldots,k$, and
$\Gamma \proves s_i : \sigma_i[\alpha_1 := \rho_1]\ldots[\alpha_n :=
  \rho_n]$ for $i=1,\ldots,k$, for an appropriate~$\Gamma$. We use
this notation to stress the fact that by default there is no explicit
application available. The application in the syntax of terms is used
just for convenience, but does not correspond to the usual application
operator since only applications of the form $\mathtt{f} \rho_1 \ldots
\rho_n u_1 \ldots u_m$ are allowed. True application can be modelled
by including the symbol ${@} : \forall\alpha\forall\beta . (\alpha
\arrtype \beta) \arrtype \alpha \arrtype \beta$ in
$\Sigma$. Similarly, type application is modelled through a symbol
$\mathtt{A} : \forall \alpha : * \arrkind * . \forall \beta . (\forall
\beta [\alpha \beta]) \arrtype \alpha \beta$.

\begin{lemma}
  If $t,s$ are PAFS terms then so is $t[\subst{x}{s}]$.
\end{lemma}

\begin{defn}\label{def_replacement}
  A \emph{$\Gamma$-replacement} is a function $\delta = \gamma \circ
  \omega$ satisfying:
  \begin{enumerate}
  \item $\omega$ is a type constructor substitution,
  \item $\gamma$ is a term substitution such that $\omega(\Gamma)
    \proves \gamma(x) : \omega(\tau)$ for every $(x : \tau) \in
    \Gamma$.
  \end{enumerate}

  For~$\tau$ a type constructor, we use $\delta(\tau)$ to denote
  $\omega(\tau)$. We use the notation $\delta[\subst{x}{t}] =
  \gamma[\subst{x}{t}] \circ \omega$. Note that if $t \in \Iterms$ and
  $\Gamma \proves t : \tau$ then $\delta(\Gamma) \proves \delta(t) :
  \delta(\tau)$.
\end{defn}

The rewrite rules are simply a set of term pairs, whose monotonic
closure generates the rewrite relation. We fix a variable
context~$\Gamma$.

\begin{defn}\label{def_rules}
  A \emph{$\sigma$-context}~$C_\sigma$ is a term with a fresh function
  symbol $\Box_\sigma \notin \Sigma$ of type~$\sigma$ occurring
  exactly once. By~$C_\sigma[t]$ we denote a term obtained
  from~$C_\sigma$ by substituting~$t$ for~$\Box_\sigma$. We drop the
  $\sigma$ subscripts when clear or irrelevant.

  A set $\Rules$ of term pairs $(\ell,r)$ is a set of \emph{rewrite
    rules} if:
  \begin{itemize}
  \item $\FV(r) \subseteq \FV(\ell) \subseteq \mathit{keys}(\Gamma)$;
  \item $\ell$ and $r$ have the same type under $\Gamma$;
  \item if $(\ell,r) \in \Rules$ then $(\delta(\ell),\delta(r)) \in
    \Rules$ for any replacement~$\delta$.
  \end{itemize}
  The reduction relation $\arr{\Rules}$ is defined by: $t \arr{\Rules}
  s$ iff $t = C[\ell]$ and $s = C[r]$ for some $(\ell,r)\in\Rules$ and
  a context~$C$.
\end{defn}

\begin{defn}\label{def_pafs}
  A \emph{Polymorphic Algebraic Functional System (PAFS)} is a triple
  $(\Sigma^T,\Sigma,\Rules)$ where~$\Sigma^T$ is a set of type
  constructor symbols, $\Sigma$ a set of function symbols (restricted
  as in Definition~\ref{def_pafs_types_terms}), and $\Rules$ is a set
  of rules as in Definition~\ref{def_rules}. A term of a given
  PAFS~$A$ is referred to as an $A$-term.
\end{defn}

\begin{example}[Fold on heterogenous lists]\label{ex_fold_pafs}
  The example from the introduction may be represented as a PAFS with
  one type symbol $\mathtt{List} : *$, the following function symbols:
  \[
  \begin{array}{rcl}
    @ & : & \forall \alpha \forall \beta . (\alpha \arrtype \beta) \arrtype \alpha \arrtype \beta \\
    \mathtt{A} & : & \forall \alpha : * \arrkind * . \forall \beta .
    (\forall \beta [\alpha \beta]) \arrtype \alpha \beta \\
    \mathtt{nil} & : & \List \\
    \mathtt{cons} & : & \forall \alpha . \alpha \arrtype \List \arrtype \List \\
    \mathtt{foldl} & : & \forall \beta . (\forall \alpha . \beta \arrtype \alpha \arrtype \beta) \arrtype \beta \arrtype \List \arrtype \List
  \end{array}
  \]
  and the following rules:
  \[
  \begin{array}{rcl}
    @_{\sigma,\tau}(\abs{x:\sigma}{s},t) & \red & s[x:=t] \\
    \mathtt{A}_{\tabs{\alpha}{\sigma},\tau}(\tabs{\alpha}{s}) & \red &
    s[\alpha:=\tau] \\
    \mathtt{foldl}_\sigma(f,a,\nil) & \red & a \\
    \mathtt{foldl}_\sigma(f,a,\cons_\tau(x,l)) & \red & \mathtt{foldl}_\sigma(f,@_{\tau,\sigma}(@_{\sigma,\tau\arrtype\sigma}(\mathtt{A}_{\tabs{\alpha}{\sigma\arrtype\alpha\arrtype\sigma},\tau}(f),a),x),l)
  \end{array}
  \]
  Formally, the above represent an infinite set of rules, one rule for
  each substitution of~PAFS terms for $s$, $t$, etc.
\end{example}

\section{A well-ordered set of interpretation terms}

In this section we define the set~$\Iterms$ of interpretation terms,
and the relations~$\succeq$ and~$\succ$ on~$\Iterms$. We also define
the set~$\Iterms^f$ of final interpretation terms as the set of all
closed interpretation terms normalized with a certain reduction
relation. We will use interpretation terms from~$\Iterms$ to interpret
terms of~PAFSs. The pair $(\succeq,\succ)$ will be used as a reduction
pair for the method of rule removal. In particular, the
relation~$\succ$ will be well-founded.

\subsection{Interpretation terms}

\begin{defn}\label{def_iterms}
  The set~$\ITypes$ of \emph{interpretation types} is the set of types
  as in Definition~\ref{def_types} with $\Sigma^T = \{ \nat : * \}$,
  i.e., there is a single type constant~$\nat$. Then $\chi_* = \nat$.

  The set~$\Iterms$ of \emph{interpretation terms} is the set of terms
  from Definition~\ref{def_terms} (see also
  Definition~\ref{def_preterms}) where as types we take the
  interpretation types and as the set~$\Sigma$ of function symbols we
  take $\Sigma = \{ n : \nat \mid n \in \Nbb \} \cup \Sigma_f$, where:
  \[
      \Sigma_f = \{ \oplus : \forall \alpha . \alpha \arrtype
                 \alpha \arrtype \alpha, \otimes : \forall \alpha . \alpha \arrtype \alpha
                 \arrtype \alpha, \flatten : \forall \alpha . \alpha \arrtype
                 \nat, \lift : \forall \alpha . \nat \arrtype \alpha
                 \}
  \]
\end{defn}

For easier presentation, we write $\oplus_\tau$, $\otimes_\tau$, etc.,
instead of $\tapp{\oplus}{\tau}$, $\tapp{\otimes}{\tau}$, etc. We will
also use $\oplus$ and $\otimes$ in \emph{infix, left-associative}
notation, and omit the type denotation where it is clear from
context. Thus, $s \oplus t \oplus u$ should be read as
$\oplus_\sigma\,s\,(\oplus_\sigma\,t\,u)$ if $s$ has type $\sigma$.

To define \emph{final interpretation terms}, we will normalise certain
elements of~$\Iterms$ using~$\arrW{}$.

\begin{defn}
  We define the relation $\arrW$ on interpretation terms as the
  smallest relation for which the following properties are satisfied:
  \begin{enumerate}
  \item\label{arrW:mono:abs}
    if $s \arrW t$ then both $\abs{x}{s} \arrW \abs{x}{t}$ and
    $\tabs{\alpha}{s} \arrW \tabs{\alpha}{t}$
  \item\label{arrW:mono:right}
    if $s \arrW t$ then $\app{u}{s} \arrW \app{u}{t}$
  \item\label{arrW:mono:left}
    if $s \arrW t$ then both $\app{s}{u} \arrW \app{t}{u}$ and
    $\tapp{s}{\sigma} \arrW \tapp{t}{\sigma}$
  \item\label{arrW:beta:abs} $\app{(\abs{x:\sigma}{s})}{t} \arrW
    s[\subst{x}{t}]$
  \item\label{arrW:beta:tabs} $\tapp{(\tabs{\alpha}{s})}{\sigma}
    \arrW s[\subst{\alpha}{\sigma}]$.
  \item\label{arrW:plus:base}
    $\app{\app{\oplus_{\nat}}{n}}{m} \arrW (n+m)$
  \item\label{arrW:times:base} $\app{\app{\otimes_{\nat}}{n}}{m}
    \arrW (n \cdot m)$
  \item\label{arrW:circ:arrow} $\app{\app{\circ_{\sigma \arrtype
        \tau}}{s}}{t} \arrW
    \abs{x:\sigma}{\app{\app{\circ_\tau}{(\app{s}{x})}}{(\app{t}{x})}}$
    for $\circ \in \{ \oplus, \otimes \}$
  \item\label{arrW:circ:forall}
    $\app{\app{\circ_{\quant{\alpha}{\sigma}}}{s}}{t} \arrW
    \tabs{\alpha}{\app{\app{\circ_\sigma}{(\tapp{s}{\alpha})}}{(
        \tapp{t}{\alpha})}}$ for $\circ \in \{ \oplus, \otimes \}$
  \item $\app{\flatten_\nat}{s} \arrW s$
  \item $\app{\flatten_{\sigma \arrtype \tau}}{s} \arrW
    \app{\flatten_\tau}{(\app{s}{(\app{\lift_\sigma}{0})})}$
  \item $\app{\flatten_{\quant{\alpha:\kappa}{\sigma}}}{s} \arrW
    \app{\flatten_{\sigma[\subst{\alpha}{\chi_\kappa}]}}{(\tapp{s}{\chi_\kappa})}$
  \item $\app{\lift_\nat}{s} \arrW s$
  \item $\app{\lift_{\sigma \arrtype \tau}}{s} \arrW
    \abs{x:\sigma}{\app{\lift_{\tau}}{s}}$
  \item $\app{\lift_{\quant{\alpha}{\sigma}}}{s} \arrW
    \tabs{\alpha}{\app{\lift_{\sigma}}{s}}$
  \end{enumerate}
  Note that the above rules are invariant under~$\equiv$ (by
  confluence of $\beta$-reduction on types), so they correctly define
  a relation on interpretation terms -- the equivalence classes
  of~$\equiv$. We say that $s$ is a \emph{redex} if $s$ reduces by one
  of the rules (\ref{arrW:plus:base}).

  A \emph{final interpretation term} is an interpretation term $s \in
  \Iterms$ such that (a) $s$ is closed, and (b) $s$ is in normal form
  with respect to $\arrW$.  We let $\Iterms^f$ be the set of all final
  interpretation terms. By~$\Iterms_\tau$ ($\Iterms^f_\tau$) we denote
  the set of all (final) interpretation terms of interpretation
  type~$\tau$ (in a fixed implicit context~$\Gamma$).
\end{defn}

An important difference with System~$\Fomega$ and related ones is that
the rules for $\oplus_\tau$, $\otimes_\tau$, $\flatten_\tau$ and
$\lift_\tau$ depend on the type~$\tau$. In particular, type
substitution in terms may create redexes. For instance, if $\alpha$ is
a type variable then $\oplus_\alpha t_1 t_2$ is not a redex, but
$\oplus_{\sigma\arrtype\tau} t_1 t_2$ is. This makes the question of
termination subtle. Indeed, System~$\Fomega$ is extremely sensitive to
modifications which are not of a logical nature. For instance, adding
a constant $\mathtt{J} : \forall \alpha \beta . \alpha \arrtype \beta$
with a reduction rule $\mathtt{J} \tau \tau \leadsto \lambda x : \tau
. x$ makes the system non-terminating~\cite{Girard1971}. Our rules do
not allow such a definition. Note that the natural number constants
are mutually indistinguishable inside our reduction system (e.g.~there
is no test for zero). For the purposes of termination, the type~$\nat$
may essentially be considered a singleton.

Now we state some properties of~$\arrW$, including strong
normalisation. Because of space limits, most proofs are delegated to
Appendix~\ref{app_proofs}.

\begin{lemma}[Subject reduction]
  If $\Gamma \proves t : \tau$ and $t \arrW t'$ then $\Gamma \proves
  t' : \tau$.
\end{lemma}

\begin{proof}
  By induction on the definition of $t \arrW t'$, using the
  generation and substitution lemmas.
\end{proof}

By~$\SN$ we denote the set of all interpretation terms terminating
w.r.t.~$\arrW$.

\begin{theorem}\label{thm_sn}
  If $\Gamma \proves t : \sigma$ then $t \in \SN$.
\end{theorem}

\begin{lemma}\label{lem_unique_final}
  Every term $s \in \Iterms$ has a unique normal form~$s\da$. If~$s$
  is closed then so is~$s\da$.
\end{lemma}

\begin{proof}
  One checks that~$\arrW$ is locally confluent. Since it is
  terminating by Theorem~\ref{thm_sn}, it is confluent by Newman's
  lemma.
\end{proof}

\begin{lemma}\label{lem_final_nat}
  The only final interpretation terms of type $\nat$ are the natural
  numbers.
\end{lemma}

\subsection{The ordering pair $(\succeq,\succ)$}

\begin{defn}\label{def_closure}
  A \emph{$\Gamma$-closure}~$\cl = \gamma \circ \omega$ is a
  $\Gamma$-replacement such that $\omega(\alpha)$ is closed for each
  type constructor variable~$\alpha$, and $\gamma(x)$ is closed for
  each term variable~$x$.
\end{defn}

We can now define the ordering pair $(\succeq,\succ)$, using
coinduction.

\begin{defn}\label{def:succ}
  Let $R \in \{ \succ,\succeq \}$. For closed~$s,t\in\Iterms_\sigma$
  and closed~$\sigma$ in $\beta$-normal form, the relation
  $s\ R_{\sigma}\ t$ is defined coinductively by the following rules.
  \[
  \begin{array}{cc}
    \infer={s\ R_\nat\ t}{s\da\ R\ t\da \text{ in natural numbers}} \quad&\quad
    \infer={s\ R_{\sigma\arrtype\tau}\ t}{\app{s}{q}\ R_{\tau}\ \app{t}{q} \text{ for all } q \in \Iterms^f_\sigma} \\ \\
    \multicolumn{2}{c}{
    \infer={s\ R_{\forall(\alpha:\kappa)[\sigma]}\ t}{\tapp{s}{\tau}\ R_{\nf_\beta(\sigma[\subst{\alpha}{\tau}])}\ \tapp{t}{\tau} \text{ for all closed } \tau \in \Tc_{\kappa}}}
  \end{array}
  \]
  We define $s \approx_\sigma t$ if both $s \succeq_\sigma t$ and $t
  \succeq_\sigma s$.

  For arbitrary types~$\sigma$ and arbitrary terms $s,t \in \Iterms$
  we define $s \succ_\sigma t$ if for every closure~$\cl$ we can
  obtain $\cl(s) \succ_{\nf_\beta(\cl(\sigma))} \cl(t)$ coinductively
  with the above rules. The relations $\succeq_\sigma$ and
  $\approx_\sigma$ are extended analogously. We drop the type
  subscripts when clear or irrelevant.
\end{defn}

Note that in the case for~$\nat$ the terms~$s\da$, $t\da$ are natural
numbers by Lemma~\ref{lem_final_nat} ($s\da,t\da$ are closed and in
normal form, so they are final interpretation terms).

Intuitively, the above definition means that e.g. $s \succ t$ iff
there exists a possibly infinite derivation tree using the above
rules. In such a derivation tree all leaves must witness $s\da > t\da$
in natural numbers. However, this also allows for infinite branches,
which solves the problem of repeating types due to impredicative
polymorphism. If e.g.~$s \succ_{\forall \alpha . \alpha} t$ then
$\tapp{s}{\forall\alpha.\alpha} \succ_{\forall \alpha . \alpha}
\tapp{t}{\forall\alpha.\alpha}$, which forces an infinite branch in
the derivation tree. According to our definition, any infinite branch
may essentially be ignored.

The coinductive definition of~$\succ$ and~$\succeq$ can be
reformulated as follows.

\begin{lemma}
  $t \succeq s$ if and only if for every closure~$\cl$ and every
  sequence $u_1,\ldots,u_n$ of closed terms and closed type
  constructors such that $\cl(t) u_1 \ldots u_n : \nat$ we have
  $(\cl(t) u_1 \ldots u_n)\da \ge (\cl(s) u_1 \ldots u_n)\da$ in
  natural numbers. An analogous result holds with $\succ$ or $\approx$
  instead of~$\succeq$.
\end{lemma}

\begin{proof}
  The direction from left to right follows by induction on~$n$. The
  other direction follows by coinduction.
\end{proof}

In what follows, all proofs by coinduction could be reformulated to
instead use the characterisation of $\succ$, $\succeq$ and~$\approx$
from the above lemma. However, this would arguably make the proofs
less perspicuous.

First we state a simple lemma that will be used implicitly.

\begin{lemma}
  If~$\tau$ is a closed interpretation type in $\beta$-normal form
  then $\tau = \nat$ or $\tau = \tau_1\arrtype\tau_2$ or $\tau =
  \forall\alpha\sigma$.
\end{lemma}

The most important property of~$\succ$ is that it is a well-founded
ordering.

\begin{lemma}\label{lem_well_founded}
  $\succ$ is well-founded.
\end{lemma}

\begin{proof}
  It suffices to show this for closed terms and closed types in
  $\beta$-normal form, because any infinite sequence $t_1 \succ_\tau
  t_2 \succ_\tau t_3 \succ_\tau \ldots$ induces an infinite sequence
  $\cl(t_1) \succ_{\nf_\beta(\cl(\tau))} \cl(t_2)
  \succ_{\nf_\beta(\cl(\tau))} \cl(t_3) \succ_{\nf_\beta(\cl(\tau))}
  \ldots$ for any closure~$\cl$. By induction on the size of a
  $\beta$-normal type~$\tau$ (with size measured as the number of the
  occurrences of~$\forall$ and~$\arrtype$) one proves that there does
  not exist an infinite sequence $t_1 \succ_\tau t_2 \succ_\tau t_3
  \succ_\tau \ldots$ For instance, if $\alpha$ has kind~$\kappa$ and
  $t_1 \succ_{\forall\alpha\tau} t_2 \succ_{\forall\alpha\tau} t_3
  \succ_{\forall\alpha\tau} \ldots$ then $\tapp{t_1}{\chi_\kappa}
  \succ_{\tau'} \tapp{t_2}{\chi_\kappa} \succ_{\tau'}
  \tapp{t_3}{\chi_\kappa} \succ_{\tau'} \ldots$, where
  $\tau'=\nf_\beta(\tau[\subst{\alpha}{\chi_\kappa}])$. Because $\tau$
  is in $\beta$-normal form, all redexes in
  $\tau[\subst{\alpha}{\chi_\kappa}]$ are created by the substitution
  and must have the form $\chi_\kappa u$. Hence, by the definition
  of~$\chi_\kappa$ (see Definition~\ref{def_types}) the
  type~$\tau'$ is smaller than~$\tau$. This is impossible by the
  inductive hypothesis.
\end{proof}

\begin{lemma}\label{lem_transitive}
  Both $\succ$ and $\succeq$ are transitive.
\end{lemma}

\begin{proof}
  We show this for~$\succ$, the proof for~$\succeq$ being
  analogous. Again, it suffices to prove this for closed terms and
  closed types in $\beta$-normal form. We proceed by coinduction.

  If $t_1 \succ_\nat t_2 \succ_\nat t_3$ then $t_1\da > t_2\da >
  t_3\da$, so $t_1\da > t_3\da$. Thus $t_1 \succ_\nat t_3$.

  If $t_1 \succ_{\sigma\arrtype\tau}t_2\succ_{\sigma\arrtype\tau}t_3$
  then $\app{t_1}{q}\succ_{\tau}\app{t_2}{q}\succ_\tau\app{t_3}{q}$
  for $q \in \Iterms^f_\sigma$. Hence
  $\app{t_1}{q}\succ_\tau\app{t_3}{q}$ for $q \in \Iterms^f_\sigma$ by
  the coinductive hypothesis. Thus $t_1\succ_{\sigma\arrtype\tau}
  t_3$.

  If $t_1
  \succ_{\forall(\alpha:\kappa)\sigma}t_2\succ_{\forall(\alpha:\kappa)\sigma}t_3$
  then
  $\tapp{t_1}{\tau}\succ_{\sigma'}\tapp{t_2}{\tau}\succ_{\sigma'}\tapp{t_3}{\tau}$
  for any closed type constructor~$\tau$ of kind~$\kappa$, where
  $\sigma' = \nf_\beta(\sigma[\subst{\alpha}{\tau}])$. Hence
  $\tapp{t_1}{\tau}\succ_{\sigma'}\tapp{t_3}{\tau}$ by the coinductive
  hypothesis. Thus $t_1\succ_{\forall\alpha\sigma} t_3$.
\end{proof}

\begin{lemma}\label{lem_reflexive}
  $\succeq$ is reflexive.
\end{lemma}

\begin{proof}
  By coinduction one shows that $\succeq_\sigma$ is reflexive on
  closed terms for closed $\beta$-normal~$\sigma$. The general case is
  then immediate from definitions.
\end{proof}

\begin{lemma}\label{lem:compatibility}
  The relations~$\succeq$ and~$\succ$ are compatible, i.e., $\succ
  \cdot \succeq\ \subseteq\ \succ$ and $\succeq \cdot
  \succ\ \subseteq\ \succ$.
\end{lemma}

\begin{proof}
  By coinduction, analogous to the transitivity proof.
\end{proof}

\begin{lemma}\label{lem_succ_to_succeq}
  If $t \succ s$ then $t \succeq s$.
\end{lemma}

\begin{proof}
  By coinduction.
\end{proof}

\begin{lemma}\label{lem_succ_red}
  Assume $t \succ s$ (resp.~$t \succeq s$).
  \begin{enumerate}
  \item If $t \leadsto t'$ or $t' \leadsto t$ then $t' \succ s$
    (resp.~$t' \succeq s$).
  \item If $s \leadsto s'$ or $s' \leadsto s$ then $t \succ s'$
    (resp.~$t \succeq s'$).
  \end{enumerate}
\end{lemma}

\begin{proof}
  We show the first point assuming $t \leadsto t'$, other cases being
  analogous. It suffices to show this for closed $t,t',s$ and the type
  closed and in $\beta$-normal form, because $t \leadsto t'$ implies
  $\cl(t) \leadsto \cl(t')$ for any closure~$\cl$.

  We proceed by coinduction. If $t \succ_\nat s$ then $t\da = t'\da$
  and the claim holds.

  If $t \succ_{\sigma\arrtype\tau} s$ then for every $q \in
  \Iterms^f_\sigma$ we have $\app{t}{q} \succ_\tau \app{s}{q}$. We
  have $\app{t}{q} \leadsto \app{t'}{q}$. Hence $\app{t'}{q}
  \succ_\tau \app{s}{q}$ by the coinductive hypothesis. Thus $t'
  \succ_{\sigma\arrtype\tau} s$.

  If $t \succ_{\forall(\alpha:\kappa)\sigma} s$ then for every closed
  type constructor~$\tau$ of kind~$\kappa$ we have $\tapp{t}{\tau}
  \succ_{\sigma'} \tapp{s}{\tau}$ where $\sigma' =
  \nf_\beta(\sigma[\subst{\alpha}{\tau}])$. We have $\tapp{t}{\tau}
  \leadsto \tapp{t'}{\tau}$. Hence $\tapp{t'}{\tau} \succ_{\sigma'}
  \tapp{s}{\tau}$ by the coinductive hypothesis. Thus $t'
  \succ_{\forall(\alpha:\kappa)\sigma} s$.
\end{proof}

\begin{corollary}\label{cor_succ_da}
  For $R \in \{\succ,\succeq,\approx\}$: $s\ R\ t$ if and only if
  $s\downarrow\ R\ t\downarrow$.
\end{corollary}


\section{Reduction pairs}\label{sec_reduction_pairs}

The goal of this section is to define an interpretation~$\Termmap$
of~PAFS function symbols and a reduction pair
$(\succeqinterpret,\succinterpret)$ on PAFS terms. In the next
section, we will show how a reduction pair may be used to prove
termination of~PAFSs.

First, we precisely define the notion of a reduction pair. We fix
a~PAFS~$A$.

\begin{defn}
  A binary relation~$R$ on $A$-terms is \emph{monotonic} if $R(s, t)$
  implies $R(C[s], C[t])$ for every context~$C$ (we assume $s,t$ have
  the same type~$\sigma$).

  A \emph{reduction pair} is a pair~$(\succeq^A,\succ^A)$ of a
  quasi-order~$\succeq^A$ on $A$-terms and a well-founded
  ordering~$\succ^A$ on $A$-terms such that:
  \begin{itemize}
  \item $\succeq^A$ and~$\succ^A$ are compatible, i.e., ${\succ^A}
    \cdot {\succeq^A} \subseteq {\succ^A}$,
  \item $\succeq^A$ and~$\succ^A$ are both monotonic.
  \end{itemize}
\end{defn}

\subsection{Weak monotonicity}

We first return to our system of interpretation terms with an aim to
show that $s \succeq s'$ implies $t[\subst{x}{s}] \succeq
t[\subst{x}{s'}]$. For this purpose, we prove a few lemmas, many of
which also apply to~$\succ$, stating the preservation of~$\succeq$
under term formation operations. We will need these results to obtain
a reduction pair.

\begin{lemma}\label{lem_app_succ}
  If $t \succeq s$ (resp.~$t \succ s$) then $t u \succeq s u$
  (resp.~$t u \succ s u$) with $u$ a term or type constructor.
\end{lemma}

\begin{proof}
  Follows from definitions.
\end{proof}

\begin{lemma}\label{lem:liftgreater}
  If $n \geq m$ (resp.~$n > m$) then $\lift_\sigma n \succeq
  \lift_\sigma m$ (resp.~$\lift_\sigma n \succ \lift_\sigma m$) for
  all types $\sigma$.
\end{lemma}

\begin{proof}
  Without loss of generality we may assume $\sigma$ closed and in
  $\beta$-normal form. By coinduction we show $\lift(n) u_1 \ldots u_k
  \succeq \lift(m) u_1 \ldots u_k$ for closed $u_1,\ldots,u_k$. First
  note that $(\lift\,t) u_1 \ldots u_k \leadsto^* \lift(t)$ (with a
  different type subscript in~$\lift$ on the right side, omitted here
  for conciseness). If $\sigma = \nat$ then $(\lift(n) u_1 \ldots
  u_k)\da = n \ge m = (\lift(m) u_1 \ldots u_k)\da$. If $\sigma =
  \tau_1\arrtype\tau_2$ then by the coinductive hypothesis $\lift(n)
  u_1 \ldots u_k q \succeq_{\tau_2} \lift(m) u_1 \ldots u_k q$ for any
  $q \in \Iterms^f_{\tau_1}$, so $\lift(n) u_1 \ldots u_k
  \succeq_{\sigma} \lift(m) u_1 \ldots u_k$ by definition. If $\sigma
  = \forall(\alpha:\kappa)\tau$ then by the coinductive hypothesis
  $\lift(n) u_1 \ldots u_k \xi \succeq_{\sigma'} \lift(m) u_1 \ldots
  u_k \xi$ for any closed $\xi \in \Tc_\kappa$, where $\sigma' =
  \tau[\subst{\alpha}{\xi}]$. Hence $\lift(n) u_1 \ldots u_k
  \succeq_{\sigma} \lift(m) u_1 \ldots u_k$ by definition
\end{proof}

\begin{lemma}\label{lem_flatten_succ}
  If $t \succeq_\sigma s$ then $\flatten_\sigma t \succeq_\nat
  \flatten_\sigma s$ for all types $\sigma$. The same holds
  with~$\succ$ instead of~$\succeq$.
\end{lemma}

\begin{proof}
  Without loss of generality we may assume~$\sigma$ is closed and in
  $\beta$-normal form. Using Lemma~\ref{lem_succ_red}, one shows by
  induction on~$\sigma$ that if $t \succeq_\sigma s$ then
  $\flatten_\sigma t \succeq_\nat \flatten_\sigma s$ (the proof
  for~$\succ$ is analogous).
\end{proof}

\begin{lemma}\label{lem_abs_succ}
  If $t \succeq s$ then $\abs{x}{t} \succeq \abs{x}{s}$ and
  $\tabs{\alpha}{t} \succeq \tabs{\alpha}{s}$. The same holds
  for~$\succ$.
\end{lemma}

\begin{proof}
  Assume $t \succeq_\tau s$ and $\Gamma \proves x : \sigma$. Let~$\cl$
  be a closure. We need to show $\cl(\abs{x}{t})
  \succeq_{\cl(\sigma\arrtype\tau)} \cl(\abs{x}{s})$. Let $u \in
  \Iterms^f_{\cl(\sigma)}$. Then $\cl' = \cl[\subst{x}{u}]$ is a
  closure and $\cl'(t) \succeq_{\cl(\tau)} \cl'(s)$. Hence
  $\cl(t)[\subst{x}{u}] \succeq_{\cl(\tau)} \cl(s)[\subst{x}{u}]$. By
  Lemma~\ref{lem_succ_red} this implies $\cl(\abs{x}{t}) u
  \succeq_{\cl(\tau)} \cl(\abs{x}{s}) u$. Therefore $\cl(\abs{x}{t})
  \succeq_{\cl(\sigma\arrtype\tau)} \cl(\abs{x}{s})$. The proof for
  the second part is analogous.
\end{proof}

\begin{lemma}\label{lem:plustimesmonotonic}
  Let $s,t,u$ be terms of type $\sigma$.
  \begin{enumerate}
  \item If $s \succeq t$ then $s \oplus_\sigma u \succeq t
    \oplus_\sigma u$, $u \oplus_\sigma s \succeq u \oplus_\sigma t$,
    $s \otimes_\sigma u \succeq t \otimes_\sigma u$, and $u
    \otimes_\sigma s \succeq u \otimes_\sigma t$.
  \item If $s \succ t$ then $s \oplus_\sigma u \succ t \oplus_\sigma
    u$ and $u \oplus_\sigma s \succ u \oplus_\sigma t$. Moreover, if
    additionally $u \succeq \lift_\sigma(1)$ then also $s
    \otimes_\sigma u \succ t \otimes_\sigma u$ and $u \otimes_\sigma s
    \succ u \otimes_\sigma t$.
  \end{enumerate}
\end{lemma}

\begin{proof}
  It suffices to prove this for closed $s,t,u$ and closed $\sigma$ in
  $\beta$-normal form. The proof is similar to the proof of
  Lemma~\ref{lem:liftgreater}. For instance, we show by coinduction
  that for closed $w_1,\ldots,w_n$ if $s w_1 \ldots w_n \succ t w_1
  \ldots w_n$ and $u w_1 \ldots w_n \succeq \lift(1) w_1 \ldots w_n$
  then $(s \otimes u) w_1 \ldots w_n \succ (t \otimes u) w_1 \ldots
  w_n$.
\end{proof}

The following lemma depends on the lemmas above. The full proof may be
found in Appendix~\ref{app_proofs}. The proof is actually quite
complex, and uses a method similar to Girard's method of candidates
for the termination proof. The difficulty comes from the fact that due
to the impredicativity of polymorphism direct induction on type
structure is not possible.

\begin{lemma}[Weak monotonicity]\label{lem_succeq_subst}
  If $s \succeq s'$ then $t[\subst{x}{s}] \succeq t[\subst{x}{s'}]$.
\end{lemma}

\begin{corollary}\label{cor_app_wm}
  If $s \succeq s'$ then $t s \succeq t s'$.
\end{corollary}

\subsection{Interpretation mapping}

\begin{defn}
  A \emph{type constructor mapping} is a function $\Typemap$ which
  maps a type constructor symbol to a closed interpretation type
  constructor of the same kind. A fixed type constructor mapping
  $\Typemap$ is extended inductively to a function from type
  constructors to closed interpretation type constructors in the
  expected way. We denote the extended \emph{interpretation (type)
    mapping} by~$\typeinterpret{\sigma}$. Thus,
  e.g.~$\typeinterpret{\quant{\alpha}{\sigma}} =
  \quant{\alpha}{\typeinterpret{\sigma}}$ and $\typeinterpret{\sigma
    \arrtype \tau} = \typeinterpret{\sigma} \arrtype
  \typeinterpret{\tau}$. Interpretation type mapping is extended to
  contexts: \( \itp{\Gamma} := \{ (x : \typeinterpret{\tau}) \mid (x :
  \tau) \in \Gamma \}.  \)
\end{defn}

\begin{lemma}\label{lem:substitutioninterpret:types}
  $\typeinterpret{\sigma}[\alpha:=\typeinterpret{\tau}] =
  \typeinterpret{\sigma[\alpha:=\tau]}$
\end{lemma}

\begin{proof}
  Induction on~$\sigma$.
\end{proof}

Similarly, we employ a \emph{symbol mapping} as an aid to interpret
PAFS terms.

\begin{defn}
  A \emph{symbol mapping} is a function $\Termmap$ which assigns to
  each function symbol $\mathtt{f} : \rho$ a closed interpretation
  term $\Termmap(\mathtt{f})$ of type~$\typeinterpret{\rho}$. For a
  fixed symbol mapping $\Termmap$, we define the \emph{interpretation
    mapping} $\interpret{s}$ inductively:
  \[
    \begin{array}{rclcrcl}
      \interpret{x} & = & x &\quad&
      \interpret{\mathtt{f}} &=& \Termmap(\mathtt{f}) \\
      \interpret{\tabs{\alpha}{s}} & = & \tabs{\alpha}{\interpret{s}} &\quad&
      \interpret{\abs{x:\sigma}{s}} & = & \abs{x:\typeinterpret{\sigma}}{
                                          \interpret{s}} \\
      \interpret{\app{t_1}{t_2}} &=& \app{\interpret{t_1}}{\interpret{t_2}} &\quad&
      \interpret{\tapp{t}{\tau}} &=& \tapp{\interpret{t}}{\typeinterpret{\tau}}
    \end{array}
  \]
\end{defn}

Interpretation mapping preserves typing:

\begin{lemma}
  If $\Gamma \vdash s : \sigma$ then $\itp{\Gamma} \vdash
  \interpret{s} : \typeinterpret{\sigma}$.
\end{lemma}

\begin{proof}
  By induction on the form of $s$, using
  Lemma~\ref{lem:substitutioninterpret:types}.
\end{proof}

\begin{defn}
  For a fixed type constructor mapping $\Typemap$ and symbol mapping
  $\Termmap$, the \emph{interpretation pair}
  $(\succeqinterpret,\succinterpret)$ is defined as follows: $s
  \succeqinterpret t$ if $\interpret{s} \succeq \interpret{t}$, and $s
  \succinterpret t$ if $\interpret{s} \succ \interpret{t}$.
\end{defn}

\begin{lemma}\label{lem:substitutioninterpret}
  \begin{enumerate}
  \item\label{lem:substitutioninterpret:mixed}
    $\interpret{s}[\alpha:=\typeinterpret{\tau}] =
    \interpret{s[\alpha:=\tau]}$.
  \item\label{lem:substitutioninterpret:terms}
    $\interpret{s}[x:=\interpret{t}] = \interpret{s[x:=t]}$.
  \end{enumerate}
\end{lemma}

\begin{proof}
  Induction on~$s$.
\end{proof}

We will show that, under certain conditions on~$\Termmap$, the
interpretation pair $(\succeqinterpret,\succinterpret)$ is a reduction
pair.

\begin{defn}\label{def_safe}
  If $s_1 \succ s_2$ implies $t[\subst{x}{s_1}] \succ
  t[\subst{x}{s_2}]$, then the interpretation term~$t$ is \emph{safe
    for~$x$}. A symbol mapping~$\Termmap$ is \emph{safe} if for all
  symbols
  \[
  \mathtt{f} : \forall (\alpha_1 : \kappa_1) \ldots \forall (\alpha_n
  : \kappa_n) . \sigma_1 \arrtype \ldots \arrtype \sigma_k \arrtype
  \tau
  \]
  with~$\tau$ a type atom we have: $\Termmap(\mathtt{f}) =
  \tabs{\alpha_1 \dots \alpha_n}{\abs{x_1 \dots x_k}{t}}$ with $t$
  safe for each~$x_i$.
\end{defn}

\begin{lemma}\label{lem_safe}
  \begin{enumerate}
  \item $x u_1 \ldots u_m$ is safe for~$x$.
  \item If $t$ is safe for~$x$ then so is~$\lift(t)$
    and~$\flatten(t)$.
  \item If $s_1$ is safe for~$x$ or $s_2$ is safe for~$x$ then $s_1
    \oplus s_2$ is safe for~$x$.
  \item If either:
    \begin{itemize}
    \item $s_1$ is safe for~$x$ and $s_2 \succeq \lift(1)$, or
    \item $s_2$ is safe for~$x$ and $s_1 \succeq \lift(1)$,
    \end{itemize}
    then $s_1 \otimes s_2$ is safe for~$x$.
  \item If~$t$ is safe for~$x$ then so is~$\tabs{\alpha}{t}$
    and~$\abs{y}{t}$ ($y \ne x$).
  \item If $t$ is safe for~$x$ then so is~$\pi^1(t)$ and~$\pi^2(t)$.
  \item If $t$ is safe for~$x$ then so is~$\xlet{}{t}{\alpha,x}{s}$.
  \item If $t$ is safe for~$x$ then so is~$\fold_\rho(f,a,t)$.
  \end{enumerate}
\end{lemma}

\begin{proof}
  Each point follows from one of the lemmas proven before,
  Lemma~\ref{lem_succ_to_succeq}, Lemma~\ref{lem_succeq_subst},
  Lemma~\ref{lem:compatibility} and the transitivity of~$\succeq$. For
  instance, assume $s_1 \succ s_2$ and let
  $u_i^j=u_i[\subst{x}{s_j}]$. Then $(x u_1 \ldots
  u_m)[\subst{x}{s_1}] = s_1 u_1^1 \ldots u_m^1$. By
  Lemma~\ref{lem_app_succ} we have $s_1 u_1^1 \ldots u_m^1 \succ s_2
  u_1^1 \ldots u_m^1$. By Lemma~\ref{lem_succ_to_succeq} and
  Lemma~\ref{lem_succeq_subst} we have $u_i^1 \succeq u_i^2$. By
  Corollary~\ref{cor_app_wm} and the transitivity of~$\succeq$ we
  obtain $s_2 u_1^1 \ldots u_m^1 \succeq s_2 u_1^2 \ldots u_m^2$. By
  Lemma~\ref{lem:compatibility} finally $(x u_1 \ldots
  u_m)[\subst{x}{s_1}] = s_1 u_1^1 \ldots u_m^1 \succ s_2 u_1^2 \ldots
  u_m^2 = (x u_1 \ldots u_m)[\subst{x}{s_2}]$.
\end{proof}

\begin{lemma}\label{lem_succinterpret_monotonic}
  If~$\Termmap$ is safe then~$\succinterpret$ is monotonic.
\end{lemma}

\begin{proof}
  Assume $s_1 \succinterpret s_2$. By induction on the structure of a
  context~$C$ we show $C[s_1] \succinterpret C[s_2]$. If $C=\Box$ then
  this is obvious. If $C = \abs{x}{C'}$ or $C = \tabs{\alpha}{C'}$
  then $C'[s_1] \succinterpret C'[s_2]$ by the inductive hypothesis,
  and thus $C[s_1] \succinterpret C[s_2]$ follows from
  Lemma~\ref{lem_abs_succ} and definitions. If $C = C' t$ then
  $C'[s_1] \succinterpret C'[s_2]$ by the inductive hypothesis, and
  thus $C[s_1] \succinterpret C[s_2]$ follows from definitions.

  Finally, assume $C = \app{t}{C'}$. Then $t = \mathtt{f} \rho_1
  \ldots \rho_n t_1 \ldots t_m$ where
  \[
  \mathtt{f} : \forall (\alpha_1 : \kappa_1) \ldots \forall (\alpha_n
  : \kappa_n) . \sigma_1 \arrtype \ldots \arrtype \sigma_k \arrtype
  \tau
  \]
  with~$\tau$ a type atom, $m < k$, and $\Termmap(\mathtt{f}) =
  \tabs{\alpha_1 \dots \alpha_n}{\abs{x_1 \dots x_k}{u}}$ with $u$
  safe for each~$x_i$. Without loss of generality assume $m=k-1$. Then
  $\interpret{C[s_i]} \leadsto u'[\subst{x_k}{\interpret{C'[s_i]}}]$
  where
  $u'=u[\subst{\alpha_1}{\typeinterpret{\rho_1}}]\ldots[\subst{\alpha_n}{\typeinterpret{\rho_n}}][\subst{x_1}{\interpret{t_1}}]\ldots[\subst{x_{k-1}}{\interpret{t_{k-1}}}]$. By
  the inductive hypothesis $\interpret{C'[s_1]} \succ
  \interpret{C'[s_2]}$. Hence $u'[\subst{x_k}{\interpret{C'[s_1]}}]
  \succ u'[\subst{x_k}{\interpret{C'[s_2]}}]$, because~$u$ is safe
  for~$x_k$. Thus $\interpret{C[s_1]} \succ \interpret{C[s_2]}$ by
  Lemma~\ref{lem_succ_red}.
\end{proof}

\begin{theorem}\label{thm_reduction_pair}
  If~$\Termmap$ is safe then $(\succeqinterpret,\succinterpret)$ is a
  reduction pair.
\end{theorem}

\begin{proof}
  If follows from Lemma~\ref{lem_transitive} and
  Lemma~\ref{lem_reflexive} that~$\succeqinterpret$ is a
  quasi-order. Lemma~\ref{lem_well_founded} and
  Lemma~\ref{lem_transitive} imply that~$\succinterpret$ is a
  well-founded ordering. Compatibility follows from
  Lemma~\ref{lem:compatibility}. Monotonicity of~$\succeqinterpret$
  follows from Lemma~\ref{lem_succeq_subst}. Monotonicity
  of~$\succinterpret$ follows from
  Lemma~\ref{lem_succinterpret_monotonic}.
\end{proof}

\begin{example}\label{ex_fold_interpretation}
  The following is a safe interpretation for the PAFS from
  Example~\ref{ex_fold_pafs}: \LC{Cynthia: TODO}.
\end{example}

\section{Proving termination with rule removal}\label{sec_rule_removal}

In this section we show how to use reduction pairs to prove
termination of a PAFS. This is achieved by the method of rule removal.

\begin{theorem}\label{thm:ruleremove}
  Let $\Rules = \Rules_1 \cup \Rules_2$, and suppose that
  $\Rules_1\subseteq{\succ^\Rules}$ and
  $\Rules_2\subseteq{\succeq^\Rules}$ for a reduction pair
  $(\succeq^\Rules,\succ^\Rules)$. Then $\arr{\Rules}$ is terminating
  if and only if $\arr{\Rules_2}$ is. In particular, $\arr{\Rules}$ is
  terminating if $\Rules_2 = \emptyset$.
\end{theorem}

\begin{proof}
  Monotonicity of~$\succeq^\Rules$ and~$\succ^\Rules$ implies that
  ${\arr{\Rules_1}}\subseteq{\succ^\Rules}$ and
  ${\arr{\Rules_2}}\subseteq{\succeq^\Rules}$.

  By well-foundedness of $\succ^\Rules$, compatibility
  of~$\succeq^\Rules$ and~$\succ^\Rules$, and transitivity
  of~$\succeq^\Rules$, every infinite $\arr{\Rules}$ sequence can
  contain only finitely many $\arr{\Rules_1}$ steps.
\end{proof}

The above theorem gives rise to the following \emph{rule removal}
algorithm:
\begin{enumerate}
\item While $\Rules$ is non-empty:
  \begin{enumerate}
  \item Orient all rules in $\Rules$ using $\succeq^\Rules$ or
    $\succ^\Rules$; at least one of them must be oriented using
    $\succ^\Rules$.
  \item Remove all rules ordered by $\succ^\Rules$ from $\Rules$.
  \end{enumerate}
\end{enumerate}
If this algorithm succeeds, we have proven termination.

We propose using the pair $(\succeqinterpret,\succinterpret)$ from
Section~\ref{sec_reduction_pairs}. To use this pair to prove
termination of $\arr{\Rules}$ with rule removal, two things must be
ensured:
\begin{itemize}
\item $\Termmap$ is safe;
\item all rules in $\Rules$ can be oriented with $\succeqinterpret$ or
  $\succinterpret$, and at least one with $\succinterpret$.
\end{itemize}

The first requirement guarantees that
$(\succeqinterpret,\succinterpret)$ is a reduction pair (by
Theorem~\ref{thm_reduction_pair}). Lemma~\ref{lem_safe} provides some
sufficient safety criteria. The second requirement has to be verified
for every individual rule.

Now, we state a few lemmas which are helpful in showing rule
orientation with~$\succeqinterpret$ and~$\succinterpret$. The proofs
may be found in Appendix~\ref{app_proofs}.

\begin{lemma}\label{lem:plusparts}
For all types $\sigma$, terms $s,t$ of type $\sigma$ and natural
numbers $n > 0$:
\begin{enumerate}
\item $s \oplus_{\sigma} t \succeq s$ and $s \oplus_{\sigma} t \succeq
  t$;
\item $s \oplus_{\sigma} (\lift_{\sigma} n) \succ s$ and
  $(\lift_{\sigma} n) \oplus_{\sigma} t \succ t$.
\end{enumerate}
\end{lemma}

\begin{lemma}\label{lem:approxproperties}
For all types $\sigma$ and all terms $s,t,u$ of type $\sigma$, we
have:
\begin{enumerate}
\item\label{lem:approx:symmetry} $s \oplus_\sigma t \approx t
  \oplus_\sigma s$ and $s \otimes_\sigma t \approx t \otimes_\sigma
  s$;
\item\label{lem:approx:assoc} $s \oplus_\sigma (t \oplus_\sigma u)
  \approx (s \oplus_\sigma t) \oplus_\sigma u$ and $s \otimes_\sigma
  (t \otimes_\sigma u) \approx (s \otimes_\sigma t) \otimes_\sigma u$;
\item\label{lem:approx:distribution} $s \otimes_\sigma (t
  \oplus_\sigma u) \approx (s \otimes_\sigma t) \oplus_\sigma (s
  \otimes_\sigma u)$;
\item\label{lem:approx:neutral} $(\lift_\sigma 0) \oplus_\sigma s
  \approx s$ and $(\lift_\sigma 1) \otimes_\sigma s \approx s$.
\end{enumerate}
\end{lemma}

\begin{lemma}\label{lem_lift_approx}
  \begin{enumerate}
  \item $\lift_\sigma(n+m) \approx_\sigma (\lift_\sigma n)
    \oplus_\sigma (\lift_\sigma n)$.
  \item $\lift_\sigma(n m) \approx_\sigma (\lift_\sigma n)
    \otimes_\sigma (\lift_\sigma n)$.
  \end{enumerate}
\end{lemma}

\begin{example}\label{ex_fold_orientation}
  We continue with our example of fold on heterogenous lists. We prove
  termination by rule removal, using the symbol mapping from
  Example~\ref{ex_fold_interpretation}. \LC{Cynthia: TODO}
\end{example}

\section{Larger examples}\label{sec:examples}

\LC{Cynthia: TODO}

\section{Conclusions and future work}

\addcontentsline{toc}{section}{References}
\bibliography{references}

\clearpage
\appendix

\section{Complete proofs}\label{app_proofs}

In many proofs below we assume the terms to be given in orthodox
Church-style (see the discussion at the end of
Section~\ref{sec_preliminaries}). We denote an occurrence of a
variable~$x$ annotated with a type~$\tau$ by~$x^\tau$. So now
e.g.~$\lambda x : \tau\arrtype\sigma . x^{\tau\arrtype\sigma}y^\tau$
is an orthodox Church-style typed term. When clear or irrelevant, we
omit the type annotations for readability, denoting the above term
by~$\lambda x : \tau\arrtype\sigma . x y$ or even~$\lambda x . x
y$. Note that now type substitution also needs to change the type
annotations. Also, each term has a unique type modulo
$\beta$-conversion. The generation and subject reduction lemmas still
hold for orthodox Church-style typed terms.

\subsection{Strong Normalisation of~$\arrW$}

For $t \in \SN$ by~$\nu(t)$ we denote the length of the longest
reduction starting at~$t$. The following lemma is obvious, but worth
stating explicitly.

\begin{lemma}\label{lem_reduce_abs}
  If $\abstraction{a}{s} \arrW^* t$, then $t = \abstraction{a}{t'}$
  and $s \arrW^* t'$.  If $s \in \SN$ then both $\abs{x}{s}$ and
  $\tabs{\alpha}{s}$ are also in $\SN$.
\end{lemma}

\begin{proof}
  We observe that every reduct of $\abstraction{x}{s}$ has the form
  $\abstraction{x}{s'}$ with $s \arrW s'$, and analogously for
  $\tabs{\alpha}{s}$.  Thus, the first statement follows by induction
  on the length of the reduction $\abstraction{a}{s} \arrW^* t$,
  and the second statement by induction on $\nu(s)$.
\end{proof}

\begin{lemma}\label{lem_circ_sn_base}
  If $t_1,t_2 \in \SN$ then $\circ_\nat t_1 t_2 \in \SN$ for $\circ
  \in \{\oplus,\otimes\}$.
\end{lemma}

\begin{proof}
  By induction on $\nu(t_1) + \nu(t_2)$. Assume $t_1,t_2 \in \SN$. To
  prove $\circ_\nat t_1 t_2 \in \SN$ it suffices to show $s \in \SN$
  for all~$s$ such that $\circ_\nat t_1 t_2 \arrW s$. If $s =
  \circ_\nat t_1' t_2$ or $s = \circ_\nat t_1 t_2'$ with $t_i \arrW
  t_i'$ then we complete by the induction hypothesis. Otherwise $s \in
  \mathbb{N}$ is obviously in $\SN$.
\end{proof}

In the rest of this section we adapt Girard's method of candidates
(which itself is based on Tait's computability method) to prove
termination of~$\arrW$. The proof is an adaptation of chapters~6
and~14 from the book~\cite{Girard1989}, and chapters~10 and~11 from
the book~\cite{SorensenUrzyczyn2006}.

\begin{defn}\label{def_candidate}
  A term~$t$ is \emph{neutral} if there does not exist a sequence of
  terms and types~$u_1,\ldots,u_n$ with $n \ge 1$ such that $t u_1
  \ldots u_n$ is a redex (by~$\arrW$).

  By induction on the kind~$\kappa$ of a type constructor~$\tau$ we
  define the set~$\Cb_\tau$ of all candidates of type
  constructor~$\tau$.

  First assume $\kappa=*$, i.e., $\tau$ is a type. A set~$X$ of
  interpretation terms of type~$\tau$ is a \emph{candidate of
    type~$\tau$} when:
  \begin{enumerate}
  \item $X \subseteq \SN$;
  \item if $t \in X$ and $t \arrW t'$ then $t' \in X$;
  \item if $t$ is neutral and for every~$t'$ with $t \arrW t'$ we
    have $t' \in X$, then $t \in X$;
  \item if $t_1,t_2 \in X$ then $\circ_\tau t_1 t_2 \in X$ for
    $\circ \in \{\oplus,\otimes\}$;
  \item if $t \in \SN$ and $t : \nat$ then $\lift_\tau t \in X$;
  \item if $t \in X$ then $\flatten_\tau t \in \SN$.
  \end{enumerate}
  Note that item~3 above implies:
  \begin{itemize}
  \item if $t$ is neutral and in normal form then $t \in X$.
  \end{itemize}

  Now assume $\kappa = \kappa_1\arrkind\kappa_2$. A function $f :
  \Tc_{\kappa_1} \times \bigcup_{\xi\in\Tc_{\kappa_1}}\Cb_\xi \to
  \bigcup_{\xi\in\Tc_{\kappa_2}}\Cb_\xi$ is a \emph{candidate of type
    constructor~$\tau$} if for every closed type constructor~$\sigma$
  of kind~$\kappa_1$ and a candidate $X \in \Cb_\sigma$ we have
  $f(\sigma,X) \in \Cb_{\tau\sigma}$.
\end{defn}

Note that the elements of a candidate of type~$\tau$ are required to
have type~$\tau$.

\begin{lemma}\label{lem_beta_candidate}
  If $\sigma =_\beta \sigma'$ then $\Cb_\sigma = \Cb_{\sigma'}$.
\end{lemma}

\begin{proof}
  Induction on the kind of~$\sigma$.
\end{proof}

\begin{defn}\label{def_computability_valuation}
  Let $\omega$ be a mapping from type constructor variables to type
  constructors (respecting kinds). The mapping~$\omega$ extends in an
  obvious way to a mapping from type constructors to type
  constructors. A mapping~$\omega$ is \emph{closed for~$\sigma$} if
  $\omega(\alpha)$ is closed for $\alpha \in \FV(\sigma)$ (then
  $\omega(\sigma)$ is closed).

  An \emph{$\omega$-valuation} is a mapping~$\xi$ from type
  constructor variables to candidates such that $\xi(\alpha) \in
  \Cb_{\omega(\alpha)}$.

  For each type constructor~$\sigma$, each mapping~$\omega$ closed
  for~$\sigma$, and each $\omega$-valuation~$\xi$, the set
  $\val{\sigma}{\xi}{\omega}$ is defined by induction on~$\sigma$:
  \begin{itemize}
  \item $\val{\alpha}{\xi}{\omega} = \xi(\alpha)$ for a type
    constructor variable~$\alpha$,
  \item $\val{\nat}{\xi}{\omega}$ is the set of all terms~$t \in \SN$
    such that $t : \nat$,
  \item $\val{\sigma \arrtype \tau}{\xi}{\omega}$ is the set of all
    terms~$t$ such that $t : \omega(\sigma\arrtype\tau)$ and for
    every~$s \in \val{\sigma}{\xi}{\omega}$ with $s : \omega(\sigma)$
    we have $\app{t}{s} \in \val{\tau}{\xi}{\omega}$,
  \item $\val{\forall(\alpha:\kappa)\sigma}{\xi}{\omega}$ is the set
    of all terms~$t$ such that $t : \omega(\forall\alpha\sigma)$ and
    for every closed type constructor~$\varphi$ of kind~$\kappa$ and
    every $X \in \Cb_\varphi$ we have $\tapp{t}{\varphi} \in
    \val{\sigma}{\xi[\subst{\alpha}{X}]}{\omega[\subst{\alpha}{\varphi}]}$,
  \item
    $\val{\varphi \psi}{\xi}{\omega} =
    \val{\varphi}{\xi}{\omega}(\omega(\psi),\val{\psi}{\xi}{\omega})$,
  \item
    $\val{\lambda(\alpha:\kappa)\varphi}{\xi}{\omega}(\psi,X) =
    \val{\varphi}{\xi[\subst{\alpha}{X}]}{\omega[\subst{\alpha}{\psi}]}$
    for closed $\psi \in \Tc_\kappa$ and $X \in \Cb_\psi$.
  \end{itemize}
  In the above, if e.g.~$\val{\psi}{\xi}{\omega} \notin
  \Cb_{\omega(\psi)}$ then $\val{\varphi \psi}{\xi}{\omega}$ is
  undefined.
\end{defn}

If~$\varphi$ is closed then $\omega,\xi$ do not affect the value
of~$\val{\varphi}{\xi}{\omega}$, so then we simply
write~$\val{\varphi}{}{}$.

\begin{lemma}\label{lem_nat_computable}
  $\val{\nat}{}{} \in \Cb_{\nat}$.
\end{lemma}

\begin{proof}
  We check the conditions in Definition~\ref{def_candidate}.
  \begin{enumerate}
  \item $\val{\nat}{}{} \subseteq \SN$ follows
    directly from Definition~\ref{def_computability_valuation}.
  \item Let $t \in \val{\nat}{}{}$ and $t \arrW t'$. Then $t :
    \nat$ and $t \in \SN$. Hence $t' \in \SN$, and $t' : \nat$ by the
    subject reduction lemma. Thus $t' \in \val{\nat}{}{}$.
  \item Let $t$ be neutral and $t : \nat$. Assume that for all~$t'$
    with $t \arrW t'$ we have $t' \in \val{\nat}{}{}$, so in
    particular $t' \in \SN$. But then $t \in \SN$. Hence $t \in
    \val{\nat}{}{}$.
  \item Let $t_1,t_2 \in \SN$ be such that $t_i : \nat$. Obviously,
    $\circ_\nat t_1 t_2 : \nat$. Also $\circ_\nat t_1 t_2 \in \SN$
    follows by Lemma~\ref{lem_circ_sn_base}. So $\circ_\nat t_1 t_2
    \in \val{\nat}{}{}$.
  \item Let $t \in \SN$ be such that $t : \nat$. Then $\lift_\nat t :
    \nat$. It remains to show $\lift_\nat t \in \SN$. Any infinite
    reduction from~$\lift_\nat t$ has the form $\lift_\nat t
    \arrW^* \lift_\nat t_0 \arrW t_1 \arrW t_2 \arrW
    \ldots$ or $\lift_\nat t \arrW \lift_\nat t_0 \arrW
    \lift_\nat t_1 \arrW \lift_\nat t_2 \arrW \ldots$, where $t
    \arrW^* t_0$ and $t_i \arrW t_{i+1}$. This contradicts $t
    \in \SN$.
  \item Let $t \in \SN$ be such that $t : \nat$. The proof of
    $\flatten_\nat t \in \SN$ is analogous to the proof of $\lift_\nat
    t \in \SN$ above.\qedhere
  \end{enumerate}
\end{proof}

\begin{lemma}\label{lem_chi_kappa_computable}
  $\val{\chi_\kappa}{}{} \in \Cb_{\chi_\kappa}$.
\end{lemma}

\begin{proof}
  Induction on~$\kappa$. If $\kappa = *$ then this follows from
  Lemma~\ref{lem_nat_computable}. If $\kappa=\kappa_1\arrkind\kappa_2$
  then $\chi_\kappa = \lambda \alpha : \kappa_1
  . \chi_{\kappa_2}$. Let~$\psi$ be a closed type constructor of
  kind~$\kappa_1$ and let $X \in \Cb_{\chi_{\kappa_1}}$. We have
  $\val{\chi_\kappa}{}{}(\psi,X) = \val{\chi_{\kappa_2}}{}{}$ because
  $\chi_{\kappa_2}$ is closed. By the inductive hypothesis
  $\val{\chi_\kappa}{}{}(\psi,X) = \val{\chi_{\kappa_2}}{}{} \in
  \Cb_{\chi_{\kappa_2}}$. This implies $\val{\chi_\kappa}{}{} \in
  \Cb_{\chi_\kappa}$.
\end{proof}

\begin{lemma}\label{lem_abstraction_computable}
  Let $\sigma,\tau$ be types. Suppose $\val{\tau}{\xi'}{\omega'} \in
  \Cb_{\omega'(\tau)}$ and $\val{\sigma}{\xi'}{\omega'} \in
  \Cb_{\omega'(\sigma)}$ for all suitable $\omega',\xi'$. Then
  \begin{itemize}
  \item
    $\abs{x}{s} \in \val{\tau \arrtype \sigma}{\xi}{\omega}$ if and
    only if $\abs{x}{s} : \omega(\tau \arrtype \sigma)$ and $s[x:=t]
    \in \val{\sigma}{\xi}{\omega}$ for all $t \in
    \val{\tau}{\xi}{\omega}$;
  \item
    $\tabs{\alpha}{s} \in
    \val{\quant{(\alpha:\kappa)}{\sigma}}{\xi}{\omega}$ if and only if
    $\tabs{\alpha}{s} : \omega(\quant{(\alpha:\kappa)}{\sigma})$ and
    for every closed type constructor~$\varphi$ of kind~$\kappa$ and
    all $X \in \Cb_\varphi$ we have $s[\alpha:=\varphi] \in
    \val{\sigma}{\xi[\subst{\alpha}{X}]}{\omega[\subst{\alpha}{\varphi}]}$.
  \end{itemize}
\end{lemma}

\begin{proof}
  First suppose
  $\abs{x:\omega(\tau)}{s} \in \val{\tau \arrtype
    \sigma}{\xi}{\omega}$. Then
  $\abs{x:\omega(\tau)}{s} : \omega(\tau\arrtype\sigma)$ and for all
  $t \in \val{\tau}{\xi}{\omega}$ we have
  $\app{(\abs{x:\omega(\tau)}{s})}{t} \in \val{\sigma}{\xi}{\omega}$.
  As this set is a candidate, it is closed under $\arrW$, so also
  $s[x:=t] \in \val{\sigma}{\xi}{\omega}$. Similarly, if
  $\tabs{\alpha}{s} \in \val{\quant{\alpha}{\sigma}}{\xi}{\omega}$,
  then $\tabs{\alpha}{s} : \quant{\alpha}{\sigma}$ and
  $\tapp{(\tabs{\alpha}{s})}{\varphi} \in
  \val{\sigma}{\xi[\subst{\alpha}{X}]}{\omega[\subst{\alpha}{\varphi}]}$,
  and we are done because
  $\tapp{(\tabs{\alpha}{s})}{\tau} \arrW s[\alpha:=\varphi]$ and
  $\val{\sigma}{\xi[\subst{\alpha}{X}]}{\omega[\subst{\alpha}{\varphi}]}$
  is a candidate, so it is closed under~$\arrW$.

  Now suppose $s[x:=t] \in \val{\sigma}{\xi}{\omega}$ for all
  $t \in \val{\tau}{\xi}{\omega}$. Let
  $t \in \val{\tau}{\xi}{\omega}$. Then $t \in \SN$ because
  $\val{\tau}{\xi}{\omega}$ is a candidate. Also $s \in \SN$ because
  every infinite reduction in $s$ induces an infinite reduction in
  $s[x:=t]$ ($\arrW$ is stable) and
  $\val{\sigma}{\xi}{\omega} \subseteq \SN$ is a candidate. For all
  $s',t'$ with $s \arrW^* s'$ and $t \arrW^* t'$, we show by
  induction on~$\nu(s') + \nu(t')$ that
  $\app{(\abs{x}{s'})} t' \in \val{\sigma}{\xi}{\omega}$. We have
  $\app{(\abs{x}{s'})} t' : \omega(\sigma)$ by definition and the
  subject reduction theorem (note that $t : \omega(\tau)$ because
  $\val{\tau}{\xi}{\omega} \in \Cb_{\omega(\tau)}$). The set
  $\val{\sigma}{\xi}{\omega}$ is a candidate, and
  $\app{(\abs{x}{s'})}{t'}$ is neutral, so in
  $\val{\sigma}{\xi}{\omega}$ if all its reducts are. Thus assume
  $\app{(\abs{x}{s'})}{t'} \arrW u$. If
  $u = \app{(\abs{x}{s'})}{t''}$ with $t' \arrW t''$ or
  $u = \app{(\abs{x}{s''})}{t'}$ with $s' \arrW s''$, then
  $u \in \val{\sigma}{\xi}{\omega}$ by the inductive hypothesis. So
  assume $u = s'[x:=t']$. We have $s[x:=t] \arrW^* s'[x:=t']$ by
  monotonicity and stability of $\arrW$. Therefore
  $u = s'[x:=t'] \in \val{\sigma}{\xi}{\omega}$, because
  $s[x:=t] \in \val{\sigma}{\xi}{\omega}$ and
  $\val{\sigma}{\xi}{\omega}$ is a candidate and hence closed under
  $\arrW$.

  A similar reasoning applies to $s[\alpha:=\varphi]$.
\end{proof}

\begin{lemma}\label{lem_val_computable}
  If $\sigma$ is a type constructor, $\omega$ is closed for~$\sigma$,
  and $\xi$ is an $\omega$-valuation, then $\val{\sigma}{\xi}{\omega}
  \in \Cb_{\omega(\sigma)}$.
\end{lemma}

\begin{proof}
  By induction on the structure of~$\sigma$ we show that
  $\val{\sigma}{\xi}{\omega} \in \Cb_{\omega(\sigma)}$ for all
  suitable $\omega,\xi$. First, if $\sigma = \alpha$ is a type
  constructor variable~$\alpha$ then $\val{\sigma}{\xi}{\omega} =
  \xi(\alpha) \in \Cb_{\omega(\sigma)}$ by definition. If $\sigma =
  \nat$ then $\val{\nat}{\xi}{\omega} \in \Cb_{\nat}$ by
  Lemma~\ref{lem_nat_computable}.

  Assume $\sigma = \tau_1 \arrtype \tau_2$. We check the conditions in
  Definition~\ref{def_candidate}.
  \begin{enumerate}
  \item Let $t \in \val{\tau_1\arrtype\tau_2}{\xi}{\omega}$ and assume
    there is an infinite reduction $t \arrW t_1 \arrW t_2
    \arrW t_3 \arrW \ldots$. By the inductive hypothesis
    $\val{\tau_1}{\xi}{\omega}$ and $\val{\tau_2}{\xi}{\omega}$ are
    candidates. Let~$x$ be a fresh variable. Then $x^{\omega(\tau_1)}
    : \omega(\tau_1)$ and $x^{\omega(\tau_1)} \in
    \val{\tau_1}{\xi}{\omega}$ because it is neutral and normal. Thus
    $t x \in \val{\tau_2}{\xi}{\omega} \subseteq \SN$. But $t x
    \arrW t_1 x \arrW t_2 x \arrW t_3 x \arrW
    \ldots$. Contradiction.
  \item Let $t \in \val{\tau_1\arrtype\tau_2}{\xi}{\omega}$ and $t
    \arrW t'$. Let $u \in \val{\tau_1}{\xi}{\omega}$ be such that
    $u : \omega(\tau_1)$. Then $t u \in \val{\tau_2}{\xi}{\omega}$. By
    the inductive hypothesis $\val{\tau_2}{\xi}{\omega}$ is a
    candidate, so $t' u \in \val{\tau_2}{\xi}{\omega}$. Also note that
    $t' : \omega(\tau_1 \arrtype \tau_2)$ by the subject reduction
    lemma. Hence $t' \in \val{\tau_1\arrtype\tau_2}{\xi}{\omega}$.
  \item Let $t$ be neutral such that $t : \omega(\tau_1 \arrtype
    \tau_2)$. Assume for every~$t'$ with $t \arrW t'$ we have $t'
    \in \val{\tau_1\arrtype\tau_2}{\xi}{\omega}$. Let $u \in
    \val{\tau_1}{\xi}{\omega}$ be such that $u : \omega(\tau_1)$. By
    the inductive hypothesis $\val{\tau_1}{\xi}{\omega}$ is a
    candidate, so $u \in \SN$. By induction on~$\nu(u)$ we show that
    $t u \in \val{\tau_2}{\xi}{\omega}$. Assume $t u \arrW t''$. We
    show $t'' \in \val{\tau_2}{\xi}{\omega}$. Because~$t$ is neutral,
    $t u$ cannot be a redex. So there are two cases.
    \begin{itemize}
    \item $t'' = t u'$ with $u \arrW u'$. Then $u' \in
      \val{\tau_1}{\xi}{\omega}$ because~$\val{\tau_1}{\xi}{\omega}$
      is a candidate, and~$u' : \omega(\tau_1)$ by the subject
      reduction lemma. So $t u' \in \val{\tau_2}{\xi}{\omega}$ by the
      inductive hypothesis for~$u$.
    \item $t'' = t' u$ with $t \arrW t'$. Then $t' \in
      \val{\tau_1\arrtype\tau_2}{\xi}{\omega}$ by point~2 above. So
      $t' u \in \val{\tau_2}{\xi}{\omega}$.
    \end{itemize}
    We have thus shown that if $t u \arrW t''$ then $t'' \in
    \val{\tau_2}{\xi}{\omega}$. By the (main) inductive hypothesis
    $\val{\tau_2}{\omega,\xi}{\Gamma}$ is a candidate. Because $t u$
    is neutral, the above implies $t u \in
    \val{\tau_2}{\xi}{\omega}$. Since $u \in
    \val{\tau_1}{\xi}{\omega}$ was arbitrary with $u :
    \omega(\tau_1)$, we have shown $t \in
    \val{\tau_1\arrtype\tau_2}{\xi}{\omega}$.
  \item Assume $t_1,t_2 \in \val{\tau_1\arrtype\tau_2}{\xi}{\omega}$.
    We have already shown that this implies $t_1,t_2 \in \SN$. Let $s
    = \circ_{\omega(\tau_1\arrtype\tau_2)} t_1 t_2$. We show $s \in
    \val{\tau_1\arrtype\tau_2}{\xi}{\omega}$ by induction on $\nu(t_1)
    + \nu(t_2)$. Note that $s : \omega(\tau_1\arrtype\tau_2)$ because
    $t_i : \omega(\tau_1\arrtype\tau_2)$. Since $s$ is neutral, we
    have already seen in point~3 above that to prove $s \in
    \val{\tau_1\arrtype\tau_2}{\xi}{\omega}$ it suffices to show that
    $s' \in \val{\tau_1\arrtype\tau_2}{\xi}{\omega}$ whenever $s
    \arrW s'$. If $s' = \circ_{\omega(\tau_1\arrtype\tau_2)} t_1'
    t_2$ with $t_1 \arrW t_1'$, then note that $t_1' \in
    \val{\tau_1\arrtype\tau_2}{\xi}{\omega}$ because we have already
    shown that $\val{\tau_1\arrtype\tau_2}{\xi}{\omega}$ is closed
    under $\arrW$; thus, we can complete by the induction
    hypothesis. If $s' = \circ_{\omega(\tau_1\arrtype\tau_2)} t_1
    t_2'$, we complete in the same way.  The only alternative is that
    $s' = \abs{x:\omega(\tau_1)}{\circ_{\omega(\tau_2)}(t_1x)(t_2x)}$.
    Let $u \in \val{\tau_1}{\xi}{\omega}$. Then $u : \omega(\tau_1)$
    because $\val{\tau_1}{\xi}{\omega} \in \Cb_{\omega(\tau_1)}$ by
    the inductive hypothesis. Since $t_1,t_2 \in
    \val{\tau_1\arrtype\tau_2}{\xi}{\omega}$, we have that $t_1 u$ and
    $t_2 u$ are in $\val{\tau_2}{\xi}{\omega}$ by definition. Since
    $\val{\tau_2}{\xi}{\omega}$ is a candidate, this means that
    $\circ_{\omega(\tau_2)} (t_1 u) (t_2 u) = (\circ_{\omega(\tau_2)}
    (t_1 x) (t_2 x))[x:=u]$ is in $\val{\tau_2}{\xi}{\omega}$ as well.
    By Lemma~\ref{lem_abstraction_computable}, we conclude that $s'
    \in \val{\tau_1\arrtype\tau_2}{\xi}{\omega}$.
  \item Let $t \in \SN$ satisfy $t : \nat$, and let $s =
    \lift_{\omega(\tau_1\arrtype\tau_2)}(t)$. We show $s \in
    \val{\tau_1\arrtype\tau_2}{\xi}{\omega}$ by induction
    on~$\nu(t)$. We have $s : \omega(\tau_1\arrtype\tau_2)$ because $t
    : \nat$. Since~$s$ is neutral, we have already proved above in
    point~3 that it suffices to show that $s' \in
    \val{\tau_1\arrtype\tau_2}{\xi}{\omega}$ whenever $s \arrW
    s'$. If $s' = \lift_{\omega(\tau_1\arrtype\tau_2)}(t')$ with $t
    \arrW t'$ then still $t' \in \SN$ and $t' : \nat$, so $s' \in
    \val{\tau_1\arrtype\tau_2}{\xi}{\omega}$ by the inductive
    hypothesis. The only alternative is that $s' = \lambda x :
    \omega(\tau_1) . \lift_{\omega(\tau_2)}(t)$. Let $u \in
    \val{\tau_1}{\xi}{\omega}$ be such that $u :
    \omega(\tau_1)$. Because $\val{\tau_2}{\xi}{\omega} \in
    \Cb_{\omega(\tau_2)}$ by the (main) inductive hypothesis
    for~$\sigma$, we have $\lift_{\omega(\tau_2)}(t) \in
    \val{\tau_2}{\xi}{\omega}$. Since $\lift_{\omega(\tau_2)}(t) =
    (\lift_{\omega(\tau_2)}x)[\subst{x}{t}]$ we obtain $s' \in
    \val{\tau_1\arrtype\tau_2}{\xi}{\omega}$ by
    Lemma~\ref{lem_abstraction_computable}.
  \item Let $t \in \val{\tau_1\arrtype\tau_2}{\xi}{\omega}$.  We show
    $s := \flatten_{\omega(\tau_1\arrtype\tau_2)}t \in \SN$. We have
    already shown $t \in \SN$ in point~1 above. Thus any infinite
    reduction starting from~$s$ must have the form $s \arrW^*
    \flatten_{\omega(\tau_1\arrtype\tau_2)}t' \arrW
    \flatten_{\omega(\tau_2)}(t' (\lift_{\omega(\tau_1)}0)) \arrW
    \ldots$ with $t \arrW^* t'$. We have already shown in point~2
    above that $\val{\tau_1\arrtype\tau_2}{\xi}{\omega}$ is closed
    under~$\arrW$, so $t' \in
    \val{\tau_1\arrtype\tau_2}{\xi}{\omega}$. By the inductive
    hypothesis $\val{\tau_1}{\xi}{\omega} \in\Cb_{\omega(\tau_1)}$, so
    $\lift_{\omega(\tau_1)}0 \in \val{\tau_1}{\xi}{\omega}$ by
    property~5 of candidates. Hence $t' (\lift_{\omega(\tau_1)}0) \in
    \val{\tau_2}{\xi}{\omega}$ by definition. But by the inductive
    hypothesis~$\val{\tau_2}{\xi}{\omega}$ is a candidate, so
    $\flatten_{\omega(\tau_2)}(t'(\lift_{\omega(\tau_1)}0))\in\SN$. Contradiction.
  \end{enumerate}

  Assume $\sigma = \forall(\alpha:\kappa)\tau$. We check the
  conditions in Definition~\ref{def_candidate}.
  \begin{enumerate}
  \item Let $t \in \val{\forall(\alpha:\kappa)\tau}{\xi}{\omega}$
    and assume there is an infinite reduction $t \arrW t_1 \arrW
    t_2 \arrW t_3 \arrW \ldots$. Recall that~$\chi_\kappa$ from
    Definition~\ref{def_types} is a closed type constructor of
    kind~$\kappa$. By Lemma~\ref{lem_chi_kappa_computable} we have
    $\val{\chi_{\kappa}}{}{} \in \Cb_{\chi_\kappa}$. Then $t
    \chi_\kappa \in
    \val{\tau}{\xi[\subst{\alpha}{\val{\chi_\kappa}{}{}}]}{\omega[\subst{\alpha}{\chi_\kappa}]}$. By
    the inductive hypothesis
    $\val{\tau}{\xi[\subst{\alpha}{\val{\chi_\kappa}{}{}}]}{\omega[\subst{\alpha}{\chi_\kappa}]}$
    is a candidate, so $t \chi_\kappa \in \SN$. But $t \chi_\kappa
    \arrW t_1 \chi_\kappa \arrW t_2 \chi_\kappa \arrW t_3
    \chi_\kappa \arrW \ldots$. Contradiction.
  \item Let $t \in \val{\forall\alpha\tau}{\xi}{\omega}$ and $t
    \arrW t'$. By the subject reduction lemma $t' :
    \omega(\forall\alpha\tau)$. Let~$\varphi$ be a closed type
    constructor of kind~$\kappa$ and~$X \in \Cb_{\varphi}$. Then $t
    \varphi \in
    \val{\tau}{\xi[\subst{\alpha}{X}]}{\omega[\subst{\alpha}{\varphi}]}$. By
    the inductive hypothesis
    $\val{\tau}{\xi[\subst{\alpha}{X}]}{\omega[\subst{\alpha}{\varphi}]}$
    is a candidate, so $t' \varphi \in
    \val{\tau}{\xi[\subst{\alpha}{X}]}{\omega[\subst{\alpha}{\varphi}]}$. Therefore
    $t' \in \val{\forall\alpha\tau}{\xi}{\omega}$.
  \item Let $t$ be neutral such that $t :
    \omega(\forall\alpha\tau)$, and assume that for every~$t'$ with
    $t \arrW t'$ we have $t' \in
    \val{\forall\alpha\tau}{\xi}{\omega}$. Let~$\varphi$ be a closed
    type constructor of kind~$\kappa$ and~$X \in
    \Cb_{\varphi}$. Assume $t \varphi \arrW t''$. Then $t'' = t'
    \varphi$ with $t \arrW t'$, because~$t$ is neutral. Hence $t
    \varphi \arrW t' \varphi \in
    \val{\tau}{\xi[\subst{\alpha}{X}]}{\omega[\subst{\alpha}{\varphi}]}$. By
    the inductive
    hypothesis~$\val{\tau}{\xi[\subst{\alpha}{X}]}{\omega[\subst{\alpha}{\varphi}]}$
    is a candidate. Also $t \varphi$ is neutral, so $t \varphi \in
    \val{\tau}{\xi[\subst{\alpha}{X}]}{\omega[\subst{\alpha}{\varphi}]}$
    because~$t''$ was arbitrary with $t \varphi \arrW t''$. This
    implies that $t \in \val{\forall\alpha\tau}{\xi}{\omega}$.
  \item Assume $t_1,t_2 \in
    \val{\forall\alpha\tau}{\xi}{\omega}$. We have already shown
    that this implies $t_1,t_2 \in \SN$. We prove
    $\circ_{\omega(\forall\alpha\tau)} t_1 t_2 \in
    \val{\forall\alpha\tau}{\xi}{\omega}$ by induction on $\nu(t_1)
    + \nu(t_2)$. Since $s := \circ_{\omega(\forall\alpha\tau)} t_1
    t_2$ is neutral, we have already proven that it suffices to show
    that $s' \in \val{\forall\alpha\tau}{\xi}{\omega}$ whenever $s
    \arrW s'$. The cases when $t_1$ or $t_2$ are reduced are
    immediate with the induction hypotheses. The only remaining case
    is when $s'=\tabs{\alpha}{\circ_{\omega(\tau)} (t_1 \alpha) (t_2
      \alpha)}$.  For all closed type constructors $\varphi$ of
    kind~$\kappa$ and all $X \in \Cb_{\varphi}^\Gamma$ we have both
    $t_1 \varphi$ and $t_2 \varphi$ in
    $\val{\tau}{\xi[\subst{\alpha}{X}]}{\omega[\subst{\alpha}{\varphi}]}$
    (by definition of $t_1,t_2 \in
    \val{\forall\alpha\tau}{\xi}{\omega}$). Let $\omega' =
    \omega[\subst{\alpha}{\varphi}]$. By bound variable renaming, we
    may assume $\omega(\alpha) = \alpha$ and $\alpha$ does not occur
    in~$t_1,t_2$. Because
    $\val{\tau}{\xi[\subst{\alpha}{X}]}{\omega[\subst{\alpha}{\varphi}]}$
    is a candidate by the inductive hypothesis for~$\sigma$, we have
    \[
    \circ_{\omega'(\tau)} (t_1 \varphi)
    (t_2\varphi) = (\circ_{\omega(\tau)} (t_1 \alpha) (t_2
    \alpha))[\subst{\alpha}{\varphi}] \in
    \val{\tau}{\xi[\subst{\alpha}{X}]}{\omega[\subst{\alpha}{\varphi}]}.
    \]
    Hence $s' \in \val{\forall\alpha\tau}{\xi}{\omega}$ by
    Lemma~\ref{lem_abstraction_computable}.
  \item Let $t \in \SN$ be such that $t : \nat$. By induction
    on~$\nu(t)$ we show $s := \lift_{\omega(\forall\alpha\tau)}(t)
    \in \val{\forall\alpha\tau}{\xi}{\omega}$. First note that $s :
    \omega(\forall\alpha\tau)$. Since~$s$ is neutral, by the already
    proven point~3 above, it suffices to show that $s' \in
    \val{\forall\alpha\tau}{\xi}{\omega}$ whenever $s \arrW
    s'$. The case when~$t$ is reduced is immediate by the inductive
    hypothesis. The only remaining case is when $s' =
    \tabs{\alpha}{\lift_{\omega(\tau)}(t)}$ (without loss of
    generality assuming $\omega(\alpha) = \alpha$). Let $\varphi$ be a
    closed type constructor of kind~$\kappa$ and let $X \in
    \Cb_\varphi$. Because
    $\val{\tau}{\xi[\subst{\alpha}{X}]}{\omega[\subst{\alpha}{\varphi}]}$
    is a candidate, we have
    \[
    \lift_{\omega[\subst{\alpha}{\varphi}](\tau)}(t) =
    (\lift_{\omega(\tau)}(t))[\subst{\alpha}{\varphi}] \in
    \val{\tau}{\xi[\subst{\alpha}{X}]}{\omega[\subst{\alpha}{\varphi}]}.
    \]
    This implies $s' \in \val{\forall\alpha\tau}{\xi}{\omega}$.
  \item Let $t \in \val{\forall\alpha\tau}{\xi}{\omega}$. We show $s
    := \flatten_{\omega(\forall\alpha\tau)}t \in \SN$. We have
    already shown $t \in \SN$ in point~1 above. Thus any infinite
    reduction starting from~$s$ must have the form $s \arrW^*
    \flatten_{\omega(\forall\alpha\tau)}t' \arrW
    \flatten_{\omega(\tau)[\subst{\alpha}{\chi_\kappa}]}(t'
    \chi_\kappa) \arrW \ldots$ with $t \arrW^* t'$ (assuming
    $\omega(\alpha) = \alpha$ without loss of generality). We have
    already shown in point~2 above that
    $\val{\forall\alpha\tau}{\xi}{\omega}$ is closed
    under~$\arrW$, so $t' \in
    \val{\forall\alpha\tau}{\xi}{\omega}$. We have
    $\val{\chi_\kappa}{}{} \in \Cb_{\chi_\kappa}$ by
    Lemma~\ref{lem_chi_kappa_computable}. Since~$\chi_\kappa$ is also
    closed, we have $t' \chi_\kappa \in
    \val{\tau}{\xi[\subst{\alpha}{\val{\chi_\kappa}{}{}}]}{\omega[\subst{\alpha}{\chi_\kappa}]}$
    by definition of $\val{\forall\alpha\tau}{\xi}{\omega}$. By the
    inductive hypothesis
    $\val{\tau}{\xi[\subst{\alpha}{\val{\chi_\kappa}{}{}}]}{\omega[\subst{\alpha}{\chi_\kappa}]}
    \in \Cb_{\omega[\subst{\alpha}{\chi_\kappa}](\tau)}$. Hence
    $\flatten_{\omega[\subst{\alpha}{\chi_\kappa}](\tau)}(t'\chi_\kappa)\in\SN$. But
    $\omega[\subst{\alpha}{\chi_\kappa}](\tau) =
    \omega(\tau)[\subst{\alpha}{\chi_\kappa}]$ because~$\chi_\kappa$
    is closed and $\omega(\alpha) = \alpha$. Contradiction.
  \end{enumerate}

  Assume $\sigma = \varphi\psi$, with $\psi$ of kind~$\kappa_1$ and
  $\varphi$ of kind~$\kappa_1\arrkind\kappa_2$. By the inductive
  hypothesis $\val{\psi}{\xi}{\omega} \in \Cb_{\omega(\psi)}$ and
  $\val{\varphi}{\xi}{\omega} \in \Cb_{\omega(\varphi)}$. Because
  applying~$\omega$ does not change kinds, we have
  $\val{\varphi\psi}{\xi}{\omega} =
  \val{\varphi}{\xi}{\omega}(\omega(\psi), \val{\psi}{\xi}{\omega})
  \in \Cb_{\omega(\varphi\psi)}$, by the definition of candidates of a
  type constructor with kind~$\kappa_1\arrkind\kappa_2$ (note that
  $\omega(\psi)$ is closed, because $\omega$ is closed for~$\sigma$).

  Finally, assume $\sigma = \lambda(\alpha:\kappa)\varphi$. Let $\psi$
  be a closed type constructor of kind~$\kappa$ and $X \in
  \Cb_{\psi}$. By the inductive hypothesis
  $\val{\lambda(\alpha:\kappa)\varphi}{\xi}{\omega}(\psi,X) =
  \val{\varphi}{\xi[\subst{\alpha}{X}]}{\omega[\subst{\alpha}{\psi}]}
  \in \Cb_{\omega[\subst{\alpha}{\psi}](\varphi)}$. Because $\psi$ is
  closed we have $\omega[\subst{\alpha}{\psi}](\varphi) =
  \omega(\varphi[\subst{\alpha}{\psi}]) =_\beta
  \omega((\lambda\alpha.\varphi)\psi) = \omega(\sigma\psi) =
  \omega(\sigma)\psi$. By Lemma~\ref{lem_beta_candidate} this implies
  that $\val{\sigma}{\xi}{\omega} \in \Cb_{\omega(\sigma)}$.
\end{proof}

\begin{lemma}\label{lem_circ}
  $\circ \in \val{\forall\alpha . \alpha \arrtype \alpha \arrtype
    \alpha}{}{}$ for $\circ \in \{ \oplus, \otimes \}$.
\end{lemma}

\begin{proof}
  Follows from definitions and property~4 of candidates.
\end{proof}

\begin{lemma}\label{lem_lift}
  $\lift \in \val{\forall\alpha.\nat\arrtype\alpha}{}{}$.
\end{lemma}

\begin{proof}
  Follows from definitions and property~5 of candidates.
\end{proof}

\begin{lemma}\label{lem_flatten}
  $\flatten \in \val{\forall\alpha.\alpha\arrtype\nat}{}{}$.
\end{lemma}

\begin{proof}
  Follows from definitions and property~6 of candidates.
\end{proof}

\begin{lemma}\label{lem_val_subst}
  For any type constructors~$\sigma,\tau$ with $\alpha \notin
  \FV(\tau)$, a mapping~$\omega$ closed for~$\sigma$ and for~$\tau$,
  and an $\omega$-valuation~$\xi$, we have:
  \[
  \val{\sigma[\subst{\alpha}{\tau}]}{\xi}{\omega} =
  \val{\sigma}{\xi[\subst{\alpha}{\val{\tau}{\xi}{\omega}}]}{\omega[\subst{\alpha}{\omega(\tau)}]}.
  \]
\end{lemma}

\begin{proof}
  Let~$\omega' = \omega[\subst{\alpha}{\omega(\tau)}]$ and $\xi' =
  \xi[\subst{\alpha}{\val{\tau}{\xi}{\omega}}]$. First note
  that~$\omega$ is closed for~$\sigma[\subst{\alpha}{\tau}]$
  and~$\omega'$ is closed for~$\sigma$. We proceed by induction
  on~$\sigma$. If $\alpha \notin \FV(\sigma)$ then the claim is
  obvious. If $\sigma = \alpha$ then
  $\val{\sigma[\subst{\alpha}{\tau}]}{\xi}{\omega} =
  \val{\tau}{\xi}{\omega} = \val{\sigma}{\xi'}{\omega'}$.

  Assume $\sigma = \sigma_1\arrtype\sigma_2$. We show
  $\val{\sigma[\subst{\alpha}{\tau}]}{\xi}{\omega} \subseteq
  \val{\sigma}{\xi'}{\omega'}$. Let $t \in
  \val{\sigma[\subst{\alpha}{\tau}]}{\xi}{\omega}$. We have $t :
  \omega(\sigma[\subst{\alpha}{\tau}])$, so $t : \omega'(\sigma)$. Let
  $u \in \val{\sigma_1}{\xi'}{\omega'}$. By the inductive hypothesis
  $u \in \val{\sigma_1[\subst{\alpha}{\tau}]}{\xi}{\omega}$. Hence $t
  u \in \val{\sigma_2[\subst{\alpha}{\tau}]}{\xi}{\omega} =
  \val{\sigma_2}{\xi'}{\omega'}$, where the last equality follows from
  the inductive hypothesis. Thus $t \in
  \val{\sigma}{\xi'}{\omega'}$. The other direction is analogous. The
  case $\sigma = \forall\alpha\sigma'$ is also analogous.

  Assume $\sigma = \varphi\psi$. We have
  $\val{\sigma[\subst{\alpha}{\tau}]}{\xi}{\omega} =
  \val{\varphi[\subst{\alpha}{\tau}]}{\xi}{\omega}(\omega(\psi[\subst{\alpha}{\tau}]),
  \val{\psi[\subst{\alpha}{\tau}]}{\xi}{\omega}) =
  \val{\varphi[\subst{\alpha}{\tau}]}{\xi}{\omega}(\omega'(\psi),
  \val{\psi[\subst{\alpha}{\tau}]}{\xi}{\omega}) =
  \val{\varphi}{\xi'}{\omega'}(\omega'(\psi),
  \val{\psi}{\xi'}{\omega'})$ where the last equality follows from the
  inductive hypothesis.

  Finally, assume $\sigma = \lambda(\beta:\kappa)\varphi$. Let $\psi
  \in \Tc_\kappa$ be closed and let $X \in \Cb_\psi$. We have
  $\val{\sigma[\subst{\alpha}{\tau}]}{\xi}{\omega}(\psi,X) =
  \val{\varphi[\subst{\alpha}{\tau}]}{\xi[\subst{\beta}{X}]}{\omega[\subst{\beta}{\tau}]}
  =
  \val{\varphi}{\xi'[\subst{\beta}{X}]}{\omega'[\subst{\beta}{\tau}]}
  = \val{\sigma}{\xi'}{\omega'}(\psi,X)$ where we use the inductive
  hypothesis in the penultimate equality.
\end{proof}

\begin{lemma}\label{lem_forall}
  Let $\tau$ be a type constructor of kind~$\kappa$. Assume $\omega$
  is closed for $\forall\alpha\sigma$ and for~$\tau$. If $t \in
  \val{\forall(\alpha:\kappa)\sigma}{\xi}{\omega}$ then $t
  (\omega(\tau)) \in \val{\sigma[\subst{\alpha}{\tau}]}{\xi}{\omega}$.
\end{lemma}

\begin{proof}
  By Lemma~\ref{lem_val_computable} we have~$\val{\tau}{\xi}{\omega}
  \in \Cb_{\omega(\tau)}$. So $t (\omega(\tau)) \in
  \val{\sigma}{\xi[\subst{\alpha}{\val{\tau}{\xi}{\omega}}]}{\omega[\subst{\alpha}{\omega(\tau)}]}$
  by $t \in \val{\forall(\alpha:\kappa)\sigma}{\xi}{\omega}$. Hence
  $t (\omega(\tau)) \in
  \val{\sigma[\subst{\alpha}{\tau}]}{\xi}{\omega}$ by
  Lemma~\ref{lem_val_subst}.
\end{proof}

\begin{lemma}\label{lem_beta_val}
  If $\omega$ is closed for~$\sigma,\sigma'$ and $\sigma =_\beta
  \sigma'$ then $\val{\sigma}{\xi}{\omega} =
  \val{\sigma'}{\xi}{\omega}$.
\end{lemma}

\begin{proof}
  It suffices to show the lemma for the case when~$\sigma$ is a
  $\beta$-redex. Then the general case follows by induction
  on~$\sigma$ and the length of reduction to a common reduct.

  So assume $(\lambda\alpha\tau)\sigma \to_\beta
  \tau[\subst{\alpha}{\sigma}]$. We have
  $\val{(\lambda\alpha\tau)\sigma}{\xi}{\omega} =
  \val{\lambda\alpha\tau}{\xi}{\omega}(\omega(\sigma),
  \val{\sigma}{\xi}{\omega}) =
  \val{\tau}{\xi[\subst{\alpha}{\val{\sigma}{\xi}{\omega}}]}{\omega[\subst{\alpha}{\omega(\sigma)}]}
  = \val{\tau[\subst{\alpha}{\sigma}]}{\xi}{\omega}$ where the last
  equality follows from Lemma~\ref{lem_val_subst}.
\end{proof}

A mapping~$\omega$ on type constructors is extended in the obvious way
to a mapping on terms. Note that $\omega$ also acts on the type
annotations of variable occurrences, e.g.~$\omega(\lambda x : \alpha
. x^\alpha) = \lambda x : \omega(\alpha) . x^{\omega(\alpha)}$.

\begin{lemma}\label{lem_typable_computable}
  If $t : \sigma$ and $\omega$ is closed for~$\sigma$ and
  $\FTV(\omega(t)) = \emptyset$ then $\omega(t) \in
  \val{\sigma}{\xi}{\omega}$.
\end{lemma}

\begin{proof}
  We prove by induction on the structure of~$t$ that if $t : \sigma$
  and $\omega$ is closed for~$\sigma$ and $\FTV(\omega(t)) =
  \emptyset$ and $x_1^{\tau_1},\ldots,x_n^{\tau_n}$ are all free
  variable occurrences in the canonical representative of~$t$ (so
  each~$\tau_i$ is $\beta$-normal), then for all
  $u_1\in\val{\tau_1}{\xi}{\omega},\ldots,u_n\in\val{\tau_n}{\xi}{\omega}$
  we have $\omega(t)[\subst{x_1}{u_1},\ldots,\subst{x_n}{u_n}] \in
  \val{\sigma}{\xi}{\omega}$. This suffices because
  $\omega(x_i^{\tau_i}) \in \val{\tau_i}{\xi}{\omega}$. Note that
  $\omega$ is closed for each~$\tau_i$ because $\FTV(\omega(t)) =
  \emptyset$ and~$t$ is typed, so no type constructor variable
  occurring free in~$\tau_i$ can be bound in~$t$ by a~$\Lambda$;
  e.g.~$\Lambda \alpha . x^\alpha$ is not a valid typed term (we
  assume~$\tau_i$ to be in $\beta$-normal form). For brevity, we use
  the notation $\omega^*(t) =
  \omega(t)[\subst{x_1}{u_1},\ldots,\subst{x_n}{u_n}]$. Note that
  $\omega^*(t) : \omega(\sigma)$.

  By the generation lemma for $t : \sigma$ there is a type~$\sigma'$
  such that $\sigma' =_\beta \sigma$ and $\FV(\sigma') \subseteq
  \FTV(t)$ and one of the cases below holds. Note that~$\omega$ is
  closed for~$\sigma'$ because it is closed for~$\sigma$ and
  $\FTV(\omega(t)) = \emptyset$. By Lemma~\ref{lem_beta_val} it
  suffices to show $\omega^*(t) \in \val{\sigma'}{\xi}{\omega}$.
  \begin{itemize}
  \item If $t = x_1^{\sigma'}$ then $\omega(t)[\subst{x_1}{u_1}] =
    (x_1^{\omega(\sigma')})[\subst{x_1}{u_1}] = u_1 \in
    \val{\sigma'}{\xi}{\omega}$ by assumption.
  \item If $t = n$ is a natural number and $\sigma' = \nat$ then $t
    \in \val{\nat}{}{}$ by definition.
  \item If $t$ is a function symbol then the claim follows from
    Lemma~\ref{lem_circ}, Lemma~\ref{lem_lift} or
    Lemma~\ref{lem_flatten}.
  \item If $t = \abs{x:\sigma_1}{s}$ then
    $\sigma' = \sigma_1\arrtype\sigma_2$ and $s : \sigma_2$. Hence
    $\omega$ is closed for~$\sigma_2$. Let
    $u \in \val{\sigma_1}{\xi}{\omega}$. By the inductive hypothesis
    $\omega^*(s)[\subst{x}{u}] \in \val{\sigma_2}{\xi}{\omega}$. Hence
    $\omega^*(t) \in \val{\sigma'}{\xi}{\omega}$ by
    Lemma~\ref{lem_abstraction_computable}.
  \item If $t = \tabs{\alpha:\kappa}{s}$ then $\sigma' =
    \forall\alpha\tau$ and $s : \tau$. Let $\psi$ be a closed type
    constructor of kind~$\kappa$ and let $X \in \Cb_\psi$. Let
    $\omega_1 = \omega[\subst{\alpha}{\psi}]$ and
    $\xi_1=\xi[\subst{\alpha}{X}]$. Then $\omega_1$ is closed
    for~$\tau$ and $\FTV(\omega_1(s)) = \emptyset$. By the inductive
    hypothesis $\omega_1^*(s) \in \val{\tau}{\xi_1}{\omega_1}$. We
    have $\omega_1^*(s) = \omega^*(s)[\subst{\alpha}{\psi}]$ (assuming
    $\alpha$ chosen fresh such that $\omega(\alpha) = \alpha$). Hence
    $\omega^*(t) \in \val{\tau}{\xi}{\omega}$ by
    Lemma~\ref{lem_abstraction_computable}.
  \item If $t = t_1 t_2$ then $t_1 : \tau\arrtype\sigma'$ and $t_2 :
    \tau$ and $\FV(\tau) \subseteq \FTV(t)$. Hence~$\omega$ is closed
    for~$\tau$ and for~$\tau\arrtype\sigma'$. By the inductive
    hypothesis $\omega^*(t_1) \in
    \val{\tau\arrtype\sigma'}{\xi}{\omega}$ and $\omega^*(t_2) \in
    \val{\tau}{\xi}{\omega}$. We have $\omega^*(t_2) :
    \omega(\tau)$. Then by definition $\omega^*(t) =
    (\omega^*(t_1))(\omega^*(t_2)) \in \val{\sigma'}{\xi}{\omega}$.
  \item If $t = s \psi$ then $s : \forall\alpha\tau$ and $\sigma' =
    \tau[\subst{\alpha}{\psi}]$. By the inductive hypothesis
    $\omega^*(s) \in \val{\forall\alpha\tau}{\xi}{\omega}$. Because
    $\FTV(\omega(t)) = \emptyset$, the mapping $\omega$ is closed
    for~$\psi$. So by Lemma~\ref{lem_forall} we have $\omega^*(t) =
    \omega^*(s) \omega(\psi) \in
    \val{\tau[\subst{\alpha}{\psi}]}{\xi}{\omega}$.\qedhere
  \end{itemize}
\end{proof}

{ \renewcommand{\thetheorem}{\ref{thm_sn}}
\begin{theorem}
  If $\Gamma \proves t : \sigma$ then $t \in \SN$.
\end{theorem}
\addtocounter{theorem}{-1}}

\begin{proof}
  For closed terms~$t$ and closed types~$\sigma$ this follows from
  Lemma~\ref{lem_typable_computable}, Lemma~\ref{lem_val_computable}
  and property~1 of candidates (Definition~\ref{def_candidate}). For
  arbitrary terms and types, this follows by closing the terms with an
  appropriate number of abstractions, and the types with corresponding
  $\forall$-quantifiers.
\end{proof}

{ \renewcommand{\thelemma}{\ref{lem_final_nat}}
\begin{lemma}
  The only final interpretation terms of type $\nat$ are the natural
  numbers.
\end{lemma}
\addtocounter{theorem}{-1}}

\begin{proof}
  We show by induction on~$t$ that if $t$ is a final interpretation
  term of type~$\nat$ then $t$ is a natural number. Because~$t$ is
  closed and in normal form, if it is not a natural number then it
  must have the form $\mathtt{f}_\sigma t_1 \ldots t_n$ for a function
  symbol $\mathtt{f}$. For concreteness assume $\mathtt{f} =
  \oplus$. Then $n \ge 2$. Because~$t$ is closed, $\sigma$ cannot be a
  type variable. It also cannot be an arrow or a $\forall$-type,
  because then $t$ would contain a redex. So $\sigma=\nat$. Then
  $t_1,t_2$ are final interpretation terms of type~$\nat$, hence
  natural numbers by the inductive hypothesis. But then $t$ contains a
  redex. Contradiction.
\end{proof}

\subsection{Weak monotonicity proof}\label{sec_weakly_monotone_proof}

We want to show that if $s \succeq s'$ then $t[\subst{x}{s}] \succeq
t[\subst{x}{s'}]$. A straightforward proof attempt runs into a problem
that, because of impredicativity of polymorphism, direct induction on
type structure is not possible. We adopt a method similar to Girard's
method of candidates from the termination proof.

\begin{defn}\label{def_wm_candidate}
  By induction on the kind~$\kappa$ of a type constructor~$\tau$ we
  define the set~$\Cb_\tau$ of all candidates of type
  constructor~$\tau$.

  First assume $\kappa=*$, i.e., $\tau$ is a type. A set~$X$ of terms
  of type~$\tau$ equipped with a binary relation~$\ge^X$ is a
  \emph{candidate of type~$\tau$} if it satisfies the following
  properties:
  \begin{enumerate}
  \item if $t \in X$ and $t' : \tau$ and $t' \leadsto t$ then $t' \in
    X$,
  \item if $t_1,t_2 \in X$ then $\circ_\tau t_1 t_2 \in X$ for $\circ
    \in \{\oplus,\otimes\}$,
  \item if $t : \nat$ then $\lift_\tau t \in X$.
  \end{enumerate}
  and the relation~$\ge^X$ satifies the following properties:
  \begin{enumerate}
  \item ${\succeq} \cap X \times X \subseteq {\ge^X}$,
  \item if $t_1 \ge^X t_2$ and $t_1' \leadsto t_1$ (resp.~$t_2'
    \leadsto t_2$) then $t_1' \ge^X t_2$ (resp.~$t_1 \ge^X t_2'$),
  \item if $t_1 \ge^X t_1'$ and $t_2 \ge^X t_2'$ then $\circ_\tau t_1
    t_2 \ge^X \circ_\tau t_1' t_2'$ for $\circ \in
    \{\oplus,\otimes\}$,
  \item if $t_1 \succeq_\nat t_2$ then $\lift_\tau(t_1) \ge^X
    \lift_\tau(t_2)$,
  \item if $t_1 \ge^X t_2$ then $\flatten_\tau(t_1) \succeq_\nat
    \flatten_\tau(t_2)$,
  \item $\ge^X$ is reflexive and transitive on~$X$.
  \end{enumerate}
  The relation~$\ge^X$ is a \emph{comparison candidate for~$X$},
  and~$X$ is a \emph{candidate set}.

  Now assume $\kappa = \kappa_1\arrkind\kappa_2$. A function $f :
  \Tc_{\kappa_1} \times \bigcup_{\xi\in\Tc_{\kappa_1}}\Cb_\xi \to
  \bigcup_{\xi\in\Tc_{\kappa_2}}\Cb_\xi$ is a \emph{candidate of type
    constructor~$\tau$} if for every closed type constructor~$\sigma$
  of kind~$\kappa_1$ and a candidate $X \in \Cb_\sigma$ we have
  $f(\sigma,X) \in \Cb_{\tau\sigma}$.
\end{defn}

\begin{lemma}\label{lem_beta_wm_candidate}
  If $\sigma =_\beta \sigma'$ then $\Cb_\sigma = \Cb_{\sigma'}$.
\end{lemma}

\begin{proof}
  Induction on the kind of~$\sigma$.
\end{proof}

\begin{defn}\label{def_wm_valuation}
  Let $\omega$ be a mapping from type constructor variables to type
  constructors (respecting kinds). The mapping~$\omega$ extends in an
  obvious way to a mapping from type constructors to type
  constructors. A mapping~$\omega$ is \emph{closed for~$\sigma$} if
  $\omega(\alpha)$ is closed for $\alpha \in \FV(\sigma)$ (then
  $\omega(\sigma)$ is closed).

  An \emph{$\omega$-valuation} is a mapping~$\xi$ on type constructor
  variables such that $\xi(\alpha) \in \Cb_{\omega(\alpha)}$.

  For each type constructor~$\sigma$, each mapping~$\omega$ closed
  for~$\sigma$, and each $\omega$-valuation~$\xi$, we define
  $\val{\sigma}{\xi}{\omega}$ by induction on~$\sigma$:
  \begin{itemize}
  \item $\val{\alpha}{\xi}{\omega} = \xi(\alpha)$ for a type
    constructor variable~$\alpha$,
  \item $\val{\nat}{\xi}{\omega}$ is the set of all terms~$t \in
    \Iterms$ such that $t : \nat$; equipped with the relation
    $\gteq{\nat}{\xi}{\omega} = \succeq_\nat$,
  \item $\val{\sigma \arrtype \tau}{\xi}{\omega}$ is the set of all
    terms~$t$ such that $t : \omega(\sigma\arrtype\tau)$ and:
    \begin{itemize}
    \item for all $s \in \val{\sigma}{\xi}{\omega}$ we have
      $\app{t}{s} \in \val{\tau}{\xi}{\omega}$, and
    \item if $s_1 \gteq{\sigma}{\xi}{\omega} s_2$ then $\app{t}{s_1}
      \gteq{\tau}{\xi}{\omega} \app{t}{s_2}$;
    \end{itemize}
    equipped with the
    relation~$\gteq{\sigma\arrtype\tau}{\xi}{\omega}$ defined by:
    \begin{itemize}
    \item $t_1 \gteq{\sigma\arrtype\tau}{\xi}{\omega} t_2$ iff
      $t_1,t_2 \in \val{\sigma\arrtype\tau}{\xi}{\omega}$ and for
      every $s \in \val{\sigma}{\xi}{\omega}$ we have $t_1 s
      \gteq{\tau}{\xi}{\omega} t_2 s$,
    \end{itemize}
  \item $\val{\forall(\alpha:\kappa)[\sigma]}{\xi}{\omega}$ is the set
    of all terms~$t$ such that $t : \omega(\forall\alpha[\sigma])$ and:
    \begin{itemize}
    \item for every closed type constructor~$\varphi$ of kind~$\kappa$
      and every $X \in \Cb_\varphi$ we have $\tapp{t}{\varphi} \in
      \val{\sigma}{\xi[\subst{\alpha}{X}]}{\omega[\subst{\alpha}{\varphi}]}$;
    \end{itemize}
    equipped with the
    relation~$\gteq{\forall\alpha[\sigma]}{\xi}{\omega}$ defined by:
    \begin{itemize}
    \item $t_1 \gteq{\forall(\alpha:\kappa)[\sigma]}{\xi}{\omega} t_2$
      iff $t_1,t_2 \in
      \val{\forall(\alpha:\kappa)[\sigma]}{\xi}{\omega}$ and for every
      closed type constructor~$\varphi$ of kind~$\kappa$ and every $X
      \in \Cb_\varphi$ we have $t_1 \varphi
      \gteq{\sigma}{\xi[\subst{\alpha}{X}]}{\omega[\subst{\alpha}{\varphi}]}
      t_2 \varphi$,
    \end{itemize}
  \item
    $\val{\varphi \psi}{\xi}{\omega} =
    \val{\varphi}{\xi}{\omega}(\omega(\psi),\val{\psi}{\xi}{\omega})$,
  \item
    $\val{\lambda(\alpha:\kappa)\varphi}{\xi}{\omega}(\psi,X) =
    \val{\varphi}{\xi[\subst{\alpha}{X}]}{\omega[\subst{\alpha}{\psi}]}$
    for closed $\psi \in \Tc_\kappa$ and $X \in \Cb_\psi$.
  \end{itemize}
  In the above, if e.g.~$\val{\psi}{\xi}{\omega} \notin
  \Cb_{\omega(\psi)}$ then $\val{\varphi \psi}{\xi}{\omega}$ is
  undefined.
\end{defn}

Note that if $t \in \val{\sigma}{\xi}{\omega}$ then $t :
\omega(\sigma)$, and if $t_1 \gteq{\sigma}{\xi}{\omega} t_2$ then
$t_1,t_2\in\val{\sigma}{\xi}{\omega}$. For brevity we
use~$\val{\sigma}{\xi}{\omega}$ to denote both the pair
$(\val{\sigma}{\xi}{\omega},{\gteq{\sigma}{\xi}{\omega}})$ and its
first element, depending on the context. For a type~$\tau$,
by~$\gteq{\tau}{\xi}{\omega}$ we always denote the second element of
the pair~$\val{\tau}{\xi}{\omega}$. If $\tau$ is closed then~$\xi$
and~$\omega$ do not matter and we simply write~$\geq_\tau$
and~$\val{\tau}{}{}$.

\begin{lemma}\label{lem_val_wm_computable}
  If $\sigma$ is a type constructor, $\omega$ is closed for~$\sigma$,
  and $\xi$ is an $\omega$-valuation, then $\val{\sigma}{\xi}{\omega}
  \in \Cb_{\omega(\sigma)}$.
\end{lemma}

\begin{proof}
  Induction on~$\sigma$. If $\sigma=\alpha$ then $\xi(\alpha) \in
  \Cb_{\omega(\alpha)}$ by definition. If $\sigma=\nat$ then this
  follows from definitions.

  Assume $\sigma=\sigma_1\arrtype\sigma_2$. We check the properties of
  a candidate set.
  \begin{enumerate}
  \item The first property follows from the inductive hypothesis and
    property~2 of comparison candidates.
  \item Let $t_1,t_2 \in \val{\sigma}{\xi}{\omega}$. We need to show
    $\circ_\omega(\sigma) t_1 t_2 \in
    \val{\sigma_1\arrtype\sigma_2}{\xi}{\omega}$.

    Let $s \in \val{\sigma_1}{\xi}{\omega}$. Then
    $\circ_{\omega(\sigma)} t_1 t_2 s \leadsto
    \circ_{\omega(\sigma_2)} (t_1 s) (t_2 s)$. Because $t_i \in
    \val{\sigma_1\arrtype\sigma_2}{\xi}{\omega}$, we have $t_i s \in
    \val{\sigma_2}{\xi}{\omega}$. By the inductive
    hypothesis~$\val{\sigma_2}{\xi}{\omega} \in
    \Cb_{\omega(\sigma_2)}$, so $\circ_{\omega(\sigma_2)} (t_1 s) (t_2
    s) \in \val{\sigma_2}{\xi}{\omega}$. Hence
    $\circ_{\omega(\sigma_2)} t_1 t_2 s \in
    \val{\sigma_2}{\xi}{\omega}$ by property~1 of candidate sets.

    Let $s_1 \gteq{\sigma_1}{\xi}{\omega} s_2$. Then $s_i \in
    \val{\sigma_1}{\xi}{\omega}$. Because $t_j \in
    \val{\sigma_1\arrtype\sigma_2}{\xi}{\omega}$, we have $t_j s_i \in
    \val{\sigma_2}{\xi}{\omega}$ and $t_j s_1
    \gteq{\sigma_2}{\xi}{\omega} t_j s_2$. By the inductive
    hypothesis~$\gteq{\sigma_2}{\xi}{\omega}$ is a comparison
    candidate for~$\val{\sigma_2}{\xi}{\omega}$. Thus
    $\circ_{\omega(\sigma_2)} (t_1 s_1) (t_2 s_1)
    \gteq{\sigma_2}{\xi}{\omega} \circ_{\omega(\sigma_2)} (t_1 s_2)
    (t_2 s_2)$ by property~3 of comparison candidates. This suffices
    by property~2 of comparison candidates.
  \item Let $t : \nat$. Then $\lift_{\omega(\sigma)} t :
    \omega(\sigma)$.

    Let $s \in \val{\sigma_1}{\xi}{\omega}$. Then
    $\lift_{\omega(\sigma)}t s \leadsto \lift_{\omega(\sigma_2)}
    t$. By the inductive hypothesis $\lift_{\omega(\sigma_2)} t \in
    \val{\sigma_2}{\xi}{\omega}$. Hence $\lift_{\omega(\sigma)}t s \in
    \val{\sigma_2}{\xi}{\omega}$ by property~1 of candidate sets.

    Let $s_1,s_2 \in \val{\sigma_1}{\xi}{\omega}$. By the inductive
    hypothesis~$\gteq{\sigma_2}{\xi}{\omega}$ is a comparison
    candidate for~$\val{\sigma_2}{\xi}{\omega}$. We have
    $\lift_{\omega(\sigma_2)} t \gteq{\sigma_2}{\xi}{\omega}
    \lift_{\omega(\sigma_2)} t$ by the reflexivity
    of~$\gteq{\sigma_2}{\xi}{\omega}$ (property~6 of comparison
    candidates). This suffices by property~2 of comparison candidates,
    because $\lift_{\omega(\sigma)}t s_i \leadsto
    \lift_{\omega(\sigma_2)} t$.
  \end{enumerate}
  Now we check the properties of a comparison candidate
  for~$\val{\sigma_1\arrtype\sigma_2}{\xi}{\omega}$.
  \begin{enumerate}
  \item Suppose $t_1 \succeq t_2$ with $t_1,t_2 \in
    \val{\sigma}{\xi}{\omega}$. Let $s \in
    \val{\sigma_1}{\xi}{\omega}$. Then $t_1 s \succeq t_2 s$ by the
    definition of~$\succeq$. Hence $t_1 s \gteq{\sigma_2}{\xi}{\omega}
    t_2 s$ by the inductive hypothesis.
  \item Follows from the inductive hypothesis and the already shown
    property~1 of candidate sets
    for~$\val{\sigma_1\arrtype\sigma_2}{\xi}{\omega}$.
  \item Assume $t_i \gteq{\sigma}{\xi}{\omega} t_i'$. Let $s \in
    \val{\sigma_1}{\xi}{\omega}$. We have $\circ_{\omega(\sigma)} t_1
    t_2 s \leadsto \circ_{\omega(\sigma_2)} (t_1 s) (t_2 s)$ and
    $\circ_{\omega(\sigma)} t_1' t_2' s \leadsto
    \circ_{\omega(\sigma_2)} (t_1' s) (t_2' s)$. Since
    $t_i,t_i'\in\val{\sigma}{\xi}{\omega}$, we have $t_i s
    \gteq{\sigma_2}{\xi}{\omega} t_i' s$ and $t_i s, t_i' s \in
    \val{\sigma_2}{\xi}{\omega}$. By the inductive hypothesis $\circ
    (t_1 s) (t_2 s) \gteq{\sigma_2}{\xi}{\omega} \circ (t_1' s) (t_2'
    s)$, so $\circ t_1 t_2 s \gteq{\sigma_2}{\xi}{\omega} \circ t_1'
    t_2' s$ by property~2 of comparison candidates. This implies
    $\circ t_1 t_2 \gteq{\sigma}{\xi}{\omega} \circ t_1' t_2'$.
  \item Follows from Lemma~\ref{lem:liftgreater} and property~1 of
    comparison candidates.
  \item Assume $t_1 \gteq{\sigma}{\xi}{\omega} t_2$. Then
    $\flatten_{\omega(\sigma)} t_i \leadsto
    \flatten_{\omega(\sigma_2)} (t_i (\lift_{\omega(\sigma_1)}0))$. By
    the inductive hypothesis and property~3 of candidate sets
    $\lift_{\omega(\sigma_1)}0 \in \val{\sigma_1}{\xi}{\omega}$. Hence
    $t_i (\lift_{\omega(\sigma_1)}0) \in \val{\sigma_2}{\xi}{\omega}$
    and $t_1 (\lift_{\omega(\sigma_1)}0) \gteq{\sigma_2}{\xi}{\omega}
    t_2 (\lift_{\omega(\sigma_1)}0)$. Thus by the inductive hypothesis
    $\flatten_{\omega(\sigma_2)} (t_1 (\lift_{\omega(\sigma_1)}0))
    \succeq_\nat \flatten_{\omega(\sigma_2)} (t_2
    (\lift_{\omega(\sigma_1)}0))$. This implies
    $\flatten_{\omega(\sigma)} t_1 \succeq_\nat
    \flatten_{\omega(\sigma)} t_2$.
  \item Follows directly from the inductive hypothesis.
  \end{enumerate}

  If $\sigma=\forall\alpha\tau$ then the proof is analogous to the
  case $\sigma=\sigma_1\arrtype\sigma_2$. If $\sigma=\varphi\psi$ or
  $\sigma=\lambda(\alpha:\kappa)\varphi$ then the claim follows from
  the inductive hypothesis and Lemma~\ref{lem_beta_wm_candidate}, like
  in the proof of Lemma~\ref{lem_val_computable}.
\end{proof}

\begin{lemma}\label{lem_wm_circ}
  $\circ \in \val{\forall \alpha . \alpha \arrtype \alpha \arrtype
    \alpha}{}{}$ for $\circ \in \{ \oplus, \otimes \}$.
\end{lemma}

\begin{proof}
  Let~$\tau$ be a closed type and let $X \in \Cb_{\tau}$. Let
  $\omega(\alpha) = \tau$ and $\xi(\alpha) = X$.

  Let $t_1,t_2 \in \val{\alpha}{\xi}{\omega} = X$. Then $\circ_{\tau}
  t_1 t_2 \in \val{\alpha}{\xi}{\omega}$ by property~2 of candidate
  sets.

  Let $t_2' \in \val{\alpha}{\xi}{\omega}$ be such that $t_2
  \gteq{\alpha}{\xi}{\omega} t_2'$, i.e., $t_2 \ge^X t_2'$. By
  properties~6 and~3 of comparison candidates we have we have
  $\circ_{\tau} t_1 t_2 \gteq{\alpha}{\xi}{\omega} \circ_{\tau} t_1
  t_2'$. This shows $\circ_{\tau} t_1 \in
  \val{\alpha\arrtype\alpha}{\xi}{\omega}$.

  Let $t_1' \in \val{\alpha}{\xi}{\omega}$ be such that $t_1
  \gteq{\alpha}{\xi}{\omega} t_1'$. Let $u \in
  \val{\alpha}{\xi}{\omega}$. By properties~6 and~3 of comparison
  candidates we have $\circ_{\tau} t_1 u \gteq{\alpha}{\xi}{\omega}
  \circ_{\tau} t_1' u$. Hence $\circ_{\tau} t_1
  \gteq{\alpha\arrtype\alpha}{\xi}{\omega} \circ_{\tau} t_1'$. This
  shows $\circ_{\tau} \in
  \val{\alpha\arrtype\alpha\arrtype\alpha}{\xi}{\omega}$.
\end{proof}

\begin{lemma}\label{lem_wm_lift}
  $\lift \in \val{\forall\alpha.\nat\arrtype\alpha}{}{}$.
\end{lemma}

\begin{proof}
  Let~$\tau$ be a closed type and let $X \in \Cb_{\tau}$. Let
  $\omega(\alpha) = \tau$ and $\xi(\alpha) = X$. By property~4 of
  comparison candidates we have $\lift_{\tau}s_1
  \gteq{\alpha}{\xi}{\omega} \lift_{\tau}s_2$ for all $s_i : \nat$
  with $s_1 \succeq_\nat s_2$. It remains to show that $\lift_{\tau}s
  \in \val{\alpha}{\xi}{\omega} = X$ for all $s : \nat$. This follows
  from property~3 of candidate sets.
\end{proof}

\begin{lemma}\label{lem_wm_flatten}
  $\flatten \in \val{\forall\alpha.\alpha\arrtype\nat}{}{}$.
\end{lemma}

\begin{proof}
  Follows from definitions and property~5 of comparison candidates.
\end{proof}

\begin{lemma}\label{lem_val_subst_wm}
  For any type constructors~$\sigma,\tau$ with $\alpha \notin
  \FV(\tau)$, a mapping~$\omega$ closed for~$\sigma$ and for~$\tau$,
  and an $\omega$-valuation~$\xi$, we have:
  \[
  \val{\sigma[\subst{\alpha}{\tau}]}{\xi}{\omega} =
  \val{\sigma}{\xi[\subst{\alpha}{\val{\tau}{\xi}{\omega}}]}{\omega[\subst{\alpha}{\omega(\tau)}]}.
  \]
\end{lemma}

\begin{proof}
  Let~$\omega' = \omega[\subst{\alpha}{\omega(\tau)}]$ and $\xi' =
  \xi[\subst{\alpha}{\val{\tau}{\xi}{\omega}}]$. The proof by
  induction on~$\sigma$ is analogous to the proof of
  Lemma~\ref{lem_val_subst}. The main difference is that in the case
  $\sigma = \sigma_1\arrtype\sigma_2$ we need to show that if e.g.~$t
  \in \val{\sigma[\subst{\alpha}{\tau}]}{\xi}{\omega}$ and $s_1
  \gteq{\sigma_1}{\xi'}{\omega'} s_2$ then $t s_1
  \gteq{\sigma_2}{\xi'}{\omega'} t s_2$. But then $s_1
  \gteq{\sigma_1[\subst{\alpha}{\tau}]}{\xi}{\omega} s_2$ by the
  inductive hypothesis, so $t s_1
  \gteq{\sigma_2[\subst{\alpha}{\tau}]}{\xi}{\omega} t s_2$ by
  definition. Hence $t s_1 \gteq{\sigma_2}{\xi'}{\omega'} t s_2$ by
  the inductive hypothesis.
\end{proof}

\begin{lemma}\label{lem_wm_forall}
  Let $\tau$ be a type constructor of kind~$\kappa$. Assume $\omega$
  is closed for $\forall\alpha[\sigma]$ and for~$\tau$.
  \begin{enumerate}
  \item If $t \in \val{\forall(\alpha:\kappa)[\sigma]}{\xi}{\omega}$
    then $t (\omega(\tau)) \in
    \val{\sigma[\subst{\alpha}{\tau}]}{\xi}{\omega}$.
  \item If $t_1 \gteq{\forall(\alpha:\kappa)[\sigma]}{\xi}{\omega}
    t_2$ then $t_1 (\omega(\tau))
    \gteq{\sigma[\subst{\alpha}{\tau}]}{\xi}{\omega} t_2
    (\omega(\tau))$.
  \end{enumerate}
\end{lemma}

\begin{proof}
  Analogous to the proof of Lemma~\ref{lem_forall}, using
  Lemma~\ref{lem_val_wm_computable} and Lemma~\ref{lem_val_subst_wm}.
\end{proof}

\begin{lemma}\label{lem_beta_val_wm}
  If $\omega$ is closed for~$\sigma,\sigma'$ and $\sigma =_\beta
  \sigma'$ then $\val{\sigma}{\xi}{\omega} =
  \val{\sigma'}{\xi}{\omega}$ and ${\gteq{\sigma}{\xi}{\omega}} =
      {\gteq{\sigma'}{\xi}{\omega}}$.
\end{lemma}

\begin{proof}
  Analogous to the proof of Lemma~\ref{lem_beta_val}, using
  Lemma~\ref{lem_val_subst_wm}.
\end{proof}

For two replacements $\delta_1 = \gamma_1 \circ \omega$ and $\delta_2
= \gamma_2 \circ \omega$ (see Definition~\ref{def_closure}) and an
$\omega$-valuation~$\xi$ we write $\delta_1 \gteq{\tau}{\xi}{\omega}
\delta_2$ iff $\delta_1(x) \gteq{\tau}{\xi}{\omega} \delta_2(x)$ for
each~$x : \tau$.

\begin{lemma}\label{lem_typable_wm_computable}
  Assume $t : \sigma$ and $\delta_1=\gamma_1\circ\omega$,
  $\delta_2=\gamma_2\circ\omega$ are replacements and~$\xi$ an
  $\omega$-valuation such that $\delta_1 \gteq{}{\xi}{\omega}
  \delta_2$ and $\omega$ is closed for~$\sigma$ and $\FTV(\omega(t)) =
  \emptyset$ and for all $x^\tau \in \FV(t)$ we have $\delta_i(x) \in
  \val{\tau}{\xi}{\omega}$. Then $\delta_i(t) \in
  \val{\sigma}{\xi}{\omega}$ and $\delta_1(t)
  \gteq{\sigma}{\xi}{\omega} \delta_2(t)$.
\end{lemma}

\begin{proof}
  Induction on the structure of~$t$. By the generation lemma for $t :
  \sigma$ there is a type~$\sigma'$ such that $\sigma' =_\beta \sigma$
  and $\FV(\sigma') \subseteq \FTV(t)$ and one of the cases below
  holds. Note that $\omega$ is closed for~$\sigma'$, because it is
  closed for~$\sigma$ and $\FTV(\omega(t)) = \emptyset$. Hence by
  Lemma~\ref{lem_beta_val_wm} it suffices to show $\delta_i(t) \in
  \val{\sigma'}{\xi}{\omega}$ and $\delta_1(t)
  \gteq{\sigma'}{\xi}{\omega} \delta_2(t)$.
  \begin{itemize}
  \item If $t = x^{\sigma'}$ then $\delta_i(t) \in
    \val{\sigma'}{\xi}{\omega}$ by assumption. Also $\delta_1(t)
    \gteq{\sigma'}{\xi}{\omega} \delta_2(t)$ by assumption.
  \item If $t = n$ is a natural number and $\sigma' = \nat$ then
    $\delta_i(t) = t$ and thus $t \in \val{\nat}{}{}$ and $\delta_1(t)
    \gteq{\nat}{\xi}{\omega} \delta_2(t)$ by definition and the
    reflexivity of~$\gteq{\nat}{\xi}{\omega}$.
  \item If $t$ is a function symbol then the claim follows from
    Lemma~\ref{lem_wm_circ}, Lemma~\ref{lem_wm_lift} or
    Lemma~\ref{lem_wm_flatten}, and the reflexivity
    of~$\gteq{}{\xi}{\omega}$.
  \item If $t = \abs{x:\sigma_1}{u}$ then $\sigma' =
    \sigma_1\arrtype\sigma_2$ and $u : \sigma_2$. Let $s \in
    \val{\sigma_1}{\xi}{\omega}$ and
    $\delta_i'=\delta_i[\subst{x}{s}]$. This is well-defined because
    $s : \omega(\sigma_1)$ and~$\omega(x)$ has
    type~$\omega(\sigma_1)$. We have $\delta_1' \gteq{}{\xi}{\omega}
    \delta_2'$ by the reflexivity of~$\gteq{\sigma_1}{\xi}{\omega}$
    (Lemma~\ref{lem_val_wm_computable} and property~6 of comparison
    candidates). Hence by the inductive hypothesis $\delta_i'(u) \in
    \val{\sigma_2}{\xi}{\omega}$. We have $\delta_i(\abs{x}{u}) s
    \leadsto \delta_i'(u)$, so $\delta_i(\abs{x}{u}) s \in
    \val{\sigma_2}{\xi}{\omega}$ by Lemma~\ref{lem_val_wm_computable}
    and property~1 of candidate sets.

    Let $s_1,s_2 \in \val{\sigma_1}{\xi}{\omega}$ be such that $s_1
    \gteq{\sigma_1}{\xi}{\omega} s_2$. Let
    $\delta_i'=\delta_i[\subst{x}{s_i}]$. We have $\delta_1
    \gteq{}{\xi}{\omega} \delta_2$. Hence by the inductive hypothesis
    $\delta_1'(u)\gteq{\sigma_2}{\xi}{\omega}\delta_2'(u)$. We have
    $\delta_i(\abs{x}{u}) s_i \leadsto \delta_i'(u)$. Thus
    $\delta_1(t) s_1 \gteq{\sigma_2}{\xi}{\omega} \delta_2(t) s_2$ by
    Lemma~\ref{lem_val_wm_computable} and property~2 of comparison
    candidates.

    Finally, we show $\delta_1(t)
    \gteq{\sigma_1\arrtype\sigma_2}{\xi}{\omega} \delta_2(t)$. Let $s
    \in \val{\sigma_1}{\xi}{\omega}$ and
    $\delta_i'=\delta_i[\subst{x}{s}]$. We have $\delta_1'
    \gteq{}{\xi}{\omega} \delta_2'$. By the inductive hypothesis
    $\delta_1'(u) \gteq{\sigma_2}{\xi}{\omega} \delta_2'(u)$. We have
    $\delta_i(\abs{x}{u}) s \leadsto \delta_i'(u)$. Thus $\delta_1(t)
    s \gteq{\sigma_2}{\xi}{\omega} \delta_2(t) s$ by
    Lemma~\ref{lem_val_wm_computable} and property~2 of comparison
    candidates.
  \item If $t = \tabs{\alpha:\kappa}{u}$ then $\sigma' =
    \forall\alpha[\tau]$ and $u : \tau$. Let $\psi$ be a closed type
    constructor of kind~$\kappa$ and let $X \in \Cb_\psi$. Let
    $\omega' = \omega[\subst{\alpha}{\psi}]$ and
    $\xi'=\xi[\subst{\alpha}{X}]$. Then $\omega'$ is closed for~$\tau$
    and $\FTV(\omega'(u)) = \emptyset$. Let
    $\delta_i'=\gamma_i\circ\omega'$. By the inductive hypothesis
    $\delta_i'(u) \in \val{\tau}{\xi'}{\omega'}$ and $\delta_1'(u)
    \gteq{\tau}{\xi'}{\omega'} \delta_2'(u)$. We have
    $\delta_i(\tabs{\alpha}{u}) \psi \leadsto \delta_i'(u)$. Hence
    $\delta_i(\tabs{\alpha}{u}) \psi \in \val{\tau}{\xi'}{\omega'}$ by
    Lemma~\ref{lem_val_wm_computable} and property~1 of candidate
    sets. Thus $\delta_i(\tabs{\alpha}{u}) \in
    \val{\forall\alpha[\tau]}{\xi}{\omega}$. Also
    $\delta_1(\tabs{\alpha}{u}) \psi \gteq{\tau}{\xi'}{\omega'}
    \delta_2(\tabs{\alpha}{u}) \psi$ by
    Lemma~\ref{lem_val_wm_computable} and property~2 of comparison
    candidates. Thus $\delta_1(\tabs{\alpha}{u})
    \gteq{\forall\alpha[\tau]}{\xi'}{\omega'}
    \delta_2(\tabs{\alpha}{u})$.
  \item If $t = t_1 t_2$ then $t_1 : \tau\arrtype\sigma'$ and $t_2 :
    \tau$ and $\FV(\tau) \subseteq \FTV(t)$. Hence~$\omega$ is closed
    for~$\tau$ and for~$\tau\arrtype\sigma'$. By the inductive
    hypothesis $\delta_i(t_1) \in
    \val{\tau\arrtype\sigma'}{\xi}{\omega}$ and $\delta_i(t_2) \in
    \val{\tau}{\xi}{\omega}$ and $\delta_1(t_1)
    \gteq{\tau\arrtype\sigma'}{\xi}{\omega} \delta_2(t_1)$ and
    $\delta_1(t_2) \gteq{\tau}{\xi}{\omega} \delta_2(t_2)$. By the
    definition of $\val{\tau\arrtype\sigma'}{\xi}{\omega}$ we have
    $\delta_i(t) = \delta_i(t_1)\delta_i(t_2) \in
    \val{\sigma'}{\xi}{\omega}$, and $\delta_1(t_1)\delta_1(t_2)
    \gteq{\sigma'}{\xi}{\omega} \delta_1(t_1)\delta_2(t_2)$. By the
    definition of~$\gteq{\tau\arrtype\sigma'}{\xi}{\omega}$ we have
    $\delta_1(t_1)\delta_2(t_2)\gteq{\sigma'}{\xi}{\omega}\delta_2(t_1)\delta_2(t_2)$. Hence
    $\delta_1(t)\gteq{\sigma'}{\xi}{\omega}\delta_2(t)$ by the
    transitivity of~$\gteq{\sigma'}{\xi}{\omega}$.
  \item If $t = s \psi$ then $s : \forall\alpha[\tau]$ and $\sigma' =
    \tau[\subst{\alpha}{\psi}]$. By the inductive hypothesis
    $\delta_i(s) \in \val{\forall\alpha[\tau]}{\xi}{\omega}$ and
    $\delta_1(s) \gteq{\forall\alpha[\tau]}{\xi}{\omega}
    \delta_2(s)$. Because $\FTV(\omega(t)) = \emptyset$, the mapping
    $\omega$ is closed for~$\psi$. So by Lemma~\ref{lem_wm_forall} we
    have $\delta_i(t) = \delta_i(s) \omega(\psi) \in
    \val{\tau[\subst{\alpha}{\psi}]}{\xi}{\omega}$ and $\delta_1(t)
    \gteq{\tau[\subst{\alpha}{\psi}]}{\xi}{\omega} \delta_2(t)$.\qedhere
  \end{itemize}
\end{proof}

\begin{corollary}\label{cor_typable_wm_computable}
  If $t$ is closed and $t : \sigma$ then $t \in \val{\sigma}{}{}$.
\end{corollary}

\begin{lemma}\label{lem_gteq_to_succeq}
  If $\sigma$ is a closed type and $t_1 \geq_\sigma t_2$ then
  $t_1 \succeq_{\sigma} t_2$.
\end{lemma}

\begin{proof}
  By coinduction. By Lemma~\ref{lem_beta_val_wm} we may assume
  that~$\sigma$ is in $\beta$-normal form. The case $\sigma=\alpha$ is
  impossible because~$\sigma$ is closed. If $\sigma = \nat$ then
  ${\geq_\nat} = {\succeq_\nat}$.

  Assume $\sigma=\sigma_1\arrtype\sigma_2$. Let $u : \sigma_1$ be
  closed. By Corollary~\ref{cor_typable_wm_computable} we have $u \in
  \val{\sigma_1}{}{}$. Hence $t_1 u \geq_{\sigma_2} t_2 u$. By the
  coinductive hypothesis $t_1 u \succeq_{\sigma_2} t_2 u$. This
  implies $t_1 \succeq_{\sigma} t_2$.

  Assume $\sigma=\forall(\alpha:\kappa)\tau$. Let $\varphi$ be a
  closed type constructor of kind~$\kappa$. By
  Lemma~\ref{lem_val_wm_computable} we have $\val{\varphi}{}{} \in
  \Cb_\varphi$. By the definition of~$\geq_{\forall\alpha\tau}$ and
  Lemma~\ref{lem_val_subst_wm} we have $t_1 \varphi
  \geq_{\tau[\subst{\alpha}{\varphi}]} t_2 \varphi$. Note that
  $\tau[\subst{\alpha}{\varphi}]$ is still closed. Hence by the
  coinductive hypothesis $t_1 \varphi
  \succeq_{\tau[\subst{\alpha}{\varphi}]} t_2 \varphi$. This implies
  $t_1 \succeq_{\sigma} t_2$.
\end{proof}

\begin{corollary}\label{cor_gteq_succeq}
  If~$\sigma$ is a closed type then ${\geq_{\sigma}} =
  {\succeq_\sigma}$.
\end{corollary}

\begin{proof}
  Follows from Lemma~\ref{lem_gteq_to_succeq},
  Lemma~\ref{lem_val_wm_computable} and property~1 of comparison
  candidates.
\end{proof}

{ \renewcommand{\thelemma}{\ref{lem_succeq_subst}}
\begin{lemma}[Weak monotonicity]
  If $s \succeq_\sigma s'$ then $t[\subst{x}{s}] \succeq_\tau t[\subst{x}{s'}]$.
\end{lemma}
\addtocounter{theorem}{-1}}

\begin{proof}
  It suffices to show this when
  $s,s',t[\subst{x}{s}],t[\subst{x}{s'}]$ and $\sigma,\tau$ are all
  closed. Assume $s \succeq_\sigma s'$. Then $s \geq_{\sigma} s'$ by
  Corollary~\ref{cor_gteq_succeq}. Thus $t[\subst{x}{s}] \geq_{\tau}
  t[\subst{x}{s'}]$ follows from
  Lemma~\ref{lem_typable_wm_computable}. Hence $t[\subst{x}{s}]
  \succeq_\tau t[\subst{x}{s'}]$ by Corollary~\ref{cor_gteq_succeq}.
\end{proof}

\subsection{Proofs for Section~\ref{sec_rule_removal}}

{ \renewcommand{\thelemma}{\ref{lem:plusparts}}
\begin{lemma}
For all types $\sigma$, terms $s,t$ of type $\sigma$ and natural
numbers $n > 0$:
\begin{enumerate}
\item $s \oplus_{\sigma} t \succeq s$ and $s \oplus_{\sigma} t \succeq
  t$;
\item $s \oplus_{\sigma} (\lift_{\sigma} n) \succ s$ and
  $(\lift_{\sigma} n) \oplus_{\sigma} t \succ t$.
\end{enumerate}
\end{lemma}
\addtocounter{theorem}{-1}}

\begin{proof}
  It suffices to prove this for closed $s,t$ and closed $\sigma$ in
  $\beta$-normal form.
  \begin{enumerate}
  \item By coinduction we show $(s \oplus t) u_1 \ldots u_m
    \succeq_\sigma s u_1 \ldots u_m$ for closed $u_1,\ldots,u_m$. The
    second case is similar.

    First note that $(s \oplus t) u_1 \ldots u_m \leadsto^* s u_1
    \ldots u_m \oplus t u_1 \ldots u_m$.

    If $\sigma = \nat$ then $((s \oplus t) u_1 \ldots u_m)\da = (s u_1
    \ldots u_m)\da + (t u_1 \ldots u_m)\da \ge (s u_1 \ldots
    u_m)\da$. Hence $(s \oplus t) u_1 \ldots u_m) \succeq_\nat s u_1
    \ldots u_m$.

    If $\sigma = \tau_1\arrtype\tau_2$ then by the coinductive
    hypothesis $(s \oplus t) u_1 \ldots u_m q \succeq_{\tau_2} s u_1
    \ldots u_m q$ for any $q \in \World_{\tau_1}$. Hence $(s \oplus t)
    u_1 \ldots u_m \succeq_\sigma s u_1 \ldots u_m$.

    If $\sigma = \forall(\alpha:\kappa)[\tau]$ then by the coinductive
    hypothesis $(s \oplus t) u_1 \ldots u_m \xi \succeq_{\sigma'} s
    u_1 \ldots u_m \xi$ for any closed $\xi \in \Tc_\kappa$, where
    $\sigma' = \tau[\subst{\alpha}{\xi}]$. Hence $(s \oplus t) u_1
    \ldots u_m \succeq_\sigma s u_1 \ldots u_m$.
  \item By coinduction we show $(s \oplus (\lift n)) u_1 \ldots u_m
    \succeq_\sigma s u_1 \ldots u_m$ for closed $u_1,\ldots,u_m$. The
    second case is similar.

    Note that $(s \oplus (\lift n)) u_1 \ldots u_m \leadsto^* s u_1
    \ldots u_m \oplus n$. From this the case $\sigma=\nat$
    follows. The other cases follow from the coinductive hypothesis,
    like in the first point above.\qedhere
  \end{enumerate}
\end{proof}

{ \renewcommand{\thelemma}{\ref{lem:approxproperties}}
\begin{lemma}
For all types $\sigma$ and all terms $s,t,u$ of type $\sigma$, we
have:
\begin{enumerate}
\item $s \oplus_\sigma t \approx t \oplus_\sigma s$ and $s
  \otimes_\sigma t \approx t \otimes_\sigma s$;
\item $s \oplus_\sigma (t \oplus_\sigma u) \approx (s \oplus_\sigma t)
  \oplus_\sigma u$ and $s \otimes_\sigma (t \otimes_\sigma u) \approx
  (s \otimes_\sigma t) \otimes_\sigma u$;
\item $s \otimes_\sigma (t \oplus_\sigma u) \approx (s \otimes_\sigma
  t) \oplus_\sigma (s \otimes_\sigma u)$;
\item $(\lift_\sigma 0) \oplus_\sigma s \approx s$ and $(\lift_\sigma
  1) \otimes_\sigma s \approx s$.
\end{enumerate}
\end{lemma}
\addtocounter{theorem}{-1}}

\begin{proof}
  The proof is again analogous to the proof of
  Lemma~\ref{lem:plusparts}. For instance, for closed $s,t$ and closed
  $\sigma$ in $\beta$-normal form, we show by coinduction that $(s
  \oplus t) w_1 \ldots w_n \succeq (t \oplus s) w_1 \ldots w_n$ for
  closed $w_1,\ldots,w_n$ (and then the same with $\preceq$).
\end{proof}

{ \renewcommand{\thelemma}{\ref{lem_lift_approx}}
\begin{lemma}
  \begin{enumerate}
  \item $\lift_\sigma(n+m) \approx_\sigma (\lift_\sigma n)
    \oplus_\sigma (\lift_\sigma n)$.
  \item $\lift_\sigma(n m) \approx_\sigma (\lift_\sigma n)
    \otimes_\sigma (\lift_\sigma n)$.
  \end{enumerate}
\end{lemma}
\addtocounter{theorem}{-1}}

\begin{proof}
  It suffices to show this for closed~$\sigma$ in $\beta$-normal
  form. Then one proves by induction on~$\sigma$ that
  $(\lift_\sigma(n+m))\da = (\lift_\sigma n \oplus_\sigma \lift_\sigma
  n)\da$ (analogously for multiplication). This suffices by
  Corollary~\ref{cor_succ_da} and the reflexivity of~$\approx$.
\end{proof}

\section{Proving the inequalities in \refsec{examples}}\label{app_ineqs}

\end{document}
