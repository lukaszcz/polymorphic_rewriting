\documentclass[a4paper,UKenglish,cleveref,autoref,numberwithinsect]{lipics-v2019}

\usepackage{stmaryrd}

%\graphicspath{{./graphics/}}%helpful if your graphic files are in another directory

\bibliographystyle{plainurl}% the mandatory bibstyle

\theoremstyle{definition}
\newtheorem{defn}[theorem]{Definition}

\newcommand{\Fomega}{\mathtt{F}_\omega}

\newcommand{\Typevars}{\mathcal{A}}
\newcommand{\Vars}{\mathcal{V}}
\newcommand{\Rules}{\mathcal{R}}
\newcommand{\World}{\mathcal{W}}
\newcommand{\Iterms}{\mathcal{I}}
\newcommand{\ITypes}{\mathcal{Y}}

\newcommand{\arrkind}{\Rightarrow}
\newcommand{\arrtype}{\rightarrow}
\newcommand{\quant}[2]{\forall #1.#2}

\newcommand{\abstraction}[2]{\backslash #1.#2}
\newcommand{\app}[2]{#1 \cdot #2}
\newcommand{\tapp}[2]{#1 * #2}
\newcommand{\subst}[2]{#1:=#2}

\newcommand{\abs}[2]{\lambda #1.#2}
\newcommand{\tabs}[2]{\Lambda #1.#2}
\newcommand{\pair}[2]{\langle #1,#2 \rangle}
\newcommand{\expair}[2]{[#1,#2]}

\newcommand{\arrW}{\leadsto}
\newcommand{\arr}[1]{\longrightarrow_{#1}}
\newcommand{\red}{\longrightarrow}

\newcommand{\nat}{\mathtt{nat}}
\newcommand{\flatten}{\mathtt{flatten}}
\newcommand{\lift}{\mathtt{lift}}

\newcommand{\typeinterpret}[1]{\llbracket #1 \rrbracket}
\newcommand{\interpret}[1]{\llbracket #1 \rrbracket}

\newcommand{\refsec}[1]{Section~\ref{sec:#1}}

\newcommand{\FTV}{\mathrm{FTV}}
\newcommand{\FV}{\mathrm{FV}}
\newcommand{\Tc}{\mathcal{T}}
\newcommand{\Vc}{\mathcal{V}}

\newcommand{\cl}{\mathcal{C}}
\newcommand{\dom}{\mathrm{dom}}
\newcommand{\nf}{\mathrm{nf}}

\newcommand{\da}{\mathord{\downarrow}}
\newcommand{\SN}{\mathrm{SN}}
\newcommand{\Cb}{\mathbb{C}}
\newcommand{\Nbb}{\mathbb{N}}
\newcommand{\val}[3]{\ensuremath{\llbracket#1\rrbracket_{#2}^{#3}}}

\newcommand{\proves}{\vdash}

\newcommand{\List}{\mathtt{List}}
\newcommand{\nil}{\mathtt{nil}}
\newcommand{\cons}{\mathtt{cons}}
\newcommand{\fold}{\mathtt{fold}}

\newcommand{\CK}[1]{\textcolor{blue}{CK: #1}}
\newcommand{\LC}[1]{\textcolor{red}{LC: #1}}

\title{Polymorphic Higher-order Termination}

%\titlerunning{Dummy short title}

\author{{\L}ukasz Czajka}{Faculty of Informatics, TU Dortmund, Germany \and \url{http://www.mimuw.edu.pl/~lukaszcz/} }{lukaszcz@mimuw.edu.pl}{https://orcid.org/0000-0001-8083-4280}{}

\author{Cynthia Kop}{Institute of Computer Science, Radboud University Nijmegen, Netherlands \and \url{https://www.cs.ru.nl/~cynthiakop/}}{c.kop@cs.ru.nl}{https://orcid.org/0000-0002-6337-2544}{}

\authorrunning{\L. Czajka and C. Kop}

\Copyright{{\L}ukasz Czajka and Cynthia Kop}

\ccsdesc[500]{Theory of computation~Rewrite systems}
\ccsdesc[500]{Theory of computation~Equational logic and rewriting}
\ccsdesc[300]{Theory of computation~Type theory}

\keywords{termination, polymorphism, higher-order rewriting, permutative conversions}

%\category{}%optional, e.g. invited paper

%\relatedversion{A full version of the paper is available at \url{...}.}
%\supplement{}%optional, e.g. related research data, source code, ... hosted on a repository like zenodo, figshare, GitHub, ...

%\acknowledgements{I want to thank \dots}%optional

%\nolinenumbers %uncomment to disable line numbering

%\hideLIPIcs  %uncomment to remove references to LIPIcs series (logo, DOI, ...), e.g. when preparing a pre-final version to be uploaded to arXiv or another public repository

%Editor-only macros:: begin (do not touch as author)%%%%%%%%%%%%%%%%%%%%%%%%%%%%%%%%%%
\EventEditors{John Q. Open and Joan R. Access}
\EventNoEds{2}
\EventLongTitle{42nd Conference on Very Important Topics (CVIT 2016)}
\EventShortTitle{CVIT 2016}
\EventAcronym{CVIT}
\EventYear{2016}
\EventDate{December 24--27, 2016}
\EventLocation{Little Whinging, United Kingdom}
\EventLogo{}
\SeriesVolume{42}
\ArticleNo{23}
%%%%%%%%%%%%%%%%%%%%%%%%%%%%%%%%%%%%%%%%%%%%%%%%%%%%%%

\begin{document}

\maketitle

\begin{abstract}
  We generalise the termination method of higher-order polynomial
  interpretations to a setting with impredicative
  polymorphism. Instead of using weakly monotonic functionals, we
  interpret terms in a suitable extension of System~$\Fomega$. In
  addition to enabling an interpretation of rewrite rules which make
  essential use of impredicative polymorphism, thanks to the
  possibility of encoding inductive data types in the polymorphic
  lambda-calculus, this generalisation increases the power of the
  method also in the non-polymorphic setting. As an illustration of
  the potential of our method, we prove termination of a substantial
  fragment of full intuitionistic second-order propositional logic
  with permutative conversions.
\end{abstract}

\section{Introduction}

\cite{pol:96}

\section{Preliminaries}\label{sec_preliminaries}

In this section we introduce the System~$\Fomega$ (see
e.g.~\cite[Section~11.7]{SorensenUrzyczyn2006}), which will form a
basis both of our interpretations and of a general syntax for the
investigated systems. First, we define the set of types.

\begin{defn}\label{def_types}
  \emph{Kinds} are defined inductively: $*$ is a kind, and if
  $\kappa_1,\kappa_2$ are kinds then so is $\kappa_1 \arrkind
  \kappa_2$.

  We assume infinitely many \emph{type constructor variables} of each
  kind. Variables of kind~$*$ are \emph{type variables}. We assume a
  fixed set~$\Sigma_T$ of \emph{type constructor symbols} paired with
  a kind, denoted $c : \kappa$. Every type constructor symbol~$c$
  occurs with only one kind declaration. We assume there exists a
  fixed type symbol~$\chi_*$ (of kind~$*$). For
  $\kappa=\kappa_1\arrkind\kappa_2$ we define $\chi_\kappa = \lambda
  \alpha:\kappa_1 . \chi_{\kappa_2}$.

  We define \emph{type constructors} of kind~$\kappa$ by induction.
  Type constructors of kind~$*$ are \emph{types}.
  \begin{itemize}
  \item A type constructor variable or symbol of kind~$\kappa$ is a
    type constructor of kind~$\kappa$.
  \item If $\varphi$ is a type constructor of kind $\kappa_1 \arrkind
    \kappa_2$ and $\psi$ is a type constructor of kind~$\kappa_1$ then
    $\varphi \psi$ is a type constructor of kind~$\kappa_2$.
  \item If $\alpha$ is a type constructor variable of kind~$\kappa_1$
    and $\varphi$ is a type constructor of kind~$\kappa_2$, then
    $\lambda\alpha . \varphi$ is a type constructor of kind $\kappa_1
    \arrkind \kappa_2$.
  \item If $\alpha$ is a type constructor variable of kind~$\kappa$
    and~$\tau$ is a type, then $\forall \alpha[\tau]$ is a type.
  \item If $\tau_1,\tau_2$ are types, then $\tau_1 \arrtype \tau_2$ is
    a type.
  \end{itemize}
  The set of type constructors of kind~$\kappa$ is denoted
  by~$\Tc_\kappa$.

  We use the standard notation $\forall \alpha . \tau$. When $\alpha$
  is of kind $\kappa$ then we use the notation $\forall \alpha :
  \kappa . \tau$. If not indicated otherwise, if the kind of~$\alpha$
  in $\forall\alpha\sigma$ is not given explicitly, we assume~$\alpha$
  to be a type variable. We treat type constructors up to
  $\alpha$-conversion. Note that~$\forall$ binds variables.

  Beta-reduction on type constructors is defined as the compatible
  closure of the rule
  \[
  (\lambda\alpha.\varphi)\psi \to \varphi[\alpha := \psi]
  \]
  Note that type constructors are simply-typed lambda-terms, so
  beta-reduction on type constructors terminates and is confluent,
  hence every type constructor~$\tau$ has a unique beta-normal
  form~$\nf_\beta(\tau)$. A \emph{type atom} is a type in
  $\beta$-normal form which is not an arrow $\tau_1\arrtype\tau_2$ or
  a quantification $\forall\alpha\tau$.

  We define $\FV(\varphi)$ -- the set of free type constructor
  variables of the type constructor~$\varphi$ -- in an obvious way by
  induction on the structure of~$\varphi$. A type
  constructor~$\varphi$ is \emph{closed} if $\FV(\varphi) =
  \emptyset$.
\end{defn}

Terms are built from a set of function symbols, using abstraction and
application.

\begin{defn}\label{def_preterms}
  We assume given an infinite set $\Vars$ of variables, and let
  $\Gamma$ refer to a mapping from a finite subset of $\Vars$ to the
  set of types. We assume given a fixed set $\Sigma$ of \emph{function
    symbols}, each paired with a type, denoted $\mathtt{f} : \tau$.
  Every function symbol $\mathtt{f}$ occurs only with one type
  declaration.

  The set of preterms consists of all expressions~$s$ such that
  $\Gamma \vdash s : \sigma$ can be inferred for some type $\sigma$
  and mapping $\Gamma$ by the following clauses:
  \begin{itemize}
  \item $\Gamma \vdash x : \sigma$ for every $(x : \sigma) \in \Gamma$.
  \item $\Gamma \vdash \mathtt{f} : \sigma$ for all
    $(\mathtt{f} : \sigma) \in \Sigma$.
  \item $\Gamma \vdash \abs{x:\sigma}{s} : \sigma \arrtype \tau$ if $x
    \in \Vars$ and $\Gamma \uplus \{ x : \sigma \} \vdash s : \tau$.
  \item $\Gamma \vdash \tabs{\alpha}{s} : \quant{\alpha}{\sigma}$ if
    $\alpha$ is a type constructor variable and $\Gamma \vdash s :
    \sigma$ and for all $(x : \tau) \in \Gamma$: $\alpha \notin
    \FV(\tau)$
  \item $\Gamma \vdash \app{s}{t} : \tau$ if $\Gamma \vdash s : \sigma
    \arrtype \tau$ and $\Gamma \vdash t : \sigma$
  \item $\Gamma \vdash \tapp{s}{\tau} : \sigma[\subst{\alpha}{\tau}]$
    if $\Gamma \vdash s : \quant{\alpha:\kappa}{\sigma}$ and~$\tau$
    is a type constructor of kind~$\kappa$,
  \item $\Gamma \vdash s : \tau$ if $\Gamma \vdash s : \tau'$ and
    $\tau =_\beta \tau'$.
  \end{itemize}
  The set of free variables of a preterm~$t$, denoted $\FV(t)$, is
  defined in the expected way. Analogously, we define the
  set~$\FTV(t)$ of type constructor variables occurring free
  in~$t$. We say that $t$ is \emph{closed} if $\FV(t) = \emptyset$ and
  $\FTV(t) = \emptyset$.
\end{defn}

If $\alpha$ is a type constructor variable of kind~$\kappa$ then we
use the notation $\tabs{\alpha:\kappa}{t}$.

\begin{defn}\label{def_type_equiv}
  We define the type equivalence relation~$\equiv$ on preterms by
  induction on preterm structure.
  \begin{itemize}
  \item $x \equiv x$, $n \equiv n$,
  \item if $\sigma =_\beta \tau$ then $f_\sigma \equiv f_\tau$ for
    $f \in \Sigma$,
  \item if $\sigma =_\beta \tau$ and $s \equiv t$ then
    $\abs{x:\sigma}{s} \equiv \abs{x:\tau}{t}$,
  \item if $s \equiv t$ then $\tabs{\alpha}{s} \equiv
    \tabs{\alpha}{t}$,
  \item if $s \equiv s'$ and $t \equiv t'$ then $s \cdot t \equiv s'
    \cdot t'$,
  \item if $s \equiv t$ and $\sigma =_\beta \tau$ then
    $\tapp{s}{\sigma} \equiv \tapp{t}{\tau}$.
  \end{itemize}
  In other words, $s \equiv t$ iff $s$ and $t$ are identical modulo
  $\beta$-conversion in types.
\end{defn}

\begin{lemma}
  If $\Gamma \vdash s : \tau$ and $s \equiv t$ then $\Gamma \vdash t :
  \tau$.
\end{lemma}

\begin{proof}
  Induction on~$s$.
\end{proof}

\begin{defn}\label{def_terms}
  The set of \emph{terms} is the set of the equivalence classes
  of~$\equiv$.
\end{defn}

Because $\beta$-reduction on types is confluent and terminating, every
term has a canonical preterm representative -- the one with all types
occurring in it $\beta$-normalized. We say that a term is
\emph{closed} if its canonical representative is. We define $\FTV(t)$
as the value of~$\FTV$ on the canonical representative of~$t$.

Because typing and term formation operations (abstraction,
application, \ldots) are invariant under~$\equiv$, we may denote terms
by their (canonical) representatives and informally treat them
interchangeably.

We will often abuse notation to omit $\cdot$ and $*$. Thus, $s t$ can
refer to both $\app{s}{t}$ and $\tapp{s}{t}$. This is not ambiguous
due to typing. We will also use $\abstraction{a}{s}$ for either
$\abs{a}{s}$ or $\tabs{a}{s}$, depending on typing.

\begin{lemma}[Substitution lemma]
  \begin{enumerate}
  \item If $\Gamma \uplus \{ x : \sigma \} \vdash s : \tau$ and
    $\Gamma \proves t : \sigma$ then $\Gamma \proves s[\subst{x}{t}] :
    \tau$.
  \item If $\Gamma \proves t : \sigma$ and~$\tau$ is a type
    constructor of the same kind as the variable~$\alpha$ then
    $\Gamma[\subst{\alpha}{\tau}] \proves t[\subst{\alpha}{\tau}] :
    \sigma[\subst{\alpha}{\tau}]$.
  \end{enumerate}
\end{lemma}

\begin{proof}
  Induction on the typing derivation.
\end{proof}

\begin{lemma}[Generation lemma]
  Assume $\Gamma \proves t : \sigma$ and let $\Vc = \FV(\sigma) \cup
  \FTV(t) \cup \FTV(\Gamma)$. Then there is a type~$\sigma'$ such that
  $\sigma' =_\beta \sigma$ and $\FV(\sigma') \subseteq \Vc$ and one of
  the following holds.
  \begin{itemize}
  \item $t \equiv x$ is a variable and $(x : \tau) \in \Gamma$ and $\tau
    =_\beta \sigma'$.
  \item $t \equiv \mathtt{f}$ is a function symbol with $\mathtt{f} :
    \sigma'$ in $\Sigma$.
  \item $t \equiv \abs{x:\tau_1}{s}$ and
    $\sigma'=\tau_1\arrtype\tau_2$ and $\Gamma \uplus \{ x : \tau_1 \}
    \vdash s : \tau_2$.
  \item $t \equiv \tabs{\alpha}{s}$ and
    $\sigma' = \quant{\alpha}{\tau}$ and $\Gamma \vdash s : \tau$ and
    for all $(x : \rho) \in \Gamma$: $\alpha \notin \FV(\rho)$.
  \item $t \equiv \app{t_1}{t_2}$ and
    $\Gamma \vdash t_1 : \tau \arrtype \sigma'$ and
    $\Gamma \vdash t_2 : \tau$ and $\FV(\tau) \subseteq \Vc$.
  \item $t \equiv \tapp{s}{\tau}$ and
    $\sigma' = \rho[\subst{\alpha}{\tau}]$ and
    $\Gamma \vdash s : \quant{(\alpha:\kappa)}{\rho}$ and~$\tau$ is a
    type constructor of kind~$\kappa$.
  \end{itemize}
\end{lemma}

\begin{proof}
  By induction on the derivation $\Gamma \proves t : \sigma$, using
  the substitution lemma. Note that if $\alpha \notin \Vc$ is of
  kind~$\kappa$ and e.g.~$\Gamma \proves s : \sigma'$ with~$s$ a
  subterm of~$t$, then $\Gamma \proves s :
  \sigma'[\subst{\alpha}{\chi_\kappa}]$ by the substitution lemma (see
  Definition~\ref{def_types}).
\end{proof}

For convenience, we sometimes assume without loss of generality that
the terms are given in orthodox Church-style, i.e., instead of using
contexts we assume that each variable occurrence is annotated with a
type (where two occurrences of the same variable must be annotated
with the same type). Note that given a context~$\Gamma$ under which
all considered terms are typable, there is a natural isomorphism
between typed terms as defined above and terms given in orthodox
Church-style.

\section{Systems of interest}\label{sec_systems}

To represent the rewrite systems whose termination we are going to
analyse, we use a syntax based on system~$\Fomega$, specialising
Section~\ref{sec_preliminaries}.

\begin{defn}
  \emph{Kinds}, \emph{type constructors} and \emph{types} are defined
  like in Definition~\ref{def_types}, parameterised by a fixed
  set~$\Sigma_T$ of type constructor symbols.

  Given a fixed set~$\Sigma$ of function symbols, we define
  \emph{terms} like in Definition~\ref{def_terms} (based on
  Definition~\ref{def_preterms}) with the following restrictions:
  \begin{itemize}
  \item if $\mathtt{f} : \sigma \in \Sigma$ then $\sigma$ is closed and
    \[
    \sigma = \forall (\alpha_1 : \kappa_1) \ldots \forall (\alpha_n : \kappa_n)
    . \sigma_1 \arrtype \ldots \arrtype \sigma_k \arrtype \tau
    \]
    with~$\tau$ a type atom,
  \item for any subterm $t_1 t_2$ of a term~$t$, the subterm~$t_1$ is
    not a variable or an abstraction.
  \end{itemize}
\end{defn}

We use the notation
$\mathtt{f}_{\rho_1,\ldots,\rho_n}(s_1,\ldots,s_k)$ for
$\mathtt{f} \rho_1 \ldots \rho_n s_1 \ldots s_k$ when
\[
  \mathtt{f} : \forall (\alpha_1 : \kappa_1) \ldots
  \forall (\alpha_n : \kappa_n) . \sigma_1 \arrtype \ldots \arrtype
  \sigma_k \arrtype \tau
\]
is a function symbol in~$\Sigma$ with~$\tau$ a type atom, and $\rho_i$
is a type constructor of kind $\kappa_i$ for $i=1,\ldots,k$, and
$\Gamma \proves s_i : \sigma_i[\alpha_1 := \rho_1]\ldots[\alpha_n :=
  \rho_n]$ for $i=1,\ldots,k$, for an appropriate~$\Gamma$. We use
this notation to stress the fact that by default there is no explicit
application available. The application in the syntax of terms is used
just as a convenient syntax, but does not correspond to the usual
application operator since only applications of the form $\mathtt{f}
u_1 \ldots u_n$ are allowed. True application can be modelled by
including the symbol ${@} : \forall\alpha\forall\beta . (\alpha
\arrtype \beta) \arrtype \alpha \arrtype \beta$ in
$\Sigma$. Similarly, type application is modelled through a symbol
$\mathtt{A} : \forall \alpha : * \arrkind * . \forall \beta . (\forall
\beta [\alpha \beta]) \arrtype \alpha \beta$.

The rewrite rules are simply a set of term pairs, whose monotonic
closure generates the rewrite relation.

\begin{defn}
  We fix a variable environment $\Gamma$, and assume given a set
  $\Rules$ of term pairs $(\ell,r)$, such that:
  \begin{itemize}
  \item $\FV(r) \subseteq \FV(\ell) \subseteq \mathit{keys}(\Gamma)$;
  \item $\ell$ and $r$ have the same type under $\Gamma$;
  \item $\Rules$ is stable: if $(\ell,r) \in \Rules$ and $\omega$ is a
    type constructor substitution and $\gamma$ is a term substitution
    such that $\omega(\Gamma) \proves \gamma(x) : \omega(\Gamma(x))$,
    then $(\gamma(\omega(\ell)),\gamma(\omega(r))) \in \Rules$.
  \end{itemize}
  The reduction relation $\arr{\Rules}$ is the smallest monotonic
  relation that contains $\Rules$.
\end{defn}

\begin{example}[Fold on heterogenous lists]
  We have the following type symbol:
  \[
  \begin{array}{c}
    \mathtt{List} : *,
  \end{array}
  \]
  the following function symbols:
  \[
  \begin{array}{rcl}
    @ & : & \forall \alpha \forall \beta . (\alpha \arrtype \beta) \arrtype \alpha \arrtype \beta \\
    \mathtt{A} & : & \forall \alpha : * \arrkind * . \forall \beta .
    (\forall \beta [\alpha \beta]) \arrtype \alpha \beta \\
    \mathtt{nil} & : & \List \\
    \mathtt{cons} & : & \forall \alpha . \alpha \arrtype \List \arrtype \List \\
    \mathtt{foldl} & : & \forall \beta . (\forall \alpha . \beta \arrtype \alpha \arrtype \beta) \arrtype \beta \arrtype \List \arrtype \List
  \end{array}
  \]
  and the following rules:
  \[
  \begin{array}{rcl}
    @_{\sigma,\tau}(\abs{x:\sigma}{s},t) & \red & s[x:=t] \\
    \mathtt{A}_{\tabs{\alpha}{\sigma},\tau}(\tabs{\alpha}{s}) & \red &
    s[\alpha:=\tau] \\
    \mathtt{foldl}_\sigma(f,a,\nil) & \red & a \\
    \mathtt{foldl}_\sigma(f,a,\cons_\tau(x,l)) & \red & \mathtt{foldl}_\sigma(f,@_{\tau,\sigma}(@_{\sigma,\tau\arrtype\sigma}(\mathtt{A}_{\tabs{\alpha}{\sigma\arrtype\alpha\arrtype\sigma},\tau}(f),a),x),l)
  \end{array}
  \]
  This example makes essential use of higher-rank polymorphism, which
  is not reducible to shallow polymorphism as used in the~ML
  programming language or in \LC{Cynthia: add here some references to
    higher-order rewriting frameworks which allow shallow
    polymorphism}. The above rules cannot be translated into an
  infinite set of simply typed rules by instantiating the type
  variables. In practice, when defining heterogenous lists in a
  functional programming language which supports higher-rank
  polymorphism (e.g.~recent extensions of Haskell), one would
  constrain the type variable~$\alpha$ above with a type class to
  guarantee the existence of certain operations on the elements of the
  list.
\end{example}

\section{A well-ordered set of interpretation terms}

In this section we define the set~$\Iterms$ of interpretation terms,
and the relations~$\succeq$ and~$\succ$ on~$\Iterms$. We also define
the set~$\World$ of final interpretation terms as the set of all
closed interpretation terms normalized with a certain reduction
relation. We will use the interpretation terms from~$\Iterms$ to
interpret the terms in the systems whose termination we
investigate. The pair $(\succeq,\succ)$ will be used as an ordering
pair for the method of rule removal. In particular, the
relation~$\succ$ will be well-founded.

\subsection{The sets~$\Iterms$ and~$\World$}

\begin{defn}\label{def_iterms}
  The set~$\ITypes$ of \emph{interpretation types} is the set of types
  as in Definition~\ref{def_types} with $\Sigma_T = \{ \nat : * \}$,
  i.e., there is a single type constant~$\nat$. Then $\chi_* = \nat$.

  The set~$\Iterms$ of \emph{interpretation terms} is the set of terms
  from Definition~\ref{def_terms} (see also
  Definition~\ref{def_preterms}) where as types we take the
  interpretation types and as the set~$\Sigma$ of function symbols we
  take $\Sigma = \Nbb \cup \Sigma_f$, where:
  \[
    \begin{array}{rcl}
      \Sigma_f &=& \{ \oplus_\sigma : \sigma \arrtype
                 \sigma \arrtype \sigma, \otimes_\sigma : \sigma \arrtype \sigma
                 \arrtype \sigma, \\ & & \flatten_{\sigma} : \sigma \arrtype
                 \nat, \lift_{\sigma} : \nat \arrtype \sigma \mid \sigma \in \ITypes
                 \}
    \end{array}
  \]
  The natural numbers in~$\Nbb$ considered as constants are assumed to
  have type~$\nat$.
\end{defn}

To define the world of \emph{final interpretation terms}, we will
normalise certain elements of~$\Iterms$ using the relation~$\arrW$.

\begin{defn}
  We define the relation $\arrW$ on interpretation terms as the
  smallest relation for which the following properties are satisfied:
  \begin{enumerate}
  \item\label{arrW:mono:abs}
    if $s \arrW t$ then both $\abs{x}{s} \arrW \abs{x}{t}$ and
    $\tabs{\alpha}{s} \arrW \tabs{\alpha}{t}$
  \item\label{arrW:mono:right}
    if $s \arrW t$ then $\app{u}{s} \arrW \app{u}{t}$
  \item\label{arrW:mono:left}
    if $s \arrW t$ then both $\app{s}{u} \arrW \app{t}{u}$ and
    $\tapp{s}{\sigma} \arrW \tapp{t}{\sigma}$
  \item\label{arrW:beta:abs} $\app{(\abs{x:\sigma}{s})}{t} \arrW
    s[\subst{x}{t}]$
  \item\label{arrW:beta:tabs} $\tapp{(\tabs{\alpha}{s})}{\sigma}
    \arrW s[\subst{\alpha}{\sigma}]$.
  \item\label{arrW:plus:base}
    $\app{\app{\oplus_{\nat}}{n}}{m} \arrW (n+m)$
  \item\label{arrW:times:base} $\app{\app{\otimes_{\nat}}{n}}{m}
    \arrW (n \cdot m)$
  \item\label{arrW:circ:arrow} $\app{\app{\circ_{\sigma \arrtype
        \tau}}{s}}{t} \arrW
    \abs{x:\sigma}{\app{\app{\circ_\tau}{(\app{s}{x})}}{(\app{t}{x})}}$
    for $\circ \in \{ \oplus, \otimes \}$
  \item\label{arrW:circ:forall}
    $\app{\app{\circ_{\quant{\alpha}{\sigma}}}{s}}{t} \arrW
    \tabs{\alpha}{\app{\app{\circ_\sigma}{(\tapp{s}{\alpha})}}{(
        \tapp{t}{\alpha})}}$ for $\circ \in \{ \oplus, \otimes \}$
  \item $\app{\flatten_\nat}{s} \arrW s$
  \item $\app{\flatten_{\sigma \arrtype \tau}}{s} \arrW
    \app{\flatten_\tau}{(\app{s}{(\app{\lift_\sigma}{0})})}$
  \item $\app{\flatten_{\quant{\alpha:\kappa}{\sigma}}}{s} \arrW
    \app{\flatten_{\sigma[\subst{\alpha}{\chi_\kappa}]}}{(\tapp{s}{\chi_\kappa})}$
  \item $\app{\lift_\nat}{s} \arrW s$
  \item $\app{\lift_{\sigma \arrtype \tau}}{s} \arrW
    \abs{x:\sigma}{\app{\lift_{\tau}}{s}}$
  \item $\app{\lift_{\quant{\alpha}{\sigma}}}{s} \arrW
    \tabs{\alpha}{\app{\lift_{\sigma}}{s}}$
  \end{enumerate}
  Note that the above rules are invariant under~$\equiv$ (by
  confluence of $\beta$-reduction on types), so they correctly define
  a relation on interpretation terms -- the equivalence classes
  of~$\equiv$. We say that $s$ is a \emph{redex} if $s$ reduces by one
  of the rules (\ref{arrW:plus:base}).

  A \emph{final interpretation term} is an interpretation term $s \in
  \Iterms$ such that (a) $s$ is closed, and (b) $s$ is in normal form
  with respect to $\arrW$.  We let $\World$ be the set of all final
  interpretation terms. By~$\World_\tau$ we denote the set of all
  final interpretation terms of interpretation type~$\tau$.
\end{defn}

An important difference with system~$\Fomega$ and related ones is that
the rules for $\oplus_\tau$, $\otimes_\tau$, $\flatten_\tau$ and
$\lift_\tau$ depend on the type~$\tau$. In particular, type
substitution in terms may create redexes. For instance, if $\alpha$ is
a type variable then $\oplus_\alpha t_1 t_2$ is not a redex, but
$\oplus_{\sigma\arrtype\tau} t_1 t_2$ is. This makes adapting the
standard computability-based termination proof method a bit more
difficult.

In the remainder of this section, we shall often speak simply of
``terms'' and ``types'' when referring to interpretation terms and
interpretation types.

\subsection{The relation $\arrW$}

Now we state some properties of~$\arrW$, including strong
normalisation. Because of space limits, most proofs are delegated to
Appendix~\ref{app_proofs}.

\begin{lemma}[Subject reduction]
  If $\Gamma \proves t : \tau$ and $t \arrW t'$ then $\Gamma \proves
  t' : \tau$.
\end{lemma}

\begin{proof}
  By induction on the definition of $t \arrW t'$, using the
  generation and substitution lemmas.
\end{proof}

By~$\SN$ we denote the set of all interpretation terms terminating
w.r.t.~$\arrW$.

\begin{theorem}\label{thm_sn}
  If $\Gamma \proves t : \sigma$ then $t \in \SN$.
\end{theorem}

\begin{lemma}\label{lem_unique_final}
  Every term $s \in \Iterms$ has a unique normal form~$s\da$. If~$s$
  is closed then so is~$s\da$.
\end{lemma}

\begin{proof}
  One checks that~$\arrW$ is locally confluent. Since it is
  terminating by Theorem~\ref{thm_sn}, it is confluent by Newman's
  lemma.
\end{proof}

\begin{lemma}\label{lem_final_nat}
  The only final interpretation terms of type $\nat$ are the natural
  numbers.
\end{lemma}

\subsection{The ordering pair $(\succeq,\succ)$}

\section{Weakly monotonic algebras}

\subsection{Weak monotonicity}

\subsection{Weakly monotonic algebras}

\section{Proving termination}

\subsection{Rule removal}

\subsection{Calculation rules}

\section{Larger examples}\label{sec:examples}

\section{Conclusions and future work}

\bibliography{references}

\clearpage
\appendix

\section{Complete proofs}\label{app_proofs}

In many proofs below we assume the terms to be given in orthodox
Church-style (see the discussion at the end of
Section~\ref{sec_preliminaries}). We denote an occurrence of a
variable~$x$ annotated with a type~$\tau$ by~$x^\tau$. So now
e.g.~$\lambda x : \tau\arrtype\sigma . x^{\tau\arrtype\sigma}y^\tau$
is an orthodox Church-style typed term. When clear or irrelevant, we
omit the type annotations for readability, denoting the above term
by~$\lambda x : \tau\arrtype\sigma . x y$ or even~$\lambda x . x
y$. Note that now type substitution also needs to change the type
annotations. Also, each term has a unique type modulo
$\beta$-conversion. We write $t : \tau$ if $t$ has type~$\tau$. The
generation and subject reduction lemmas still hold for orthodox
Church-style typed terms.

\subsection{Strong Normalisation of~$\arrW$}

For $t \in \SN$ by~$\nu(t)$ we denote the length of the longest
reduction starting at~$t$. The following lemma is obvious, but worth
stating explicitly.

\begin{lemma}\label{lem_reduce_abs}
  If $\abstraction{a}{s} \arrW^* t$, then $t = \abstraction{a}{t'}$
  and $s \arrW^* t'$.  If $s \in \SN$ then both $\abs{x}{s}$ and
  $\tabs{\alpha}{s}$ are also in $\SN$.
\end{lemma}

\begin{proof}
  We observe that every reduct of $\abstraction{x}{s}$ has the form
  $\abstraction{x}{s'}$ with $s \arrW s'$, and analogously for
  $\tabs{\alpha}{s}$.  Thus, the first statement follows by induction
  on the length of the reduction $\abstraction{a}{s} \arrW^* t$,
  and the second statement by induction on $\nu(s)$.
\end{proof}

\begin{lemma}\label{lem_circ_sn_base}
  If $t_1,t_2 \in \SN$ then $\circ_\nat t_1 t_2 \in \SN$ for $\circ
  \in \{\oplus,\otimes\}$.
\end{lemma}

\begin{proof}
  By induction on $\nu(t_1) + \nu(t_2)$. Assume $t_1,t_2 \in \SN$. To
  prove $\circ_\nat t_1 t_2 \in \SN$ it suffices to show $s \in \SN$
  for all~$s$ such that $\circ_\nat t_1 t_2 \arrW s$. If $s =
  \circ_\nat t_1' t_2$ or $s = \circ_\nat t_1 t_2'$ with $t_i \arrW
  t_i'$ then we complete by the induction hypothesis. Otherwise $s \in
  \mathbb{N}$ is obviously in $\SN$.
\end{proof}

In the rest of this section we adapt Girard's method of candidates
(which itself is based on Tait's computability method) to prove
termination of~$\arrW$. The proof is an adaptation of chapters~6
and~14 from the book~\cite{Girard1989}, and chapters~10 and~11 from
the book~\cite{SorensenUrzyczyn2006}.

\begin{defn}\label{def_candidate}
  A term~$t$ is \emph{neutral} if there does not exist a sequence of
  terms and types~$u_1,\ldots,u_n$ with $n \ge 1$ such that $t u_1
  \ldots u_n$ is a redex (by~$\arrW$).

  By induction on the kind~$\kappa$ of a type constructor~$\tau$ we
  define the set~$\Cb_\tau$ of all candidates of type
  constructor~$\tau$.

  First assume $\kappa=*$, i.e., $\tau$ is a type. A set~$X$ of
  interpretation terms of type~$\tau$ is a \emph{candidate of
    type~$\tau$} when:
  \begin{enumerate}
  \item $X \subseteq \SN$;
  \item if $t \in X$ and $t \arrW t'$ then $t' \in X$;
  \item if $t$ is neutral and for every~$t'$ with $t \arrW t'$ we
    have $t' \in X$, then $t \in X$;
  \item if $t_1,t_2 \in X$ then $\circ_\tau t_1 t_2 \in X$ for
    $\circ \in \{\oplus,\otimes\}$;
  \item if $t \in \SN$ and $t : \nat$ then $\lift_\tau t \in X$;
  \item if $t \in X$ then $\flatten_\tau t \in \SN$.
  \end{enumerate}
  Note that item~3 above implies:
  \begin{itemize}
  \item if $t$ is neutral and in normal form then $t \in X$.
  \end{itemize}

  Now assume $\kappa = \kappa_1\arrkind\kappa_2$. A function $f :
  \Tc_{\kappa_1} \times \bigcup_{\xi\in\Tc_{\kappa_1}}\Cb_\xi \to
  \bigcup_{\xi\in\Tc_{\kappa_2}}\Cb_\xi$ is a \emph{candidate of type
    constructor~$\tau$} if for every closed type constructor~$\sigma$
  of kind~$\kappa_1$ and a candidate $X \in \Cb_\sigma$ we have
  $f(\sigma,X) \in \Cb_{\tau\sigma}$.
\end{defn}

Note that the elements of a candidate of type~$\tau$ are required to
have type~$\tau$.

\begin{lemma}\label{lem_beta_candidate}
  If $\sigma =_\beta \sigma'$ then $\Cb_\sigma = \Cb_{\sigma'}$.
\end{lemma}

\begin{proof}
  Induction on the kind of~$\sigma$.
\end{proof}

\begin{defn}\label{def_computability_valuation}
  Let $\omega$ be a mapping from type constructor variables to type
  constructors (respecting kinds). The mapping~$\omega$ extends in an
  obvious way to a mapping from type constructors to type
  constructors. A mapping~$\omega$ is \emph{closed for~$\sigma$} if
  $\omega(\alpha)$ is closed for $\alpha \in \FV(\sigma)$ (then
  $\omega(\sigma)$ is closed).

  An \emph{$\omega$-valuation} is a mapping~$\xi$ from type
  constructor variables to candidates such that $\xi(\alpha) \in
  \Cb_{\omega(\alpha)}$.

  For each type constructor~$\sigma$, each mapping~$\omega$ closed
  for~$\sigma$, and each $\omega$-valuation~$\xi$, the set
  $\val{\sigma}{\xi}{\omega}$ is defined by induction on~$\sigma$:
  \begin{itemize}
  \item $\val{\alpha}{\xi}{\omega} = \xi(\alpha)$ for a type
    constructor variable~$\alpha$,
  \item $\val{\nat}{\xi}{\omega}$ is the set of all terms~$t \in \SN$
    such that $t : \nat$,
  \item $\val{\sigma \arrtype \tau}{\xi}{\omega}$ is the set of all
    terms~$t$ such that $t : \omega(\sigma\arrtype\tau)$ and for
    every~$s \in \val{\sigma}{\xi}{\omega}$ with $s : \omega(\sigma)$
    we have $\app{t}{s} \in \val{\tau}{\xi}{\omega}$,
  \item $\val{\forall(\alpha:\kappa)[\sigma]}{\xi}{\omega}$ is the set
    of all terms~$t$ such that $t : \omega(\forall\alpha[\sigma])$ and
    for every closed type constructor~$\varphi$ of kind~$\kappa$ and
    every $X \in \Cb_\varphi$ we have $\tapp{t}{\varphi} \in
    \val{\sigma}{\xi[\subst{\alpha}{X}]}{\omega[\subst{\alpha}{\varphi}]}$,
  \item
    $\val{\varphi \psi}{\xi}{\omega} =
    \val{\varphi}{\xi}{\omega}(\omega(\psi),\val{\psi}{\xi}{\omega})$,
  \item
    $\val{\lambda(\alpha:\kappa)\varphi}{\xi}{\omega}(\psi,X) =
    \val{\varphi}{\xi[\subst{\alpha}{X}]}{\omega[\subst{\alpha}{\psi}]}$
    for closed $\psi \in \Tc_\kappa$ and $X \in \Cb_\psi$.
  \end{itemize}
  In the above, if e.g.~$\val{\psi}{\xi}{\omega} \notin
  \Cb_{\omega(\psi)}$ then $\val{\varphi \psi}{\xi}{\omega}$ is
  undefined.
\end{defn}

If~$\varphi$ is closed then $\omega,\xi$ do not affect the value
of~$\val{\varphi}{\xi}{\omega}$, so then we simply
write~$\val{\varphi}{}{}$.

\begin{lemma}\label{lem_nat_computable}
  $\val{\nat}{}{} \in \Cb_{\nat}$.
\end{lemma}

\begin{proof}
  We check the conditions in Definition~\ref{def_candidate}.
  \begin{enumerate}
  \item $\val{\nat}{}{} \subseteq \SN$ follows
    directly from Definition~\ref{def_computability_valuation}.
  \item Let $t \in \val{\nat}{}{}$ and $t \arrW t'$. Then $t :
    \nat$ and $t \in \SN$. Hence $t' \in \SN$, and $t' : \nat$ by the
    subject reduction lemma. Thus $t' \in \val{\nat}{}{}$.
  \item Let $t$ be neutral and $t : \nat$. Assume that for all~$t'$
    with $t \arrW t'$ we have $t' \in \val{\nat}{}{}$, so in
    particular $t' \in \SN$. But then $t \in \SN$. Hence $t \in
    \val{\nat}{}{}$.
  \item Let $t_1,t_2 \in \SN$ be such that $t_i : \nat$. Obviously,
    $\circ_\nat t_1 t_2 : \nat$. Also $\circ_\nat t_1 t_2 \in \SN$
    follows by Lemma~\ref{lem_circ_sn_base}. So $\circ_\nat t_1 t_2
    \in \val{\nat}{}{}$.
  \item Let $t \in \SN$ be such that $t : \nat$. Then $\lift_\nat t :
    \nat$. It remains to show $\lift_\nat t \in \SN$. Any infinite
    reduction from~$\lift_\nat t$ has the form $\lift_\nat t
    \arrW^* \lift_\nat t_0 \arrW t_1 \arrW t_2 \arrW
    \ldots$ or $\lift_\nat t \arrW \lift_\nat t_0 \arrW
    \lift_\nat t_1 \arrW \lift_\nat t_2 \arrW \ldots$, where $t
    \arrW^* t_0$ and $t_i \arrW t_{i+1}$. This contradicts $t
    \in \SN$.
  \item Let $t \in \SN$ be such that $t : \nat$. The proof of
    $\flatten_\nat t \in \SN$ is analogous to the proof of $\lift_\nat
    t \in \SN$ above.
  \end{enumerate}
\end{proof}

\begin{lemma}\label{lem_chi_kappa_computable}
  $\val{\chi_\kappa}{}{} \in \Cb_{\chi_\kappa}$.
\end{lemma}

\begin{proof}
  Induction on~$\kappa$. If $\kappa = *$ then this follows from
  Lemma~\ref{lem_nat_computable}. If $\kappa=\kappa_1\arrkind\kappa_2$
  then $\chi_\kappa = \lambda \alpha : \kappa_1
  . \chi_{\kappa_2}$. Let~$\psi$ be a closed type constructor of
  kind~$\kappa_1$ and let $X \in \Cb_{\chi_{\kappa_1}}$. We have
  $\val{\chi_\kappa}{}{}(\psi,X) = \val{\chi_{\kappa_2}}{}{}$ because
  $\chi_{\kappa_2}$ is closed. By the inductive hypothesis
  $\val{\chi_\kappa}{}{}(\psi,X) = \val{\chi_{\kappa_2}}{}{} \in
  \Cb_{\chi_{\kappa_2}}$. This implies $\val{\chi_\kappa}{}{} \in
  \Cb_{\chi_\kappa}$.
\end{proof}

\begin{lemma}\label{lem_abstraction_computable}
  Let $\sigma,\tau$ be types. Suppose $\val{\tau}{\xi'}{\omega'} \in
  \Cb_{\omega'(\tau)}$ and $\val{\sigma}{\xi'}{\omega'} \in
  \Cb_{\omega'(\sigma)}$ for all suitable $\omega',\xi'$. Then
  \begin{itemize}
  \item
    $\abs{x}{s} \in \val{\tau \arrtype \sigma}{\xi}{\omega}$ if and
    only if $\abs{x}{s} : \omega(\tau \arrtype \sigma)$ and $s[x:=t]
    \in \val{\sigma}{\xi}{\omega}$ for all $t \in
    \val{\tau}{\xi}{\omega}$;
  \item
    $\tabs{\alpha}{s} \in
    \val{\quant{(\alpha:\kappa)}{\sigma}}{\xi}{\omega}$ if and only if
    $\tabs{\alpha}{s} : \omega(\quant{(\alpha:\kappa)}{\sigma})$ and
    for every closed type constructor~$\varphi$ of kind~$\kappa$ and
    all $X \in \Cb_\varphi$ we have $s[\alpha:=\varphi] \in
    \val{\sigma}{\xi[\subst{\alpha}{X}]}{\omega[\subst{\alpha}{\varphi}]}$.
  \end{itemize}
\end{lemma}

\begin{proof}
  First suppose
  $\abs{x:\omega(\tau)}{s} \in \val{\tau \arrtype
    \sigma}{\xi}{\omega}$. Then
  $\abs{x:\omega(\tau)}{s} : \omega(\tau\arrtype\sigma)$ and for all
  $t \in \val{\tau}{\xi}{\omega}$ we have
  $\app{(\abs{x:\omega(\tau)}{s})}{t} \in \val{\sigma}{\xi}{\omega}$.
  As this set is a candidate, it is closed under $\arrW$, so also
  $s[x:=t] \in \val{\sigma}{\xi}{\omega}$. Similarly, if
  $\tabs{\alpha}{s} \in \val{\quant{\alpha}{\sigma}}{\xi}{\omega}$,
  then $\tabs{\alpha}{s} : \quant{\alpha}{\sigma}$ and
  $\tapp{(\tabs{\alpha}{s})}{\varphi} \in
  \val{\sigma}{\xi[\subst{\alpha}{X}]}{\omega[\subst{\alpha}{\varphi}]}$,
  and we are done because
  $\tapp{(\tabs{\alpha}{s})}{\tau} \arrW s[\alpha:=\varphi]$ and
  $\val{\sigma}{\xi[\subst{\alpha}{X}]}{\omega[\subst{\alpha}{\varphi}]}$
  is a candidate, so it is closed under~$\arrW$.

  Now suppose $s[x:=t] \in \val{\sigma}{\xi}{\omega}$ for all
  $t \in \val{\tau}{\xi}{\omega}$. Let
  $t \in \val{\tau}{\xi}{\omega}$. Then $t \in \SN$ because
  $\val{\tau}{\xi}{\omega}$ is a candidate. Also $s \in \SN$ because
  every infinite reduction in $s$ induces an infinite reduction in
  $s[x:=t]$ ($\arrW$ is stable) and
  $\val{\sigma}{\xi}{\omega} \subseteq \SN$ is a candidate. For all
  $s',t'$ with $s \arrW^* s'$ and $t \arrW^* t'$, we show by
  induction on~$\nu(s') + \nu(t')$ that
  $\app{(\abs{x}{s'})} t' \in \val{\sigma}{\xi}{\omega}$. We have
  $\app{(\abs{x}{s'})} t' : \omega(\sigma)$ by definition and the
  subject reduction theorem (note that $t : \omega(\tau)$ because
  $\val{\tau}{\xi}{\omega} \in \Cb_{\omega(\tau)}$). The set
  $\val{\sigma}{\xi}{\omega}$ is a candidate, and
  $\app{(\abs{x}{s'})}{t'}$ is neutral, so in
  $\val{\sigma}{\xi}{\omega}$ if all its reducts are. Thus assume
  $\app{(\abs{x}{s'})}{t'} \arrW u$. If
  $u = \app{(\abs{x}{s'})}{t''}$ with $t' \arrW t''$ or
  $u = \app{(\abs{x}{s''})}{t'}$ with $s' \arrW s''$, then
  $u \in \val{\sigma}{\xi}{\omega}$ by the inductive hypothesis. So
  assume $u = s'[x:=t']$. We have $s[x:=t] \arrW^* s'[x:=t']$ by
  monotonicity and stability of $\arrW$. Therefore
  $u = s'[x:=t'] \in \val{\sigma}{\xi}{\omega}$, because
  $s[x:=t] \in \val{\sigma}{\xi}{\omega}$ and
  $\val{\sigma}{\xi}{\omega}$ is a candidate and hence closed under
  $\arrW$.

  A similar reasoning applies to $s[\alpha:=\varphi]$.
\end{proof}

\begin{lemma}\label{lem_val_computable}
  If $\sigma$ is a type constructor, $\omega$ is closed for~$\sigma$,
  and $\xi$ is an $\omega$-valuation, then $\val{\sigma}{\xi}{\omega}
  \in \Cb_{\omega(\sigma)}$.
\end{lemma}

\begin{proof}
  By induction on the structure of~$\sigma$ we show that
  $\val{\sigma}{\xi}{\omega} \in \Cb_{\omega(\sigma)}$ for all
  suitable $\omega,\xi$. First, if $\sigma = \alpha$ is a type
  constructor variable~$\alpha$ then $\val{\sigma}{\xi}{\omega} =
  \xi(\alpha) \in \Cb_{\omega(\sigma)}$ by definition. If $\sigma =
  \nat$ then $\val{\nat}{\xi}{\omega} \in \Cb_{\nat}$ by
  Lemma~\ref{lem_nat_computable}.

  Assume $\sigma = \tau_1 \arrtype \tau_2$. We check the conditions in
  Definition~\ref{def_candidate}.
  \begin{enumerate}
  \item Let $t \in \val{\tau_1\arrtype\tau_2}{\xi}{\omega}$ and assume
    there is an infinite reduction $t \arrW t_1 \arrW t_2
    \arrW t_3 \arrW \ldots$. By the inductive hypothesis
    $\val{\tau_1}{\xi}{\omega}$ and $\val{\tau_2}{\xi}{\omega}$ are
    candidates. Let~$x$ be a fresh variable. Then $x^{\omega(\tau_1)}
    : \omega(\tau_1)$ and $x^{\omega(\tau_1)} \in
    \val{\tau_1}{\xi}{\omega}$ because it is neutral and normal. Thus
    $t x \in \val{\tau_2}{\xi}{\omega} \subseteq \SN$. But $t x
    \arrW t_1 x \arrW t_2 x \arrW t_3 x \arrW
    \ldots$. Contradiction.
  \item Let $t \in \val{\tau_1\arrtype\tau_2}{\xi}{\omega}$ and $t
    \arrW t'$. Let $u \in \val{\tau_1}{\xi}{\omega}$ be such that
    $u : \omega(\tau_1)$. Then $t u \in \val{\tau_2}{\xi}{\omega}$. By
    the inductive hypothesis $\val{\tau_2}{\xi}{\omega}$ is a
    candidate, so $t' u \in \val{\tau_2}{\xi}{\omega}$. Also note that
    $t' : \omega(\tau_1 \arrtype \tau_2)$ by the subject reduction
    lemma. Hence $t' \in \val{\tau_1\arrtype\tau_2}{\xi}{\omega}$.
  \item Let $t$ be neutral such that $t : \omega(\tau_1 \arrtype
    \tau_2)$. Assume for every~$t'$ with $t \arrW t'$ we have $t'
    \in \val{\tau_1\arrtype\tau_2}{\xi}{\omega}$. Let $u \in
    \val{\tau_1}{\xi}{\omega}$ be such that $u : \omega(\tau_1)$. By
    the inductive hypothesis $\val{\tau_1}{\xi}{\omega}$ is a
    candidate, so $u \in \SN$. By induction on~$\nu(u)$ we show that
    $t u \in \val{\tau_2}{\xi}{\omega}$. Assume $t u \arrW t''$. We
    show $t'' \in \val{\tau_2}{\xi}{\omega}$. Because~$t$ is neutral,
    $t u$ cannot be a redex. So there are two cases.
    \begin{itemize}
    \item $t'' = t u'$ with $u \arrW u'$. Then $u' \in
      \val{\tau_1}{\xi}{\omega}$ because~$\val{\tau_1}{\xi}{\omega}$
      is a candidate, and~$u' : \omega(\tau_1)$ by the subject
      reduction lemma. So $t u' \in \val{\tau_2}{\xi}{\omega}$ by the
      inductive hypothesis for~$u$.
    \item $t'' = t' u$ with $t \arrW t'$. Then $t' \in
      \val{\tau_1\arrtype\tau_2}{\xi}{\omega}$ by point~2 above. So
      $t' u \in \val{\tau_2}{\xi}{\omega}$.
    \end{itemize}
    We have thus shown that if $t u \arrW t''$ then $t'' \in
    \val{\tau_2}{\xi}{\omega}$. By the (main) inductive hypothesis
    $\val{\tau_2}{\omega,\xi}{\Gamma}$ is a candidate. Because $t u$
    is neutral, the above implies $t u \in
    \val{\tau_2}{\xi}{\omega}$. Since $u \in
    \val{\tau_1}{\xi}{\omega}$ was arbitrary with $u :
    \omega(\tau_1)$, we have shown $t \in
    \val{\tau_1\arrtype\tau_2}{\xi}{\omega}$.
  \item Assume $t_1,t_2 \in \val{\tau_1\arrtype\tau_2}{\xi}{\omega}$.
    We have already shown that this implies $t_1,t_2 \in \SN$. Let $s
    = \circ_{\omega(\tau_1\arrtype\tau_2)} t_1 t_2$. We show $s \in
    \val{\tau_1\arrtype\tau_2}{\xi}{\omega}$ by induction on $\nu(t_1)
    + \nu(t_2)$. Note that $s : \omega(\tau_1\arrtype\tau_2)$ because
    $t_i : \omega(\tau_1\arrtype\tau_2)$. Since $s$ is neutral, we
    have already seen in point~3 above that to prove $s \in
    \val{\tau_1\arrtype\tau_2}{\xi}{\omega}$ it suffices to show that
    $s' \in \val{\tau_1\arrtype\tau_2}{\xi}{\omega}$ whenever $s
    \arrW s'$. If $s' = \circ_{\omega(\tau_1\arrtype\tau_2)} t_1'
    t_2$ with $t_1 \arrW t_1'$, then note that $t_1' \in
    \val{\tau_1\arrtype\tau_2}{\xi}{\omega}$ because we have already
    shown that $\val{\tau_1\arrtype\tau_2}{\xi}{\omega}$ is closed
    under $\arrW$; thus, we can complete by the induction
    hypothesis. If $s' = \circ_{\omega(\tau_1\arrtype\tau_2)} t_1
    t_2'$, we complete in the same way.  The only alternative is that
    $s' = \abs{x:\omega(\tau_1)}{\circ_{\omega(\tau_2)}(t_1x)(t_2x)}$.
    Let $u \in \val{\tau_1}{\xi}{\omega}$. Then $u : \omega(\tau_1)$
    because $\val{\tau_1}{\xi}{\omega} \in \Cb_{\omega(\tau_1)}$ by
    the inductive hypothesis. Since $t_1,t_2 \in
    \val{\tau_1\arrtype\tau_2}{\xi}{\omega}$, we have that $t_1 u$ and
    $t_2 u$ are in $\val{\tau_2}{\xi}{\omega}$ by definition. Since
    $\val{\tau_2}{\xi}{\omega}$ is a candidate, this means that
    $\circ_{\omega(\tau_2)} (t_1 u) (t_2 u) = (\circ_{\omega(\tau_2)}
    (t_1 x) (t_2 x))[x:=u]$ is in $\val{\tau_2}{\xi}{\omega}$ as well.
    By Lemma~\ref{lem_abstraction_computable}, we conclude that $s'
    \in \val{\tau_1\arrtype\tau_2}{\xi}{\omega}$.
  \item Let $t \in \SN$ satisfy $t : \nat$, and let $s =
    \lift_{\omega(\tau_1\arrtype\tau_2)}(t)$. We show $s \in
    \val{\tau_1\arrtype\tau_2}{\xi}{\omega}$ by induction
    on~$\nu(t)$. We have $s : \omega(\tau_1\arrtype\tau_2)$ because $t
    : \nat$. Since~$s$ is neutral, we have already proved above in
    point~3 that it suffices to show that $s' \in
    \val{\tau_1\arrtype\tau_2}{\xi}{\omega}$ whenever $s \arrW
    s'$. If $s' = \lift_{\omega(\tau_1\arrtype\tau_2)}(t')$ with $t
    \arrW t'$ then still $t' \in \SN$ and $t' : \nat$, so $s' \in
    \val{\tau_1\arrtype\tau_2}{\xi}{\omega}$ by the inductive
    hypothesis. The only alternative is that $s' = \lambda x :
    \omega(\tau_1) . \lift_{\omega(\tau_2)}(t)$. Let $u \in
    \val{\tau_1}{\xi}{\omega}$ be such that $u :
    \omega(\tau_1)$. Because $\val{\tau_2}{\xi}{\omega} \in
    \Cb_{\omega(\tau_2)}$ by the (main) inductive hypothesis
    for~$\sigma$, we have $\lift_{\omega(\tau_2)}(t) \in
    \val{\tau_2}{\xi}{\omega}$. Since $\lift_{\omega(\tau_2)}(t) =
    (\lift_{\omega(\tau_2)}x)[\subst{x}{t}]$ we obtain $s' \in
    \val{\tau_1\arrtype\tau_2}{\xi}{\omega}$ by
    Lemma~\ref{lem_abstraction_computable}.
  \item Let $t \in \val{\tau_1\arrtype\tau_2}{\xi}{\omega}$.  We show
    $s := \flatten_{\omega(\tau_1\arrtype\tau_2)}t \in \SN$. We have
    already shown $t \in \SN$ in point~1 above. Thus any infinite
    reduction starting from~$s$ must have the form $s \arrW^*
    \flatten_{\omega(\tau_1\arrtype\tau_2)}t' \arrW
    \flatten_{\omega(\tau_2)}(t' (\lift_{\omega(\tau_1)}0)) \arrW
    \ldots$ with $t \arrW^* t'$. We have already shown in point~2
    above that $\val{\tau_1\arrtype\tau_2}{\xi}{\omega}$ is closed
    under~$\arrW$, so $t' \in
    \val{\tau_1\arrtype\tau_2}{\xi}{\omega}$. By the inductive
    hypothesis $\val{\tau_1}{\xi}{\omega} \in\Cb_{\omega(\tau_1)}$, so
    $\lift_{\omega(\tau_1)}0 \in \val{\tau_1}{\xi}{\omega}$ by
    property~5 of candidates. Hence $t' (\lift_{\omega(\tau_1)}0) \in
    \val{\tau_2}{\xi}{\omega}$ by definition. But by the inductive
    hypothesis~$\val{\tau_2}{\xi}{\omega}$ is a candidate, so
    $\flatten_{\omega(\tau_2)}(t'(\lift_{\omega(\tau_1)}0))\in\SN$. Contradiction.
  \end{enumerate}

  Assume $\sigma = \forall(\alpha:\kappa)[\tau]$. We check the
  conditions in Definition~\ref{def_candidate}.
  \begin{enumerate}
  \item Let $t \in \val{\forall(\alpha:\kappa)[\tau]}{\xi}{\omega}$
    and assume there is an infinite reduction $t \arrW t_1 \arrW
    t_2 \arrW t_3 \arrW \ldots$. Recall that~$\chi_\kappa$ from
    Definition~\ref{def_types} is a closed type constructor of
    kind~$\kappa$. By Lemma~\ref{lem_chi_kappa_computable} we have
    $\val{\chi_{\kappa}}{}{} \in \Cb_{\chi_\kappa}$. Then $t
    \chi_\kappa \in
    \val{\tau}{\xi[\subst{\alpha}{\val{\chi_\kappa}{}{}}]}{\omega[\subst{\alpha}{\chi_\kappa}]}$. By
    the inductive hypothesis
    $\val{\tau}{\xi[\subst{\alpha}{\val{\chi_\kappa}{}{}}]}{\omega[\subst{\alpha}{\chi_\kappa}]}$
    is a candidate, so $t \chi_\kappa \in \SN$. But $t \chi_\kappa
    \arrW t_1 \chi_\kappa \arrW t_2 \chi_\kappa \arrW t_3
    \chi_\kappa \arrW \ldots$. Contradiction.
  \item Let $t \in \val{\forall\alpha[\tau]}{\xi}{\omega}$ and $t
    \arrW t'$. By the subject reduction lemma $t' :
    \omega(\forall\alpha[\tau])$. Let~$\varphi$ be a closed type
    constructor of kind~$\kappa$ and~$X \in \Cb_{\varphi}$. Then $t
    \varphi \in
    \val{\tau}{\xi[\subst{\alpha}{X}]}{\omega[\subst{\alpha}{\varphi}]}$. By
    the inductive hypothesis
    $\val{\tau}{\xi[\subst{\alpha}{X}]}{\omega[\subst{\alpha}{\varphi}]}$
    is a candidate, so $t' \varphi \in
    \val{\tau}{\xi[\subst{\alpha}{X}]}{\omega[\subst{\alpha}{\varphi}]}$. Therefore
    $t' \in \val{\forall\alpha[\tau]}{\xi}{\omega}$.
  \item Let $t$ be neutral such that $t :
    \omega(\forall\alpha[\tau])$, and assume that for every~$t'$ with
    $t \arrW t'$ we have $t' \in
    \val{\forall\alpha[\tau]}{\xi}{\omega}$. Let~$\varphi$ be a closed
    type constructor of kind~$\kappa$ and~$X \in
    \Cb_{\varphi}$. Assume $t \varphi \arrW t''$. Then $t'' = t'
    \varphi$ with $t \arrW t'$, because~$t$ is neutral. Hence $t
    \varphi \arrW t' \varphi \in
    \val{\tau}{\xi[\subst{\alpha}{X}]}{\omega[\subst{\alpha}{\varphi}]}$. By
    the inductive
    hypothesis~$\val{\tau}{\xi[\subst{\alpha}{X}]}{\omega[\subst{\alpha}{\varphi}]}$
    is a candidate. Also $t \varphi$ is neutral, so $t \varphi \in
    \val{\tau}{\xi[\subst{\alpha}{X}]}{\omega[\subst{\alpha}{\varphi}]}$
    because~$t''$ was arbitrary with $t \varphi \arrW t''$. This
    implies that $t \in \val{\forall\alpha[\tau]}{\xi}{\omega}$.
  \item Assume $t_1,t_2 \in
    \val{\forall\alpha[\tau]}{\xi}{\omega}$. We have already shown
    that this implies $t_1,t_2 \in \SN$. We prove
    $\circ_{\omega(\forall\alpha[\tau])} t_1 t_2 \in
    \val{\forall\alpha['tau]}{\xi}{\omega}$ by induction on $\nu(t_1)
    + \nu(t_2)$. Since $s := \circ_{\omega(\forall\alpha[\tau])} t_1
    t_2$ is neutral, we have already proven that it suffices to show
    that $s' \in \val{\forall\alpha[\tau]}{\xi}{\omega}$ whenever $s
    \arrW s'$. The cases when $t_1$ or $t_2$ are reduced are
    immediate with the induction hypotheses. The only remaining case
    is when $s'=\tabs{\alpha}{\circ_{\omega(\tau)} (t_1 \alpha) (t_2
      \alpha)}$.  For all closed type constructors $\varphi$ of
    kind~$\kappa$ and all $X \in \Cb_{\varphi}^\Gamma$ we have both
    $t_1 \varphi$ and $t_2 \varphi$ in
    $\val{\tau}{\xi[\subst{\alpha}{X}]}{\omega[\subst{\alpha}{\varphi}]}$
    (by definition of $t_1,t_2 \in
    \val{\forall\alpha[\tau]}{\xi}{\omega}$). Let $\omega' =
    \omega[\subst{\alpha}{\varphi}]$. By bound variable renaming, we
    may assume $\omega(\alpha) = \alpha$ and $\alpha$ does not occur
    in~$t_1,t_2$. Because
    $\val{\tau}{\xi[\subst{\alpha}{X}]}{\omega[\subst{\alpha}{\varphi}]}$
    is a candidate by the inductive hypothesis for~$\sigma$, we have
    \[
    \circ_{\omega'(\tau)} (t_1 \varphi)
    (t_2\varphi) = (\circ_{\omega(\tau)} (t_1 \alpha) (t_2
    \alpha))[\subst{\alpha}{\varphi}] \in
    \val{\tau}{\xi[\subst{\alpha}{X}]}{\omega[\subst{\alpha}{\varphi}]}.
    \]
    Hence $s' \in \val{\forall\alpha[\tau]}{\xi}{\omega}$ by
    Lemma~\ref{lem_abstraction_computable}.
  \item Let $t \in \SN$ be such that $t : \nat$. By induction
    on~$\nu(t)$ we show $s := \lift_{\omega(\forall\alpha[\tau])}(t)
    \in \val{\forall\alpha[\tau]}{\xi}{\omega}$. First note that $s :
    \omega(\forall\alpha[\tau])$. Since~$s$ is neutral, by the already
    proven point~3 above, it suffices to show that $s' \in
    \val{\forall\alpha[\tau]}{\xi}{\omega}$ whenever $s \arrW
    s'$. The case when~$t$ is reduced is immediate by the inductive
    hypothesis. The only remaining case is when $s' =
    \tabs{\alpha}{\lift_{\omega(\tau)}(t)}$ (without loss of
    generality assuming $\omega(\alpha) = \alpha$). Let $\varphi$ be a
    closed type constructor of kind~$\kappa$ and let $X \in
    \Cb_\varphi$. Because
    $\val{\tau}{\xi[\subst{\alpha}{X}]}{\omega[\subst{\alpha}{\varphi}]}$
    is a candidate, we have
    \[
    \lift_{\omega[\subst{\alpha}{\varphi}](\tau)}(t) =
    (\lift_{\omega(\tau)}(t))[\subst{\alpha}{\varphi}] \in
    \val{\tau}{\xi[\subst{\alpha}{X}]}{\omega[\subst{\alpha}{\varphi}]}.
    \]
    This implies $s' \in \val{\forall\alpha[\tau]}{\xi}{\omega}$.
  \item Let $t \in \val{\forall\alpha[\tau]}{\xi}{\omega}$. We show $s
    := \flatten_{\omega(\forall\alpha[\tau])}t \in \SN$. We have
    already shown $t \in \SN$ in point~1 above. Thus any infinite
    reduction starting from~$s$ must have the form $s \arrW^*
    \flatten_{\omega(\forall\alpha[\tau])}t' \arrW
    \flatten_{\omega(\tau)[\subst{\alpha}{\chi_\kappa}]}(t'
    \chi_\kappa) \arrW \ldots$ with $t \arrW^* t'$ (assuming
    $\omega(\alpha) = \alpha$ without loss of generality). We have
    already shown in point~2 above that
    $\val{\forall\alpha[\tau]}{\xi}{\omega}$ is closed
    under~$\arrW$, so $t' \in
    \val{\forall\alpha[\tau]}{\xi}{\omega}$. We have
    $\val{\chi_\kappa}{}{} \in \Cb_{\chi_\kappa}$ by
    Lemma~\ref{lem_chi_kappa_computable}. Since~$\chi_\kappa$ is also
    closed, we have $t' \chi_\kappa \in
    \val{\tau}{\xi[\subst{\alpha}{\val{\chi_\kappa}{}{}}]}{\omega[\subst{\alpha}{\chi_\kappa}]}$
    by definition of $\val{\forall\alpha[\tau]}{\xi}{\omega}$. By the
    inductive hypothesis
    $\val{\tau}{\xi[\subst{\alpha}{\val{\chi_\kappa}{}{}}]}{\omega[\subst{\alpha}{\chi_\kappa}]}
    \in \Cb_{\omega[\subst{\alpha}{\chi_\kappa}](\tau)}$. Hence
    $\flatten_{\omega[\subst{\alpha}{\chi_\kappa}](\tau)}(t'\chi_\kappa)\in\SN$. But
    $\omega[\subst{\alpha}{\chi_\kappa}](\tau) =
    \omega(\tau)[\subst{\alpha}{\chi_\kappa}]$ because~$\chi_\kappa$
    is closed and $\omega(\alpha) = \alpha$. Contradiction.
  \end{enumerate}

  Assume $\sigma = \varphi\psi$, with $\psi$ of kind~$\kappa_1$ and
  $\varphi$ of kind~$\kappa_1\arrkind\kappa_2$. By the inductive
  hypothesis $\val{\psi}{\xi}{\omega} \in \Cb_{\omega(\psi)}$ and
  $\val{\varphi}{\xi}{\omega} \in \Cb_{\omega(\varphi)}$. Because
  applying~$\omega$ does not change kinds, we have
  $\val{\varphi\psi}{\xi}{\omega} =
  \val{\varphi}{\xi}{\omega}(\omega(\psi), \val{\psi}{\xi}{\omega})
  \in \Cb_{\omega(\varphi\psi)}$, by the definition of candidates of a
  type constructor with kind~$\kappa_1\arrkind\kappa_2$ (note that
  $\omega(\psi)$ is closed, because $\omega$ is closed for~$\sigma$).

  Finally, assume $\sigma = \lambda(\alpha:\kappa)\varphi$. Let $\psi$
  be a closed type constructor of kind~$\kappa$ and $X \in
  \Cb_{\psi}$. By the inductive hypothesis
  $\val{\lambda(\alpha:\kappa)\varphi}{\xi}{\omega}(\psi,X) =
  \val{\varphi}{\xi[\subst{\alpha}{X}]}{\omega[\subst{\alpha}{\psi}]}
  \in \Cb_{\omega[\subst{\alpha}{\psi}](\varphi)}$. Because $\psi$ is
  closed we have $\omega[\subst{\alpha}{\psi}](\varphi) =
  \omega(\varphi[\subst{\alpha}{\psi}]) =_\beta
  \omega((\lambda\alpha.\varphi)\psi) = \omega(\sigma\psi) =
  \omega(\sigma)\psi$. By Lemma~\ref{lem_beta_candidate} this implies
  that $\val{\sigma}{\xi}{\omega} \in \Cb_{\omega(\sigma)}$.
\end{proof}

\begin{lemma}\label{lem_circ}
  If $\omega$ is closed for~$\sigma$ then $\circ_{\omega(\sigma)} \in
  \val{\sigma \arrtype \sigma \arrtype \sigma}{\xi}{\omega}$ for
  $\circ \in \{ \oplus, \otimes \}$.
\end{lemma}

\begin{proof}
  Follows from definitions, Lemma~\ref{lem_val_computable} and
  property~4 of candidates.
\end{proof}

\begin{lemma}\label{lem_lift}
  If $\omega$ is closed for~$\sigma$ then $\lift_{\omega(\sigma)} \in
  \val{\nat\arrtype\sigma}{\xi}{\omega}$.
\end{lemma}

\begin{proof}
  Follows from definitions, Lemma~\ref{lem_val_computable} and
  property~5 of candidates.
\end{proof}

\begin{lemma}\label{lem_flatten}
  If $\omega$ is closed for~$\sigma$ then $\flatten_{\omega(\sigma)}
  \in \val{\sigma\arrtype\nat}{\xi}{\omega}$.
\end{lemma}

\begin{proof}
  Follows from definitions, Lemma~\ref{lem_val_computable} and
  property~6 of candidates.
\end{proof}

\begin{lemma}\label{lem_val_subst}
  For any type constructors~$\sigma,\tau$ with $\alpha \notin
  \FV(\tau)$, a mapping~$\omega$ closed for~$\sigma$ and for~$\tau$,
  and an $\omega$-valuation~$\xi$, we have:
  \[
  \val{\sigma[\subst{\alpha}{\tau}]}{\xi}{\omega} =
  \val{\sigma}{\xi[\subst{\alpha}{\val{\tau}{\xi}{\omega}}]}{\omega[\subst{\alpha}{\omega(\tau)}]}.
  \]
\end{lemma}

\begin{proof}
  Let~$\omega' = \omega[\subst{\alpha}{\omega(\tau)}]$ and $\xi' =
  \xi[\subst{\alpha}{\val{\tau}{\xi}{\omega}}]$. First note
  that~$\omega$ is closed for~$\sigma[\subst{\alpha}{\tau}]$
  and~$\omega'$ is closed for~$\sigma$. We proceed by induction
  on~$\sigma$. If $\alpha \notin \FV(\sigma)$ then the claim is
  obvious. If $\sigma = \alpha$ then
  $\val{\sigma[\subst{\alpha}{\tau}]}{\xi}{\omega} =
  \val{\tau}{\xi}{\omega} = \val{\sigma}{\xi'}{\omega'}$.

  Assume $\sigma = \sigma_1\arrtype\sigma_2$. We show
  $\val{\sigma[\subst{\alpha}{\tau}]}{\xi}{\omega} \subseteq
  \val{\sigma}{\xi'}{\omega'}$. Let $t \in
  \val{\sigma[\subst{\alpha}{\tau}]}{\xi}{\omega}$. We have $t :
  \omega(\sigma[\subst{\alpha}{\tau}])$, so $t : \omega'(\sigma)$. Let
  $u \in \val{\sigma_1}{\xi'}{\omega'}$. By the inductive hypothesis
  $u \in \val{\sigma_1[\subst{\alpha}{\tau}]}{\xi}{\omega}$. Hence $t
  u \in \val{\sigma_2[\subst{\alpha}{\tau}]}{\xi}{\omega} =
  \val{\sigma_2}{\xi'}{\omega'}$, where the last equality follows from
  the inductive hypothesis. Thus $t \in
  \val{\sigma}{\xi'}{\omega'}$. The other direction is analogous. The
  case $\sigma = \forall\alpha[\sigma']$ is also analogous.

  Assume $\sigma = \varphi\psi$. We have
  $\val{\sigma[\subst{\alpha}{\tau}]}{\xi}{\omega} =
  \val{\varphi[\subst{\alpha}{\tau}]}{\xi}{\omega}(\omega(\psi[\subst{\alpha}{\tau}]),
  \val{\psi[\subst{\alpha}{\tau}]}{\xi}{\omega}) =
  \val{\varphi[\subst{\alpha}{\tau}]}{\xi}{\omega}(\omega'(\psi),
  \val{\psi[\subst{\alpha}{\tau}]}{\xi}{\omega}) =
  \val{\varphi}{\xi'}{\omega'}(\omega'(\psi),
  \val{\psi}{\xi'}{\omega'})$ where the last equality follows from the
  inductive hypothesis.

  Finally, assume $\sigma = \lambda(\beta:\kappa)\varphi$. Let $\psi
  \in \Tc_\kappa$ be closed and let $X \in \Cb_\psi$. We have
  $\val{\sigma[\subst{\alpha}{\tau}]}{\xi}{\omega}(\psi,X) =
  \val{\varphi[\subst{\alpha}{\tau}]}{\xi[\subst{\beta}{X}]}{\omega[\subst{\beta}{\tau}]}
  =
  \val{\varphi}{\xi'[\subst{\beta}{X}]}{\omega'[\subst{\beta}{\tau}]}
  = \val{\sigma}{\xi'}{\omega'}(\psi,X)$ where we use the inductive
  hypothesis in the penultimate equality.
\end{proof}

\begin{lemma}\label{lem_forall}
  Let $\tau$ be a type constructor of kind~$\kappa$. Assume $\omega$
  is closed for $\forall\alpha[\sigma]$ and for~$\tau$. If $t \in
  \val{\forall(\alpha:\kappa)[\sigma]}{\xi}{\omega}$ then $t
  (\omega(\tau)) \in \val{\sigma[\subst{\alpha}{\tau}]}{\xi}{\omega}$.
\end{lemma}

\begin{proof}
  By Lemma~\ref{lem_val_computable} we have~$\val{\tau}{\xi}{\omega}
  \in \Cb_{\omega(\tau)}$. So $t (\omega(\tau)) \in
  \val{\sigma}{\xi[\subst{\alpha}{\val{\tau}{\xi}{\omega}}]}{\omega[\subst{\alpha}{\omega(\tau)}]}$
  by $t \in \val{\forall(\alpha:\kappa)[\sigma]}{\xi}{\omega}$. Hence
  $t (\omega(\tau)) \in
  \val{\sigma[\subst{\alpha}{\tau}]}{\xi}{\omega}$ by
  Lemma~\ref{lem_val_subst}.
\end{proof}

\begin{lemma}\label{lem_beta_val}
  If $\omega$ is closed for~$\sigma,\sigma'$ and $\sigma =_\beta
  \sigma'$ then $\val{\sigma}{\xi}{\omega} =
  \val{\sigma'}{\xi}{\omega}$.
\end{lemma}

\begin{proof}
  It suffices to show the lemma for the case when~$\sigma$ is a
  $\beta$-redex. Then the general case follows by induction
  on~$\sigma$ and the length of reduction to a common reduct.

  So assume $(\lambda\alpha\tau)\sigma \to_\beta
  \tau[\subst{\alpha}{\sigma}]$. We have
  $\val{(\lambda\alpha\tau)\sigma}{\xi}{\omega} =
  \val{\lambda\alpha\tau}{\xi}{\omega}(\omega(\sigma),
  \val{\sigma}{\xi}{\omega}) =
  \val{\tau}{\xi[\subst{\alpha}{\val{\sigma}{\xi}{\omega}}]}{\omega[\subst{\alpha}{\omega(\sigma)}]}
  = \val{\tau[\subst{\alpha}{\sigma}]}{\xi}{\omega}$ where the last
  equality follows from Lemma~\ref{lem_val_subst}.
\end{proof}

A mapping~$\omega$ on type constructors is extended in the obvious way
to a mapping on terms. Note that $\omega$ also acts on the type
annotations of variable occurrences, e.g.~$\omega(\lambda x : \alpha
. x^\alpha) = \lambda x : \omega(\alpha) . x^{\omega(\alpha)}$.

\begin{lemma}\label{lem_typable_computable}
  If $t : \sigma$ and $\omega$ is closed for~$\sigma$ and
  $\FTV(\omega(t)) = \emptyset$ then $\omega(t) \in
  \val{\sigma}{\xi}{\omega}$.
\end{lemma}

\begin{proof}
  We prove by induction on the structure of~$t$ that if $t : \sigma$
  and $\omega$ is closed for~$\sigma$ and $\FTV(\omega(t)) =
  \emptyset$ and $x_1^{\tau_1},\ldots,x_n^{\tau_n}$ are all free
  variable occurrences in the canonical representative of~$t$ (so
  each~$\tau_i$ is $\beta$-normal), then for all
  $u_1\in\val{\tau_1}{\xi}{\omega},\ldots,u_n\in\val{\tau_n}{\xi}{\omega}$
  we have $\omega(t)[\subst{x_1}{u_1},\ldots,\subst{x_n}{u_n}] \in
  \val{\sigma}{\xi}{\omega}$. This suffices because
  $\omega(x_i^{\tau_i}) \in \val{\tau_i}{\xi}{\omega}$. Note that
  $\omega$ is closed for each~$\tau_i$ because $\FTV(\omega(t)) =
  \emptyset$ and~$t$ is typed, so no type constructor variable
  occurring free in~$\tau_i$ can be bound in~$t$ by a~$\Lambda$;
  e.g.~$\Lambda \alpha . x^\alpha$ is not a valid typed term (we
  assume~$\tau_i$ to be in $\beta$-normal form). For brevity, we use
  the notation $\omega^*(t) =
  \omega(t)[\subst{x_1}{u_1},\ldots,\subst{x_n}{u_n}]$. Note that
  $\omega^*(t) : \omega(\sigma)$.

  By the generation lemma for $t : \sigma$ there is a type~$\sigma'$
  such that $\sigma' =_\beta \sigma$ and $\FV(\sigma') \subseteq
  \FV(\sigma) \cup \FTV(t)$ and one of the cases below holds. Note
  that~$\omega$ is closed for~$\sigma'$ because it is closed
  for~$\sigma$ and $\FTV(\omega(t)) = \emptyset$. By
  Lemma~\ref{lem_beta_val} it suffices to show $\omega^*(t) \in
  \val{\sigma'}{\xi}{\omega}$.
  \begin{itemize}
  \item If $t = x_1^{\tau_1}$ and $\tau_1 =_\beta \sigma'$ then
    $\omega(t)[\subst{x_1}{u_1}] =
    (x_1^{\omega(\tau_1)})[\subst{x_1}{u_1}] = u_1 \in
    \val{\tau_1}{\xi}{\omega} = \val{\sigma'}{\xi}{\omega}$ by
    assumption and Lemma~\ref{lem_beta_val}.
  \item If $t = n$ is a natural number and $\sigma' = \nat$ then $t
    \in \val{\nat}{}{}$ by definition.
  \item If $t$ is a function symbol then the claim follows from
    Lemma~\ref{lem_circ}, Lemma~\ref{lem_lift} or
    Lemma~\ref{lem_flatten}.
  \item If $t = \abs{x:\sigma_1}{s}$ then
    $\sigma' = \sigma_1\arrtype\sigma_2$ and $s : \sigma_2$. Hence
    $\omega$ is closed for~$\sigma_2$. Let
    $u \in \val{\sigma_1}{\xi}{\omega}$. By the inductive hypothesis
    $\omega^*(s)[\subst{x}{u}] \in \val{\sigma_2}{\xi}{\omega}$. Hence
    $\omega^*(t) \in \val{\sigma'}{\xi}{\omega}$ by
    Lemma~\ref{lem_abstraction_computable}.
  \item If $t = \tabs{\alpha:\kappa}{s}$ then $\sigma' =
    \forall\alpha[\tau]$ and $s : \tau$. Let $\psi$ be a closed type
    constructor of kind~$\kappa$ and let $X \in \Cb_\psi$. Let
    $\omega_1 = \omega[\subst{\alpha}{\psi}]$ and
    $\xi_1=\xi[\subst{\alpha}{X}]$. Then $\omega_1$ is closed
    for~$\tau$ and $\FTV(\omega_1(s)) = \emptyset$. By the inductive
    hypothesis $\omega_1^*(s) \in \val{\tau}{\xi_1}{\omega_1}$. We
    have $\omega_1^*(s) = \omega^*(s)[\subst{\alpha}{\psi}]$ (assuming
    $\alpha$ chosen fresh such that $\omega(\alpha) = \alpha$). Hence
    $\omega^*(t) \in \val{\tau}{\xi}{\omega}$ by
    Lemma~\ref{lem_abstraction_computable}.
  \item If $t = t_1 t_2$ then $t_1 : \tau\arrtype\sigma'$ and
    $t_2 : \tau$ and $\FV(\tau) \subseteq \FV(\sigma) \cup
    \FTV(t)$. Hence~$\omega$ is closed for~$\tau$ and
    for~$\tau\arrtype\sigma'$. By the inductive hypothesis
    $\omega^*(t_1) \in \val{\tau\arrtype\sigma'}{\xi}{\omega}$ and
    $\omega^*(t_2) \in \val{\tau}{\xi}{\omega}$. We have
    $\omega^*(t_2) : \omega(\tau)$. Then by definition
    $\omega^*(t) = (\omega^*(t_1))(\omega^*(t_2)) \in
    \val{\sigma'}{\xi}{\omega}$.
  \item If $t = s \psi$ then $s : \forall\alpha[\tau]$ and $\sigma' =
    \tau[\subst{\alpha}{\psi}]$. By the inductive hypothesis
    $\omega^*(s) \in \val{\forall\alpha[\tau]}{\xi}{\omega}$. Because
    $\FTV(\omega(t)) = \emptyset$, the mapping $\omega$ is closed
    for~$\psi$. So by Lemma~\ref{lem_forall} we have $\omega^*(t) =
    \omega^*(s) \omega(\psi) \in
    \val{\tau[\subst{\alpha}{\psi}]}{\xi}{\omega}$.
  \end{itemize}
\end{proof}

{ \renewcommand{\thetheorem}{\ref{thm_sn}}
\begin{theorem}
  If $\Gamma \proves t : \sigma$ then $t \in \SN$.
\end{theorem}
\addtocounter{theorem}{-1}}

\begin{proof}
  For closed terms~$t$ and closed types~$\sigma$ this follows from
  Lemma~\ref{lem_typable_computable}, Lemma~\ref{lem_val_computable}
  and property~1 of candidates (Definition~\ref{def_candidate}). For
  arbitrary terms and types, this follows by closing the terms with an
  appropriate number of abstractions, and the types with corresponding
  $\forall$-quantifiers.
\end{proof}

{ \renewcommand{\thetheorem}{\ref{lem_final_nat}}
\begin{lemma}
  The only final interpretation terms of type $\nat$ are the natural
  numbers.
\end{lemma}
\addtocounter{theorem}{-1}}

\begin{proof}
  We show by induction on~$t$ that if $t$ is a final interpretation
  term of type~$\nat$ then $t$ is a natural number. Because~$t$ is
  closed and in normal form, if it is not a natural number then it
  must have the form $\mathtt{f}_\sigma t_1 \ldots t_n$ for a function
  symbol $\mathtt{f}$. For concreteness assume $\mathtt{f} =
  \oplus$. Then $n \ge 2$. Because~$t$ is closed, $\sigma$ cannot be a
  type variable. It also cannot be an arrow or a $\forall$-type,
  because then $t$ would contain a redex. So $\sigma=\nat$. Then
  $t_1,t_2$ are final interpretation terms of type~$\nat$, hence
  natural numbers by the inductive hypothesis. But then $t$ contains a
  redex. Contradiction.
\end{proof}

\section{Proving the inequalities in \refsec{examples}}\label{app_ineqs}

\end{document}
