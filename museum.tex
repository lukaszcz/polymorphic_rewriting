% A museum of no-longer-needed lemmas and such (fragments of proofs of
% which may however become useful later)

\cite[Chapter 6,7]{Kop2012}).
\CK{If space happens to permit, we can replace the pointer to my thesis by
cite\{kop:raa:12,suz:kus:bla:11\}, but this does require 4 extra lines
of references.}

From the definition we immediately see that all encodings have the expected types:

\begin{lemma}\label{lem:encodings_types}
  \begin{enumerate}
  \item If $\Gamma \proves t_1 : \sigma$ and $\Gamma \proves t_2 :
    \tau$ then $\Gamma \proves \pair{t_1}{t_2} : \sigma \times \tau$.
  \item If $\Gamma \proves t : \tau_1 \times \tau_2$ then $\Gamma
    \proves \pi^i(t) : \tau_i$.
  \item If $\Gamma \proves t : \sigma[\subst{\alpha}{\tau}]$ then
    $\Gamma \proves \expair{\tau}{t} : \Sigma \alpha . \sigma$.
  \item If $\Gamma \proves t : \Sigma \alpha . \sigma$ and
    $\Gamma,x:\sigma \proves s : \rho$ and $\alpha \notin
    \FTV(\Gamma,\rho)$ then $\xlet{}{t}{\alpha,x}{s} : \rho$.
  \end{enumerate}
\end{lemma}

In addition, $\beta$-reduction provides the desired reduction rules:

\begin{lemma}\label{lem:encodings_reduce}
  $\pi^i(\pair{t_1}{t_2}) \leadsto_\beta^* t_i$ and
  $\xlet{}{\expair{\tau}{t}}{\alpha,x}{s} \leadsto_\beta^*
  s[\subst{\alpha}{\tau}][\subst{x}{t}]$.
\end{lemma}



Polymorphism in type theory is parametric, which means that a
polymorphic function behaves uniformly with respect to its generic
arguments, not depending on their actual types. This makes it
impossible to directly extract elements from a heterogenous list -- we
can only perform some operations on them under the assumption that we
know nothing about their types. In practice, one would constrain the
type variable~$\alpha$ above with a type class, in order to guarantee
the existence of certain functions on the elements of a list.


\subsection{Parametricity}

\begin{definition}\label{def_param_candidate}\normalfont
  For two type constructors~$\tau_1,\tau_2$ of the same kind~$\kappa$,
  we define by induction on~$\kappa$ the set~$\Rel_{\tau_1,\tau_2}$ of
  $\tau_1,\tau_2$-candidates.

  Assume $\kappa=*$, i.e., $\tau_1,\tau_2$ are types. A relation~$R$
  between terms of type~$\tau_1$ and terms of type~$\tau_2$ is a
  \emph{candidate relation} in $\Rel_{\tau_1,\tau_2}$ when the
  following properties are satisfied:
  \begin{enumerate}
  \item if $R(t_1,t_2)$ and $t_1' \leadsto t_1$ (resp.~$t_2' \leadsto
    t_2$) then $R(t_1',t_2)$ (resp.~$R(t_1,t_2')$),
  \item \ldots
  \end{enumerate}
\end{definition}

\begin{lemma}\label{lem_beta_param_candidate}
  If $\sigma_1 =_\beta \sigma_1'$ and $\sigma_2 =_\beta \sigma_2'$
  then $\Rel_{\sigma_1,\sigma_2} = \Rel_{\sigma_1',\sigma_2'}$.
\end{lemma}

\begin{proof}
  Induction on the kind of~$\sigma_1,\sigma_2$.
\end{proof}

\begin{definition}\label{def_param_valuation}\normalfont
  Let $\omega$ be a mapping from type constructor variables to pairs
  of type constructors (respecting kinds). The mapping~$\omega$
  extends in an obvious way to a mapping from type constructors to
  pairs type constructors. We write $\omega_i(\tau)$ for the $i$-th
  component of the pair~$\omega(\tau)$. A mapping~$\omega$ is
  \emph{closed for~$\sigma$} if $\omega_i(\alpha)$ is closed for
  $\alpha \in \FV(\sigma)$ (then $\omega_i(\sigma)$ is closed), for
  $i=1,2$.

  An \emph{$\omega$-valuation} is a mapping~$\xi$ on type constructor
  variables such that $\xi(\alpha) \in
  \Rel_{\omega_1(\alpha),\omega_2(\alpha)}$.

  For each type constructor~$\sigma$, each mapping~$\omega$ closed
  for~$\sigma$, and each $\omega$-valuation~$\xi$, we define
  $\rval{\sigma}{\xi}{\omega}$ by induction on~$\sigma$:
  \begin{itemize}
  \item $\rval{\alpha}{\xi}{\omega} = \xi(\alpha)$ for a type
    constructor variable~$\alpha$,
  \item $\rval{\nat}{\xi}{\omega}(t_1,t_2)$ iff $t_1 \approx_\nat t_2$,
  \item $\rval{\sigma \arrtype \tau}{\xi}{\omega}(t_1,t_2)$ iff $t_i :
    \omega_i(\sigma\arrtype\tau)$ and for all $s_1,s_2$ such that
    $\rval{\sigma}{\xi}{\omega}(s_1,s_2)$ we have
    $\rval{\tau}{\xi}{\omega}(t_1 s_1, t_2 s_2)$,
  \item $\rval{\forall(\alpha:\kappa)[\sigma]}{\xi}{\omega}(t_1,t_2)$
    iff $t_i : \omega_i(\forall\alpha\sigma)$ and for all closed type
    constructors~$\tau_1,\tau_2$ of kind~$\kappa$ and every $R \in
    \Rel_{\tau_1,\tau_2}$ we have
    $\rval{\sigma}{\xi[\subst{\alpha}{R}]}{\omega[\subst{\alpha}{(\tau_1,\tau_2)}]}$.
  \end{itemize}
\end{definition}


  \begin{enumerate}
  \item By coinduction we show $(s \oplus t) u_1 \ldots u_m
    \succeq_\sigma s u_1 \ldots u_m$ for closed $u_1,\ldots,u_m$. The
    second case is similar.

    First note that $(s \oplus t) u_1 \ldots u_m \leadsto^* s u_1
    \ldots u_m \oplus t u_1 \ldots u_m$.

    If $\sigma = \nat$ then $((s \oplus t) u_1 \ldots u_m)\da = (s u_1
    \ldots u_m)\da + (t u_1 \ldots u_m)\da \ge (s u_1 \ldots
    u_m)\da$. Hence $(s \oplus t) u_1 \ldots u_m) \succeq_\nat s u_1
    \ldots u_m$.

    If $\sigma = \tau_1\arrtype\tau_2$ then by the coinductive
    hypothesis $(s \oplus t) u_1 \ldots u_m q \succeq_{\tau_2} s u_1
    \ldots u_m q$ for any $q \in \World_{\tau_1}$. Hence $(s \oplus t)
    u_1 \ldots u_m \succeq_\sigma s u_1 \ldots u_m$.

    If $\sigma = \forall(\alpha:\kappa)[\tau]$ then by the coinductive
    hypothesis $(s \oplus t) u_1 \ldots u_m \xi \succeq_{\sigma'} s
    u_1 \ldots u_m \xi$ for any closed $\xi \in \Tc_\kappa$, where
    $\sigma' = \tau[\subst{\alpha}{\xi}]$. Hence $(s \oplus t) u_1
    \ldots u_m \succeq_\sigma s u_1 \ldots u_m$.
  \item By coinduction we show $(s \oplus (\lift n)) u_1 \ldots u_m
    \succeq_\sigma s u_1 \ldots u_m$ for closed $u_1,\ldots,u_m$. The
    second case is similar.

    Note that $(s \oplus (\lift n)) u_1 \ldots u_m \leadsto^* s u_1
    \ldots u_m \oplus n$. From this the base case $\sigma=\nat$
    follows. The other cases follow from the induction hypothesis,
    like in the first point above.
  \end{enumerate}


$\{ \oplus_\sigma : \sigma \arrtype
  \sigma \arrtype \sigma,\ \otimes_\sigma : \sigma \arrtype \sigma
  \arrtype \sigma,\ \flatten_{\sigma} : \sigma \arrtype
  \nat,\ \lift_{\sigma} : \nat \arrtype \sigma \mid \sigma \in \ITypes
  \}$

\section{TODO list}

\renewcommand{\theenumii}{\arabic{enumi}.\arabic{enumii}}

Here's what we need to do:
\begin{enumerate}
\item Define a world $\World$ and a well-founded ordering $\succ$ on
  $\World$:
  \begin{enumerate}
  \item Define a set of terms $\World$ typed under some variation of
    System F-$\omega$.
  \item Define relations $\succ$ and $\succeq$ on the elements of $\World$.
  \item Prove that $\succ$ is a well-founded ordering relation and that
    $\succeq$ is a compatible quasi-ordering.
  \end{enumerate}
\item Specify what systems we are interested in analysing, and prove
  standard results which will make their analysis doable.
  \begin{enumerate}
  \item Specify a form of system which includes all the systems of interest.
  \item Specify a default way of interpreting terms in these systems.
  \item Prove that in all such systems, using our way of interpreting
    terms: if $\interpret{\ell} \succ \interpret{r}$ (resp.\ $\interpret{
    \ell} \succeq \interpret{r}$) for a rule  $\ell \to r$, then
    $\interpret{s} \succ \interpret{t}$ whenever $s \arr{\Rules} t$ by
    this rule (resp.\ $\interpret{s} \succeq \interpret{t}$).
  \end{enumerate}
\item Obtain useful lemmas regarding these defaults.
  \begin{enumerate}
  \item $\interpret{s[x:=t]} = \interpret{s}[x:=\interpret{t}]$.
  \item $\interpret{s\sigma} = \interpret{s}\sigma$.
  \item \dots?
  \end{enumerate}
\item For some system of interest, prove its termination:
  \begin{enumerate}
  \item Present the system and give interpretations (following the
    default scheme) for all ways of constructing terms.
  \item Show that $\ell \succ r$ or $\ell \succeq r$ for all rules.
    Remove the rules which are oriented using $\succ$ and repeat,
    until all rules have been removed.
  \end{enumerate}
\end{enumerate}

\renewcommand{\theenumii}{\alph{enumii}}

We use \emph{rule removal}:

\begin{theorem}\label{thm:ruleremove}
Let $\Rules = \Rules_1 \cup \Rules_2$, and suppose that $\arr{\Rules_1}\:
\subseteq\:\succ$ and $\arr{\Rules_2}\:\subseteq\:\succeq$ for a
well-founded ordering $\succ$ and a compatible quasi-ordering $\succeq$.
Then $\arr{\Rules}$ is terminating if and only if $\arr{\Rules_2}$ is
(which is certainly the case if $\Rules_2 = \emptyset$).
\end{theorem}

\begin{proof}
By well-foundedness of $\succ$, every infinite decreasing $\arr{\Rules}$
sequence can only use finitely many steps using $\arr{\Rules_1}$.
\qed
\end{proof}

This gives rise to the following algorithm:
\begin{enumerate}
\item While $\Rules$ is non-empty:
  \begin{enumerate}
  \item Orient all rules in $\Rules$ using $\succeq$ or $\succ$; at least
    one of them must be oriented using $\succ$.
  \item Remove all rules ordered by $\succ$ from $\Rules$.
  \end{enumerate}
\end{enumerate}
If this algorithm succeeds, we have proven termination.

\section{Lemmas}

\begin{lemma}\label{lem_reducible_type}
  If $\tau$ is a type and $t \in \val{\tau}{\xi}{\omega}$ then
  $t : \omega(\tau)$.
\end{lemma}

\begin{proof}
  Let~$\varphi$ be a type constructor. By induction on the kind
  of~$\varphi$ we define when $\val{\varphi}{\omega}{\xi}$ is
  \emph{good}:
  \begin{itemize}
  \item if $\varphi$ is a type then $\val{\varphi}{\xi}{\omega}$ is
    good when $t \in \val{\varphi}{\xi}{\omega}$ implies
    $t : \omega(\varphi)$,
  \item if $\varphi$ has kind~$\kappa_1\arrkind\kappa_2$ then
    $\val{\varphi}{\xi}{\omega}$ is good when
    $\val{\varphi}{\xi[\subst{\alpha}{X}]}{\omega[\subst{\alpha}{\tau}]}$
    is good for every $\tau \in \Tc_{\kappa_1}$ and every
    $X \in \Cb_\tau$.
  \end{itemize}
  Now prove by induction on~$\varphi$
  that~$\val{\varphi}{\omega,\xi}{}$ is good.

  TODO
\end{proof}

Thus, we have:

\begin{lemma}\label{lem_abusive_notation}
  Every interpretation term has the form $s t_1 \dots t_n$ with $s$ a
  variable, function symbol or abstraction $\abstraction{a}{s'}$ (for
  $n \geq 0$), or $s$ a natural number and $n = 0$.
\end{lemma}

\begin{proof}
By induction on the size of interpretation terms (and a simple case
analysis).\qed
\end{proof}

\newcommand{\arreta}{\rightarrow_\eta}
\newcommand{\etalong}[1]{#1\uparrow^\eta}
\newcommand{\almostetalong}[1]{\overline{#1}}

In addition, we are particularly interested in interpretation terms in
\emph{$\eta$-long form}.

\begin{definition}
For a given variable environment $\Gamma$ and interpretation term $s$
which is well-typed under $\Gamma$, the \emph{$\eta$-long form} of $s$,
denoted $\etalong{s}$, is defined as follows:
\begin{itemize}
\item if $\Gamma \vdash s : \sigma \arrtype \tau$ and $s = \abs{x}{s'}$,
  then $\etalong{s} =  \abs{x}{(\etalong{s'})}$
\item if $\Gamma \vdash s : \sigma \arrtype \tau$ and $s$ does not have
  the form $\abs{x}{s'}$, then $\etalong{s} =  \abs{x}{(
  \etalong{(\app{s}{x})})}$
\item if $\Gamma \vdash s : \quant{\alpha}{\sigma}$ and $s = \tabs{
  \alpha}{s'}$, then $\etalong{s} = \tabs{\alpha}{(\etalong{s'})}$
\item if $\Gamma \vdash s : \quant{\alpha}{\sigma}$ and $s$ does not have
  the form $\tabs{\alpha}{s'}$, then $\etalong{s} = \tabs{\alpha}{
  (\etalong{(\tapp{s}{\alpha})})}$
\item if $\Gamma \vdash s : \nat$ or $\Gamma \vdash s : \alpha$ or
  $\Gamma \vdash s : \sigma \times \tau$, then $\etalong{s} =
  \almostetalong{s}$, where
  \begin{itemize}
  \item $\almostetalong{s} = s$ if $s$ is a natural number, variable or
    function symbol
  \item $\almostetalong{s} = \pair{\etalong{s_1}}{\etalong{s_2}}$ if
    $s = \pair{s_1}{s_2}$
  \item $\almostetalong{s} = \etalong{s}$ if $s$ has the form
    $\abs{x}{s'}$ or $\tabs{\alpha}{s'}$
  \item $\almostetalong{s} = \app{\almostetalong{s_1}}{(\etalong{s_2})}$
    if $s = \app{s_1}{s_2}$
  \item $\almostetalong{s} = \tapp{\almostetalong{s'}}{\tau}$
    if $s = \tapp{s'}{\tau}$
  \end{itemize}
\end{itemize}
A term $s$ is in \emph{$\eta$-long form} if $s = \etalong{s}$.
\end{definition}

It is clear that the $\eta$-long form of an interpretation term is
unique; the following lemma shows that it is also always well-defined.

\begin{lemma}
For all interpretation terms $s$, the function $\etalong{s}$ is
well-defined.
\end{lemma}

\begin{proof}
Define here the \emph{depth} function on interpretation terms as
follows:
\begin{itemize}
\item $\mathit{depth}(n) = \mathit{depth}(x) = \mathit{depth}(\mathtt{f})
  = 0$
\item $\mathit{depth}(\pair{s}{t}) = \max(\mathit{depth}(s),\mathit{
  depth}(t)) + 1$
\item $\mathit{depth}(\abs{x}{s}) = \mathit{depth}(\tabs{\alpha}{s}) =
  \mathit{depth}(s) + 1$
\item $\mathit{depth}(\app{s}{t}) = \max(\mathit{depth}(s),\mathit{
  depth}(t)+1)$
\item $\mathit{depth}(\tapp{s}{\tau}) = \mathit{depth}(s)$
\end{itemize}
(This definition is clearly well-defined as the term under consideration
becomes smaller each step.)
We can see that $\etalong{x}$ is well-defined for all variables by
induction on their type.  This we use to see that $\etalong{(\apps{s}{
x_1}{x_n})}$ is well-defined for $s$ a variable, natural number or
function symbol by induction on its type.  Thus, well-definedness of
$\etalong{s}$ for all $s$ with depth $0$ follows; also $\almostetalong{
s} = s$ is well-defined for such $s$.
For interpretation terms of depth $\geq 1$, well-definedness of
$\etalong{s}$ and $\overline{s}$ follows by induction on
$(\mathit{depth}(s)$, $1$, the size of the type of $s)$ in the case
of $\etalong{s}$ and
$(\mathit{depth}(s)$, $0$, the size of $s)$ in the case of
$\almostetalong{s}$.
\end{proof}

\begin{lemma}
If $s$ is a closed interpretation term in $\eta$-long form which is
terminating under $\leadsto$, then there is a final interpretation term
$t$ such that $s \leadsto^* t$.
\end{lemma}

\begin{proof}
TODO
\CK{And honestly not sure whether we even need this.}
\end{proof}

\LC{What do we need these $\eta$-expansions for? I think this just
  works now with the modified $\leadsto$-rules for $\oplus,\otimes$.}

\begin{lemma}
 If for all $u \in \val{\tau_1}{\omega,\xi}{\Gamma}$ we have
 $t[\subst{x}{u}] \in \val{\tau_2}{\omega,\xi}{\Gamma}$, then
 $\lambda x : \omega(\tau_1) . t \in
 \val{\tau_1\to\tau_2}{\omega,\xi}{\Gamma}$.
\end{lemma}

\begin{proof}
 By Lemma~\ref{lem_val_computable} the
 sets~$\val{\tau_i}{\omega,\xi}{\Gamma}$ are candidates, so
 $\val{\tau_i}{\omega,\xi}{\Gamma} \subseteq \SN$. Let
 $u \in \val{\tau_1}{\omega,\xi}{\Gamma}$. We show
 $(\lambda x : \omega(\tau_1) . t) u \in
 \val{\tau_2}{\omega,\xi}{\Gamma}$ by induction on $\nu(t) +
 \nu(u)$. Suppose $(\lambda x : \omega(\tau_1) . t) u \leadsto
 w$. There are three cases.
 \begin{itemize}
 \item $w = t[\subst{x}{u}]$. Then
   $w \in \val{\tau_2}{\omega,\xi}{\Gamma}$ by assumption.
 \item $w = (\lambda x : \omega(\tau_1) . t') u$ with
   $t \leadsto t'$. Because $\val{\tau_2}{\omega,\xi}{\Gamma}$ is a
   candidate, still
   $t'[\subst{x}{u}] \in \val{\tau_2}{\omega,\xi}{\Gamma}$. Also
   $\nu(t') < \nu(t)$, so $w \in \val{\tau_2}{\omega,\xi}{\Gamma}$
   follows from the inductive hypothesis.
 \item $w = (\lambda x : \omega(\tau_1) . t) u'$ with
   $u \leadsto u'$. Because $\val{\tau_i}{\omega,\xi}{\Gamma}$ are
   candidates, still $u' \in \val{\tau_1}{\omega,\xi}{\Gamma}$ and
   $t[\subst{x}{u'}] \in \val{\tau_2}{\omega,\xi}{\Gamma}$. Also
   $\nu(u') < \nu(u)$, so $w \in \val{\tau_2}{\omega,\xi}{\Gamma}$
   follows from the inductive hypothesis.
 \end{itemize}
 Because $(\lambda x : \omega(\tau_1) . t) u$ is neutral and
 $\val{\tau_2}{\omega,\xi}{\Gamma}$ is a candidate, this shows
 $(\lambda x : \omega(\tau_1) . t) u \in
 \val{\tau_2}{\omega,\xi}{\Gamma}$.
\end{proof}

\begin{lemma}
 If for every type~$\tau$ and every $X \in \Cb_\tau^\Gamma$ we have
 $t \in
 \val{\sigma}{\omega[\subst{\alpha}{\tau}],\xi[\subst{\alpha}{X}]}{\Gamma}$,
 then
 $\Lambda \alpha . t \in
 \val{\forall\alpha[\sigma]}{\omega,\xi}{\Gamma}$.
\end{lemma}

\begin{proof}
 Let~$\tau$ be a type and $X \in \Cb_\tau^\Gamma$. By
 Lemma~\ref{lem_val_computable} the
 set~$\val{\forall\alpha[\sigma]}{\omega,\xi}{\Gamma}$ is a
 candidate, so $t \in \SN$. By induction on~$\nu(t)$ we show
 $(\Lambda \alpha . t) \alpha \in
 \val{\sigma}{\omega[\subst{\alpha}{\tau}],\xi[\subst{\alpha}{X}]}{\Gamma}$.
 Because~$\val{\sigma}{\omega[\subst{\alpha}{\tau}],\xi[\subst{\alpha}{X}]}{\Gamma}$
 is a candidate by Lemma~\ref{lem_val_computable}, it suffices to
 show that for every~$w$ with $(\Lambda \alpha . t) \leadsto w$ we
 have
 $w \in
 \val{\sigma}{\omega[\subst{\alpha}{\tau}],\xi[\subst{\alpha}{X}]}{\Gamma}$. There
 are two cases.
 \begin{itemize}
 \item $w = (\Lambda \alpha . t') \alpha$. Then $\nu(t') < \nu(t)$
   and
   $w \in
   \val{\sigma}{\omega[\subst{\alpha}{\tau}],\xi[\subst{\alpha}{X}]}{\Gamma}$
   follows from the inductive hypothesis.
 \item $w = t$. Then
   $w \in
   \val{\sigma}{\omega[\subst{\alpha}{\tau}],\xi[\subst{\alpha}{X}]}{\Gamma}$
   by assumption.
 \end{itemize}
\end{proof}
