
% A museum of no-longer-needed lemmas and such (fragments of proofs of which may however become useful later)

\newcommand{\arreta}{\rightarrow_\eta}
\newcommand{\etalong}[1]{#1\uparrow^\eta}
\newcommand{\almostetalong}[1]{\overline{#1}}

In addition, we are particularly interested in interpretation terms in
\emph{$\eta$-long form}.

\begin{definition}
For a given variable environment $\Gamma$ and interpretation term $s$
which is well-typed under $\Gamma$, the \emph{$\eta$-long form} of $s$,
denoted $\etalong{s}$, is defined as follows:
\begin{itemize}
\item if $\Gamma \vdash s : \sigma \arrtype \tau$ and $s = \abs{x}{s'}$,
  then $\etalong{s} =  \abs{x}{(\etalong{s'})}$
\item if $\Gamma \vdash s : \sigma \arrtype \tau$ and $s$ does not have
  the form $\abs{x}{s'}$, then $\etalong{s} =  \abs{x}{(
  \etalong{(\app{s}{x})})}$
\item if $\Gamma \vdash s : \quant{\alpha}{\sigma}$ and $s = \tabs{
  \alpha}{s'}$, then $\etalong{s} = \tabs{\alpha}{(\etalong{s'})}$
\item if $\Gamma \vdash s : \quant{\alpha}{\sigma}$ and $s$ does not have
  the form $\tabs{\alpha}{s'}$, then $\etalong{s} = \tabs{\alpha}{
  (\etalong{(\tapp{s}{\alpha})})}$
\item if $\Gamma \vdash s : \nat$ or $\Gamma \vdash s : \alpha$ or
  $\Gamma \vdash s : \sigma \times \tau$, then $\etalong{s} =
  \almostetalong{s}$, where
  \begin{itemize}
  \item $\almostetalong{s} = s$ if $s$ is a natural number, variable or
    function symbol
  \item $\almostetalong{s} = \pair{\etalong{s_1}}{\etalong{s_2}}$ if
    $s = \pair{s_1}{s_2}$
  \item $\almostetalong{s} = \etalong{s}$ if $s$ has the form
    $\abs{x}{s'}$ or $\tabs{\alpha}{s'}$
  \item $\almostetalong{s} = \app{\almostetalong{s_1}}{(\etalong{s_2})}$
    if $s = \app{s_1}{s_2}$
  \item $\almostetalong{s} = \tapp{\almostetalong{s'}}{\tau}$
    if $s = \tapp{s'}{\tau}$
  \end{itemize}
\end{itemize}
A term $s$ is in \emph{$\eta$-long form} if $s = \etalong{s}$.
\end{definition}

It is clear that the $\eta$-long form of an interpretation term is
unique; the following lemma shows that it is also always well-defined.

\begin{lemma}
For all interpretation terms $s$, the function $\etalong{s}$ is
well-defined.
\end{lemma}

\begin{proof}
Define here the \emph{depth} function on interpretation terms as
follows:
\begin{itemize}
\item $\mathit{depth}(n) = \mathit{depth}(x) = \mathit{depth}(\mathtt{f})
  = 0$
\item $\mathit{depth}(\pair{s}{t}) = \max(\mathit{depth}(s),\mathit{
  depth}(t)) + 1$
\item $\mathit{depth}(\abs{x}{s}) = \mathit{depth}(\tabs{\alpha}{s}) =
  \mathit{depth}(s) + 1$
\item $\mathit{depth}(\app{s}{t}) = \max(\mathit{depth}(s),\mathit{
  depth}(t)+1)$
\item $\mathit{depth}(\tapp{s}{\tau}) = \mathit{depth}(s)$
\end{itemize}
(This definition is clearly well-defined as the term under consideration
becomes smaller each step.)
We can see that $\etalong{x}$ is well-defined for all variables by
induction on their type.  This we use to see that $\etalong{(\apps{s}{
x_1}{x_n})}$ is well-defined for $s$ a variable, natural number or
function symbol by induction on its type.  Thus, well-definedness of
$\etalong{s}$ for all $s$ with depth $0$ follows; also $\almostetalong{
s} = s$ is well-defined for such $s$.
For interpretation terms of depth $\geq 1$, well-definedness of
$\etalong{s}$ and $\overline{s}$ follows by induction on
$(\mathit{depth}(s)$, $1$, the size of the type of $s)$ in the case
of $\etalong{s}$ and
$(\mathit{depth}(s)$, $0$, the size of $s)$ in the case of
$\almostetalong{s}$.
\end{proof}

\begin{lemma}
If $s$ is a closed interpretation term in $\eta$-long form which is
terminating under $\leadsto$, then there is a final interpretation term
$t$ such that $s \leadsto^* t$.
\end{lemma}

\begin{proof}
TODO
\CK{And honestly not sure whether we even need this.}
\end{proof}

\LC{What do we need these $\eta$-expansions for? I think this just
  works now with the modified $\leadsto$-rules for $\oplus,\otimes$.}

\begin{lemma}
 If for all $u \in \val{\tau_1}{\omega,\xi}{\Gamma}$ we have
 $t[\subst{x}{u}] \in \val{\tau_2}{\omega,\xi}{\Gamma}$, then
 $\lambda x : \omega(\tau_1) . t \in
 \val{\tau_1\to\tau_2}{\omega,\xi}{\Gamma}$.
\end{lemma}

\begin{proof}
 By Lemma~\ref{lem_val_computable} the
 sets~$\val{\tau_i}{\omega,\xi}{\Gamma}$ are candidates, so
 $\val{\tau_i}{\omega,\xi}{\Gamma} \subseteq \SN$. Let
 $u \in \val{\tau_1}{\omega,\xi}{\Gamma}$. We show
 $(\lambda x : \omega(\tau_1) . t) u \in
 \val{\tau_2}{\omega,\xi}{\Gamma}$ by induction on $\nu(t) +
 \nu(u)$. Suppose $(\lambda x : \omega(\tau_1) . t) u \leadsto
 w$. There are three cases.
 \begin{itemize}
 \item $w = t[\subst{x}{u}]$. Then
   $w \in \val{\tau_2}{\omega,\xi}{\Gamma}$ by assumption.
 \item $w = (\lambda x : \omega(\tau_1) . t') u$ with
   $t \leadsto t'$. Because $\val{\tau_2}{\omega,\xi}{\Gamma}$ is a
   candidate, still
   $t'[\subst{x}{u}] \in \val{\tau_2}{\omega,\xi}{\Gamma}$. Also
   $\nu(t') < \nu(t)$, so $w \in \val{\tau_2}{\omega,\xi}{\Gamma}$
   follows from the inductive hypothesis.
 \item $w = (\lambda x : \omega(\tau_1) . t) u'$ with
   $u \leadsto u'$. Because $\val{\tau_i}{\omega,\xi}{\Gamma}$ are
   candidates, still $u' \in \val{\tau_1}{\omega,\xi}{\Gamma}$ and
   $t[\subst{x}{u'}] \in \val{\tau_2}{\omega,\xi}{\Gamma}$. Also
   $\nu(u') < \nu(u)$, so $w \in \val{\tau_2}{\omega,\xi}{\Gamma}$
   follows from the inductive hypothesis.
 \end{itemize}
 Because $(\lambda x : \omega(\tau_1) . t) u$ is neutral and
 $\val{\tau_2}{\omega,\xi}{\Gamma}$ is a candidate, this shows
 $(\lambda x : \omega(\tau_1) . t) u \in
 \val{\tau_2}{\omega,\xi}{\Gamma}$.
\end{proof}

\begin{lemma}
 If for every type~$\tau$ and every $X \in \Cb_\tau^\Gamma$ we have
 $t \in
 \val{\sigma}{\omega[\subst{\alpha}{\tau}],\xi[\subst{\alpha}{X}]}{\Gamma}$,
 then
 $\Lambda \alpha . t \in
 \val{\forall\alpha[\sigma]}{\omega,\xi}{\Gamma}$.
\end{lemma}

\begin{proof}
 Let~$\tau$ be a type and $X \in \Cb_\tau^\Gamma$. By
 Lemma~\ref{lem_val_computable} the
 set~$\val{\forall\alpha[\sigma]}{\omega,\xi}{\Gamma}$ is a
 candidate, so $t \in \SN$. By induction on~$\nu(t)$ we show
 $(\Lambda \alpha . t) \alpha \in
 \val{\sigma}{\omega[\subst{\alpha}{\tau}],\xi[\subst{\alpha}{X}]}{\Gamma}$.
 Because~$\val{\sigma}{\omega[\subst{\alpha}{\tau}],\xi[\subst{\alpha}{X}]}{\Gamma}$
 is a candidate by Lemma~\ref{lem_val_computable}, it suffices to
 show that for every~$w$ with $(\Lambda \alpha . t) \leadsto w$ we
 have
 $w \in
 \val{\sigma}{\omega[\subst{\alpha}{\tau}],\xi[\subst{\alpha}{X}]}{\Gamma}$. There
 are two cases.
 \begin{itemize}
 \item $w = (\Lambda \alpha . t') \alpha$. Then $\nu(t') < \nu(t)$
   and
   $w \in
   \val{\sigma}{\omega[\subst{\alpha}{\tau}],\xi[\subst{\alpha}{X}]}{\Gamma}$
   follows from the inductive hypothesis.
 \item $w = t$. Then
   $w \in
   \val{\sigma}{\omega[\subst{\alpha}{\tau}],\xi[\subst{\alpha}{X}]}{\Gamma}$
   by assumption.
 \end{itemize}
\end{proof}
