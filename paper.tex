\documentclass[runningheads,a4paper]{llncs}
\pdfoutput=1

\bibliographystyle{plainurl}

\usepackage{amssymb}
\setcounter{tocdepth}{3}
\usepackage{enumerate}
\usepackage[colorlinks=true]{hyperref}
\usepackage{tikz}
\usepackage{xcolor,latexsym,amsmath,extarrows,alltt}
\usepackage{xspace}
\usepackage{booktabs}
\usepackage{mathtools}
\usepackage{enumitem}
\usepackage{stmaryrd}
\usepackage{microtype}
\usepackage{bussproofs}
\usepackage{multirow}

\newcommand{\World}{\mathcal{W}}

\begin{document}

\mainmatter

\title{TODO list
  \thanks{The authors are supported by lots of people.}}
\subtitle{What we need to do to get where we want to be}

\author{Lukasz Czajka and Cynthia Kop}
\authorrunning{L. Czajka and C. Kop}
\institute{
Department of Computer Science, University of Copenhagen (DIKU)
\quad\quad\quad
Institute of Computer Science, Radboud University Nijmegen (RU)
\\
\email{lukaszcz@mimuw.edu.pl}
\quad\quad\quad
\email{C.Kop@cs.ru.nl}
}

\maketitle

\begin{abstract}
The introduction of this paper is a TODO list.
The remainder is meant for filling in the TODOs with what we already
have.
The hope is that the end result is something useful. :)
\end{abstract}

\section*{THE LIST}

Here's what we need to do:
\begin{enumerate}
\item Define a world $\World$ and a well-founded ordering $\succ$ on
  $\World$:
  \begin{enumerate}
  \item Define a set of terms $\World$ typed under some variation of
    System F-$\omega$.
  \item Define relations $\succ$ and $\succeq$ on the elements of $\World$.
  \item Prove that $\succ$ is a well-founded ordering relation and that
    $\succeq$ is a compatible quasi-ordering.
  \end{enumerate}
\item Specify what systems we are interested in analysing, and prove
  standard results which will make their analysis doable.
\end{enumerate}

\end{document}

