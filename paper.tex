\documentclass[runningheads,a4paper]{llncs}
\pdfoutput=1

\bibliographystyle{plainurl}

\usepackage{amssymb}
\setcounter{tocdepth}{3}
\usepackage{enumerate}
\usepackage[colorlinks=true]{hyperref}
\usepackage{tikz}
\usepackage{xcolor,latexsym,amsmath,extarrows,alltt}
\usepackage{xspace}
\usepackage{booktabs}
\usepackage{mathtools}
\usepackage{enumitem}
\usepackage{stmaryrd}
\usepackage{microtype}
\usepackage{bussproofs}
\usepackage{multirow}

\newcommand{\Iterms}{\mathcal{I}}
\newcommand{\World}{\mathcal{W}}
\newcommand{\Rules}{\mathcal{R}}
\newcommand{\Typevars}{\mathcal{A}}
\newcommand{\Vars}{\mathcal{V}}
\newcommand{\ITypes}{\mathcal{Y}}
\newcommand{\Types}{\mathcal{T}}
\newcommand{\Terms}{\mathcal{T}\!\mathit{erms}}
\newcommand{\TypeConstructors}{\mathcal{C}}
\newcommand{\Typemap}{\mathcal{T\!M}}
\newcommand{\Termmap}{\mathcal{J}}

\newcommand{\quant}[2]{\forall #1[#2]}
\newcommand{\interpret}[1]{\llparenthesis #1 \rrparenthesis}
\newcommand{\arr}[1]{\to_{#1}}
\newcommand{\arrtype}{\Rightarrow}
\newcommand{\abs}[2]{\lambda #1.#2}
\newcommand{\tabs}[2]{\Lambda #1.#2}
\newcommand{\app}[2]{#1 \cdot #2}
\newcommand{\apps}[3]{#1 \cdot #2 \cdots #3}
\newcommand{\tapp}[2]{#1 * #2}
\newcommand{\subst}[2]{#1:=#2}

\newcommand{\FTV}{\mathit{FTV}}
\newcommand{\FV}{\mathit{FV}}

\newcommand{\nat}{\mathtt{nat}}
\newcommand{\proj}{\pi}
\newcommand{\flatten}{\mathtt{flatten}}
\newcommand{\lift}{\mathtt{lift}}
\newcommand{\con}{\mathtt{c}}

\begin{document}

\mainmatter

\title{TODO list
  \thanks{The authors are supported by lots of people.}}
\subtitle{What we need to do to get where we want to be}

\author{{\L}ukasz Czajka and Cynthia Kop}
\authorrunning{{\L}. Czajka and C. Kop}
\institute{
Department of Computer Science, University of Copenhagen (DIKU)
\quad\quad\quad
Institute of Computer Science, Radboud University Nijmegen (RU)
\\
\email{lukaszcz@mimuw.edu.pl}
\quad\quad\quad
\email{C.Kop@cs.ru.nl}
}

\maketitle

\begin{abstract}
The introduction of this paper is a TODO list.
The remainder is meant for filling in the TODOs with what we already
have.
The hope is that the end result is something useful. :)
\end{abstract}

\section*{THE LIST}

\renewcommand{\theenumii}{\arabic{enumi}.\arabic{enumii}}

Here's what we need to do:
\begin{enumerate}
\item Define a world $\World$ and a well-founded ordering $\succ$ on
  $\World$:
  \begin{enumerate}
  \item Define a set of terms $\World$ typed under some variation of
    System F-$\omega$.
  \item Define relations $\succ$ and $\succeq$ on the elements of $\World$.
  \item Prove that $\succ$ is a well-founded ordering relation and that
    $\succeq$ is a compatible quasi-ordering.
  \end{enumerate}
\item Specify what systems we are interested in analysing, and prove
  standard results which will make their analysis doable.
  \begin{enumerate}
  \item Specify a form of system which includes all the systems of interest.
  \item Specify a default way of interpreting terms in these systems.
  \item Prove that in all such systems, using our way of interpreting
    terms: if $\interpret{\ell} \succ \interpret{r}$ (resp.\ $\interpret{
    \ell} \succeq \interpret{r}$) for a rule  $\ell \to r$, then
    $\interpret{s} \succ \interpret{t}$ whenever $s \arr{\Rules} t$ by
    this rule (resp.\ $\interpret{s} \succeq \interpret{t}$).
  \end{enumerate}
\item Obtain useful lemmas regarding these defaults.
  \begin{enumerate}
  \item $\interpret{s[x:=t]} = \interpret{s}[x:=\interpret{t}]$.
  \item $\interpret{s\sigma} = \interpret{s}\sigma$.
  \item \dots?
  \end{enumerate}
\item For some system of interest, prove its termination:
  \begin{enumerate}
  \item Present the system and give interpretations (following the
    default scheme) for all ways of constructing terms.
  \item Show that $\ell \succ r$ or $\ell \succeq r$ for all rules.
    Remove the rules which are oriented using $\succ$ and repeat,
    until all rules have been removed.
  \end{enumerate}
\end{enumerate}

\renewcommand{\theenumii}{\alph{enumii}}

We use \emph{rule removal}:

\begin{theorem}
Let $\Rules = \Rules_1 \cup \Rules_2$, and suppose that $\arr{\Rules_1}\:
\subseteq\:\succ$ and $\arr{\Rules_2}\:\subseteq\:\succeq$ for a
well-founded ordering $\succ$ and a compatible quasi-ordering $\succeq$.
Then $\arr{\Rules}$ is terminating if and only if $\arr{\Rules_2}$ is
(which is certainly the case if $\Rules_2 = \emptyset$).
\end{theorem}

\begin{proof}
By well-foundedness of $\succ$, every infinite decreasing $\arr{\Rules}$
sequence can only use finitely many steps using $\arr{\Rules_1}$.
\qed
\end{proof}

This gives rise to the following algorithm:
\begin{enumerate}
\item While $\Rules$ is non-empty:
  \begin{enumerate}
  \item Orient all rules in $\Rules$ using $\succeq$ or $\succ$; at least
    one of them must be oriented using $\succ$.
  \item Remove all rules ordered by $\succ$ from $\Rules$.
  \end{enumerate}
\end{enumerate}
If this algorithm succeeds, we have proven termination.

\section{Defining a world}

\subsection{Defining the set $\World$}

We define the set of types for interpretation terms.

\begin{definition}\label{def:itypes}
We assume given an infinite set $\Typevars$ of \emph{type variables}.
The set $\ITypes$ of \emph{interpretation types} is given by:
\begin{itemize}
\item $\alpha \in \ITypes$ for all $\alpha \in \Typevars$, and
  $\FTV(\alpha) = \{ \alpha \}$.
\item $\nat \in \ITypes$, and $\FTV(\nat) = \emptyset$.
\item $\sigma \arrtype \tau \in \ITypes$ if both $\sigma \in \ITypes$
  and $\tau \in \ITypes$, and $\FTV(\sigma \arrtype \tau) = \FTV(\sigma)
  \cup \FTV(\tau)$.
\item $\sigma \times \tau \in \ITypes$ if both $\sigma \in \ITypes$
  and $\tau \in \ITypes$, and $\FTV(\sigma \times \tau) = \FTV(\sigma)
  \cup \FTV(\tau)$.
\item $\quant{\alpha}{\sigma} \in \ITypes$ if $\alpha \in \Typevars$ and
  $\sigma \in \ITypes$, and $\FTV(\quant{\alpha}{\sigma}) =
  \FTV(\sigma) \setminus \{ \alpha \}$.
\end{itemize}
\end{definition}

The set $\Iterms$ of interpretation terms is now defined as follows.

\begin{definition}
We assume given an infinite set $\Vars$ of variables, and let $\Gamma$
refer to a mapping from a finite subset of $\Vars$ to the set of
interpretation types.  The set $\Iterms$ of interpretation terms consists
of all expressions $s$ such that $\Gamma \vdash s : \sigma$ can be
inferred for some interpretation type $\sigma$ and mapping $\Gamma$ by
the following clauses:
\begin{itemize}
\item $\Gamma \vdash n : \nat$ for every natural number $n$.
\item $\Gamma \vdash x : \sigma$ for every $(x : \sigma) \in \Gamma$.
\item $\Gamma \vdash \mathtt{f} : \sigma$ for all $(\mathtt{f} :
  \sigma)$ in the following set: $\{ \oplus_\sigma : \sigma \arrtype
  \sigma \arrtype \sigma,\ \otimes_\sigma : \sigma \arrtype \sigma \arrtype
  \sigma,\ \proj^1_{\sigma,\tau} : (\sigma \times \tau) \arrtype
  \sigma,\ \proj^2_{\sigma,\tau} : (\sigma \times \tau) \arrtype \tau,\ 
  \flatten_{\sigma} : \sigma \arrtype \nat,\ 
  \lift_{\sigma} : \nat \arrtype \sigma
  \mid \sigma \in \ITypes \}$.
\item $\Gamma \vdash (s,t) : \sigma \times \tau$ if $\Gamma \vdash s :
  \sigma$ and $\Gamma \vdash t : \tau$.
\item $\Gamma \vdash \abs{x:\sigma}{s} : \sigma \arrtype \tau$ if $x
  \in \Vars$ and $\Gamma \uplus \{ x : \sigma \} \vdash s : \tau$.
\item $\Gamma \vdash \tabs{\alpha}{s} : \quant{\alpha}{\sigma}$ if
  $\alpha \in \Typevars$ and $\Gamma \vdash s : \sigma$ and for all
  $(x : \tau) \in \Gamma$: $\alpha \notin \FTV(\tau)$
\item $\Gamma \vdash \app{s}{t} : \tau$ if $\Gamma \vdash s :
  \sigma \arrtype \tau$ and $\Gamma \vdash t : \sigma$
\item $\Gamma \vdash \tapp{s}{\tau} : \sigma[\subst{\alpha}{\tau}]$ if
  $\Gamma \vdash s : \quant{\alpha}{\sigma}$\footnote{TODO: a
  definition of type substitution}
\end{itemize}
We say that $s$ is \emph{closed} if $\emptyset \vdash s : \sigma$.
\end{definition}

Note that for a given $\Gamma$, if $s$ is typable under $\Gamma$, then
there is only one choice for the type (this is easily proved by
induction on the form of $s$).  Thus, all closed terms have a unique
type.

\medskip
To define the world of \emph{final interpretation terms}, we will
normalise certain elements of $\Iterms$ using the relation $\leadsto$:

\begin{definition}
Let $s,t$ be closed interpretation terms of the same type $\sigma$ such
that $\FTV(\sigma) = \emptyset$.  We say $s \leadsto t$ if we can denote
$s = C[\ell]$ and $t = C[r]$\footnote{TODO: a definition of contexts.}
for some rule scheme $\ell \leadsto r$ in the following list:
\begin{itemize}
\item $\app{\app{\oplus_{\nat}}{n}}{m} \leadsto (n+m)$ 
\item $\app{\app{\otimes_{\nat}}{n}}{m} \leadsto (n \cdot m)$ 
\item $\app{\app{\circ_{\sigma \arrtype \tau}}{(\abs{x:\sigma}{s})}}{
  (\abs{x:\sigma}{t})} \leadsto \abs{x:\sigma}{\app{\app{\circ_\tau}{
  s}}{t}}$ for $\circ \in \{ \oplus, \otimes \}$
\item $\app{\app{\circ_{\quant{\alpha}{\sigma}}}{(\tabs{\alpha}{s})}}{
  (\tabs{\alpha}{t})} \leadsto \tabs{\alpha}{\app{\app{\circ}{s}}{t}}$
  for $\circ \in \{ \oplus, \otimes \}$
\item $\app{\proj^1_{\sigma,\tau}}{(s,t)} \leadsto s$
\item $\app{\proj^2_{\sigma,\tau}}{(s,t)} \leadsto t$
\item $\app{\flatten_\nat}{s} \leadsto s$
\item $\app{\flatten_{\sigma \arrtype \tau}}{s} \leadsto
  \app{s}{(\app{\lift_\sigma}{0})}$
\item $\app{\flatten_{\quant{\alpha}{\sigma}}}{s} \leadsto
  \app{\flatten_{\sigma[\subst{\alpha}{\nat}]}}{(\tapp{s}{\nat})}$
\item $\app{\lift_\nat}{s} \leadsto s$
\item $\app{\lift_{\sigma \arrtype \tau}}{s} \leadsto
  \abs{x:\sigma}{\app{\lift_{\tau}}{x}}$
\item $\app{\lift_{\quant{\alpha}{\sigma}}}{s} \leadsto
  \tabs{\alpha}{\app{\lift_{\sigma}}{s}}$
\item $\app{(\abs{x:\sigma}{s})}{t} \leadsto s[\subst{x}{t}]$
\item $\tapp{(\tabs{\alpha}{s})}{\sigma} \leadsto
  s[\subst{\alpha}{\sigma}]$.
\end{itemize}
A \emph{final interpretation term} is a term $s \in \Iterms$ such that
$\FV(s) = \FTV(s) = \emptyset$ and $s$ is in normal form with respect
to $\leadsto$.  We let $\World$ be the set of all final interpretation
terms.
\end{definition}

\begin{lemma}
Every term $s \in \Iterms$ with $\FV(s) = \FTV(s) = \emptyset$ reduces
to a unique final interpretation term $s\downarrow$.
\end{lemma}

\begin{proof}
By proving termination and local confluence of $\leadsto$ (although it
actually suffices to only prove local confluence for fully closed terms).

TODO.
\qed
\end{proof}

\begin{lemma}
The only final interpretation terms of type $\nat$ are the natural
numbers.
\end{lemma}

\begin{proof}
TODO.
\qed
\end{proof}

\subsection{Defining $\succ$ and $\succeq$}

As an aide, we will need \emph{valuations}:

\begin{definition}
A valuation for $\Gamma$ is a substitution which, for each $(x:\sigma)
\in \Gamma$, maps $x$ to a final interpretation term of type $\sigma$.
\end{definition}

We will define $\succ$ and $\succeq$ on elements of $\World$, using a
recursive procedure: $s \succ t$ (resp.\ $s \succeq t$) if $[] \Vdash
s \succ_\sigma t$ (resp.\ $[] \Vdash s \succeq_\sigma t$) can be
obtained for $\sigma$ the type of $s,t$, following
Definition~\ref{def:succ}.

\begin{definition}\label{def:succ}
Let $\Gamma$ be an environment\footnote{TODO: introduce the word
environment sometime before this point.}, $\gamma$ a valuation for $\Gamma$ and
$s,t$ interpretation terms such that (1) $s$ and $t$ have the same type
under $\Gamma$ (so $\Gamma \vdash s : \sigma$ and $\Gamma \vdash t :
\sigma$ for the same type $\sigma$), and (2) $\abs{x_1:\sigma_1\dots
x_n:\sigma_n}{s}$ and $\abs{x_1:\sigma_1\dots x_n:\sigma_n}{t}$ are final
interpretation terms if $\Gamma = \{ (x_1:\sigma_1),\dots,(x_n:\sigma_n)
\}$.  For $R \in \{ \succ,\succeq \}$, we say $\gamma \Vdash s\ R_{
\sigma}\ t$ if this can be obtained using the following recursive clauses:
\begin{itemize}
\item $\gamma \Vdash s\ R_\nat\ t$ if $s\gamma\downarrow\ R\ t\gamma
  \downarrow$ in the natural numbers
\item $\gamma \Vdash s\ R_{\sigma \arrtype \tau} t$ if:
  \begin{itemize}
  \item $s = \abs{x:\sigma}{s'}$ and $t = \abs{x:\sigma}{t'}$ and
    $\gamma \cup [x:=u] \Vdash s'\ R_{\tau}\ t'$ for all final
    interpretation terms~$u$ of type $\sigma$
  \item TODO: cases for $s$ or $t$ having a form $\apps{x}{s_1}{s_n}$
    with $x \in \Vars$ and $n \geq 0$
  \end{itemize}
\item $\gamma \Vdash s\ R_{\quant{\alpha}{\sigma}}\ t$ if:
  \begin{itemize}
  \item $s = \tabs{\alpha}{s'}$ and $t = \tabs{\alpha}{t'}$ and
    $s'[\subst{\alpha}{\tau}]\ R_{\sigma[\subst{\alpha}{\tau}]}\ t'[\subst{\alpha}{\tau}]$ for all types $\tau$
  \item TODO: perhaps cases for $s$ or $t$ having a variable at the
    head as well
  \end{itemize}
\end{itemize}
\end{definition}

\subsection{Properties of $\succ$ and $\succeq$}

We're going to at least need to see that $\succ$ is a well-founded ordering:

\begin{lemma}
$\succ$ is well-founded.
\end{lemma}

\begin{proof}
TODO.
\qed
\end{proof}

\begin{lemma}
Both $\succ$ and $\succeq$ are transitive.
\end{lemma}

\begin{proof}
TODO.
\qed
\end{proof}

And that $\succeq$ is a quasi-ordering:

\begin{lemma}
$\succeq$ is reflexive.
\end{lemma}

\begin{proof}
TODO.
\qed
\end{proof}

Finally, they must be compatible:

\begin{lemma}
We have $\succ \cdot \succeq\ \subseteq\ \succ$
\end{lemma}

\begin{proof}
TODO.  Note that proving $\succeq \cdot \succ\ \subseteq\ \succ$ is
just as useful, if that so happens to be easier (but I imagine it does
not make the slightest difference, given the definition).
\qed
\end{proof}

I think that's all that absolutely \emph{has} to be proven to make the
argument work.  However, it would be great if we could also for
instance obtain that always $\app{\app{\oplus}{s}}{t} \succeq s$ and
that $\app{\app{\oplus}{s}}{(\app{\lift}{1})} \succ s$.

\section{Systems of interest and their properties}

The systems of interest are sets of terms with a rewriting relation on
them.  Interestingly, $\Iterms$ can be seen of an instance of this general
scheme.

\subsection{Systems of interest}

Types are built from a set of type constructors, using $\arrtype$ and
$\forall$.

\begin{definition}
We assume given a fixed set $\TypeConstructors$ of \emph{type
constructors}, each paired with an integer \emph{arity} $\geq 0$, as
well as the infinite set $\Typevars$ of type variables.  The set
$\Types$ of types is given by:
\begin{itemize}
\item $\alpha \in \Types$ for all $\alpha \in \Typevars$
\item $\con(\sigma_1,\dots,\sigma_n) \in \Types$ if $(\con,n) \in
  \TypeConstructors$ and $\sigma_1,\dots,\sigma_n \in \Types$
\item $\sigma \arrtype \tau \in \Types$ if $\sigma,\tau \in \Types$
\item $\quant{\alpha}{\sigma} \in \Types$ if $\alpha \in \Typevars$ and
  $\sigma \in \Types$
\end{itemize}
We let $\FTV(\sigma)$ be defined in the obvious way (much like
Definition~\ref{def:itypes}).
\end{definition}

Terms are built from a set of function symbols, using (type) abstraction
and (type) application as for interpretation terms.

\begin{definition}
We assume given a fixed set $\Sigma$ of \emph{function symbols}, each
paired with an arity $n$ and a type $\sigma_1 \arrtype \dots \arrtype
\sigma_n \arrtype \tau$, as well as the infinite set $\Vars$ of variables.
The set $\Terms$ of terms consists of those expressions $s$ such that
$\Gamma \vdash s : \sigma$ can be derived for some type $\sigma$ and
environment $\Gamma$ using the following clauses:
\begin{itemize}
\item $\Gamma \vdash x : \sigma$ for every $(x : \sigma) \in \Gamma$.
\item $\Gamma \vdash \mathtt{f}(s_1,\dots,s_n) : \tau$ if
  $(\mathtt{f} : (n,\sigma_1 \arrtype \dots \arrtype \sigma_n \arrtype
  \tau))$ in $\Sigma$ and $\Gamma \vdash s_i : \sigma_i$ for $1 \leq
  i \leq n$
\item $\Gamma \vdash \abs{x:\sigma}{s} : \sigma \arrtype \tau$ if $x
  \in \Vars$ and $\Gamma \uplus \{ x : \sigma \} \vdash s : \tau$.
\item $\Gamma \vdash \tabs{\alpha}{s} : \quant{\alpha}{\sigma}$ if
  $\alpha \in \Typevars$ and $\Gamma \vdash s : \sigma$ and for all
  $(x : \tau) \in \Gamma$: $\alpha \notin \FTV(\tau)$
\item $\Gamma \vdash \tapp{s}{\tau} : \sigma[\subst{\alpha}{\tau}]$ if
  $\Gamma \vdash s : \quant{\alpha}{\sigma}$
\end{itemize}
\end{definition}

Note that there is no explicit application; systems which include a form
of application should be modelled simply using a set of symbols $\{
@_{\sigma,\tau}: (2,(\sigma \arrtype \tau) \arrtype \sigma \arrtype \tau)
\mid \sigma,\tau \in \Types \} \subseteq \Sigma$.

TODO: are there interesting systems where the type application $*$ is
not the only way for types to appear in terms?  If so, we might as well
not have the exceptional case, easier to go general straight away.

LC: yes, e.g., when you have explicit existential quantification over
types, like in urzy\_emb.

I'm thinking that maybe here it will be easier to just go with
system $F_\omega$ from the beginning instead of system~$F$, because
there you have in-built type constructors.

TODO: describe what rules look like.  Are the cases where substitution
/ type substitution is used limited to a few special cases (in which
case we can just add those cases as special) or is it very typical?

LC: (type) substitution occurs with more rules, in e.g. urzy\_emb

Should we describe rules using variables which are instantiated, or
simply assume that typical systems have infinitely many rules (e.g.,
all instances of $\beta$-reductions are a separate rule)? We're
already using infinite signatures -- should we head straight for
polymorphism?

\subsubsection{Some example systems.}

TODO: describe here how system F and urzy\_emb can be seen as instances
of this general case.

\subsection{Interpreting interesting terms}

To start, all types must be mapped to interesting types.  This is
easily done:

\begin{definition}
A \emph{type constructor mapping} is a function $\Typemap$ which assigns
to each $(\con:n) \in \TypeConstructors$ a type $\sigma$ with
$\FTV(\sigma) \subseteq \{ \alpha_1,\dots,\alpha_n \}$.  A type
constructor mapping is extended to a function operating on types as
follows:
\[
\begin{array}{rcl}
\Typemap(\alpha) & = & \alpha \\
\Typemap(\con(\sigma_1,\dots,\sigma_n)) & = &
  \tau[\alpha_1:=\Typemap(\sigma_1),\dots,\alpha_n:=\Typemap(\sigma_n)]\ 
  \text{if}\ \Typemap(\con) = \tau \\
\Typemap(\sigma \arrtype \tau) & = & \Typemap(\sigma) \arrtype
  \Typemap(\tau) \\
\Typemap(\quant{\alpha}{\sigma}) & = & \quant{\alpha}{\Typemap(\sigma)} \\
\end{array}
\]
\end{definition}

Clearly, a type constructor mapping maps types to interpretation types,
and $\FTV(\Typemap(\sigma)) \subseteq \FTV(\sigma)$.

Terms are mapped to interpretation terms in a very similar way:

\begin{definition}
A \emph{symbol mapping} is a function $\Termmap$ which assigns to each
$(\mathtt{f}:(n,\sigma)) \in \Sigma$ a final interpretation term
$\Termmap(\mathtt{f})$ of the form $\abs{x_1:\sigma_1 \dots x_n:
\sigma_n}{t}$, such that $\emptyset \vdash \Termmap(\mathtt{f}) :
\Typemap(\sigma)$.  Considering $\Termmap$ fixed, we define the
\emph{interpretation} of each term $s$, $\interpret{s}$, as follows:
\[
\begin{array}{rcl}
\interpret{x} & = & x \\
\interpret{\mathtt{f}(s_1,\dots,s_n)} & = &
  t[x_1:=\interpret{s_1},\dots,x_n:=\interpret{s_n}]\ \text{if}\ 
  \Termmap(\mathtt{f}) = \abs{x_1:\sigma_1 \dots x_n:\sigma_n}{t} \\
\interpret{\abs{x:\sigma}{s}} & = & \abs{x:\Termmap(\sigma)}{
  \interpret{s}} \\
\interpret{\tabs{\alpha}{s}} & = & \tabs{\alpha}{\interpret{s}} \\
\interpret{\tapp{s}{\tau}} & = & \tapp{\interpret{s}}{\Termmap(\tau)} \\
\end{array}
\]
\end{definition}

TODO: prove that this indeed maps to well-typed interpretation terms of
the expected type.

TODO: observe that this choice does not really allow us to order the
type instantiation rule with $\succ$, just with $\succeq$.  However, this
isn't actually a problem since, if all other rules can be removed, we have
termination anyway.  However, we could allow for a custom interpretation
there as well, or consider the $*$ merely one case of a more general kind
of function symbol.

\subsection{Monotonicity}

This relies on the definition of rules (a form of stability may also be
required), but we really want to have that if $\interpret{\ell} \succ
\interpret{r}$ for some rule, then $\interpret{s} \succ \interpret{t}$
whenever $s \arr{\Rules} t$ by that rule.

\section{Some useful lemmas}

\subsection{Orienting $\beta$-reduction}

\subsection{Orienting type instantiation}

\subsection{Other}

\section{Some system of interest -- urzy\_emb?}

\subsection{Interpretations}

\subsection{Rule orientation}

\end{document}

