\documentclass[runningheads,a4paper]{llncs}
\pdfoutput=1

\bibliographystyle{plainurl}

\usepackage{amssymb}
\setcounter{tocdepth}{3}
\usepackage{enumerate}
\usepackage[colorlinks=true]{hyperref}
\usepackage{tikz}
\usepackage{xcolor,latexsym,amsmath,extarrows,alltt}
\usepackage{xspace}
\usepackage{booktabs}
\usepackage{mathtools}
\usepackage{enumitem}
\usepackage{stmaryrd}
\usepackage{microtype}
\usepackage{bussproofs}
\usepackage{multirow}
\usepackage{proof}
\usepackage[T1]{fontenc}

\newcommand{\Iterms}{\mathcal{I}}
\newcommand{\World}{\mathcal{W}}
\newcommand{\Rules}{\mathcal{R}}
\newcommand{\Typevars}{\mathcal{A}}
\newcommand{\Vars}{\mathcal{V}}
\newcommand{\ITypes}{\mathcal{Y}}
\newcommand{\Types}{\mathcal{T}}
\newcommand{\Terms}{\mathcal{T}\!\mathit{erms}}
\newcommand{\TypeConstructors}{\mathcal{C}}
\newcommand{\TypeQuantifiers}{\mathcal{Q}}
\newcommand{\Typemap}{\mathcal{T\!M}}
\newcommand{\Termmap}{\mathcal{J}}

\newcommand{\quant}[2]{\forall #1[#2]}
\newcommand{\qquant}[3]{#1 #2[#3]}
\newcommand{\typeinterpret}[1]{\text{\guillemotleft} #1\! \text{\guillemotright}}
\newcommand{\interpret}[1]{\llparenthesis #1 \rrparenthesis}
\newcommand{\fullinterpret}[1]{\Phi(#1)}
\newcommand{\arr}[1]{\to_{#1}}
\newcommand{\arrtype}{\Rightarrow}
\newcommand{\abs}[2]{\lambda #1.#2}
\newcommand{\tabs}[2]{\Lambda #1.#2}
\newcommand{\abstraction}[2]{\backslash #1.#2}
\newcommand{\app}[2]{#1 \cdot #2}
\newcommand{\apps}[3]{#1 \cdot #2 \cdots #3}
\newcommand{\tapp}[2]{#1 * #2}
\newcommand{\pair}[2]{\langle #1,#2 \rangle}
\newcommand{\subst}[2]{#1:=#2}
\newcommand{\meta}[2]{#1\langle#2\rangle}

\newcommand{\FTV}{\mathit{FTV}}
\newcommand{\FV}{\mathit{FV}}

\newcommand{\nat}{\mathtt{nat}}
\newcommand{\proj}{\pi}
\newcommand{\flatten}{\mathtt{flatten}}
\newcommand{\lift}{\mathtt{lift}}
\newcommand{\con}{\mathtt{c}}

\newcommand{\ur}{\upharpoonright}
\newcommand{\da}{\downarrow}
\newcommand{\SN}{\mathrm{SN}}
\newcommand{\Cb}{\mathbb{C}}
\newcommand{\Nbb}{\mathbb{N}}
\newcommand{\val}[3]{\ensuremath{\llbracket#1\rrbracket_{#2}^{#3}}}
\newcommand{\proves}{\vdash}

\newcommand{\CK}[1]{\textcolor{blue}{CK: #1}}
\newcommand{\LC}[1]{\textcolor{purple}{LC: #1}}

\begin{document}

\mainmatter

\title{TODO list
  \thanks{The authors are supported by lots of people.}}
\subtitle{What we need to do to get where we want to be}

\author{{\L}ukasz Czajka and Cynthia Kop}
\authorrunning{{\L}. Czajka and C. Kop}
\institute{
Department of Computer Science, University of Copenhagen (DIKU)
\\
Institute of Computer Science, Radboud University Nijmegen (RU)
\\
\email{lukaszcz@mimuw.edu.pl}
\quad\quad\quad
\email{C.Kop@cs.ru.nl}
}

\maketitle

\begin{abstract}
The introduction of this paper is a TODO list.
The remainder is meant for filling in the TODOs with what we already have.
The hope is that the end result is something useful. :)
\end{abstract}

\section*{THE LIST}

\renewcommand{\theenumii}{\arabic{enumi}.\arabic{enumii}}

Here's what we need to do:
\begin{enumerate}
\item Define a world $\World$ and a well-founded ordering $\succ$ on
  $\World$:
  \begin{enumerate}
  \item Define a set of terms $\World$ typed under some variation of
    System F-$\omega$.
  \item Define relations $\succ$ and $\succeq$ on the elements of $\World$.
  \item Prove that $\succ$ is a well-founded ordering relation and that
    $\succeq$ is a compatible quasi-ordering.
  \end{enumerate}
\item Specify what systems we are interested in analysing, and prove
  standard results which will make their analysis doable.
  \begin{enumerate}
  \item Specify a form of system which includes all the systems of interest.
  \item Specify a default way of interpreting terms in these systems.
  \item Prove that in all such systems, using our way of interpreting
    terms: if $\interpret{\ell} \succ \interpret{r}$ (resp.\ $\interpret{
    \ell} \succeq \interpret{r}$) for a rule  $\ell \to r$, then
    $\interpret{s} \succ \interpret{t}$ whenever $s \arr{\Rules} t$ by
    this rule (resp.\ $\interpret{s} \succeq \interpret{t}$).
  \end{enumerate}
\item Obtain useful lemmas regarding these defaults.
  \begin{enumerate}
  \item $\interpret{s[x:=t]} = \interpret{s}[x:=\interpret{t}]$.
  \item $\interpret{s\sigma} = \interpret{s}\sigma$.
  \item \dots?
  \end{enumerate}
\item For some system of interest, prove its termination:
  \begin{enumerate}
  \item Present the system and give interpretations (following the
    default scheme) for all ways of constructing terms.
  \item Show that $\ell \succ r$ or $\ell \succeq r$ for all rules.
    Remove the rules which are oriented using $\succ$ and repeat,
    until all rules have been removed.
  \end{enumerate}
\end{enumerate}

\renewcommand{\theenumii}{\alph{enumii}}

We use \emph{rule removal}:

\begin{theorem}\label{thm:ruleremove}
Let $\Rules = \Rules_1 \cup \Rules_2$, and suppose that $\arr{\Rules_1}\:
\subseteq\:\succ$ and $\arr{\Rules_2}\:\subseteq\:\succeq$ for a
well-founded ordering $\succ$ and a compatible quasi-ordering $\succeq$.
Then $\arr{\Rules}$ is terminating if and only if $\arr{\Rules_2}$ is
(which is certainly the case if $\Rules_2 = \emptyset$).
\end{theorem}

\begin{proof}
By well-foundedness of $\succ$, every infinite decreasing $\arr{\Rules}$
sequence can only use finitely many steps using $\arr{\Rules_1}$.
\qed
\end{proof}

This gives rise to the following algorithm:
\begin{enumerate}
\item While $\Rules$ is non-empty:
  \begin{enumerate}
  \item Orient all rules in $\Rules$ using $\succeq$ or $\succ$; at least
    one of them must be oriented using $\succ$.
  \item Remove all rules ordered by $\succ$ from $\Rules$.
  \end{enumerate}
\end{enumerate}
If this algorithm succeeds, we have proven termination.

\section{Defining a world}

\subsection{Defining the set $\World$}

\subsubsection{Interpretation terms}
We define the set of types for interpretation terms.

\begin{definition}\label{def:itypes}
We assume given an infinite set $\Typevars$ of \emph{type variables}.
The set $\ITypes$ of \emph{interpretation types} is given by:
\begin{itemize}
\item $\alpha \in \ITypes$ for all $\alpha \in \Typevars$, and
  $\FTV(\alpha) = \{ \alpha \}$.
\item $\nat \in \ITypes$, and $\FTV(\nat) = \emptyset$.
\item $\sigma \arrtype \tau \in \ITypes$ if both $\sigma \in \ITypes$
  and $\tau \in \ITypes$, and $\FTV(\sigma \arrtype \tau) = \FTV(\sigma)
  \cup \FTV(\tau)$.
\item $\sigma \times \tau \in \ITypes$ if both $\sigma \in \ITypes$
  and $\tau \in \ITypes$, and $\FTV(\sigma \times \tau) = \FTV(\sigma)
  \cup \FTV(\tau)$.
\item $\quant{\alpha}{\sigma} \in \ITypes$ if $\alpha \in \Typevars$ and
  $\sigma \in \ITypes$, and $\FTV(\quant{\alpha}{\sigma}) =
  \FTV(\sigma) \setminus \{ \alpha \}$.
\end{itemize}
\end{definition}

\begin{definition}\label{def:typesubst}
A \emph{type substitution} is a partial function $[\alpha_1:=\sigma_1,
\dots,\alpha_n:=\sigma_n]$ mapping a finite set of type variables to
types.  We let $\tau[\alpha_1:=\sigma_1,\dots,\alpha_n:=\sigma_n]$
denote $\tau$ with all occurrences of some $\alpha_i$ replaced by the
corresponding $\sigma_i$.  We use alpha-conversion to guarantee that
substitution does not capture variables in any of the $\sigma_i$.
\end{definition}

The set $\Iterms$ of interpretation terms is now defined as follows.

\begin{definition}\label{def_typing}
We assume given an infinite set $\Vars$ of variables, and let $\Gamma$
refer to a mapping from a finite subset of $\Vars$ to the set of
interpretation types.  The set $\Iterms$ of interpretation terms consists
of all expressions $s$ such that $\Gamma \vdash s : \sigma$ can be
inferred for some interpretation type $\sigma$ and mapping $\Gamma$ by
the following clauses:
\begin{itemize}
\item $\Gamma \vdash n : \nat$ for every natural number $n$.
\item $\Gamma \vdash x : \sigma$ for every $(x : \sigma) \in \Gamma$.
\item $\Gamma \vdash \mathtt{f} : \sigma$ for all $(\mathtt{f} :
  \sigma)$ in the following set: $\{ \oplus_\sigma : \sigma \arrtype
  \sigma \arrtype \sigma,\ \otimes_\sigma : \sigma \arrtype \sigma \arrtype
  \sigma,\ \proj^1_{\sigma,\tau} : (\sigma \times \tau) \arrtype
  \sigma,\ \proj^2_{\sigma,\tau} : (\sigma \times \tau) \arrtype \tau,\ 
  \flatten_{\sigma} : \sigma \arrtype \nat,\ 
  \lift_{\sigma} : \nat \arrtype \sigma
  \mid \sigma \in \ITypes \}$.
\item $\Gamma \vdash \pair{s}{t} : \sigma \times \tau$ if $\Gamma \vdash
  s : \sigma$ and $\Gamma \vdash t : \tau$.
\item $\Gamma \vdash \abs{x:\sigma}{s} : \sigma \arrtype \tau$ if $x
  \in \Vars$ and $\Gamma \uplus \{ x : \sigma \} \vdash s : \tau$.
\item $\Gamma \vdash \tabs{\alpha}{s} : \quant{\alpha}{\sigma}$ if
  $\alpha \in \Typevars$ and $\Gamma \vdash s : \sigma$ and for all
  $(x : \tau) \in \Gamma$: $\alpha \notin \FTV(\tau)$
\item $\Gamma \vdash \app{s}{t} : \tau$ if $\Gamma \vdash s :
  \sigma \arrtype \tau$ and $\Gamma \vdash t : \sigma$
\item $\Gamma \vdash \tapp{s}{\tau} : \sigma[\subst{\alpha}{\tau}]$ if
  $\Gamma \vdash s : \quant{\alpha}{\sigma}$
\end{itemize}
We say that $s$ is \emph{closed} if $\emptyset \vdash s : \sigma$.
\end{definition}

Note that for a given $\Gamma$, if $s$ is typable under $\Gamma$, then
there is only one choice for the type (this is easily proved by
induction on the form of $s$). Thus, all closed terms have a unique
type.

\subsubsection{Normalising interpretation terms}
Term equality is considered modulo $\alpha$-conversion. To define the
world of \emph{final interpretation terms}, we will normalise certain
elements of $\Iterms$ using the relation $\leadsto$.

\begin{definition}
We define the relation $\leadsto$ on interpretation terms as the
smallest relation for which the following properties are satisfied:
\begin{enumerate}
\item\label{leadsto:mono:abs}
  if $s \leadsto t$ then both $\abs{x}{s} \leadsto \abs{x}{t}$ and
  $\tabs{\alpha}{s} \leadsto \tabs{\alpha}{t}$
\item\label{leadsto:mono:right}
  if $s \leadsto t$ then $\app{u}{s} \leadsto \app{u}{t}$
\item\label{leadsto:mono:left}
  if $s \leadsto t$ then both $\app{s}{u} \leadsto \app{t}{u}$ and
  $\tapp{s}{\sigma} \leadsto \tapp{t}{\sigma}$
\item\label{leadsto:plus:base}
  $\app{\app{\oplus_{\nat}}{n}}{m} \leadsto (n+m)$ 
\item\label{leadsto:times:base}
  $\app{\app{\otimes_{\nat}}{n}}{m} \leadsto (n \cdot m)$ 
\item\label{leadsto:circ:arrow}
  $\app{\app{\circ_{\sigma \arrtype \tau}}{s}}{t} \leadsto \abs{x:
  \sigma}{\app{\app{\circ_\tau}{(\app{s}{x})}}{(\app{t}{x})}}$ for
  $\circ \in \{ \oplus, \otimes \}$
\item\label{leadsto:circ:forall}
  $\app{\app{\circ_{\quant{\alpha}{\sigma}}}{s}}{t} \leadsto
  \tabs{\alpha}{\app{\app{\circ_\sigma}{(\tapp{s}{\alpha})}}{(
  \tapp{t}{\alpha})}}$ for $\circ \in \{ \oplus, \otimes \}$
\item $\app{\proj^1_{\sigma,\tau}}{(s,t)} \leadsto s$
\item $\app{\proj^2_{\sigma,\tau}}{(s,t)} \leadsto t$
\item $\app{\flatten_\nat}{s} \leadsto s$
\item $\app{\flatten_{\sigma \arrtype \tau}}{s} \leadsto
  \app{s}{(\app{\lift_\sigma}{0})}$
\item $\app{\flatten_{\quant{\alpha}{\sigma}}}{s} \leadsto
  \app{\flatten_{\sigma[\subst{\alpha}{\nat}]}}{(\tapp{s}{\nat})}$
\item $\app{\lift_\nat}{s} \leadsto s$
\item $\app{\lift_{\sigma \arrtype \tau}}{s} \leadsto
  \abs{x:\sigma}{\app{\lift_{\tau}}{x}}$
\item $\app{\lift_{\quant{\alpha}{\sigma}}}{s} \leadsto
  \tabs{\alpha}{\app{\lift_{\sigma}}{s}}$
\item\label{leadsto:beta:abs}
  $\app{(\abs{x:\sigma}{s})}{t} \leadsto s[\subst{x}{t}]$
\item\label{leadsto:beta:tabs}
  $\tapp{(\tabs{\alpha}{s})}{\sigma} \leadsto
  s[\subst{\alpha}{\sigma}]$.
\end{enumerate}
We say that $s$ is a \emph{redex} if $s$ reduces by one of the rules
(\ref{leadsto:plus:base})--(\ref{leadsto:beta:tabs}), and that $s$
\emph{reduces at the head} to $t$ if the derivation of $s \leadsto t$
does not use (\ref{leadsto:mono:abs}) or (\ref{leadsto:mono:right}).

A \emph{final interpretation term} is a term $s \in \Iterms$ such that
(a) $\FV(s) = \FTV(s) = \emptyset$, and (b) $s$ is in normal form with
respect to $\leadsto$.  We let $\World$ be the set of all final
interpretation terms. By~$\World_\tau$ we denote the set of all final
interpretation terms of type~$\tau$.
\end{definition}

NOTE: in the remainder of this section, we shall often speak simply of
``terms'' when referring to interpretation terms.

\subsubsection{Key properties of $\leadsto$}
In the remainder of this section, we will often abuse notation to omit
$\cdot$ and $*$.  Thus, $s t$ can refer to both $\app{s}{t}$ and
$\tapp{s}{t}$.  Due to typing, this notation is not ambiguous.  We will
also denote $\abstraction{a}{s}$ for either $\abs{a}{s}$ or
$\tabs{a}{s}$, depending on typing. Thus, we have:

\begin{lemma}\label{lem_abusive_notation}
Every interpretation term has the form $s t_1 \dots t_n$ with
$s$ a variable, function symbol or abstraction $\abstraction{a}{s'}$
(for $n \geq 0$).
\end{lemma}

\begin{proof}
By induction on the size of interpretation terms (and a simple case
analysis).
\qed
\end{proof}

\begin{lemma}[Subject reduction]
  If $\Gamma \vdash t : \tau$ and $t \leadsto t'$ then
  $\Gamma \vdash t' : \tau$.
\end{lemma}

\begin{proof}
  TODO
\end{proof}

By~$\SN$ we denote the set of all terminating interpretation
terms. For $t \in \SN$ by~$\nu(t)$ we denote the length of the longest
reduction starting at~$t$. We use the notation $t u$ for either
$\app{t}{u}$ when~$u$ is a term, or $\tapp{t}{u}$ when~$u$ is a type.

The following lemma is obvious, but worth stating explicitly.

\begin{lemma}\label{lem_reduce_abs}
If $\abstraction{a}{s} \leadsto^* t$, then $t = \abstraction{a}{
t}$ and $s \leadsto^* t'$.
If $s \in \SN$ then both $\abs{x}{s}$ and $\tabs{\alpha}{s}$ are also
in $\SN$.
\end{lemma}

\begin{proof}
We observe that every reduct of $\abs{x}{s}$ has the form $\abs{x}{s'}$
with $s \leadsto s'$, and similar for $\tabs{\alpha}{s}$.
Thus, the first statement follows by induction on the length of the
reduction $\abstraction{a}{s} \leadsto^* t$, and the second by induction
on $s$ using $\leadsto$.
\qed
\end{proof}

\begin{lemma}\label{lem_circ_sn_base}
  If $t_1,t_2 \in \SN$ then $\circ_\nat t_1 t_2 \in \SN$ for
  $\circ \in \{\oplus,\otimes\}$.
  \CK{Changed this because I don't think it'll work out to do this for
  higher types without assuming a kind of computability of $t_1,t_2$ --
  not with the changed definition of the $\circ$ rules (it could be
  that $t_1$ is terminating but $t_1 x$ is not).} \LC{Right, with the
  previous rules this was a relatively simple induction on~$\tau$, now
  it won't work.}
\end{lemma}

\begin{proof}
  By induction on $t_1,t_2$ oriented with $\leadsto$.
  $\circ_\nat t_1 t_2 \in \SN$ if $s \in \SN$ for all $s$ such that
  $\circ_\nat t_1 t_2 \in \SN \leadsto s$.  If
  $s = \circ_\nat t_1' t_2$ or $s = \circ_\nat t_1 t_2'$ then we
  complete by the induction hypothesis. Otherwise $s \in \mathbb{N}$
  is obviously in $\SN$.
\end{proof}

\subsubsection{Computability}
In the rest of this section we adapt the Tait-Girard computability
method to prove termination of~$\leadsto$. The proof is an adaptation
of chapters~6 and~14 from the book ``Proofs and Types'' by Girard, and
chapters~10 and~11 from the book ``Lectures on the Curry-Howard
Isomorphism'' by Sorensen and Urzyczyn.

NOTE: For now the proof does not take product types into account. I'm
not sure if explicit products are necessary in the definition, because
in contrast to the simply-typed lambda-calculus they may be encoded in
the polymorphic lambda-calculus.

NOTE: Currently, $\flatten$ and $\lift$ are not accounted for either,
but adding them should be similar to~$\oplus$ and~$\otimes$.

\begin{definition}\label{def_candidate}
  A term~$t$ is \emph{neutral} if there does not exists a sequence of
  terms or types~$u_1,\ldots,u_n$ with $n \ge 1$ such that
  $t u_1 \ldots u_n$ is a redex (by~$\leadsto$).

  A set~$X$ of interpretation terms of type~$\tau$ in context~$\Gamma$
  is a \emph{candidate of type~$\tau$} when:
  \begin{enumerate}
  \item $X \subseteq \SN$;
  \item if $t \in X$ and $t \leadsto t'$ then $t' \in X$;
  \item if $t$ is neutral and for every~$t'$ with $t \leadsto t'$ we
    have $t' \in X$, then $t \in X$;
  \item if $t_1,t_2 \in X$ then $\circ_\tau t_1 t_2 \in X$ for $\circ
    \in \{\oplus,\otimes\}$;
  \item conditions analogous to the previous one for $\flatten$ and
    $\lift$ \ldots
  \end{enumerate}
  Note that item~3 above implies:
  \begin{itemize}
  \item if $t$ is neutral and in normal form then $t \in X$.
  \end{itemize}
  The set of all candidates of type~$\tau$ in context~$\Gamma$ is
  denoted by~$\Cb_\tau^\Gamma$.
\end{definition}

\begin{definition}\label{def_reducibility_valuation}
  Let $\omega$ be a mapping from type variables to types. A
  \emph{$\Gamma,\omega$-valuation} is a mapping~$\xi$ from type
  variables to candidates such that $\xi(\alpha)$ is a candidate of
  type~$\omega(\alpha)$ (in context~$\Gamma$).

  For each type~$\sigma$, context~$\Gamma$, mapping~$\omega$
  and each $\Gamma,\omega$-valuation~$\xi$, the set $\val{\sigma}{\omega,\xi}{\Gamma}$ is
  defined by induction on~$\sigma$:
  \begin{itemize}
  \item $\val{\alpha}{\omega,\xi}{\Gamma} = \xi(\alpha)$ for a type
    variable~$\alpha$,
  \item $\val{\nat}{\omega,\xi}{\Gamma}$ is the set of all
    terms~$t \in \SN$ such that $\Gamma \proves t : \nat$,
  \item $\val{\sigma \to \tau}{\omega,\xi}{\Gamma}$ is the set of all
    terms~$t$ such that $\Gamma \proves t : \omega(\sigma \to \tau)$,
    and for every~$s \in \val{\sigma}{\omega,\xi}{\Gamma}$ we have
    $\app{t}{s} \in \val{\tau}{\omega,\xi}{\Gamma}$,
  \item $\val{\forall\alpha[\sigma]}{\omega,\xi}{\Gamma}$ is the set
    of all terms~$t$ such that
    $\Gamma \proves t : \omega(\forall\alpha[\sigma])$ and for every
    type~$\tau$ and every $X \in \Cb_\tau^\Gamma$ we have
    $\tapp{t}{\alpha} \in
    \val{\sigma}{\omega[\subst{\alpha}{\tau}],\xi[\subst{\alpha}{X}]}{\Gamma}$.
  \end{itemize}
\end{definition}

\begin{lemma}\label{lem_val_typable}
  If $t \in \val{\sigma}{\omega,\xi}{\Gamma}$ then
  $\Gamma \proves t : \omega(\sigma)$.
\end{lemma}

\begin{proof}
  Follows from definitions.
\end{proof}

\begin{lemma}\label{lem_nat_reducible}
  $\val{\nat}{\omega,\xi}{\Gamma} \in \Cb_{\nat}^\Gamma$.
\end{lemma}

\begin{proof}
  We check the conditions in Definition~\ref{def_candidate}.
  \begin{enumerate}
  \item $\val{\nat}{\omega,\xi}{\Gamma} \subseteq \SN$ follows
    directly from Definition~\ref{def_reducibility_valuation}.
  \item Let $t \in \val{\nat}{\omega,\xi}{\Gamma}$ and
    $t \leadsto t'$. Then $\Gamma \proves t : \nat$ and $t \in
    \SN$. Hence $t' \in \SN$, and $\Gamma \proves t' : \nat$ by the
    subject reduction lemma. Thus
    $t' \in \val{\nat}{\omega,\xi}{\Gamma}$.
  \item Let $t$ be neutral and $\Gamma \proves t : \nat$. Assume that
    for all~$t'$ with $t \leadsto t'$ we have
    $t' \in \val{\nat}{\omega,\xi}{\Gamma}$, so in particular
    $t' \in \SN$. But then $t \in \SN$. Hence
    $t \in \val{\nat}{\omega,\xi}{\Gamma}$.
  \item Let $t_1,t_2 \in \SN$ be such that
    $\Gamma \proves t_i : \nat$. Obviously,
    $\Gamma \proves \circ_\nat t_1 t_2 : \nat$. Also
    $\circ_\nat t_1 t_2 \in \SN$ follows by Lemma~\ref{lem_circ_sn_base}.
    So $\circ_\nat t_1 t_2 \in \val{\nat}{\omega,\xi}{\Gamma}$.
  \end{enumerate}
\end{proof}

\LC{The following lemma is essentially just the two lemmas you
  commented out, plus using property 2 of candidates for the other
  direction. I corrected the statement (the statements of the
  commented-out lemmas were also incomplete -- I forgot about the
  typing).}
\CK{Yeah, I realised that afterwards -- I needed them \emph{before}
  the main proof, and hadn't realised that they were given later
  as well.}
\begin{lemma}\label{lem_abstraction_computable}
  Suppose it is given that $\val{\sigma}{\omega',\xi'}{\Gamma}$ and
  $\val{\tau}{\omega,\xi}{\Gamma}$ are candidates for all suitable
  $\omega',\xi'$.  Then
  \begin{itemize}
  \item
    $\abs{x:\omega(\tau)}{s} \in \val{\tau \arrtype
      \sigma}{\omega,\xi}{ \Gamma}$ if and only if
    $\Gamma \proves \abs{x:\omega(\tau)}{s} : \omega(\tau \arrtype
    \sigma)$ and $s[x:=t] \in \val{\sigma}{\omega,\xi}{\Gamma}$ for
    all $t \in \val{ \tau}{\omega,\xi}{\Gamma}$;
  \item
    $\tabs{\alpha}{s} \in \val{\quant{\alpha}{\sigma}}{\omega,\xi}{
      \Gamma}$ if and only if
    $\Gamma \proves \tabs{\alpha}{s} : \omega(\quant{\alpha}{\sigma})$
    and for every type~$\tau$ and all $X \in \Cb_\tau^\Gamma$ we have
    $s[\alpha:=\tau] \in
    \val{\sigma}{\omega[\subst{\alpha}{\tau}],\xi[\subst{\alpha}{X}]}{\Gamma}$.
  \end{itemize}
\end{lemma}

\begin{proof}
  First suppose $\abs{x:\sigma}{s} \in \val{\sigma \arrtype \tau}{\omega,
  \xi}{\Gamma}$.  Then for all $t \in \val{\sigma}{\omega,\xi}{\Gamma}$ we
  have $\app{(\abs{x:\sigma}{s})}{t} \in \val{\tau}{\omega,\xi}{\Gamma}$.
  As this set is a candidate, it is closed under $\leadsto$, so also
  $s[x:=t] \in \val{\tau}{\omega,\xi}{\Gamma}$.
  Similarly, if $\tabs{\alpha}{s} \in \quant{\alpha}{\sigma}$, then
  $\tapp{(\tabs{\alpha}{s})}{\tau} \in \val{\sigma}{
  \omega[\subst{\alpha}{\tau}],\xi[\subst{\alpha}{X}]}{\Gamma}$, and we are
  done because $\tapp{(\tabs{\alpha}{s})}{\tau} \leadsto s[\alpha:=\tau]$.

  Now suppose (**): $s[x:=t] \in \val{\sigma}{\omega,\xi}{\Gamma}$ for all
  $t \in \val{\tau}{\omega,\xi}{\Gamma}$.  Let $t \in \val{\tau}{\omega,
  \xi}{\Gamma}$.  Then:
  \begin{itemize}
  \item $t \in \SN$ because $\val{\tau}{\omega,\xi}{\Gamma}$ is a candidate;
  \item $s \in \SN$ because every infinite reduction in $s$ induces an
    infinite reduction in $s[x:=t]$ ($\leadsto$ is stable);
  \item if $s \leadsto^* s'$ and $t \leadsto^* t'$, then:
    \begin{itemize}
    \item $s[x:=t] \leadsto^* s'[x:=t']$ by monotonicity and stability of
      $\leadsto$;
    \item therefore $s'[x:=t'] \in \val{\sigma}{\omega,\xi}{\Gamma}$, because
      $\val{\sigma}{\omega,\xi}{\Gamma}$ is a candidate and therefore closed
      under $\leadsto$;
    \item $\app{(\abs{x}{s'})}{t'}$ is neutral, so in
      $\val{\sigma}{\omega,\xi}{\Gamma}$ if all its reducts are;
    \item by induction on $s',t'$ oriented with $\leadsto$ therefore
      $\app{(\abs{x}{s'})}{t'} \in \val{\sigma}{\omega,\xi}{\Gamma}$.
    \end{itemize}
  \end{itemize}
  A similar reasoning applies to $s[\alpha:=\tau]$.
\end{proof}

\begin{lemma}\label{lem_val_computable}
  $\val{\sigma}{\omega,\xi}{\Gamma} \in \Cb_{\omega(\sigma)}^\Gamma$.
\end{lemma}

\begin{proof}
  Induction on~$\sigma$. First, if $\sigma = \alpha$ for a type
  variable~$\alpha$ then
  $\val{\sigma}{\omega,\xi}{\Gamma} = \xi(\alpha) \in
  \Cb_{\omega(\sigma)}^\Gamma$ by definition. If $\sigma = \nat$ then
  $\val{\nat}{\omega,\xi}{\Gamma} \in \Cb_{\nat}^\Gamma$ by
  Lemma~\ref{lem_nat_reducible}.

  Assume $\sigma = \tau_1 \to \tau_2$. We check the conditions in
  Definition~\ref{def_candidate}.
  \begin{enumerate}
  \item Let $t \in \val{\tau_1\to\tau_2}{\omega,\xi}{\Gamma}$ and
    assume there is an infinite reduction
    $t \leadsto t_1 \leadsto t_2 \leadsto t_3 \leadsto \ldots$. By the
    inductive hypothesis $\val{\tau_1}{\omega,\xi}{\Gamma}$ and
    $\val{\tau_2}{\omega,\xi}{\Gamma}$ are candidates. So
    $x \in \val{\tau_1}{\omega,\xi}{\Gamma}$ because it is neutral and
    normal. Then $t x \in \val{\tau_2}{\omega,\xi}{\Gamma}$. Thus
    $t x$~is strongly normalising. But
    $t x \leadsto t_1 x \leadsto t_2 x \leadsto t_3 x \leadsto
    \ldots$. Contradiction.
  \item Let $t \in \val{\tau_1\to\tau_2}{\omega,\xi}{\Gamma}$ and
    $t \leadsto t'$. Let $u \in
    \val{\tau_1}{\omega,\xi}{\Gamma}$. Then
    $t u \in \val{\tau_2}{\omega,\xi}{\Gamma}$. By the inductive
    hypothesis $\val{\tau_2}{\omega,\xi}{\Gamma}$ is a candidate, so
    $t' u \in \val{\tau_2}{\omega,\xi}{\Gamma}$. Also note that
    $\Gamma \proves t' : \omega(\tau_1 \to \tau_2)$ by the subject
    reduction lemma. Hence
    $t' \in \val{\tau_1\to\tau_2}{\omega,\xi}{\Gamma}$.
  \item Let $t$ be neutral and
    $\Gamma \proves t : \omega(\tau_1 \to \tau_2)$ and assume for
    every~$t'$ with $t \leadsto t'$ we have
    $t' \in \val{\tau_1\to\tau_2}{\omega,\xi}{\Gamma}$. Let
    $u \in \val{\tau_1}{\omega,\xi}{\Gamma}$. By the inductive
    hypothesis $\val{\tau_1}{\omega,\xi}{\Gamma}$ is a candidate, so
    $u \in \SN$. By induction on~$\nu(u)$ we show that
    $t u \in \val{\tau_2}{\omega,\xi}{\Gamma}$. Assume
    $t u \leadsto t''$. We show
    $t'' \in \val{\tau_2}{\omega,\xi}{\Gamma}$. Because~$t$ is
    neutral, $t u$ cannot be a redex. So there are two cases.
    \begin{itemize}
    \item $t'' = t u'$ with $u \leadsto u'$. Then
      $t u' \in \val{\tau_2}{\omega,\xi}{\Gamma}$ by the inductive
      hypothesis for~$u$.
    \item $t'' = t' u$ with $t \leadsto t'$. Then
      $t' \in \val{\tau_1\to\tau_2}{\omega,\xi}{\Gamma}$, so
      $t' u \in \val{\tau_2}{\omega,\xi}{\Gamma}$.
    \end{itemize}
    We have thus shown that if $t u \leadsto t''$ then
    $t'' \in \val{\tau_2}{\omega,\xi}{\Gamma}$. By the (main)
    inductive hypothesis $\val{\tau_2}{\omega,\xi}{\Gamma}$ is a
    candidate, and $t u$ is neutral, so also
    $t u \in \val{\tau_2}{\omega,\xi}{\Gamma}$. Since
    $u \in \val{\tau_1}{\omega,\xi}{\Gamma}$ was arbitrary, we have
    shown $t \in \val{\tau_1\to\tau_2}{\omega,\xi}{\Gamma}$.
  \item Assume $t_1,t_2 \in \val{\tau_1\to\tau_2}{\omega,\xi}{\Gamma}$.
    We have already shown that this implies $t_1,t_2 \in \SN$.
    Since $s := \circ_{\omega(\tau_1\to\tau_2)} t_1 t_2$ is neutral,
    we have also already seen that $s \in \val{\tau_1\to\tau_2}{\omega,
    \xi}{\Gamma}$ if $s' \in \val{\tau_1\to\tau_2}{\omega,\xi}{\Gamma}$
    whenever $s \leadsto s'$.
    This we show by induction on $t_1,t_2$ (oriented with $\leadsto$).
    If $s' = \circ_{\omega(\tau_1\to\tau_2)} t_1' t_2$, then note that
    $t_1' \in \val{\tau_1\to\tau_2}{\omega,\xi}{\Gamma}$ because we
    have already shown that $\val{\tau_1\to\tau_2}{\omega,\xi}{\Gamma}$
    is closed under $\leadsto$; thus, we can complete by the induction
    hypothesis.
    If $s' = \circ_{\omega(\tau_1\to\tau_2)} t_1 t_2'$, we complete in
    the same way.
    The only alternative is that $s' = \abs{x:\omega(\tau_1)}{\circ_{
    \omega(\tau_2)} (t_1 x) (t_2 x)}$.
    Let $u \in \val{\tau_1}{\omega,\xi}{\Gamma}$.  Since $t_1,t_2 \in
    \val{\tau_1\to\tau_2}{\omega,\xi}{\Gamma}$, we have that $t_1 u$
    and $t_2 u$ are in $\val{\tau_2}{\omega,\xi}{\Gamma}$ by
    definition.  Since, $\val{\tau_2}{\omega,\xi}{\Gamma}$ is a
    candidate, this means that $\circ_{\omega(\tau_2)} (t_1 u) (t_2 u) =
    (\circ_{\omega(\tau_2)} (t_1 x) (t_2 x))[x:=u]$ is in
    $\val{\tau_2}{\omega,\xi}{\Gamma}$ as well.  By
    Lemma~\ref{lem_abstraction_computable}, we conclude that $s' \in
    \val{\tau_1\to\tau_2}{\omega,\xi}{\Gamma}$.
  \end{enumerate}

  Assume $\sigma = \forall\alpha[\tau]$. We check the conditions in
  Definition~\ref{def_candidate}.
  \begin{enumerate}
  \item Let $t \in \val{\forall\alpha[\tau]}{\omega,\xi}{\Gamma}$ and
    assume there is an infinite reduction
    $t \leadsto t_1 \leadsto t_2 \leadsto t_3 \leadsto
    \ldots$. Let~$\tau'$ be an arbitrary type and let
    $S \in \Cb_{\tau'}^\Gamma$ (e.g.~take~$S$ to be the set of
    terminating terms of type~$\tau'$ in context~$\Gamma$).
    \CK{It is not guaranteed that this will work! For that we really
    do need that $\circ s t$ is SN if $s$ and $t$ are.  However, can't
    we just take $\tau' := \nat$?} \LC{I think $\tau' := \nat$ indeed works here.}
    Then
    $t \tau' \in
    \val{\tau}{\omega[\subst{\alpha}{\tau'}],\xi[\subst{\alpha}{S}]}{\Gamma}$. By
    the inductive hypothesis
    $\val{\tau}{\omega[\subst{\alpha}{\tau'}],\xi[\subst{\alpha}{S}]}{\Gamma}$
    is a candidate, so $t \tau' \in \SN$. But
    $t \tau' \leadsto t_1 \tau' \leadsto t_2 \tau' \leadsto t_3 \tau'
    \leadsto \ldots$. Contradiction.
  \item Let $t \in \val{\forall\alpha[\tau]}{\omega,\xi}{\Gamma}$ and
    $t \leadsto t'$. By the subject reduction lemma
    $\Gamma \proves t' : \omega(\forall\alpha[\tau])$. Let~$\tau'$ be
    a type and~$X \in \Cb_{\tau'}^\Gamma$. Then
    $t \tau' \in
    \val{\tau}{\omega[\subst{\alpha}{\tau'}],\xi[\subst{\alpha}{X}]}{\Gamma}$. By
    the inductive hypothesis
    $\val{\tau}{\omega[\subst{\alpha}{\tau'}],\xi[\subst{\alpha}{X}]}{\Gamma}$
    is a candidate, so
    $t' \tau' \in
    \val{\tau}{\omega[\subst{\alpha}{\tau'}],\xi[\subst{\alpha}{X}]}{\Gamma}$. Therefore
    $t' \in \val{\forall\alpha[\tau]}{\omega,\xi}{\Gamma}$.
  \item Let $t$ be neutral and
    $\Gamma \proves t : \omega(\forall\alpha[\tau])$ and assume for
    every~$t'$ with $t \leadsto t'$ we have
    $t' \in
    \val{\forall\alpha[\tau]}{\omega,\xi}{\Gamma}$. Let~$\tau'$ be a
    type and~$X \in \Cb_{\tau'}^\Gamma$. Assume
    $t \tau' \leadsto t''$. Then $t'' = t' \tau'$ with
    $t \leadsto t'$, because~$t$ is neutral. Hence
    $t \tau' \leadsto t' \tau' \in
    \val{\tau}{\omega[\subst{\alpha}{\tau'}],\xi[\subst{\alpha}{X}]}{\Gamma}$. By
    the inductive
    hypothesis~$\val{\tau}{\omega[\subst{\alpha}{\tau'}],\xi[\subst{\alpha}{X}]}{\Gamma}$
    is a candidate, and also $t \tau'$ is neutral, so
    $t \tau' \in
    \val{\tau}{\omega[\subst{\alpha}{\tau'}],\xi[\subst{\alpha}{X}]}{\Gamma}$. This
    implies that
    $t \in \val{\forall\alpha[\tau]}{\omega,\xi}{\Gamma}$.
  \item Assume
    $t_1,t_2 \in \val{\forall\alpha[\tau]}{\omega,\xi}{\Gamma}$. We
    have already shown that this implies $t_1,t_2 \in \SN$, and since
    $s := \circ_{\omega(\forall\alpha[\tau])} t_1 t_2$ is neutral, that
    $s \in \val{\forall\alpha[\tau]}{\omega,\xi}{\Gamma}$ if $s' \in
    \val{\forall\alpha[\tau]}{\omega,\xi}{\Gamma}$ whenever $s \leadsto
    s'$.  This we show by induction on $t_1,t_2$.
    The cases when $t_1$ or $t_2$ are reduced are immediate with the
    induction hypotheses; the only remaining case is when $s' =
    \tabs{\alpha}{\circ_{\omega(\tau)} (t_1 \alpha) (t_2 \alpha)}$.
    For all types $\sigma$ and $X \in \Cb_{\sigma}^\Gamma$ we have both
    $t_1 \sigma$ and $t_2 \sigma$ in $\val{\tau}{\omega[\subst{\alpha}{
    \sigma}],\xi[\subst{\alpha}{X}]}{\Gamma}$ (by definition of
    $t_1,t_2 \in \val{\forall\alpha[\tau]}{\omega,\xi}{\Gamma}$).
    But then, because $\val{\tau}{\omega[\subst{\alpha}{\sigma}],
    \xi[\subst{\alpha}{X}]}{\Gamma}$ is a candidate by the inductive
    hypothesis, we have $\circ_{\omega(\tau[\subst{\alpha}{\sigma})}
    (t_1 \sigma) (t_2\sigma) = (\circ_{\omega(\tau)} (t_1 \alpha) (t_2
    \alpha))[\subst{\alpha}{\sigma}] \in \val{\tau}{\omega[\subst{
    \alpha}{\sigma}],\xi[\subst{\alpha}{X}]}{\Gamma}$ for all types
    $\sigma$, so $s' \in \val{\forall\alpha[\tau]}{\omega,\xi}{\Gamma}$
    by Lemma~\ref{lem_abstraction_computable}.
  \end{enumerate}
\end{proof}

\begin{lemma}\label{lem_val_subst}
  $\val{\sigma[\subst{\alpha}{\tau}]}{\omega,\xi}{\Gamma} =
  \val{\sigma}{\omega[\subst{\alpha}{\tau}],\xi[\subst{\alpha}{\val{\tau}{\omega,\xi}{\Gamma}}]}{\Gamma}$.
\end{lemma}

\begin{proof}
  TODO
\end{proof}

\begin{lemma}
  $\circ_{\omega(\sigma)} \in \val{\sigma \to \sigma \to
    \sigma}{\omega,\xi}{\Gamma}$ for
  $\circ \in \{ \oplus, \otimes \}$.
\end{lemma}

\begin{proof}
  Follows from Definition~\ref{def_typing} and
  Lemma~\ref{lem_val_computable}.
\end{proof}

\begin{lemma}
  $\flatten_{\omega(\sigma)} \in
  \val{\sigma\to\nat}{\omega,\xi}{\Gamma}$.
\end{lemma}

\begin{proof}
  TODO
\end{proof}

\begin{lemma}
  $\lift_{\omega(\sigma)} \in
  \val{\nat\to\sigma}{\omega,\xi}{\Gamma}$.
\end{lemma}

\begin{proof}
  TODO
\end{proof}

\begin{lemma}
  If $t \in \val{\forall\alpha[\sigma]}{\omega,\xi}{\Gamma}$ then
  $t \tau \in \val{\sigma[\subst{\alpha}{\tau}]}{\omega,\xi}{\Gamma}$
  for any type~$\tau$.
\end{lemma}

\begin{proof}
  By Lemma~\ref{lem_val_computable} the
  set~$\val{\tau}{\omega,\xi}{\Gamma}$ is a candidate of
  type~$\tau$. By hypothesis
  $t \tau \in
  \val{\sigma}{\omega[\subst{\alpha}{\tau}],\xi[\subst{\alpha}{\val{\tau}{\omega,\xi}{\Gamma}}]}{\Gamma}$. Hence
  $t \tau \in \val{\sigma[\subst{\alpha}{\tau}]}{\omega,\xi}{\Gamma}$
  by Lemma~\ref{lem_val_subst}.
\end{proof}

\begin{lemma}\label{lem_typable_computable}
  If $\Gamma \proves t : \sigma$ then
  $\omega(t) \in \val{\sigma}{\omega,\xi}{\Gamma}$.
\end{lemma}

\begin{proof}
  Idea: by induction on~$t$, using the previous lemmas. One
  generalizes the inductive hypothesis to: if
  $x_1 : \tau_1,\ldots,x_n:\tau_n \proves t : \sigma$ then for all
  $u_1\in\val{\tau_1}{\omega,\xi}{\Gamma},\ldots,u_n\in\val{\tau_n}{\omega,\xi}{\Gamma}$
  we have
  $\omega(t[\subst{x_1}{u_1},\ldots,\subst{x_n}{u_n}]) \in
  \val{\sigma}{\omega,\xi}{\Gamma}$.
\end{proof}

\begin{corollary}
  If $\Gamma \proves t : \sigma$ then $t \in \SN$.
\end{corollary}

\begin{proof}
  Follows from Lemma~\ref{lem_typable_computable},
  Lemma~\ref{lem_val_computable} and Definition~\ref{def_candidate}.
\end{proof}

\begin{lemma}
Every term $s \in \Iterms$ with $\FV(s) = \FTV(s) = \emptyset$ reduces
to a unique final interpretation term $s\downarrow$.
\end{lemma}

\begin{proof}
By proving termination and local confluence of $\leadsto$ (although it
actually suffices to only prove local confluence for fully closed terms).

TODO.
\qed
\end{proof}

\begin{lemma}
The only final interpretation terms of type $\nat$ are the natural
numbers.
\end{lemma}

\begin{proof}
  This is~NOT~TRUE. Consider
  $\oplus_{\nat\to\nat} \lift_\nat \lift_\nat n$. Need to add more
  rewrite rules?
  \CK{Better now?} \LC{Yes, I think it works now.}
\end{proof}

\subsection{Defining $\succ$ and $\succeq$}

We will define $\succ$ and $\succeq$ on elements of $\World$, using
coinduction: $s \succ t$ (resp.\ $s \succeq t$) if $s \succ_\sigma t$
(resp.\ $s \succeq_\sigma t$) can be obtained for $\sigma$ the type of
$s,t$, following Definition~\ref{def:succ}.

\begin{definition}\label{def:succ}
  Let $R \in \{ \succ,\succeq \}$. For~$s,t \in \World_\sigma$, the
  relation $s\ R_{\sigma}\ t$ is defined coinductively by the
  following rules.
  \[
    \begin{array}{c}
    \infer={s\ R_\nat\ t}{s\da\ R\ t\da \text{ in natural
        numbers}}\quad\quad
    \infer={s\ R_{\sigma\to\tau}\ t}{\app{s}{q}\ R_{\tau}\ \app{t}{q}
      \text{ for every } q \in \World_\sigma} \\ \\
    \infer={s\ R_{\forall\alpha[\sigma]}\ t}{\tapp{s}{\tau}\ R_{\sigma[\subst{\alpha}{\tau}]}\ \tapp{t}{\tau}
      \text{ for every closed type } \tau}
    \end{array}
  \]
\end{definition}

\CK{Actually, by definition of ``final interpretation terms'' you
don't need to normalise in the natural number case.}

\subsection{Properties of $\succ$ and $\succeq$}

We're going to at least need to see that $\succ$ is a well-founded ordering:

\begin{lemma}
$\succ$ is well-founded.
\end{lemma}

\begin{proof}
  By induction on the size of the type~$\tau$ one shows that there does
  not exist an infinite sequence
  $t_1 \succ_\tau t_2 \succ_\tau t_3 \succ_\tau \ldots$ For instance,
  if
  $t_1 \succ_{\forall\alpha[\tau]} t_2 \succ_{\forall\alpha[\tau]} t_3
  \succ_{\forall\alpha[\tau]} \ldots$ then
  $\tapp{t_1}{\nat} \succ_{\tau[\subst{\alpha}{\nat}]}
  \tapp{t_2}{\nat} \succ_{\tau[\subst{\alpha}{\nat}]} \tapp{t_3}{\nat}
  \succ_{\tau[\subst{\alpha}{\nat}]} \ldots$, which is impossible by
  the inductive hypothesis.
\end{proof}

\begin{lemma}
Both $\succ$ and $\succeq$ are transitive.
\end{lemma}

\begin{proof}
  We show this for~$\succeq$, the proof for~$\succ$ being
  analogous. We proceed by coinduction.

  If $t_1 \succeq_\nat t_2 \succeq_\nat t_3$ then
  $t_1\da \ge t_2\da \ge t_3\da$, so $t_1\da \ge t_3\da$. Thus
  $t_1 \succeq_\nat t_3$.

  If $t_1 \succeq_{\sigma\to\tau}t_2\succeq_{\sigma\to\tau}t_3$ then
  $\app{t_1}{q}\succeq_{\tau}\app{t_2}{q}\succeq_\tau\app{t_3}{q}$ for
  $q \in \World_\sigma$. Hence $\app{t_1}{q}\succeq_\tau\app{t_3}{q}$
  for $q \in \World_\sigma$ by the coinductive hypothesis. Thus
  $t_1\succeq_{\sigma\to\tau} t_3$.

  If $t_1 \succeq_{\forall\alpha[\sigma]}t_2\succeq_{\forall\alpha[\sigma]}t_3$ then
  $\tapp{t_1}{\tau}\succeq_{\sigma[\subst{\alpha}{\tau}]}\tapp{t_2}{\tau}\succeq_{\sigma[\subst{\alpha}{\tau}]}\tapp{t_3}{\tau}$ for
  any closed type~$\tau$. Hence
  $\tapp{t_1}{\tau}\succeq_{\sigma[\subst{\alpha}{\tau}]}\tapp{t_3}{\tau}$
  by the coinductive hypothesis. Thus $t_1\succeq_{\forall\alpha[\sigma]} t_3$.\qed
\end{proof}

And that $\succeq$ is a quasi-ordering:

\begin{lemma}
$\succeq$ is reflexive.
\end{lemma}

\begin{proof}
  By coinduction.\qed
\end{proof}

Finally, they must be compatible:

\begin{lemma}
We have $\succ \cdot \succeq\ \subseteq\ \succ$
\end{lemma}

\begin{proof}
By coinduction, analogous to the transitivity proof.\qed
\end{proof}

I think that's all that absolutely \emph{has} to be proven to make the
argument work.  However, it would be great if we could also for
instance obtain that always $\app{\app{\oplus}{s}}{t} \succeq s$ and
that $\app{\app{\oplus}{s}}{(\app{\lift}{1})} \succ s$.

\LC{I think these also hold by coinduction because they hold for
type~$\nat$. But one needs to generalize the coinductive hypothesis to
$\app{\app{\oplus}{s}}{t} \cdot t_1 \cdot \ldots \cdot t_n
\succeq_\sigma s \cdot t_1 \cdot \ldots \cdot t_n$ (and with some
additional type arguments also).}

\section{Systems of interest and their properties}\label{sec:systems}

The systems of interest are sets of terms with a rewriting relation on
them.  Interestingly, $\Iterms$ can be seen as an instance of this general
scheme.

\subsection*{\ref{sec:systems}.0\quad Type context variables}

In the following, we will use \emph{type contexts}.  A type context
is an expression $\meta{\alpha}{\rho_1,\dots,\rho_n}$ where $\alpha$
is a type variable and $\rho_1,\dots,\rho_n$ are types.  An
\emph{interpretation type scheme} is generated by the clauses of
Definition~\ref{def:itypes} and in addition:
\[
\meta{\alpha}{\rho_1,\dots,\rho_n} \in \ITypes'\ \text{if}\ 
  \alpha \in \Typevars\ \text{and}\ \rho_1,\dots,\rho_n \in \ITypes'
\]
A type substitution $\gamma$ can only be applied to an interpretation
type scheme $\sigma$ if for every occurrence $\meta{\alpha}{\rho_1,
\dots,\rho_n}$ in $\sigma$, the type variable $\alpha$ is in the domain
of $\gamma$ and $\gamma(\alpha)$ has the form $\tabs{\beta_1 \dots
\beta_n}{\tau}$.  The type substitution $\sigma\gamma$ is obtained as
follows:
\begin{itemize}
\item $\alpha\gamma = \gamma(\alpha)$ if $\alpha$ is in the domain of
  $\gamma$;
\item $\alpha\gamma = \alpha$ if $\alpha \in \Typevars$ and $\alpha$
  is not in the domain of $\gamma$;
\item $\meta{\alpha}{\rho_1,\dots,\rho_n}\gamma = \subst{\tau}{[
  \beta_1:=\rho_1\gamma,\dots,\beta_n:=\rho_n\gamma]}$ if $\gamma(\alpha) =
  \tabs{\beta_1\dots \beta_n}{\tau}$;
\item $(\tabs{\alpha}{\tau})\gamma = \tabs{\alpha}{(\tau\gamma)}$ if
  $\alpha$ is not in the domain of $\gamma$;
  if $\alpha$ is in this domain, then $\alpha$-conversion can be used
  to rename it to a fresh variable first;
\item $(\sigma \arrtype \tau)\gamma = (\sigma\gamma) \arrtype (\tau
  \gamma)$
\item $(\sigma \times \tau)\gamma = (\sigma\gamma) \times (\tau\gamma)$
\end{itemize}

In this way, a type variable context $\meta{\alpha}{\rho_1,\dots,
\rho_n}$ is essentially instantiated by a type in which $\rho_1,\dots,
\rho_n$ may (but don't have to) occur.  This is important when the
$\rho_i$ contain bound type variables.  For example, the
interpretation type scheme $\tabs{\alpha}{\beta}$ can only be
instantiated by interpretation types of the form $\tabs{\alpha}{\tau}$
where the bound variable $\alpha$ does not occur in $\tau$, while the
interpretation type scheme $\tabs{\alpha}{\beta[\alpha]}$ is
instantiated by all interpretation types of the form
$\tabs{\alpha}{\sigma}$.

\subsection{Systems of interest}

Types are built from two sets of type constructors (first-order and
second-order respectively) which must include the type constructors
$\arrtype$ and $\forall$ respectively.

\begin{definition}
We assume given a fixed set $\TypeConstructors$ of \emph{type
constructors}, each paired with an integer \emph{arity}, and a set
$\TypeQuantifiers$ of \emph{type quantifiers}, each quantifier also
paired with an integer arity; $\TypeConstructors$ must contain
$\arrtype : 2$ and $\TypeQuantifiers$ must include $\forall : 1$.
The set $\Types$ of \emph{type schemes} is given by:
\begin{itemize}
\item $\alpha \in \Types$ for all $\alpha \in \Typevars$
\item $\con(\sigma_1,\dots,\sigma_n) \in \Types$ if $\con : n \in
  \TypeConstructors$ and $\sigma_1,\dots,\sigma_n \in \Types$
  (but we denote $\arrtype$ infix)
\item $\qquant{?}{\alpha_1 \dots \alpha_n}{\sigma} \in \Types$ if
  $? : n \in \TypeQuantifiers$, each $\alpha_i \in \Typevars$ and
  $\sigma \in \Types$
\item $\meta{\alpha}{\sigma_1,\dots,\sigma_n} \in \Types$ for all
  $\alpha \in \Typevars$ and $\sigma_1,\dots,\sigma_n \in \Types$.
\end{itemize}
\emph{Types} are type schemes without occurrences of type variable
contexts $\meta{\alpha}{\sigma_1,\dots,\sigma_n}$.
We will not consider type schemes where type variable occurrences
use a \emph{bound} type variable $\alpha$.
We let $\FTV(\sigma)$ be defined in the obvious way (much like
Definition~\ref{def:itypes}, and also including type variables at
the head of a type variable context).
\end{definition}

Terms are built from a set of (possibly type-indexed) function symbols,
using abstraction and application as for interpretation terms.

\begin{definition}
We assume given a fixed set $\Sigma$ of \emph{function symbols}, each
indexed with $0$ or more type variables and paired with an arity $k$
as well as $k$ input types and one output types, denoted
$\mathtt{f}_{\alpha_1,\dots,\alpha_n} : [\sigma_1 \times \dots \times
\sigma_k] \arrtype \tau$; we require that $\FTV(\sigma_1) \cup \dots
\cup \FTV(\sigma_k) \cup \FTV(\tau) \subseteq \{ \alpha_1,\dots,
\alpha_n\}$.
Every function symbol $\mathtt{f}$ occurs only with one type
declaration.

The set $\Terms$ of terms consists of those expressions $s$ such that
$\Gamma \vdash s : \sigma$ can be derived for some type $\sigma$ and
environment $\Gamma$ using the following clauses:
\begin{itemize}
\item $\Gamma \vdash x : \sigma$ for every $(x : \sigma) \in \Gamma$.
\item $\Gamma \vdash \mathtt{f}_{\vec{\rho}}(s_1,\dots,s_k) :
  \tau\gamma$ if $\mathtt{f}_{\alpha_1,\dots,\alpha_n} : [\sigma_1
  \times \dots \times \sigma_k] \arrtype \tau \in \Sigma$ and
  $\gamma$ is a type substitution and $\vec{\rho} =
  [\alpha_1\gamma,\dots,\alpha_n\gamma]$ and $\Gamma \vdash s_i :
  \sigma_i\gamma$ for $1 \leq i \leq k$
\item $\Gamma \vdash \abs{x:\sigma}{s} : \sigma \arrtype \tau$ if $x
  \in \Vars$ and $\Gamma \uplus \{ x : \sigma \} \vdash s : \tau$.
\item $\Gamma \vdash \tabs{\alpha}{s} : \quant{\alpha}{\sigma}$ if
  $\alpha \in \Typevars$ and $\Gamma \vdash s : \sigma$ and for all
  $(x : \tau) \in \Gamma$: $\alpha \notin \FTV(\tau)$
%\item $\Gamma \vdash \tapp{s}{\tau} : \sigma[\subst{\alpha}{\tau}]$ if
%  $\Gamma \vdash s : \quant{\alpha}{\sigma}$
\end{itemize}
%Rather than using a pair $(k,\sigma_1 \arrtype \dots \arrtype \sigma_k
%\arrtype \tau)$, we will denote the type declaration of a function
%symbol as $[\sigma_1 \times \dots \times \sigma_k] \arrtype \tau$.
\end{definition}

Note that there is no explicit application; application can be modelled
by including symbol $@_{\alpha,\beta} : [\alpha \arrtype \beta \times
\alpha] \arrtype \beta$ in $\Sigma$.
Similarly, type application is modelled through a symbol
$\mathtt{A}_{\alpha,\beta} : [\quant{\xi}{\meta{\alpha}{\xi}} \arrtype
\meta{\alpha}{\beta}$.

The rules are simply an infinite set of term pairs, whose monotonic
closure generates the rewrite relation.

\begin{definition}
We fix an infinite variable environment $\Gamma$,
and assume given a set $\Rules$ of term pairs $(\ell,r)$, such that:
\begin{itemize}
\item $\FV(r) \subseteq \FV(\ell) \subseteq \mathit{keys}(\Gamma)$;
\item $\ell$ and $r$ have the same type under $\Gamma$;
\item $\FTV(\ell) = \FTV(r) = \emptyset$;
  %, and all type variables in
  %the range of $\Gamma$ are bound;
\item $\Rules$ is stable: if $(\ell,r) \in \Rules$ and $\gamma$ is a
  substitution, then $(\ell\gamma,r\gamma) \in \Rules$.
\end{itemize}
The reduction relation $\arr{\Rules}$ is the smallest monotonic
relation that contains $\Rules$.
\end{definition}

\subsubsection{An example system.}

For the system of urzy\_emb, we have the following type constructors
and quantifiers:
\[
\begin{array}{c}
\TypeConstructors = \{\quad
  \bot : 0,\quad
  \mathtt{or} : 2,\quad
  \mathtt{and} : 2,\quad
  \Rightarrow : 2\quad
  \}\quad \cup\quad
  \{\quad \mathtt{prop}_i \mid i \in \mathbb{N}\quad \} \\
\TypeQuantifiers = \{\quad
  \forall : 1,\quad
  \exists : 1\quad
  \} \\
\end{array}
\]

We have the following function symbols:
\[
\begin{array}{rclrcl}
\epsilon_\alpha & : & [\bot] \arrtype \alpha &
\pi^1_{\alpha,\beta} & : & [\mathtt{and}(\alpha,\beta)] \arrtype \alpha \\
@_{\alpha,\beta} & : & [\alpha \arrtype \beta \times \alpha] \arrtype \beta &
\pi^2_{\alpha,\beta} & : & [\mathtt{and}(\alpha,\beta)] \arrtype \beta \\
\mathtt{in}^1_{\alpha,\beta} & : & [\alpha] \arrtype
  \mathtt{or}(\alpha,\beta) &
\mathtt{pair}_{\alpha,\beta} & : & [\alpha \times \beta] \arrtype
  \mathtt{and}(\alpha,\beta) \\
\mathtt{in}^2_{\alpha,\beta} & : & [\beta] \arrtype
  \mathtt{or}(\alpha,\beta) &
\mathtt{tapp}_{\alpha,\beta} & : &
  [\quant{\xi}{\meta{\alpha}{\xi}}] \arrtype \meta{\alpha}{\beta} \\
\mathtt{case}_{\alpha,\beta,\xi} & : & [\mathtt{or}(\alpha,\beta) \times
  \alpha \arrtype \xi \times \beta \arrtype \xi] \arrtype \xi\ \ \ & 
\mathtt{let}_{\alpha,\beta} & : &
  \qquant{\exists}{\xi}{\meta{\alpha}{\xi}} \times 
  \quant{\xi}{\xi \arrtype \beta}] \arrtype \beta \\
\mathtt{ext}_{\alpha,\beta} & : & [\meta{\alpha}{\beta}] \arrtype
  \qquant{\exists}{\xi}{\meta{\alpha}{\xi}} \\
\end{array}
\]
(Note that in the $\mathtt{let}$ symbol, the definition of type
substitution implies that in all instances $\mathtt{let}_{\sigma,\tau}$
the type $\tau$ cannot contain the bound type variable $\xi$.)

And finally the rules.  These are represented as \emph{rule schemes},
representing infinitely many rules at once.  Type substitution is also
done on indexes of function symbols (thus giving essentially
different symbols, but with correct types).
\[
\begin{array}{rcl}
@_{\sigma,\tau}(\abs{x}{s},t) & \to & s[x:=t] \\
\mathtt{tapp}_{\quant{\alpha}{\sigma},\tau}(\tabs{\alpha}{s}) & \to &
  s[\alpha:=\tau] \\
\pi^1_{\sigma,\tau}(\mathtt{pair}_{\sigma,\tau}(s,t)) & \to & s \\
\pi^2_{\sigma,\tau}(\mathtt{pair}_{\sigma,\tau}(s,t)) & \to & s \\
\mathtt{case}_{\sigma,\tau,\rho}(\mathtt{in}^1_{\sigma,\tau}(u),
  \abs{x}{s},\abs{y}{t}) & \to & s[x:=u] \\
\mathtt{case}_{\sigma,\tau,\rho}(\mathtt{in}^2_{\sigma,\tau}(u),
  \abs{x}{s},\abs{y}{t}) & \to & t[x:=u] \\
\mathtt{let}_{\qquant{\forall}{\alpha}{\sigma},\rho}(
  \mathtt{ext}_{\qquant{\forall}{\alpha}{\sigma},\tau}(s),\tabs{\alpha}{
    \abs{x}{t}}) & \to & t[\alpha:=\tau][x:=s] \\
\end{array}
\]
\[
\begin{array}{rcl}
\epsilon_\tau(\epsilon_\bot(s)) & \to & \epsilon_\tau(s) \\
@_{\sigma,\tau}(\epsilon_{\sigma \arrtype \tau}(s),t) & \to &
  \epsilon_\tau(s) \\
\mathtt{tapp}_{\quant{\alpha}{\sigma},\tau}(
  \epsilon_{\quant{\alpha}{\sigma}}(s)) & \to &
  \epsilon_{\sigma[\alpha:=\tau]}(s) \\
\pi^1_{\sigma,\tau}(\epsilon_{\mathtt{and}(\sigma,\tau)}(s)) & \to &
  \epsilon_\sigma(s) \\
\pi^2_{\sigma,\tau}(\epsilon_{\mathtt{and}(\sigma,\tau)}(s)) & \to &
  \epsilon_\tau(s) \\
\mathtt{case}_{\sigma,\tau,\rho}(\epsilon_{\mathtt{or}(\sigma,\tau)}(
  u),\abs{x}{s},\abs{y}{t}) & \to & \epsilon_\rho(s) \\
\mathtt{let}_{\qquant{\forall}{\alpha}{\sigma},\rho}(\epsilon_{\qquant{
  \forall}{\alpha}{\sigma}}(s),\tabs{\alpha}{\abs{x}{t}}) & \to &
  \epsilon_\rho(s) \\
\end{array}
\]
\begin{itemize}
%\[
%\begin{array}{rcl}
\item $
\epsilon_\rho(\mathtt{case}_{\sigma,\tau,\bot}(u,\abs{x}{s},\abs{y}{t}))
  %& \to &
  \to
  \mathtt{case}_{\sigma,\tau,\rho}(u,\abs{x}{\epsilon_\rho(s)},
  \abs{y}{\epsilon_\rho(t)}) $%\\
\item $
@_{\rho,\pi}(\mathtt{case}_{\sigma,\tau,\rho \arrtype \pi}(u,
  \abs{x}{s},\abs{y}{t}),v) %& \to &
  \to
  \mathtt{case}_{\sigma,\tau,\pi}(u,
  \abs{x}{@_{\rho,\pi}(s,v)},\abs{y}{@_{\rho,\pi}(t,v)}) $%\\
\item $
\mathtt{tapp}_{\quant{\alpha}{\rho},\pi}(\mathtt{case}_{\sigma,\tau,
  \quant{\alpha}{\rho}}(u,\abs{x}{s},\abs{y}{t})) %& \to &
  \to
  \mathtt{case}_{\sigma,\tau,\rho[\alpha:=\pi]}(u,
  \abs{x}{\mathtt{tapp}_{\quant{\alpha}{\rho},\pi}(s)},\\
  \abs{y}{\mathtt{tapp}_{\quant{\alpha}{\rho},\pi}(t)}) $%\\
\item $
\pi^1_{\rho,\pi}(\mathtt{case}_{\sigma,\tau,\mathtt{and}(\rho,\pi)}(u,
  \abs{x}{s},\abs{y}{t})) %& \to &
  \to
  \mathtt{case}_{\sigma,\tau,\rho}(u,\abs{x}{\pi^1_{\rho,\pi}(s)},
  \abs{y}{\pi^1_{\rho,\pi}(t)}) $%\\
\item $
\pi^2_{\rho,\pi}(\mathtt{case}_{\sigma,\tau,\mathtt{and}(\rho,\pi)}(u,
  \abs{x}{s},\abs{y}{t})) %& \to &
  \to
  \mathtt{case}_{\sigma,\tau,\pi}(u,\abs{x}{\pi^2_{\rho,\pi}(s)},
  \abs{y}{\pi^2_{\rho,\pi}(t)}) $%\\
\item $
\mathtt{case}_{\rho,\pi,\xi}(\mathtt{case}_{\sigma,\tau,\mathtt{or}(
  \rho,\pi)}(u,\abs{x}{s},\abs{y}{t}),\abs{z}{v},\abs{a}{w}) %& \to &
  \to\\
  \mathtt{case}_{\sigma,\tau,\xi}(u,
    \abs{x}{\mathtt{case}_{\rho,\pi,\xi}(s,\abs{z}{v},\abs{a}{w})},
    \abs{y}{\mathtt{case}_{\rho,\pi,\xi}(t,\abs{z}{v},\abs{a}{w})}) $%\\
\item $
\mathtt{let}_{\qquant{\forall}{\alpha}{\rho}}(
  \mathtt{case}_{\sigma,\tau,\qquant{\forall}{\alpha}{\rho}}(
  u,\abs{x}{s},\abs{y}{t}),v) %& \to &
  \to\\
  \mathtt{case}_{\sigma,\tau,\rho}(u,
  \abs{x}{\mathtt{let}_{\qquant{\forall}{\alpha}{\rho}}(s,v)},
  \abs{y}{\mathtt{let}_{\qquant{\forall}{\alpha}{\rho}}(t,v)})
  $%\\
%\end{array}
%\]
\end{itemize}
\begin{itemize}
\item $\epsilon_\tau(\mathtt{let}_{\qquant{\forall}{\alpha}{\sigma},
  \bot}(s,t)) \to
  \mathtt{let}_{\qquant{\forall}{\alpha}{\sigma},\tau}(s,\epsilon_\tau(t))$
\item $@_{\tau,\rho}(\mathtt{let}_{\qquant{\forall}{\alpha}{\sigma},
  \tau \arrtype \rho}(s,t),u) \to
  \mathtt{let}_{\qquant{\forall}{\alpha}{\sigma},\rho}(s,@_{\tau,\rho}(t,
  u))$
\item $\mathtt{tapp}_{\quant{\alpha}{\tau},\rho}(
\mathtt{let}_{\qquant{\forall}{\alpha}{\sigma},\quant{\alpha}{\tau}}(s,t))
  \to
  \mathtt{let}_{\qquant{\forall}{\alpha}{\sigma},\tau[\alpha:=\rho]}(s,
  \mathtt{tapp}_{\quant{\alpha}{\tau},\rho}(t))$
\item $\pi^1_{\tau,\rho}(\mathtt{let}_{\qquant{\forall}{\alpha}{\sigma},
  \mathtt{and}(\tau,\rho)}(s,t)) \to
  \mathtt{let}_{\qquant{\forall}{\alpha}{\sigma},\tau}(s,\pi^1_{\tau,
  \rho}(t))$
\item $\pi^2_{\tau,\rho}(\mathtt{let}_{\qquant{\forall}{\alpha}{\sigma},
  \mathtt{and}(\tau,\rho)}(s,t)) \to
  \mathtt{let}_{\qquant{\forall}{\alpha}{\sigma},\rho}(s,\pi^2_{\tau,
  \rho}(t))$
\item $\mathtt{case}_{\tau,\rho,\pi}(
  \mathtt{let}_{\qquant{\forall}{\alpha}{\sigma},\mathtt{or}(\tau,
  \rho)}(s,t),\abs{x}{u},\abs{y}{v}) \to
  \mathtt{let}_{\qquant{\forall}{\alpha}{\sigma},\pi}(s,\mathtt{case}_{
  \tau,\rho,\pi}(t,\abs{x}{u},\abs{y}{v}))$
\item $\mathtt{let}_{\qquant{\forall}{\beta}{\tau},\rho}(\mathtt{let}_{\qquant{\forall}{\alpha}{\sigma},\qquant{\forall}{\beta}{\tau}}(s,t),u) \to
  \mathtt{let}_{\qquant{\forall}{\alpha}{\sigma},\rho}(s,\mathtt{let}_{\qquant{\forall}{\beta}{\tau},\rho}(t,u))$
\end{itemize}

%\subsection*{General proposal}
%
%The following is a generalised alternative, that I think could also work.
%However, if we don't actually get any additional systems out of it, then
%that is probably not worth the effort of exploring.
%
%\begin{definition}
%We assume given the following sets of type constructors:
%\begin{itemize}
%\item $\TypeConstructors_{\mathsf{func}}$: a set of \emph{function
%  type constructors} (possibly indexed with a number of types) along
%  with an integer arity;
%\item $\TypeConstructors_{\mathsf{abs}}$: a set of \emph{abstraction
%  type constructors}
%\end{itemize}
%\end{definition}

\subsection{Interpreting interesting terms}

All terms will be mapped to interpretation terms, which are compared
using $\succ$ to hopefully obtain a decrease that can be used with
rule removal.

To start, all types must be mapped to the types of interpretation
terms.  This we do as follows:

\begin{definition}
A \emph{type mapping} is a function $\typeinterpret{\cdot}$ which,
given a type $\sigma$ returns an interpretation type $\typeinterpret{
\sigma}$ such that:
\begin{itemize}
\item $\typeinterpret{\sigma \arrtype \tau} =
  \quant{\alpha_1 \dots \alpha_n}{\typeinterpret{\sigma}[\beta_1:=
  \alpha_1,\dots,\beta_n:=\alpha_n] \arrtype
  \typeinterpret{\tau}}$, \\
  where $\{\beta_1,\dots,\beta_n\} = \FTV(\typeinterpret{\sigma})
  \setminus \FTV(\sigma)$ and $\alpha_1,\dots,\alpha_n$ fresh
  variables
\item $\typeinterpret{\quant{\alpha}{\sigma}} =
  \quant{\alpha}{\typeinterpret{\sigma}}$ \\
  (where fresh variables in $\FTV(\typeinterpret{\sigma})$ must also
  be chosen distinct from $\alpha$)
\end{itemize}

\CK{For now, I will not be any more specific than that.  It would be
  good to give a more well-defined recipe, but first I want to see how
  this will work out.  If it works out well, it will probably be
  something where you can indicate exactly how you can translate types
  recursively based on the root constructor/quantifier, much like the
  way the interpretation of terms is defined.}
%A \emph{type constructor mapping} is a function $\Typemap$ which assigns
%to each $(\con:n) \in \TypeConstructors$ a type $\sigma$ with
%$\FTV(\sigma) \subseteq \{ \alpha_1,\dots,\alpha_n \}$,
%and to each $(\con : n) \in \TypeQuantifiers$ a ``type'' $\sigma$ which
%may include one \emph{type context variable}\footnote{TODO: define this
%-- I basically mean a ``variable'' with given open spots to be
%substituted.} of arity $n$, denoted $\alpha[?_1,\dots,?_n]$ (with the
%$?_i$ replaced by types).  It is allowed for $\sigma$ to have
%additional free type variables.  We require that
%$\Typemap(\arrtype) = \alpha_1 \arrtype \alpha_2$ and $\Typemap(\forall)
%= \quant{\beta}{\alpha[\beta]}$.
%
%A type constructor mapping is extended to a function on types as follows:
%\[
%\begin{array}{rcl}
%\Typemap(\alpha) & = & \alpha \\
%\Typemap(\con(\sigma_1,\dots,\sigma_n)) & = &
%  \tau[\alpha_1:=\Typemap(\sigma_1),\dots,\alpha_n:=\Typemap(\sigma_n)]\ 
%  \text{if}\ \Typemap(\con) = \tau \\
%\Typemap(\qquant{\con}{\alpha_1 \dots \alpha_n}{\sigma}) & = &
%  \tau\ \text{with}\ \alpha[\rho_1,\dots,\rho_n]\ \text{replaced
%  by}\ \sigma[\rho_1,\dots,\rho_n] \\
%  & & \phantom{x}\hfill\text{if}\ \Typemap(\con) = \tau
%\end{array}
%\]
\end{definition}

To basis for mapping terms to interpretation terms is a function that
defines what should be done to function symbols:

\begin{definition}
A \emph{symbol mapping} is a function $\Termmap$ which assigns to each
instantiated function symbol $\mathtt{f}_{\rho_1,\dots,\rho_n} :
[\sigma_1 \times \dots \times \sigma_k] \arrtype \tau \in \Sigma$ an
interpretation term $\Termmap(\mathtt{f}_{\rho_1,\dots,\rho_n})$ such
that $\Termmap(\mathtt{f}_{\rho_1,\dots,\rho_n})$ has the following
form:
\[
\tabs{\alpha_1^1 \dots \alpha_1^{n_1} \dots \alpha_k^1 \dots \alpha_k^{n_k}}{
\abs{x_1:\typeinterpret{\sigma_1}[\vec{\beta_i}:=\vec{\alpha_i}] \dots
x_k:\typeinterpret{\sigma_k}[\vec{\beta_k}:=\vec{\alpha_k}]}{t}}
\]
Where for all $1 \leq i \leq k$:
\begin{itemize}
\item $\FTV(\typeinterpret{\sigma_i}) \setminus \FTV(\sigma_i) =
  \{ \beta_i^1,\dots,\beta_i^{n_i} \}$ (occurring in that order in
  $\typeinterpret{\sigma_i}$);
\item $\alpha_i^1,\dots,\alpha_i^{n_i}$ are fresh variables (disjoint
    from all other $\alpha_j^m$);
\end{itemize}
And moreover, if $\FTV(\typeinterpret{\tau}) \setminus \FTV(\tau) =
\{ \beta_0^1,\dots,\beta_0^{n_i} \}$, then there exist types
$\rho_1,\dots,\rho_{n_i}$ (possibly containing the variables
$\alpha_i^j$, but not $\beta_i^j$) such that:
\[
\{ x_1 : \typeinterpret{\sigma_1}[\vec{\beta_i}:=\vec{\alpha_i}],
\dots, x_k : \typeinterpret{\sigma_k}[\vec{\beta_k}:=\vec{\alpha_k}]
\} \vdash t : \typeinterpret{\tau}[\vec{\beta_0}:=\vec{\rho}]
\]
\end{definition}

Term interpretations are done in a type-conscious way.  Importantly,
terms are mapped not only to an interpretation term, but rather to a
pair of an interpretation term and a type substitution, as follows
below.
In the following, we say that a type substitution $\gamma$ \emph{fills
in the blanks caused by interpretation} for a type $\sigma$ if the
domain of $\gamma$ is included in $\FTV(\typeinterpret{\sigma})
\setminus \FTV(\sigma)$.

\begin{lemma}
Assume given a fixed symbol mapping $\Termmap$.

Additionally assume given environments $\Gamma,\Delta$, a term $s$
and a type $\sigma$ such that (1) $\Gamma \vdash s : \sigma$ and (2)
for all $(x : \tau) \in \Gamma$ there exists a substitution $\delta$
that fills in the blanks caused by interpretation of $\tau$ such
that $(x : \tau\delta) \in \Delta$.

Then the following recursive procedure defines a pair
$\fullinterpret{s,\Gamma,\Delta} = (\ \interpret{s},\ \gamma\ )$
of an interpretatin term $\interpret{s}$ and a type substitution
$\gamma$ that fills in the blanks caused by interpretation of
$\sigma$, such that $\Delta \vdash \interpret{s} :
\typeinterpret{\sigma}\gamma$:

\begin{itemize}
\item If $s$ is a variable, then let $\interpret{x} := x$ and let
  $\gamma$ be such that $(x : \sigma\gamma) \in \Delta$.
\item If $s = \tabs{\alpha}{s'}$ -- so $\sigma = \quant{\alpha}{
  \sigma'}$ -- then denote $\fullinterpret{s',\Gamma,\Delta} = (\ 
  \interpret{s'},\ \gamma\ )$, and let
  $\mathit{interpret}(s,\Gamma,\Delta) := (\ 
    \tabs{\alpha}{\interpret{s}},\ \gamma\ )$ if $\alpha \in \FTV(
    \interpret{\sigma'})$.
\item If $s = \abs{x:\tau}{s'}$ -- so $\sigma = \tau \arrtype \rho$,
  say -- then let $\FTV(\interpret{\tau}) \setminus \FTV(\tau) =
  \{ \beta_1,\dots,\beta_n \}$ and $\alpha_1,\dots,\alpha_n$ be fresh
  type variables.  If we denote $\fullinterpret{s',\Gamma \cup \{x:\tau\},
  \Delta \cup \{ x : \typeinterpret{\tau}[\beta_1:=\alpha_1,\dots,
  \beta_n:=\alpha_n]\}} = (\ \interpret{s'},\ \gamma\ )$, then we let
  $\fullinterpret{s,\Gamma,\Delta} := (\ \tabs{\alpha_1 \dots \alpha_n}{
  \abs{x:\typeinterpret{\tau}[\beta_1:=\alpha_1,\dots,\beta_n:=
  \alpha_n]}{\interpret{s}}},\ \gamma\ )$.
\item If $s = \mathtt{f}(s_1,\dots,s_k)$ with $\mathtt{f} :
  [\tau_1 \times \dots \times \tau_k] \arrtype \sigma \in \Sigma$ then
  denote $\fullinterpret{s_i,\Gamma,\Delta} = (\ \interpret{s_i},\ 
  \gamma_i\ )$ and $\FTV(\typeinterpret{\tau_i}) \setminus \FTV(\tau_i)
  = \{ \beta_i^1,\dots,\beta_i^{n_i} \}$ for $1 \leq i \leq k$.
  Write $\Termmap(\mathtt{f}) = \tabs{\alpha_1^1 \dots \alpha_k^{n_k}}{
  \abs{x_1 \dots x_k}{t}}$, and let $\delta$ be the substitution such
  that $\{ x_1 : \typeinterpret{\tau_1}[\vec{\beta_1}:=\vec{\alpha_1}]
  \dots x_k : \typeinterpret{\tau_k}[\vec{\beta_k}:=\vec{\alpha_k}] \}
  \vdash t : \typeinterpret{\sigma}\delta$.  Then
  $\fullinterpret{s,\Gamma,\Delta} := (\ \interpret{s}, \gamma\ )$,
  where:
  \begin{itemize}
  \item $\interpret{s} = t[\alpha_i^j:=\gamma_i(\beta_i^j) \mid 1 \leq
    i \leq k \wedge 1 \leq j \leq n_i][x_i:=\interpret{s_i} \mid
    1 \leq i \leq k]$
  \item $\gamma = \delta[\alpha_i^j:=\gamma_i(\beta_i^j) \mid 1 \leq
    i \leq k \wedge 1 \leq j \leq n_i][x_i:=\interpret{s_i} \mid
    1 \leq i \leq k]$
  \end{itemize}
\end{itemize}

\end{lemma}

\begin{proof}
Correctness is shown by induction on the form of $s$.

The variable case is obvious.

If $s = \tabs{\alpha}{s'}$, then note that $\typeinterpret{\sigma} =
\typeinterpret{\quant{\alpha}{\sigma'}} = \quant{\alpha}{
\typeinterpret{\sigma'}}$.  By the induction hypothesis,
$\Delta \vdash \interpret{s'} : \typeinterpret{\sigma'}\gamma$.
By definition, $\alpha \notin \FTV(\typeinterpret{\sigma'}) \setminus
\FTV(\sigma')$, so $\alpha \notin \mathit{domain}(\gamma)$.  Therefore,
$\Delta \vdash \tabs{\alpha}{\interpret{s}} :
\quant{\alpha}{\interpret{\sigma'}\gamma} = \interpret{\sigma}\gamma$.

If $s = \abs{x:\tau}{s'}$ with $\Gamma \cup \{ x : \tau \} \vdash
s' : \rho$, then $\typeinterpret{\sigma} = \quant{\alpha_1 \dots
\alpha_n}{\interpret{\tau}[\beta_1:=\alpha_1,\dots,\beta_n:=\alpha_n]
\arrtype \interpret{\rho}}$ with $\{\beta_1,\dots,\beta_n]\} =
\FTV(\interpret{\tau}) \setminus \FTV(\tau)$.  As the type variables
$\beta_1,\dots,\beta_n$ are chosen fresh, they do not occur in
$\typeinterpret{\rho}$.
By the induction hypothesis, $\Delta \cup \{ x :
\typeinterpret{\tau}[\beta_1:=\alpha_1,\dots,\beta_n:=\alpha_n] \} \vdash
\interpret{s'} : \interpret{\rho}\gamma$.
Thus, $\Delta \vdash \abs{x:\typeinterpret{\sigma}[\vec{\beta}:=
\vec{\alpha}]}{\interpret{s'}} : \typeinterpret{\tau}[\vec{\beta}:=
\vec{\alpha}] \arrtype (\typeinterpret{\rho}\gamma)$.
Since the variables that occur in $\typeinterpret{\rho}$ but not in
$\rho$ are chosen fresh, we can safely assume that they do not occur in
either $\tau$ or in $\Delta$, nor are they equal to any $\alpha_i$.
Thus, the type $\typeinterpret{\tau}[\vec{\beta}:=\vec{\alpha}]
\arrtype (\typeinterpret{\rho}\gamma)$ is equal to
$(\typeinterpret{\tau}[\vec{\beta}:=\vec{\alpha}] \arrtype
\typeinterpret{\rho})\gamma$.
We then easily obtain that $\tabs{\alpha_1 \dots \alpha_n}{\abs{x}{s'}} :
\quant{\alpha_1 \dots \alpha_n}{\interpret{\tau}[\vec{\beta}:=\vec{
\alpha}] \arrtype \rho}\gamma = \typeinterpret{\sigma}\gamma$.

Finally, suppose $s = \mathtt{f}(s_1,\dots,s_k)$ with $\mathtt{f} :
[\tau_1 \times \dots \times \tau_k] \arrtype \sigma \in \Sigma$ and
$\Gamma \vdash s_i : \tau_i$ for all $1 \leq i \leq k$.  By the
induction hypothesis, $\Delta \vdash \interpret{s_i} : \typeinterpret{
\tau_i}\gamma_i$ forall $i$.
Since all $\alpha_i^j$ are distinct by definition, we can define
$\chi$ as the type substitution that maps each $\alpha_i^j$ to
$\gamma_i(\beta_i^j)$; then we have $\Delta \vdash \interpret{s_i} :
(\typeinterpret{\tau_i}[\vec{\beta_i}:=\vec{\alpha_i}])\chi$.
We also have $\{ x_1 : \typeinterpret{\tau_1}[\vec{\beta_1}:=
\vec{\alpha_1}]\chi, \dots, x_k : \typeinterpret{\tau_k}
[\vec{\beta_k}:=\vec{\alpha_k}]\chi \} \vdash t\chi :
\typeinterpret{\sigma}\delta\chi$.
Substituting each $x_i$ by $\interpret{s_i}$, we thus have:
$\Delta \vdash t\chi[x_1:=\interpret{s_1},\dots,x_k:=\interpret{s_k}]
: \typeinterpret{\sigma}\delta\chi$.
\qed
\end{proof}

Thus, without reference to the type environment, we have:
\[
\begin{array}{rcl}
\interpret{x} & = & x \\
\interpret{\tabs{\alpha}{s}} & = & \tabs{\alpha}{\interpret{s}} \\
\interpret{\abs{x:\sigma}{s}} & = & \tabs{\alpha_1 \dots \alpha_n}{
  \abs{x:\typeinterpret{\sigma}[\beta_1:=\alpha_1,\dots,\beta_n:=
  \alpha_n]}{\interpret{s}}} \\
  & & \ \ \ \ \hfill
  \text{if}\ \FTV(\typeinterpret{\sigma}) \setminus \FTV(\sigma) =
  \{ \beta_1,\dots,\beta_n \}\ \text{and}\ \alpha_1,\dots,\alpha_n\ 
  \text{fresh} \\
\interpret{\mathtt{f}(s_1,\dots,s_k)} & = &
  t\gamma[x_1:=\interpret{s_1},\dots,x_n:=\interpret{s_n}] \\
  & & \ \ \ \ \hfill \text{if}\ 
  \Termmap(\mathtt{f}) = \tabs{\beta_1^1 \dots \beta_k^{
  n_k}}{\abs{x_1:\sigma_1 \dots x_n:\sigma_n}{t}}\ \text{and}\ \gamma\ 
  \text{is a type} \\
  & & \ \ \ \ \hfill \text{substitution
  on domain}\ \{ \beta_1^1,\dots,\beta_k^{n_k} \}\ \text{so this is
  well-typed} \\
\end{array}
\]

\subsection{Monotonicity}

By imposing certain monotonicity requirements on the choices for
$\Termmap$, we obtain a pair $(\succ,\succeq)$ that can be used with
rule removal (Theorem \ref{thm:ruleremove}).

\begin{lemma}
TODO
\end{lemma}

Thus, if $\interpret{\ell} \succ \interpret{r}$ for some rule, then
$\interpret{s}\downarrow \succ \interpret{t}\downarrow$ whenever
$s \arr{\Rules} t$ by that rule, and similar for $\succeq$.  This
provides a strategy for termination analysis using rule removal.

\section{Some useful lemmas}

\subsection{Orienting $\beta$-reduction}

\subsection{Orienting type instantiation}

\subsection{Other}

\section{Some systems of interest -- urzy\_emb?}

\subsection{Interpretations}

We use the following interpretation of the \emph{types}:

\begin{itemize}
\item $\typeinterpret{\bot} = \nat$
\item $\typeinterpret{\mathtt{or}(\sigma,\tau)} =
  \typeinterpret{\sigma} \times \typeinterpret{\tau}$
\item $\typeinterpret{\mathtt{and}(\sigma,\tau)} =
  \typeinterpret{\sigma} \times \typeinterpret{\tau}$
\item $\typeinterpret{\mathtt{prop}_i} = \nat$
\item $\typeinterpret{\quant{\alpha}{\sigma}} =
  \quant{\alpha}{\typeinterpret{\sigma}}$
\item $\typeinterpret{\qquant{\exists}{\alpha}{\sigma}} = \sigma$
\end{itemize}

We use the following symbol mapping:

\subsection{Rule orientation}

\end{document}

