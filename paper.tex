\documentclass[runningheads,a4paper]{llncs}
\pdfoutput=1

\bibliographystyle{plainurl}

\usepackage{amssymb}
\setcounter{tocdepth}{3}
\usepackage{enumerate}
\usepackage[colorlinks=true]{hyperref}
\usepackage{tikz}
\usepackage{xcolor,latexsym,amsmath,extarrows,alltt}
\usepackage{xspace}
\usepackage{booktabs}
\usepackage{mathtools}
\usepackage{enumitem}
\usepackage{stmaryrd}
\usepackage{microtype}
\usepackage{bussproofs}
\usepackage{multirow}

\newcommand{\Iterms}{\mathcal{I}}
\newcommand{\World}{\mathcal{W}}
\newcommand{\Rules}{\mathcal{R}}
\newcommand{\Typevars}{\mathcal{A}}
\newcommand{\Vars}{\mathcal{V}}
\newcommand{\Types}{\mathcal{T}}

\newcommand{\quant}[2]{\forall #1[#2]}
\newcommand{\interpret}[1]{\llparenthesis #1 \rrparenthesis}
\newcommand{\arr}[1]{\to_{#1}}
\newcommand{\arrtype}{\Rightarrow}
\newcommand{\abs}[2]{\lambda #1.#2}
\newcommand{\tabs}[2]{\Lambda #1.#2}
\newcommand{\app}[2]{#1 \cdot #2}
\newcommand{\tapp}[2]{#1 * #2}
\newcommand{\subst}[2]{#1:=#2}

\newcommand{\FTV}{\mathit{FTV}}
\newcommand{\FV}{\mathit{FV}}

\newcommand{\nat}{\mathtt{nat}}
\newcommand{\proj}{\pi}
\newcommand{\flatten}{\mathtt{flatten}}
\newcommand{\lift}{\mathtt{lift}}

\begin{document}

\mainmatter

\title{TODO list
  \thanks{The authors are supported by lots of people.}}
\subtitle{What we need to do to get where we want to be}

\author{Lukasz Czajka and Cynthia Kop}
\authorrunning{L. Czajka and C. Kop}
\institute{
Department of Computer Science, University of Copenhagen (DIKU)
\quad\quad\quad
Institute of Computer Science, Radboud University Nijmegen (RU)
\\
\email{lukaszcz@mimuw.edu.pl}
\quad\quad\quad
\email{C.Kop@cs.ru.nl}
}

\maketitle

\begin{abstract}
The introduction of this paper is a TODO list.
The remainder is meant for filling in the TODOs with what we already
have.
The hope is that the end result is something useful. :)
\end{abstract}

\section*{THE LIST}

\renewcommand{\theenumii}{\arabic{enumi}.\arabic{enumii}}

Here's what we need to do:
\begin{enumerate}
\item Define a world $\World$ and a well-founded ordering $\succ$ on
  $\World$:
  \begin{enumerate}
  \item Define a set of terms $\World$ typed under some variation of
    System F-$\omega$.
  \item Define relations $\succ$ and $\succeq$ on the elements of $\World$.
  \item Prove that $\succ$ is a well-founded ordering relation and that
    $\succeq$ is a compatible quasi-ordering.
  \end{enumerate}
\item Specify what systems we are interested in analysing, and prove
  standard results which will make their analysis doable.
  \begin{enumerate}
  \item Specify a form of system which includes all the systems of interest.
  \item Specify a default way of interpreting terms in these systems.
  \item Prove that in all such systems, using our way of interpreting
    terms: if $\interpret{\ell} \succ \interpret{r}$ (resp.\ $\interpret{
    \ell} \succeq \interpret{r}$) for a rule  $\ell \to r$, then
    $\interpret{s} \succ \interpret{t}$ whenever $s \arr{\Rules} t$ by
    this rule (resp.\ $\interpret{s} \succeq \interpret{t}$).
  \end{enumerate}
\item Obtain useful lemmas regarding these defaults.
  \begin{enumerate}
  \item $\interpret{s[x:=t]} = \interpret{s}[x:=\interpret{t}]$.
  \item $\interpret{s\sigma} = \interpret{s}\sigma$.
  \item \dots?
  \end{enumerate}
\item For some system of interest, prove its termination:
  \begin{enumerate}
  \item Present the system and give interpretations (following the
    default scheme) for all ways of constructing terms.
  \item Show that $\ell \succ r$ or $\ell \succeq r$ for all rules.
    Remove the rules which are oriented using $\succ$ and repeat,
    until all rules have been removed.
  \end{enumerate}
\end{enumerate}

\renewcommand{\theenumii}{\alph{enumii}}

We use \emph{rule removal}:

\begin{theorem}
Let $\Rules = \Rules_1 \cup \Rules_2$, and suppose that $\arr{\Rules_1}\:
\subseteq\:\succ$ and $\arr{\Rules_2}\:\subseteq\:\succeq$ for a
well-founded ordering $\succ$ and a compatible quasi-ordering $\succeq$.
Then $\arr{\Rules}$ is terminating if and only if $\arr{\Rules_2}$ is
(which is certainly the case if $\Rules_2 = \emptyset$).
\end{theorem}

\begin{proof}
By well-foundedness of $\succ$, every infinite decreasing $\arr{\Rules}$
sequence can only use finitely many steps using $\arr{\Rules_1}$.
\qed
\end{proof}

This gives rise to the following algorithm:
\begin{enumerate}
\item While $\Rules$ is non-empty:
  \begin{enumerate}
  \item Orient all rules in $\Rules$ using $\succeq$ or $\succ$; at least
    one of them must be oriented using $\succ$.
  \item Remove all rules ordered by $\succ$ from $\Rules$.
  \end{enumerate}
\end{enumerate}
If this algorithm succeeds, we have proven termination.

\section{Defining a world}

\subsection{Defining the set $\World$}

We define the set of types for interpretation terms.

\begin{definition}
We assume given an infinite set $\Typevars$ of \emph{type variables}.
The set $\Types$ of types is given by:
\begin{itemize}
\item $\alpha \in \Types$ for all $\alpha \in \Typevars$, and
  $\FTV(\alpha) = \{ \alpha \}$.
\item $\nat \in \Types$, and $\FTV(\nat) = \emptyset$.
\item $\sigma \arrtype \tau \in \Types$ if both $\sigma \in \Types$
  and $\tau \in \Types$, and $\FTV(\sigma \arrtype \tau) = \FTV(\sigma)
  \cup \FTV(\tau)$.
\item $\sigma \times \tau \in \Types$ if both $\sigma \in \Types$
  and $\tau \in \Types$, and $\FTV(\sigma \times \tau) = \FTV(\sigma)
  \cup \FTV(\tau)$.
\item $\quant{\alpha}{\sigma} \in \Types$ if $\alpha \in \Typevars$ and
  $\sigma \in \Types$, and $\FTV(\quant{\alpha}{\sigma}) =
  \FTV(\sigma) \setminus \{ \alpha \}$.
\end{itemize}
\end{definition}

The set $\Iterms$ of interpretation terms is now defined as follows.

\begin{definition}
We assume given an infinite set $\Vars$ of variables, and let $\Gamma$
refer to a mapping from a finite subset of $\Vars$ to the set of
interpretation types.  The set $\Iterms$ of interpretation terms consists
of all expressions $s$ such that $\Gamma \vdash s : \sigma$ can be
inferred for some interpretation type $\sigma$ and mapping $\Gamma$ by
the following clauses:
\begin{itemize}
\item $\Gamma \vdash n : \nat$ for every natural number $n$.
\item $\Gamma \vdash x : \sigma$ for every $(x : \sigma) \in \Gamma$.
\item $\Gamma \vdash \mathtt{f} : \sigma$ for all $(\mathtt{f} :
  \sigma)$ in the following set: $\{ \oplus_\sigma : \sigma \arrtype
  \sigma \arrtype \sigma,\ \otimes : \sigma \arrtype \sigma \arrtype
  \sigma,\ \proj^1_{\sigma,\tau} : (\sigma \times \tau) \arrtype
  \sigma,\ \proj^2_{\sigma,\tau} : (\sigma \times \tau) \arrtype \tau,\ 
  \flatten_{\sigma} : \sigma \arrtype \nat,\ 
  \lift_{\sigma} : \nat \arrtype \sigma
  \mid \sigma \in \Types \}$
\item $\Gamma \vdash \abs{x:\sigma}{s} : \sigma \arrtype \tau$ if $x
  \in \Vars$ and $\Gamma \uplus \{ x : \sigma \} \vdash s : \tau$.
\item $\Gamma \vdash \tabs{\alpha}{s} : \quant{\alpha}{\sigma}$ if
  $\alpha \in \Typevars$ and $\Gamma \vdash s : \sigma$ and for all
  $(x : \tau) \in \Gamma$: $\alpha \notin \FTV(\tau)$
\item $\Gamma \vdash \app{s}{t} : \tau$ if $\Gamma \vdash s :
  \sigma \arrtype \tau$ and $\Gamma \vdash t : \sigma$
\item $\Gamma \vdash \tapp{s}{\tau} : \sigma[\subst{\alpha}{\tau}]$ if
  $\Gamma \vdash s : \quant{\alpha}{\sigma}$\footnote{TODO: a
  definition of type substitution}
\end{itemize}
We say that $s$ is \emph{closed} if $\emptyset \vdash s : \sigma$.
\end{definition}

Note that for a given $\Gamma$, if $s$ is typable under $\Gamma$, then
there is only one choice for the type (this is easily proved by
induction on the form of $s$).  Thus, all closed terms have a unique
type.

\medskip
To define the world of \emph{final interpretation terms}, we will
normalise certain elements of $\Iterms$ using the relation $\leadsto$:

\begin{definition}
Let $s,t$ be closed interpretation terms of the same type $\sigma$ such
that $\FTV(\sigma) = \emptyset$.  We say $s \leadsto t$ if we can denote
$s = C[\ell]$ and $t = C[r]$\footnote{TODO: a definition of contexts.}
for some rule scheme $\ell \leadsto r$ in the following list:
\begin{itemize}
\item $\app{\app{\oplus_{\nat}}{n}}{m} \leadsto (n+m)$ 
\item $\app{\app{\otimes_{\nat}}{n}}{m} \leadsto (n \cdot m)$ 
\item $\app{\app{\oplus_{\sigma \arrtype \tau}}{(\abs{x:\sigma}{s})}}{
  (\abs{x:\sigma}{t})} \leadsto \abs{x:\sigma}{\app{\app{\oplus_\tau}{
  s}}{t}}$
\item TODO
\end{itemize}
\end{definition}

\subsection{Defining $\succ$ and $\succeq$}

\subsection{Properties of $\succ$ and $\succeq$}

\section{Systems of interest and their properties}

\subsection{Systems of interest}

\subsection{Interpreting interesting terms}

\subsection{Monotonicity}

\section{Some useful lemmas}

\subsection{Orienting $\beta$-reduction}

\subsection{Orienting type instantiation}

\subsection{Other}

\section{Some system of interest -- urzy\_emb?}

\subsection{Interpretations}

\subsection{Rule orientation}

\end{document}

